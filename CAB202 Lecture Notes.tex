%!TEX TS-program = xelatex
%!TEX options = -aux-directory=Debug -shell-escape -file-line-error -interaction=nonstopmode -halt-on-error -synctex=1 "%DOC%"
\documentclass{article}
% Packages

%% Math enhancements
\usepackage{amsmath} % Misc enhancements to math equations
\usepackage{cancel} % Draw diagonal lines and arrows in math equations
\usepackage{mathtools} % Starred versions of amsmath matrix environments; Multiline, cases, gathered environment
\usepackage{chngcntr} % Reset counter within sections
\usepackage{interval} % Format intervals
\intervalconfig{
    soft open fences
}

%% Symbols
\usepackage{amssymb} % Extended symbol collection - also loads amsfonts
\usepackage{stmaryrd} % Extra symbols

%% Fonts
\usepackage{mathrsfs} % Support \mathcal and \mathscr

%% Environments
\usepackage{amsthm} % Use theorems

%% Tables and arrays
\usepackage{booktabs} % Top and bottom rule for tabular
\usepackage{tabularx} % Advanced Tables

%% Lists
\usepackage{enumitem} % Itemize, enumerate, description environments

%% Page layout
\usepackage{geometry} % Page layout customisation
\usepackage{fancyhdr} % Page headers and footers
\usepackage{float} % Float objects such as figures and tables
\usepackage{tcolorbox} % Create boxed environments

%% Text enhancements
\usepackage[none]{hyphenat} % Disable hyphenation of long text
\usepackage{ragged2e} % Text alignment options

%% Referencing
\usepackage{tocbibind} % Adds bibliography to the Table of Contents
\usepackage{url} % Define urls

%% Graphics
\usepackage{graphicx} % Extension to graphics
% \graphicspath{ {./figures/} }

%% Miscellaneous
\usepackage[outputdir=Debug]{minted} % Typeset programming code
\usepackage{siunitx} % SI units package
\usepackage{derivative} % Derivative notation
\usepackage{pdfpages} % Import PDFs into document

\usepackage[hidelinks]{hyperref} % Handle cross-referencing
\usepackage{bookmark} % New bookmark organisation for hyperref

%% Unicode setup
\usepackage[warnings-off={mathtools-colon, mathtools-overbracket}]{unicode-math}
\setmathfont{Latin Modern Math}
\setmathfont[range={bb, bbit}, Scale=MatchUppercase]{TeX Gyre Pagella Math}
\setmathfont[range={\mathcal, \mathbfcal}, StylisticSet=1]{XITS Math}
\setmathfont[range={\mathscr}]{XITS Math}
\setmathfont[range={"2205}]{XITS Math} % chktex 18

% Preamble

%% Misc Commands

%%% Number Sets
\newcommand*{\N}{\mathbb{N}}
\newcommand*{\Z}{\mathbb{Z}}
\newcommand*{\Q}{\mathbb{Q}}
\newcommand*{\I}{\mathbb{I}}
\newcommand*{\R}{\mathbb{R}}
\newcommand*{\C}{\mathbb{C}}

%%% Empty set character
\let\oldemptyset\emptyset
\let\varnothing\relax
\newcommand{\varnothing}{\char"2205} % chktex 18

%%% Contradiction
\newcommand{\contradiction}{
\hspace{-1em}
{\hbox{
\setbox0=\hbox{\(\mkern-3mu{\times}\mkern-3mu\)}
\setbox1=\hbox to0pt{\hss\copy0\hss}
\copy0\raisebox{0.5\wd0}{\copy1}\raisebox{-0.5\wd0}{\box1}\box0}}
}

%%% Lines for matrices
\newcommand*{\vertbar}{\rule[-1ex]{0.5pt}{2.5ex}}
\newcommand*{\horzbar}{\rule[.5ex]{2.5ex}{0.5pt}}

%% Paired Delimiters
\DeclarePairedDelimiter{\ceil}{\lceil}{\rceil}
\DeclarePairedDelimiter{\floor}{\lfloor}{\rfloor}
\DeclarePairedDelimiter{\abracket}{\langle}{\rangle}
\DeclarePairedDelimiter{\abs}{\lvert}{\rvert}
\DeclarePairedDelimiter{\norm}{\lVert}{\rVert}

%% Probability Functions
\let\Pr\relax
\DeclareMathOperator{\Pr}{Pr}
\DeclareMathOperator{\E}{E}
\DeclareMathOperator{\Var}{Var}
\DeclareMathOperator{\Cov}{Cov}
\DeclareMathOperator{\Corr}{Corr}

\newcommand{\Perm}[2]{\prescript{#1}{}{P}_{#2}}

%% Hyperbolic Functions
\DeclareMathOperator{\arcsinh}{arcsinh}
\DeclareMathOperator{\arccosh}{arccosh}
\DeclareMathOperator{\arctanh}{arctanh}
\DeclareMathOperator{\arccoth}{arccoth}
\DeclareMathOperator{\arcsech}{arcsech}
\DeclareMathOperator{\arccsch}{arccsch}

%% Linear Algebra
%%% Augmented matrices
\makeatletter
\renewcommand*\env@matrix[1][*\c@MaxMatrixCols c]{%
    \hskip -\arraycolsep
    \let\@ifnextchar\new@ifnextchar
    \array{#1}}
\makeatother

%%% Operators
\let\det\relax
\DeclareMathOperator{\det}{det}
\DeclareMathOperator{\Tr}{Tr}
\DeclareMathOperator{\diag}{diag}
\DeclareMathOperator{\adj}{adj}

\DeclareMathOperator{\vspan}{span}
\DeclareMathOperator{\vref}{ref}
\DeclareMathOperator{\vrref}{rref}

\DeclareMathOperator{\vrank}{rank}
\DeclareMathOperator{\vnull}{null}

\DeclareMathOperator{\proj}{proj}

\DeclareMathOperator{\vim}{im}
\DeclareMathOperator{\vcoim}{coim}
\DeclareMathOperator{\vker}{ker}
\DeclareMathOperator{\vcoker}{coker}

\newcommand{\columnspace}[1]{\mathcal{C}\left(\symbf{#1}\right)}
\newcommand{\rowspace}[1]{\mathcal{C}\left(\symbf{#1}^{\top}\right)}
\newcommand{\nullspace}[1]{\mathcal{N}\left(\symbf{#1}\right)}
\newcommand{\leftnullspace}[1]{\mathcal{N}\left(\symbf{#1}^{\top}\right)}

%% Additional operators
\DeclareMathOperator{\erf}{erf}

% Theorems
\theoremstyle{definition}
\newtheorem{definition}{Definition}[section]

\theoremstyle{plain}
\newtheorem{theorem}{Theorem}[subsection]
\newtheorem{corollary}{Corollary}[theorem]
\newtheorem{lemma}{Lemma}[theorem]
\newtheorem{axiom}{Axiom}

\theoremstyle{remark}
\newtheorem{remark}{Remark}
\newtheorem{note}{Note}[subsection]
\newtheorem*{statement}{Statement}

\newenvironment{examples}[1][Examples]{\let\qed\relax\proof[#1]\mbox{}\\*}{\endproof}
\newenvironment{question}[1][Question]{\let\qed\relax\proof[#1]\mbox{}\\*}{\endproof}
\newenvironment{solution}[1][Solution]{\let\qed\relax\proof[#1]\mbox{}\\*}{\endproof}

\newenvironment{proofcase}[1]{\proof[Case #1]\mbox{}}{\endproof}

%% Box styles
\tcbuselibrary{skins}
\newtcolorbox{tcolorboxlarge}[1][]{
    skin=enhanced,
    boxrule=0pt,
    frame hidden,
    sharp corners,
    borderline west={0.5pt}{0pt}{black},
    borderline east={0.5pt}{0pt}{black},
    enlarge left by=10pt,
    width=\linewidth-20pt,
    opacityback=0,
    coltitle=black,
    fonttitle=\large\bfseries,
    #1
}

\newtcolorbox{tcolorboxcols}[1][]{
    skin=enhanced,
    boxrule=0pt,
    frame hidden,
    sharp corners,
    borderline west={0.5pt}{0pt}{black},
    opacityback=0,
    coltitle=black,
    fonttitle=\large\bfseries,
    #1
}

%% Reset counter within subsections
\counterwithin*{equation}{section}
\counterwithin*{equation}{subsection}
\counterwithin*{remark}{subsection}

%% Page layout setup
\pagestyle{fancy}
\setlength\headheight{24pt}
\setlength\parindent{0pt} % Indent first line of new paragraphs


% Additional packages & macros
\usepackage{xcolor}
\newcommand{\keyword}[1]{\textcolor[rgb]{0.00,0.50,0.00}{\textbf{#1}}}

\setminted{
    escapeinside=||
}

% Header and footer
\newcommand{\unitName}{Microprocessors and Digital Systems}
\newcommand{\unitTime}{Semester 2, 2022}
\newcommand{\unitCoordinator}{Dr Mark Broadmeadow}
\newcommand{\documentAuthors}{Tarang Janawalkar}

\fancyhead[L]{\unitName}
\fancyhead[R]{\leftmark}
\fancyfoot[C]{\thepage}

% Copyright
\usepackage[
    type={CC},
    modifier={by-nc-sa},
    version={4.0},
    imagewidth={5em},
    hyphenation={raggedright}
]{doclicense}

\date{}

\begin{document}
%
\begin{titlepage}
    \vspace*{\fill}
    \begin{center}
        \LARGE{\textbf{\unitName}} \\[0.1in]
        \normalsize{\unitTime} \\[0.2in]
        \normalsize\textit{\unitCoordinator} \\[0.2in]
        \documentAuthors
    \end{center}
    \vspace*{\fill}
    \doclicenseThis
    \thispagestyle{empty}
\end{titlepage}
\newpage
%
\tableofcontents
\newpage
%
\section{Microcontroller Fundamentals}
\subsection{Architecture of a Computer}
\begin{definition}[Computer]
    A computer is a digital electronic machine that can be programmed to carry
    out sequences of arithmetic or logical operations (computation) automatically.
\end{definition}
\begin{definition}[Control unit]
    The control unit interprets the instructions and decides what actions to take.
\end{definition}
\begin{definition}[Arithmetic logic unit]
    The arithmetic logic unit (ALU) performs computations required by the control unit.
\end{definition}
\subsection{Microprocessors \& Microcontrollers}
While a microcontroller puts the CPU and all peripherals onto the same chip,
a microprocessor houses a more powerful CPU on a single chip that connects to external peripherals.
The peripherals include memory, I/O, and control units.

The QUTy uses a microcontroller called ATtiny1626, that are within a family of microcontrollers called AVRs.
\subsection{ATtiny1626 Microcontroller}
The ATtiny1626 microcontroller has the following features:
\begin{itemize}
    \item CPU:\@ AVR Core (AVRxt variant)
    \item Memory:\@
          \begin{itemize}
              \item Flash memory (16KB) used to store program instructions in memory
              \item SRAM (2KB) used to store data in memory
              \item EEPROM (256B)
          \end{itemize}
    \item Peripherals:\@ Implemented in hardware (part of the chip) in order to offload complexity
\end{itemize}
\subsubsection{Flash Memory}
\begin{itemize}
    \item Non-volatile --- memory is not lost when power is removed
    \item Inexpensive
    \item Slower than SRAM
    \item Can only erase in large chunks
    \item Typically used to store programme data
    \item Generally read-only. Programmed via an external tool, which is loaded once and remains static during the lifetime of the program
    \item Writing is slow
\end{itemize}
\subsubsection{SRAM}
\begin{itemize}
    \item Volatile --- memory is lost when power is removed
    \item Expensive
    \item Faster than flash memory and is used to store variables and temporary data
    \item Can access individual bytes (large chunk erases are not required)
\end{itemize}
\subsubsection{EEPROM}
\begin{itemize}
    \item Older technology
    \item Expensive
    \item Non-volatile
    \item Can erase individual bytes
\end{itemize}
\subsection{AVR Core}
\begin{definition}[Computer programme]
    A computer programme is a sequence or set of instructions in a programming language
    for a computer to execute.
\end{definition}
The main function of the AVR Core Central Processing Unit (CPU) is to ensure correct program execution.
The CPU must, therefore, be able to access memory, perform calculations, control peripherals, and handle interrupts.
Some key characteristics of the AVR Core are:
\begin{itemize}
    \item 8-bit Reduced Instruction Set Computer (RISC)
    \item 32 working registers (R0 to R31)
    \item Program Counter (PC) --- location in memory where the program is stored
    \item Status Register (SREG) --- stores key information from calculations performed by the ALU (i.e., whether a result is negative)
    \item Stack Pointer --- temporary data that doesn't fit into the registers
    \item 8-bit core --- all data, registers, and operations, operate within 8-bits
\end{itemize}
\subsection{Programme Execution}
At the time of reset, PC = 0. The following steps are then performed:
\begin{enumerate}
    \item Fetch instruction (from memory)
    \item Decode instruction (decode binary instruction)
    \item Execute instruction:
          \begin{itemize}
              \item Execute an operation
              \item Store data in data memory, the ALU, a register, or update the stack pointer
          \end{itemize}
    \item Store result
    \item Update PC (move to next instruction or if instruction is longer than 1 word, increment twice. The program can also move to another point in the program that has an address \(k\), through jumps.)
\end{enumerate}
This is illustrated in the following figure:
\begin{figure}[H]
    \centering
    \includegraphics[height = 12cm, keepaspectratio = true]{figures/AVR_CPU.pdf}
    % \caption{} % \label{}
\end{figure}
\subsection{Instructions}
\begin{itemize}
    \item The CPU understands and can execute a limited set of instructions --- \textasciitilde88 unique instructions for the ATtiny1626
    \item Instructions are encoded in programme memory as opcodes. Most instructions are two bytes long, but some instructions are four bytes long
    \item The AVR Instruction Set Manual describes all of the available instructions, and how they are translated into opcodes
    \item Instructions fall loosely into five categories:
          \begin{itemize}
              \item Arithmetic and logic --- ALU
              \item Change of flow --- jumping to different sections of the code or making decisions
              \item Data transfer --- moving data in/out of registers, into the data space, or into RAM
              \item Bit and bit-test --- looking at data in registers (which bits are set or not set)
              \item Control --- changing what the CPU is doing
          \end{itemize}
\end{itemize}
\subsection{Interacting with memory and peripherals}
\begin{itemize}
    \item The CPU interacts with both memory and peripherals via the data space
    \item From the perspective of the CPU, the data space is single large array of
          locations that can be read from, or written to, using an address
    \item We control peripherals by reading from, and writing to, their registers
    \item Each peripheral register is assigned a unique address in the data space
    \item When a peripheral is accessed in this manner we refer to it as being
          memory mapped, as we access them as if they were normal memory
    \item Different devices, peripherals and memory can be included in a memory map
          (and sometimes a device can be accessed at multiple different addresses)
\end{itemize}
\subsection{Memory map}
\begin{figure}[H]
    \centering
    \includegraphics[height = 8cm, keepaspectratio = true]{figures/memory_map.pdf}
    % \caption{Memory map.} % \label{}
\end{figure}
\subsection{Assembly code}
\begin{itemize}
    \item The opcodes placed into programme memory are called
          machine code (i.e., code the machine operates on directly)
    \item We don't write machine code directly as it is:
          \begin{itemize}
              \item Not human readable
              \item Prone to errors (swapping a single bit can completely change the operation)
          \end{itemize}
    \item Instead we can write instructions directly in assembly code
    \item We use instruction mnemonics to specify each instruction:\@ \mintinline{ca65}{|\keyword{ldi}|}, \mintinline{ca65}{|\keyword{add}|}, \mintinline{ca65}{|\keyword{sts}|}, \mintinline{ca65}{jmp}, \dots
    \item An assembler takes assembly code and translates it into opcodes that can
          be loaded into programme memory
\end{itemize}
\section{Digital Representations and Operations}
\subsection{Digital Systems}
A \textbf{bit}\footnote{The term \textit{bit} comes from \textbf{b}inary dig\textbf{it}.}
is the most basic unit of information in a digital system.
A bit encodes a logical state with one of two possible values (i.e., binary).
These states are often referred to as:
\begin{itemize}
    \item true, false
    \item high, low (voltage states)
    \item on, off (logical states)
    \item set, reset
    \item 1, 0
\end{itemize}
A sequence of \textit{eight} bits is known as a \textbf{byte}, and it is the most
common representation of data in digital systems.
A sequence of \textit{four} bits is known as a \textbf{nibble}.

A sequence of \(n\) bits can represent up to \(2^n\) states.
\subsection{Representation}
\subsubsection{Binary Representation}
The \textbf{binary system} is a base-2 system that uses a sequence of bits to represent a number.
Bits are written left-to-right from \textbf{most significant} to \textbf{least significant} bit.

The first bit is the ``most significant'' bit because it is associated with the highest value in the sequence (coefficient of the highest power of two).
\begin{itemize}
    \item The \textbf{least significant bit} (LSB) is at bit index 0.
    \item The \textbf{most significant bit} (MSB) is at bit index \(n - 1\) in an \(n\)-bit sequence.
\end{itemize}
\begin{align*}
    0000_2 & = 0 & 0100_2 & = 4 & 1000_2 & = 8  & 1100_2 = 12 \\
    0001_2 & = 1 & 0101_2 & = 5 & 1001_2 & = 9  & 1101_2 = 13 \\
    0010_2 & = 2 & 0110_2 & = 6 & 1010_2 & = 10 & 1110_2 = 14 \\
    0011_2 & = 3 & 0111_2 & = 7 & 1011_2 & = 11 & 1111_2 = 15
\end{align*}
The subscript 2 indicates that the number is represented using a base-2 system.
\subsubsection{Hexadecimal Representation}
The \textbf{hexadecimal system} (hex) is a base-16 system. As we need 16 digits in this system, we use the letters A to F to represent digits 10 to 15.

Hex is a convenient notation when working with digital systems as each hex digit maps to a nibble.
\begin{align*}
    0_{16} & = 0000_2 & 4_{16} & = 0100_2 & 8_{16} & = 1000_2 & C_{16} = 1100_2 \\
    1_{16} & = 0001_2 & 5_{16} & = 0101_2 & 9_{16} & = 1001_2 & D_{16} = 1101_2 \\
    2_{16} & = 0010_2 & 6_{16} & = 0110_2 & A_{16} & = 1010_2 & E_{16} = 1110_2 \\
    3_{16} & = 0011_2 & 7_{16} & = 0111_2 & B_{16} & = 1011_2 & F_{16} = 1111_2
\end{align*}
\subsubsection{Numeric Literals}
When a fixed value is declared directly in a program, it is referred to as a \textbf{literal}.
Generally, numeric literals can be expressed as either binary, decimal, or hexadecimal, so we
use prefixes to denote various bases. Typically,
\begin{itemize}
    \item \textbf{Binary} notation requires the prefix \mintinline{ca65}{0b}
    \item \textbf{Decimal} notation does not require prefixes
    \item \textbf{Hexadecimal} notation requires the prefix \mintinline{ca65}{0x}
\end{itemize}
For example, \mintinline{ca65}{0x80 |=| 0b10000000 |=| 128}.
\subsection{Unsigned Integers}
The \textbf{unsigned integers} represent the set of counting (natural) numbers, starting at 0.
In the \textbf{decimal system} (base-10), the unsigned integers are encoded using a sequence of decimal digits (0--9).

The decimal system is a \textbf{positional numeral system}, where the contribution of each digit is determined by its position.
For example,
\begin{align*}
    278_{10} & = 2 \times 10^2 &  & + 7 \times 10^1 &  & + 8 \times 10^0 \\
             & = 2 \times 100  &  & + 7 \times 10   &  & + 8 \times 1    \\
             & = 200           &  & + 70            &  & + 8             \\
\end{align*}
In the \textbf{binary system} (base-2) the unsigned integers are encoded using a sequence of binary digits (0--1)
in the same manner. For example,
\begin{align*}
    10101_2 & = 1 \times 2^4 &  & + 0 \times 2^3 &  & + 1 \times 2^2 &  & + 0 \times 2^1 &  & + 1 \times 2^0 \\
            & = 1 \times 16  &  & + 0 \times 8   &  & + 1 \times 4   &  & + 0 \times 2   &  & + 1 \times 1   \\
            & = 16           &  & + 0            &  & + 4            &  & + 0            &  & + 1            \\
            & = 21_{10}
\end{align*}
The range of values an \(n\)-bit binary number can hold when encoding an unsigned integer is 0 to \(2^n - 1\).
\begin{table}[H]
    \centering
    \begin{tabular}{c c}
        \toprule
        \textbf{No.\ of Bits} & \textbf{Range}                        \\
        \midrule
        8                     & \(0\)--\(255\)                        \\
        16                    & \(0\)--\(\num{65535}\)                \\
        32                    & \(0\)--\(\num{4294967295}\)           \\
        64                    & \(0\)--\(\num{18446744073709551615}\) \\
        \bottomrule
    \end{tabular}
    \caption{Range of available values in binary representations.} % \label{}
\end{table}
\subsection{Signed Integers}
Signed integers are used to represent integers that can be positive or negative.
The following representations allow us to encode negative integers using a sequence of binary bits:
\begin{itemize}
    \item Sign-magnitude
    \item One's complement
    \item Two's complement (most common)
\end{itemize}
\subsubsection{Sign-Magnitude}
In sign-magnitude representation, the most significant bit encodes the sign of the
integer. In an 8-bit sequence, the remaining 7-bits are used to
encode the value of the bit.
\begin{itemize}
    \item If the sign bit is 0, the remaining bits represent a positive value,
    \item If the sign bit is 1, the remaining bits represent a negative value.
\end{itemize}
As the sign bit consumes one bit from the sequence, the range of values that can be
represented by an \(n\)-bit sign-magnitude encoded bit sequence is:
\begin{equation*}
    -\left( 2^{n - 1} - 1 \right) \text{ to } 2^{n - 1} - 1
\end{equation*}
For 8-bit sequences, this range is: \(-127\) to \(127\).

However, there are some issues with this representation.
\begin{enumerate}
    \item There are two ways to represent zero: \mintinline{ca65}{0b10000000 |=| 0}, or \mintinline{ca65}{0b00000000 |=| -0}.
    \item Arithmetic and comparison requires inspecting the sign bit
    \item The range is reduced by 1 (due to the redundant zero representation)
\end{enumerate}
\subsubsection{One's Complement}
In one's complement representation, a negative number is represented by
inverting the bits of a positive number (i.e., \(0 \to 1\) and \(1 \to 0\)).

The range of values are still the same:
\begin{equation*}
    -\left( 2^{n - 1} - 1 \right) \text{ to } 2^{n - 1} - 1
\end{equation*}
however, this representation tackles the second problem in the previous representation as
addition is performed via standard binary addition with \textit{end-around carry} (carry bit is added onto result).
\begin{equation*}
    a - b = a + \left( \text{\textasciitilde} b \right) + C.
\end{equation*}
\subsubsection{Two's Complement}
In two's complement representation, the most significant bit encodes a negative weighting of
\(-2^{n - 1}\). For example, in 8-bit sequences, index-7 represents a value of \(-128\).

The two's complement is calculated by adding 1 to the one's complement.

The range of values are:
\begin{equation*}
    -2^{n - 1} \text{ to } 2^{n - 1} - 1
\end{equation*}
This representation is more efficient than the previous because \mintinline{ca65}{0} has a single representation
and subtraction is performed by adding the two's complement of the subtrahend.
\begin{equation*}
    a - b = a + \left( \text{\textasciitilde} b + 1 \right).
\end{equation*}
\subsection{Logical Operators}
\subsubsection{Boolean Functions}
A Boolean function is a function whose arguments and results assume values
from a two-element set, (usually \(\left\{ 0,\: 1 \right\}\) or \mintinline{text}{{false, true}}).

These functions are also referred to as \textit{logical functions} when they operate on bits.
The most common logical functions available to microprocessors and most programming languages are:
\begin{itemize}
    \item Negation: \mintinline{ca65}{|\keyword{NOT}|}
    \item Conjunction: \mintinline{ca65}{AND}
    \item Disjunction: \mintinline{ca65}{|\keyword{OR}|}
    \item Exclusive disjunction: \mintinline{ca65}{|\keyword{XOR}|}
\end{itemize}
By convention, we map a bit value of \mintinline{ca65}{0} to \mintinline{ca65}{false}, and a bit value of \mintinline{ca65}{1} to \mintinline{ca65}{true}.
\subsubsection{Negation}
\mintinline{ca65}{NOT} is a unary operator that is used to \textbf{invert} a bit.
It is typically expressed as:
\begin{itemize}
    \item \mintinline{ca65}{|\keyword{NOT}|} \(a\)
    \item \textasciitilde\(a\)
    \item \(\overline{a}\)
\end{itemize}
Truth table:
\begin{table}[H]
    \centering
    \begin{tabular}{c c}
        \toprule
        \textbf{\(a\)} & \mintinline{ca65}{|\keyword{NOT}|} \(a\) \\
        \midrule
        0              & 1                                        \\
        1              & 0                                        \\
        \bottomrule
    \end{tabular}
\end{table}
\subsubsection{Conjunction}
\mintinline{ca65}{AND} is a binary operator whose output is true if \textbf{both} inputs are \textbf{true}.
It is typically expressed as:
\begin{itemize}
    \item \(a\) \mintinline{ca65}{AND} \(b\)
    \item \(a\) \mintinline{ca65}{&} \(b\)
    \item \(a \cdot b\)
    \item \(a \land b\)
\end{itemize}
Truth table:
\begin{table}[H]
    \centering
    \begin{tabular}{c c c}
        \toprule
        \textbf{\(a\)} & \textbf{\(b\)} & \textbf{\(a\) \mintinline{ca65}{AND} \(b\)} \\
        \midrule
        0              & 0              & 0                                           \\
        0              & 1              & 0                                           \\
        1              & 0              & 0                                           \\
        1              & 1              & 1                                           \\
        \bottomrule
    \end{tabular}
\end{table}
\subsubsection{Disjunction}
\mintinline{ca65}{|\keyword{OR}|} is a binary operator whose output is true if \textbf{either} input is \textbf{true}.
It is typically expressed as:
\begin{itemize}
    \item \(a\) \mintinline{ca65}{|\keyword{OR}|} \(b\)
    \item \(a\) \mintinline{ca65}{|\vert|} \(b\)
    \item \(a + b\)
    \item \(a \lor b\)
\end{itemize}
Truth table:
\begin{table}[H]
    \centering
    \begin{tabular}{c c c}
        \toprule
        \textbf{\(a\)} & \textbf{\(b\)} & \(a\) \mintinline{ca65}{|\keyword{OR}|} \(b\) \\
        \midrule
        0              & 0              & 0                                             \\
        0              & 1              & 0                                             \\
        1              & 0              & 1                                             \\
        1              & 1              & 1                                             \\
        \bottomrule
    \end{tabular}
\end{table}
\subsubsection{Exclusive Disjunction}
\mintinline{ca65}{|\keyword{XOR}|} (Exclusive \mintinline{ca65}{|\keyword{OR}|}) is a binary operator whose output is true if \textbf{only one} input is \textbf{true}.
It is typically expressed as:
\begin{itemize}
    \item \(a\) \mintinline{ca65}{|\keyword{XOR}|} \(b\)
    \item \(a\) \mintinline{ca65}{^} \(b\)
    \item \(a \oplus b\)
\end{itemize}
Truth table:
\begin{table}[H]
    \centering
    \begin{tabular}{c c c}
        \toprule
        \textbf{\(a\)} & \textbf{\(b\)} & \(a\) \mintinline{ca65}{|\keyword{XOR}|} \(b\) \\
        \midrule
        0              & 0              & 0                                              \\
        0              & 1              & 1                                              \\
        1              & 0              & 1                                              \\
        1              & 1              & 0                                              \\
        \bottomrule
    \end{tabular}
\end{table}
\subsubsection{Bitwise Operations}
When applying logical operators to a sequence of bits, the operation is performed in a \textbf{bitwise} manner. The result of each operation is stored in the corresponding bit index also.
\subsection{Bit Manipulation}
Often we need to modify individual bits within a byte, \textbf{without} modifying other bits.
This is accomplished by performing a bitwise operation on the byte using a \textbf{bit mask} or \textbf{bit field}.

These operations can:
\begin{itemize}
    \item \textbf{Set} specific bits (change value to \mintinline{ca65}{1})
    \item \textbf{Clear} specific bits (change value to \mintinline{ca65}{0})
    \item \textbf{Toggle} specific bits (change values from \(0 \to 1\), or \(1 \to 0\))
\end{itemize}
\subsubsection{Setting Bits}
To \textbf{set} a bit, we take the bitwise \mintinline{ca65}{|\keyword{OR}|} of the byte, with a bit mask
that has a \textbf{1} in each position where the bit should be set.
\begin{figure}[H]
    \centering
    \includegraphics[height = 4cm, keepaspectratio = true]{figures/bit_set.pdf}
    \caption{Setting bits using the logical or.} % \label{}
\end{figure}
\subsubsection{Clearing Bits}
To \textbf{clear} a bit, we take the bitwise \mintinline{ca65}{AND} of the byte, with a bit mask
that has a \textbf{0} in each position where the bit should be cleared.
\begin{figure}[H]
    \centering
    \includegraphics[height = 4cm, keepaspectratio = true]{figures/bit_clear.pdf}
    \caption{Clearing bits using the logical and.} % \label{}
\end{figure}
\subsubsection{Toggling Bits}
To \textbf{toggle} a bit, we take the bitwise \mintinline{ca65}{|\keyword{XOR}|} of the byte, with a bit mask
that has a \textbf{1} in each position where the bit should be toggled.
\begin{figure}[H]
    \centering
    \includegraphics[height = 4cm, keepaspectratio = true]{figures/bit_toggle.pdf}
    \caption{Toggling bits using the logical exclusive or.} % \label{}
\end{figure}
Other bitwise operations act on the entire byte.
\begin{itemize}
    \item One's complement (bitwise \mintinline{ca65}{|\keyword{NOT}|})
    \item Two's complement (bitwise \mintinline{ca65}{|\keyword{NOT}|} + 1)
    \item Shifts
          \begin{itemize}
              \item Logical
              \item Arithmetic (for signed integers)
          \end{itemize}
    \item Rotations
\end{itemize}
\subsubsection{One's Complement}
The one's complement of a byte inverts every bit in the operand. This is done by
taking the bitwise \mintinline{ca65}{|\keyword{NOT}|} of the byte.

Similarly, we can subtract the byte from \mintinline{ca65}{0xFF} to get the one's complement.
\subsubsection{Two's Complement}
The two's complement of a byte is the one's complement of the byte plus one.
Therefore, we can apply take the bitwise \mintinline{ca65}{|\keyword{NOT}|} of the byte, and then add one to it.
\subsubsection{Shifts}
Shifts are used to move bits within a byte. In many programming languages this is represented by two greater than \mintinline{ca65}{>>} or two less than \mintinline{ca65}{<<} characters.
\begin{equation*}
    a \gg s
\end{equation*}
shifts the bits in \(a\) by \(s\) places to the right while adding \mintinline{ca65}{0}'s to the MSB.\
\begin{figure}[H]
    \centering
    \includegraphics[height = 2cm, keepaspectratio = true]{figures/logical_right_shift.pdf}
    \caption{Right shift using \mintinline{ca65}{lsr} in AVR Assembly.} % \label{}
\end{figure}
Similarly
\begin{equation*}
    a \ll s
\end{equation*}
shifts the bits in \(a\) by \(s\) places to the left while adding \mintinline{ca65}{0}'s to the LSB.\
\begin{figure}[H]
    \centering
    \includegraphics[height = 2cm, keepaspectratio = true]{figures/logical_left_shift.pdf}
    \caption{Left shift using \keyword{\ttfamily{lsl}} in AVR Assembly.} % \label{}
\end{figure}
When using signed integers, the arithmetic shift is used to preserve the value of the sign bit when shifting.
\begin{figure}[H]
    \centering
    \includegraphics[height = 2cm, keepaspectratio = true]{figures/arithmetic_right_shift.pdf}
    \caption{Arithmetic right shift using \keyword{\ttfamily{asr}} in AVR Assembly.} % \label{}
\end{figure}
Left shifts are used to multiply numbers by 2, whereas right shifts are used to divide numbers by 2 (with truncation).
\subsubsection{Rotations}
Rotatations are used to shift bits with a carry from the previous instruction.
\begin{figure}[H]
    \centering
    \includegraphics[height = 2cm, keepaspectratio = true]{figures/rotate_left.pdf}
    \caption{Rotate left using \mintinline{ca65}{rol} in AVR Assembly.} % \label{}
\end{figure}
\begin{figure}[H]
    \centering
    \includegraphics[height = 2cm, keepaspectratio = true]{figures/rotate_right.pdf}
    \caption{Rotate right using \mintinline{ca65}{ror} in AVR Assembly.} % \label{}
\end{figure}
Here the blue bit is carried from the previous instruction, and the carry bit is updated
to the value of the bit that was shifted out.
Rotations are used to perform multi-byte shifts and arithmetic operations.
\subsection{Arithmetic Operations}
\subsubsection{Addition}
Addition is performed using the same process as decimal addition except we only use two digits, 0 and 1.
\begin{enumerate}
    \item \mintinline{ca65}{0b0 + 0b0 = 0b0}
    \item \mintinline{ca65}{0b0 + 0b1 = 0b1}
    \item \mintinline{ca65}{0b1 + 0b1 = 0b10}
\end{enumerate}
When adding two 1's, we carry the result into the next bit position as we would with a 10 in decimal addition.
In AVR Assembly, we can use the \mintinline{ca65}{|\keyword{add}|} instruction to add two bytes. The following
example adds two bytes.
\begin{minted}{ca65}
; Accumulator
|\keyword{ldi}| r16, 0

; First number
|\keyword{ldi}| r17, 29
|\keyword{add}| r16, 17 ; R16 <- R16 + R17 = 0 + 29 = 29

; Second number
|\keyword{ldi}| r17, 118
|\keyword{add}| r16, 17 ; R16 <- R16 + R17 = 29 + 118 = 147
\end{minted}
Below is a graphical illustration of the above code.
\begin{figure}[H]
    \centering
    \includegraphics[height = 3cm, keepaspectratio = true]{figures/add.pdf}
    \caption{Overflow addition using \keyword{\ttfamily{add}} in AVR Assembly.} % \label{}
\end{figure}
\subsubsection{Overflows}
When the sum of two 8-bit numbers is greater than 8-bit (255), an \textbf{overflow} occurs.
Here we must utilise a second register to store the high byte so that the result is represented as
a 16-bit number.

To avoid loss of information, a \textbf{carry bit} is used to indicate when an overflow has occurred.
This carry bit can be added to the high byte in the event that an overflow occurs.
This is because the carry bit is 0 when the sum is less than 256, and 1 when the sum is greater than 255.

The following example shows how to use the \mintinline{ca65}{adc} instruction to carry the carry bit when an overflow occurs.
\begin{minted}{ca65}
; Low byte
|\keyword{ldi}| r30, 0
; High byte
|\keyword{ldi}| r31, 0

; Empty byte for adding carry bit
|\keyword{ldi}| r29, 0

; First number
|\keyword{ldi}| r16, 0b11111111
; Add to low byte
|\keyword{add}| r30, r16 ; R30 <- R30 + R16 = 0 + 255 = 255, C <- 0
; Add to high byte
adc r31, r29 ; R31 <- R31 + R29 + C = 0 + 0 + 0 = 0

; Second number
|\keyword{ldi}| r16, 0b00000001
; Add to low byte
|\keyword{add}| r30, r16 ; R30 <- R30 + R16 = 255 + 1 = 0, C <- 1
; Add to high byte
adc r31, r29 ; R31 <- R31 + R29 + C = 0 + 0 + 1 = 1
\end{minted}
Therefore the final result is: \mintinline{ca65}{R31|:|R30 |=| 0b00000001|:|0b00000001 |=| 256}.
Below is a graphical representation of the above code.
\begin{figure}[H]
    \centering
    \includegraphics[height = 6cm, keepaspectratio = true]{figures/adc.pdf}
    \caption{Overflow addition using \mintinline{ca65}{adc} in AVR Assembly.} % \label{}
\end{figure}
\subsubsection{Subtraction}
Subtraction is performed using the same process as binary addition, with the
subtrahend in two's complement form.
In the case of overflows, the carry bit is discarded.
\subsubsection{Multiplication}
Multiplication is understood as the sum of a set of partial products, similar to the process used in decimal multiplication.
Here each digit of the multiplier is multiplied to the multipicand and each partial product is added to the result.

Given an \(m\)-bit and an \(n\)-bit number, the product is at most \((m+n)\)-bits wide.
\begin{align*}
    13 \times 43 & = 00001101_2 \times 00101011_2          \\
                 & = \begin{aligned}[t]
                          &   & 00001101_2 & & \times & & 1_2      \\
                          & + & 00001101_2 & & \times & & 10_2     \\
                          & + & 00001101_2 & & \times & & 1000_2   \\
                          & + & 00001101_2 & & \times & & 100000_2
                     \end{aligned} \\
                 & = \begin{aligned}[t]
                          &   & 00001101_2  \\
                          & + & 00011010_2  \\
                          & + & 01101000_2  \\
                          & + & 110100000_2
                     \end{aligned}                    \\
                 & = 1000101111
\end{align*}
Using AVR assembly, we can use the \mintinline{ca65}{|\keyword{mul}|} instruction to perform multiplication.
\begin{minted}{ca65}
; First number
|\keyword{ldi}| r16, 13
; Second number
|\keyword{ldi}| r17, 43

; Multiply
|\keyword{mul}| r16, r17 ; R1:R0 <- 0b00000010:0b00101111 = 2:47
\end{minted}
The result is stored in the register pair \mintinline{text}{R1:R0}.
\subsubsection{Division}
Division, square roots and many other functions are very expensive to implement in hardware,
and thus are typically not found in conventional ALUs, but rather
implemented in software.

However, there are other techniques that can be used to implement division in hardware.
By representing the divisor in reciprocal form, we can try to represent the number as
the sum of powers of 2.

For example, the divisor \(6.4\) can be represented as:
\begin{equation*}
    \frac{1}{6.4} = \frac{10}{64} = 10 \times 2^{-6}
\end{equation*}
so that dividing an integer \(n\) by \(6.4\) is approximately equivalent to:
\begin{equation*}
    \frac{n}{6.4} \approx \left( n \times 10 \right) \gg 6
\end{equation*}
When the divisor is not exactly representable as a power of 2 we can use fractional exponents to represent the divisor, however this requires a floating point system implementation which is not provided on the AVR.
\end{document}

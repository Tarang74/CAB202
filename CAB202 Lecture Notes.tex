%!TEX TS-program = xelatex
%!TEX options = -aux-directory=Debug -shell-escape -file-line-error -interaction=nonstopmode -halt-on-error -synctex=1 "%DOC%"
\documentclass{article}
% Packages

%% Math enhancements
\usepackage{amsmath} % Misc enhancements to math equations
\usepackage{cancel} % Draw diagonal lines and arrows in math equations
\usepackage{mathtools} % Starred versions of amsmath matrix environments; Multiline, cases, gathered environment
\usepackage{chngcntr} % Reset counter within sections
\usepackage{interval} % Format intervals
\intervalconfig{
    soft open fences
}

%% Symbols
\usepackage{amssymb} % Extended symbol collection - also loads amsfonts
\usepackage{stmaryrd} % Extra symbols

%% Fonts
\usepackage{mathrsfs} % Support \mathcal and \mathscr

%% Environments
\usepackage{amsthm} % Use theorems

%% Tables and arrays
\usepackage{booktabs} % Top and bottom rule for tabular
\usepackage{tabularx} % Advanced Tables

%% Lists
\usepackage{enumitem} % Itemize, enumerate, description environments

%% Page layout
\usepackage{geometry} % Page layout customisation
\usepackage{fancyhdr} % Page headers and footers
\usepackage{float} % Float objects such as figures and tables
\usepackage{tcolorbox} % Create boxed environments

%% Text enhancements
\usepackage[none]{hyphenat} % Disable hyphenation of long text
\usepackage{ragged2e} % Text alignment options

%% Referencing
\usepackage{tocbibind} % Adds bibliography to the Table of Contents
\usepackage{url} % Define urls

%% Graphics
\usepackage{graphicx} % Extension to graphics
% \graphicspath{ {./figures/} }

%% Miscellaneous
\usepackage[outputdir=Debug]{minted} % Typeset programming code
\usepackage{siunitx} % SI units package
\usepackage{derivative} % Derivative notation
\usepackage{pdfpages} % Import PDFs into document

\usepackage[hidelinks]{hyperref} % Handle cross-referencing
\usepackage{bookmark} % New bookmark organisation for hyperref

%% Unicode setup
\usepackage[warnings-off={mathtools-colon, mathtools-overbracket}]{unicode-math}
\setmathfont{Latin Modern Math}
\setmathfont[range={bb, bbit}, Scale=MatchUppercase]{TeX Gyre Pagella Math}
\setmathfont[range={\mathcal, \mathbfcal}, StylisticSet=1]{XITS Math}
\setmathfont[range={\mathscr}]{XITS Math}
\setmathfont[range={"2205}]{XITS Math} % chktex 18

% Preamble

%% Misc Commands

%%% Number Sets
\newcommand*{\N}{\mathbb{N}}
\newcommand*{\Z}{\mathbb{Z}}
\newcommand*{\Q}{\mathbb{Q}}
\newcommand*{\I}{\mathbb{I}}
\newcommand*{\R}{\mathbb{R}}
\newcommand*{\C}{\mathbb{C}}

%%% Empty set character
\let\oldemptyset\emptyset
\let\varnothing\relax
\newcommand{\varnothing}{\char"2205} % chktex 18

%%% Contradiction
\newcommand{\contradiction}{
\hspace{-1em}
{\hbox{
\setbox0=\hbox{\(\mkern-3mu{\times}\mkern-3mu\)}
\setbox1=\hbox to0pt{\hss\copy0\hss}
\copy0\raisebox{0.5\wd0}{\copy1}\raisebox{-0.5\wd0}{\box1}\box0}}
}

%%% Lines for matrices
\newcommand*{\vertbar}{\rule[-1ex]{0.5pt}{2.5ex}}
\newcommand*{\horzbar}{\rule[.5ex]{2.5ex}{0.5pt}}

%% Paired Delimiters
\DeclarePairedDelimiter{\ceil}{\lceil}{\rceil}
\DeclarePairedDelimiter{\floor}{\lfloor}{\rfloor}
\DeclarePairedDelimiter{\abracket}{\langle}{\rangle}
\DeclarePairedDelimiter{\abs}{\lvert}{\rvert}
\DeclarePairedDelimiter{\norm}{\lVert}{\rVert}

%% Probability Functions
\let\Pr\relax
\DeclareMathOperator{\Pr}{Pr}
\DeclareMathOperator{\E}{E}
\DeclareMathOperator{\Var}{Var}
\DeclareMathOperator{\Cov}{Cov}
\DeclareMathOperator{\Corr}{Corr}

\newcommand{\Perm}[2]{\prescript{#1}{}{P}_{#2}}

%% Hyperbolic Functions
\DeclareMathOperator{\arcsinh}{arcsinh}
\DeclareMathOperator{\arccosh}{arccosh}
\DeclareMathOperator{\arctanh}{arctanh}
\DeclareMathOperator{\arccoth}{arccoth}
\DeclareMathOperator{\arcsech}{arcsech}
\DeclareMathOperator{\arccsch}{arccsch}

%% Linear Algebra
%%% Augmented matrices
\makeatletter
\renewcommand*\env@matrix[1][*\c@MaxMatrixCols c]{%
    \hskip -\arraycolsep
    \let\@ifnextchar\new@ifnextchar
    \array{#1}}
\makeatother

%%% Operators
\let\det\relax
\DeclareMathOperator{\det}{det}
\DeclareMathOperator{\Tr}{Tr}
\DeclareMathOperator{\diag}{diag}
\DeclareMathOperator{\adj}{adj}

\DeclareMathOperator{\vspan}{span}
\DeclareMathOperator{\vref}{ref}
\DeclareMathOperator{\vrref}{rref}

\DeclareMathOperator{\vrank}{rank}
\DeclareMathOperator{\vnull}{null}

\DeclareMathOperator{\proj}{proj}

\DeclareMathOperator{\vim}{im}
\DeclareMathOperator{\vcoim}{coim}
\DeclareMathOperator{\vker}{ker}
\DeclareMathOperator{\vcoker}{coker}

\newcommand{\columnspace}[1]{\mathcal{C}\left(\symbf{#1}\right)}
\newcommand{\rowspace}[1]{\mathcal{C}\left(\symbf{#1}^{\top}\right)}
\newcommand{\nullspace}[1]{\mathcal{N}\left(\symbf{#1}\right)}
\newcommand{\leftnullspace}[1]{\mathcal{N}\left(\symbf{#1}^{\top}\right)}

%% Additional operators
\DeclareMathOperator{\erf}{erf}

% Theorems
\theoremstyle{definition}
\newtheorem{definition}{Definition}[section]

\theoremstyle{plain}
\newtheorem{theorem}{Theorem}[subsection]
\newtheorem{corollary}{Corollary}[theorem]
\newtheorem{lemma}{Lemma}[theorem]
\newtheorem{axiom}{Axiom}

\theoremstyle{remark}
\newtheorem{remark}{Remark}
\newtheorem{note}{Note}[subsection]
\newtheorem*{statement}{Statement}

\newenvironment{examples}[1][Examples]{\let\qed\relax\proof[#1]\mbox{}\\*}{\endproof}
\newenvironment{question}[1][Question]{\let\qed\relax\proof[#1]\mbox{}\\*}{\endproof}
\newenvironment{solution}[1][Solution]{\let\qed\relax\proof[#1]\mbox{}\\*}{\endproof}

\newenvironment{proofcase}[1]{\proof[Case #1]\mbox{}}{\endproof}

%% Box styles
\tcbuselibrary{skins}
\newtcolorbox{tcolorboxlarge}[1][]{
    skin=enhanced,
    boxrule=0pt,
    frame hidden,
    sharp corners,
    borderline west={0.5pt}{0pt}{black},
    borderline east={0.5pt}{0pt}{black},
    enlarge left by=10pt,
    width=\linewidth-20pt,
    opacityback=0,
    coltitle=black,
    fonttitle=\large\bfseries,
    #1
}

\newtcolorbox{tcolorboxcols}[1][]{
    skin=enhanced,
    boxrule=0pt,
    frame hidden,
    sharp corners,
    borderline west={0.5pt}{0pt}{black},
    opacityback=0,
    coltitle=black,
    fonttitle=\large\bfseries,
    #1
}

%% Reset counter within subsections
\counterwithin*{equation}{section}
\counterwithin*{equation}{subsection}
\counterwithin*{remark}{subsection}

%% Page layout setup
\pagestyle{fancy}
\setlength\headheight{24pt}
\setlength\parindent{0pt} % Indent first line of new paragraphs


% Additional packages & macros
\usepackage{xcolor}
\newcommand{\keyword}[1]{\textcolor[rgb]{0.00,0.50,0.00}{\textbf{#1}}}
\newcommand{\keywordinline}[1]{\textcolor[rgb]{0.00,0.50,0.00}{\textbf{\mintinline{ca65}{#1}}}}

\setminted{
    escapeinside=||,
    frame=lines,
    breaklines
}

\usepackage{subcaption}
\usepackage{multicol}
\usepackage{multirow}

% Header and footer
\newcommand{\unitName}{Microprocessors and Digital Systems}
\newcommand{\unitTime}{Semester 2, 2022}
\newcommand{\unitCoordinator}{Dr Mark Broadmeadow}
\newcommand{\documentAuthors}{Tarang Janawalkar}

\fancyhead[L]{\unitName}
\fancyhead[R]{\parbox[t]{0.6\textwidth}{\raggedleft\leftmark\strut}}
\fancyfoot[C]{\thepage}

% Copyright
\usepackage[
    type={CC},
    modifier={by-nc-sa},
    version={4.0},
    imagewidth={5em},
    hyphenation={raggedright}
]{doclicense}

\date{}

\begin{document}
%
\begin{titlepage}
    \vspace*{\fill}
    \begin{center}
        \LARGE{\textbf{\unitName}} \\[0.1in]
        \normalsize{\unitTime} \\[0.2in]
        \normalsize\textit{\unitCoordinator} \\[0.2in]
        \documentAuthors
    \end{center}
    \vspace*{\fill}
    \doclicenseThis
    \thispagestyle{empty}
\end{titlepage}
\newpage
%
\tableofcontents
\newpage
%
\part{Foundations of Microcontrollers}
\section{Computer Systems Architecture}
\subsection{Introduction to Computers}
\begin{definition}[Computer]
    A computer is a digital electronic machine that can be programmed to
    carry out sequences of arithmetic or logical operations
    (computations) automatically.
\end{definition}
\begin{definition}[Central Processing Unit]
    The central processing unit (CPU) is a system of components that
    processes instructions, performs calculations, and manages the flow
    of data through a computer. It consists of several components such
    as the control unit, arithmetic logic unit, and processing core(s).
\end{definition}
\begin{definition}[Control unit]
    The control unit is a component of the CPU responsible for managing
    the flow of instructions and data within the processor. It
    interprets program instructions, coordinates the activities of other
    CPU components, and ensures that operations are carried out in the
    correct sequence.
\end{definition}
\begin{definition}[Arithmetic logic unit]
    The arithmetic logic unit (ALU) is a subsystem of the CPU that
    performs mathematical operations, such as addition and subtraction,
    as well as logical comparisons, such as equality or inequality
    checks. It is a critical component for executing calculations and
    decision-making tasks within a computer.
\end{definition}
\begin{definition}[Processing core]
    A core is an independent processing unit within the CPU, capable of
    executing its own set of instructions. Modern CPUs may contain
    multiple cores, enabling parallel execution of tasks to improve
    efficiency and performance. Each core operates as a self-contained
    processor within the larger CPU system.
\end{definition}
\subsection{Microprocessors \& Microcontrollers}
While a microcontroller puts the CPU and all peripherals onto the same
chip, a microprocessor houses a more powerful CPU on a single chip that
connects to external peripherals. The peripherals include memory, I/O,
and control units. The QUTy board used in this unit houses an AVR
ATtiny1626 microcontroller. Some of the key features of this
microcontroller are provided in the next section.
\subsection{The AVR ATtiny1626 Microcontroller}
The ATtiny1626 microcontroller has the following features:
\begin{itemize}
    \item CPU:\@ AVR Core (AVRxt variant)
    \item Memory:
          \begin{itemize}
              \item Flash memory (\qty{16}{K.B}) used to store program
                    instructions in memory
              \item SRAM (\qty{2}{K.B}) used to store data in memory
              \item EEPROM (\qty{256}{B})
          \end{itemize}
    \item Peripherals:\@ Implemented in hardware (part of the chip) in
          order to offload complexity
\end{itemize}
\subsubsection{Flash Memory}
\begin{itemize}
    \item Non-volatile --- memory is not lost when power is removed
    \item Inexpensive
    \item Slower than SRAM
    \item Can only be erased in large chunks
    \item Typically used to store program data
    \item Generally read-only. Programmed via an external tool, which
          is loaded once and remains static during the lifetime of the
          program
    \item Writes are slow
\end{itemize}
\subsubsection{SRAM}
\begin{itemize}
    \item Volatile --- memory is lost when power is removed
    \item Expensive
    \item Faster than flash memory and is used to store variables and
          temporary data
    \item Can access individual bytes (large chunk erases are not
          required)
\end{itemize}
\subsubsection{EEPROM}
\begin{itemize}
    \item Older technology
    \item Expensive
    \item Non-volatile
    \item Can erase individual bytes
\end{itemize}
\subsubsection{The AVR Core}
\begin{itemize}
    \item 8-bit Reduced Instruction Set Computer (RISC)
    \item 32 working registers (R0 to R31)
    \item Program Counter (PC) --- location in memory of the next
          instruction to execute
    \item Status Register (SREG) --- stores key information from
          calculations performed by the ALU (i.e., whether a result is
          negative)
    \item Stack Pointer --- location in memory of the top of the stack
    \item 8-bit core --- all data, registers, and operations, operate within 8-bits
\end{itemize}
\subsubsection{Status Register}
The status register is an 8-bit register that stores the result of the
last operation performed by the ALU. It has the following flags:
\begin{itemize}
    \item[\textbf{C}] Carry Flag
    \item[\textbf{Z}] Zero Flag
    \item[\textbf{N}] Negative Flag
    \item[\textbf{V}] Two's Complement Overflow Flag
    \item[\textbf{S}] Sign Flag
    \item[\textbf{H}] Half Carry Flag
    \item[\textbf{T}] Transfer Bit
    \item[\textbf{I}] Global Interrupt Enable Bit
\end{itemize}
\subsection{Computer Programming Basics}
A computer program is a set of instructions written to perform a
specific task or solve a problem. These instructions are processed by
the CPU, which executes them step by step to produce the desired
outcome. Programs are typically written in high-level programming
languages, which are then translated into machine-readable formats for
execution.
\begin{definition}[Machine code]
    Machine code is the lowest-level representation of a program,
    consisting of binary instructions (sequences of 0s and 1s) that the
    CPU can execute directly. Each instruction corresponds to a specific
    operation, such as arithmetic, memory access, or control flow,
    defined by the CPU's architecture.
\end{definition}
\begin{definition}[Assembly language]
    Assembly language is a low-level programming language that provides
    a human-readable representation of machine code. It uses symbolic
    names (called operational codes) for operations and memory
    addresses, making it easier for programmers to write and understand
    code that is closely tied to the CPU's architecture. Assembly
    language must be translated into machine code by an assembler for
    execution.
\end{definition}
\subsection{Program Execution}
At the time of reset, PC = 0 and the following steps are performed:
\begin{enumerate}
    \item Fetch instruction (from memory)
    \item Decode instruction (decode binary instruction)
    \item Execute instruction:
          \begin{itemize}
              \item Execute an operation
              \item Store data in data memory, the ALU, a register, or
                    update the stack pointer
          \end{itemize}
    \item Store result
    \item Update PC: increment once if the instruction is one word,
          otherwise increment twice. Control flow instructions may move
          the program to another location and, as a result, set the PC
          to a specific address.
\end{enumerate}
This is illustrated in the following figure:
\begin{figure}[H]
    \centering
    \includegraphics[height = 12cm, keepaspectratio = true]{figures/AVR_CPU.pdf}
    \caption{Program execution on the ATtiny1626.} % \label{}
\end{figure}
\subsubsection{Instructions}
\begin{itemize}
    \item The CPU understands and can execute a limited set of
          instructions --- \textasciitilde88 unique instructions for
          the ATtiny1626
    \item Instructions are encoded in program memory as opcodes. Most
          instructions are two bytes long, but some instructions are
          four bytes long
    \item The AVR Instruction Set Manual describes all the available
          instructions, and how they are translated into opcodes
    \item Instructions fall into five categories:
          \begin{itemize}
              \item Arithmetic and logic --- arithmetic and logical
                    operations performed by the ALU
              \item Change of flow --- jumping to specific locations of
                    the program unconditionally, or conditionally, by
                    testing bits in the status register
              \item Data transfer --- moving data in/out of registers,
                    into the data space, or into RAM
              \item Bit and bit-test --- inspecting data in registers
                    (specifically for bit-level operations)
              \item Control --- Special microcontroller instructions
          \end{itemize}
\end{itemize}
\subsubsection{Memory and Peripherals}
The CPU interacts with both memory and peripherals via the data space.
From the perspective of the CPU, the data space is a large array of
locations that can be read from, or written to, using an address.
Peripherals can be controlled by reading from, and writing to, their
registers which have a unique address in the data space. When
peripherals are accessed in this manner we refer to them as being
memory mapped. Different devices, peripherals, and memory can be
included in a memory map (and sometimes a device can be accessed at
multiple different addresses).
\section{Digital Representations and Operations}
\subsection{Bits, Bytes, and Nibbles}
A \textbf{bit}\footnote{The term \textit{bit} comes from
\textbf{b}inary dig\textbf{it}.} is the most basic unit of information
in a digital system. A bit encodes a logical state with one of two
possible values. These states can represent a variety of concepts:
\begin{itemize}
    \item true, false (Boolean states)
    \item high, low (voltage states)
    \item on, off (switch states)
    \item set, reset (memory states)
    \item 1, 0 (binary states)
\end{itemize}
A sequence of \textit{eight} bits is known as a \textbf{byte}, and it is the most
common representation of data in digital systems.
A sequence of \textit{four} bits is known as a \textbf{nibble}.
A sequence of \(n\) bits can represent up to \(2^n\) states.
\subsection{Number Representations}
\subsubsection{Binary}
The \textbf{binary system} is a base-2 system that uses a sequence of
bits to represent a number. Bits are written right-to-left from
\textbf{least significant} to \textbf{most significant} bit. The
left-most bit is the ``most significant'' bit because it is associated
with the highest value in the sequence (coefficient of the highest
power of two).
\begin{itemize}
    \item The \textbf{least significant bit} (LSB) is at bit index 0.
    \item The \textbf{most significant bit} (MSB) is at bit index \(n -
          1\) in an \(n\)-bit sequence.
\end{itemize}
\begin{align*}
    0000_2 & = 0 & 0100_2 & = 4 & 1000_2 & = 8  & 1100_2 = 12 \\
    0001_2 & = 1 & 0101_2 & = 5 & 1001_2 & = 9  & 1101_2 = 13 \\
    0010_2 & = 2 & 0110_2 & = 6 & 1010_2 & = 10 & 1110_2 = 14 \\
    0011_2 & = 3 & 0111_2 & = 7 & 1011_2 & = 11 & 1111_2 = 15
\end{align*}
The subscript 2 indicates that the number is represented using a base-2
system. As with the familiar decimal system, left-padded zeros do not
change the value of a number, but are included here for formatting
purposes.
\subsubsection{Octal}
The \textbf{octal system} is a base-8 system. It is most notably used
as a shorthand for representing file permissions on UNIX systems, where
three bits are used to represent read, write, execute permissions for
the owner of a file, the groups the owner is part of, and other users.
\begin{align*}
    0_8 & = 000_2 & 4_8 & = 100_2 \\
    1_8 & = 001_2 & 5_8 & = 101_2 \\
    2_8 & = 010_2 & 6_8 & = 110_2 \\
    3_8 & = 011_2 & 7_8 & = 111_2
\end{align*}
As each octal digit maps to three bits, it is not very convenient for
systems with byte-sized data. Despite this, it is still available in
many programming languages for historical reasons.
\subsubsection{Hexadecimal}
The \textbf{hexadecimal system} (hex) is a base-16 system. As we need
more than 10 digits in this system, we use the letters A-F to represent
digits 10 to 15. Hex is a convenient notation when working with digital
systems as each hexadecimal digit maps to a nibble.
\begin{align*}
    0_{16} & = 0000_2 & 4_{16} & = 0100_2 & 8_{16} & = 1000_2 & C_{16} = 1100_2 \\
    1_{16} & = 0001_2 & 5_{16} & = 0101_2 & 9_{16} & = 1001_2 & D_{16} = 1101_2 \\
    2_{16} & = 0010_2 & 6_{16} & = 0110_2 & A_{16} & = 1010_2 & E_{16} = 1110_2 \\
    3_{16} & = 0011_2 & 7_{16} & = 0111_2 & B_{16} & = 1011_2 & F_{16} = 1111_2
\end{align*}
\subsubsection{Numeric Literals}
When a fixed value is declared directly in a program, it is referred to
as a \textbf{literal}. Here we must use prefixes to denote the base of
the number:
\begin{itemize}
    \item \textbf{Binary} notation requires the prefix \mintinline{ca65}{0b}
    \item \textbf{Decimal} notation does not require prefixes
    \item \textbf{Octal} notation requires the prefix \mintinline{ca65}{0o}
    \item \textbf{Hexadecimal} notation requires the prefix \mintinline{ca65}{0x}
\end{itemize}
For example, \mintinline{ca65}{0x80 |=| 0o200 |=| 0b10000000 |=| 128}.
\subsection{Unsigned Integers}
The \textbf{unsigned integers} represent the set of counting (natural)
numbers, starting at 0. In the \textbf{decimal system} (base-10), the
unsigned integers are encoded using a sequence of decimal digits
(0--9). The decimal system is a \textbf{positional numeral system},
where the contribution of each digit is determined by its position. For
example,
\begin{align*}
    278_{10} & = 2 \times 10^2 &  & + 7 \times 10^1 &  & + 8 \times 10^0 \\
             & = 2 \times 100  &  & + 7 \times 10   &  & + 8 \times 1    \\
             & = 200           &  & + 70            &  & + 8             \\
\end{align*}
In the \textbf{binary system} (base-2) the unsigned integers are encoded using a sequence of binary digits (0--1)
in the same manner. For example,
\begin{align*}
    10101_2 & = 1 \times 2^4 &  & + 0 \times 2^3 &  & + 1 \times 2^2 &  & + 0 \times 2^1 &  & + 1 \times 2^0 \\
            & = 1 \times 16  &  & + 0 \times 8   &  & + 1 \times 4   &  & + 0 \times 2   &  & + 1 \times 1   \\
            & = 16           &  & + 0            &  & + 4            &  & + 0            &  & + 1            \\
            & = 21_{10}
\end{align*}
The range of values an \(n\)-bit binary number can hold when encoding an unsigned integer is 0 to \(2^n - 1\).
\begin{table}[H]
    \centering
    \begin{tabular}{c c}
        \toprule
        \textbf{No.\ of Bits} & \textbf{Range}                        \\
        \midrule
        8                     & \(0\)--\(255\)                        \\
        16                    & \(0\)--\(\num{65535}\)                \\
        32                    & \(0\)--\(\num{4294967295}\)           \\
        64                    & \(0\)--\(\num{18446744073709551615}\) \\
        \bottomrule
    \end{tabular}
    \caption{Range of available values in binary representations.} % \label{}
\end{table}
\subsection{Signed Integers}
Signed integers are used to represent integers that can be positive or
negative. The following representations allow us to encode negative
integers using a sequence of binary bits:
\begin{itemize}
    \item Sign-magnitude
    \item One's complement
    \item Two's complement (most common)
\end{itemize}
\subsubsection{Sign-Magnitude}
In sign-magnitude representation, the most significant bit encodes the
sign of the integer. In an 8-bit sequence, the remaining 7-bits are
used to encode the value of the bit.
\begin{itemize}
    \item If the sign bit is 0, the remaining bits represent a positive
          value,
    \item If the sign bit is 1, the remaining bits represent a negative
          value.
\end{itemize}
As the sign bit consumes one bit from the sequence, the range of values that can be
represented by an \(n\)-bit sign-magnitude encoded bit sequence is:
\begin{equation*}
    -\left( 2^{n - 1} - 1 \right) \text{ to } 2^{n - 1} - 1.
\end{equation*}
For 8-bit sequences, this range is: \(-127\) to \(127\).
However, this presents several issues:
\begin{enumerate}
    \item There are two ways to represent zero:
          \mintinline{ca65}{0b10000000 |=| 0}, or
          \mintinline{ca65}{0b00000000 |=| -0}.
    \item Arithmetic and comparison requires inspecting the sign bit
    \item The range is reduced by 1 (due to the redundant zero
          representation)
\end{enumerate}
\subsubsection{One's Complement}
In one's complement representation, a negative number is represented by
inverting the bits of a positive number (i.e., \(0 \to 1\) and \(1 \to
0\)). While the range of representable values are still the same:
\begin{equation*}
    -\left( 2^{n - 1} - 1 \right) \text{ to } 2^{n - 1} - 1
\end{equation*}
this representation tackles the second problem in the previous representation as
addition is performed via standard binary addition with \textit{end-around carry} (carry bit is added onto result).
\begin{equation*}
    a - b = a + \left( \text{\textasciitilde} b \right) + C.
\end{equation*}
\subsubsection{Two's Complement}
In two's complement representation, the most significant bit encodes a
negative weighting of \(2^{n - 1}\). For example, in 8-bit sequences,
index-7 represents a value of \(-128\). It can be shown that the two's
complement is calculated by adding 1 to the one's complement. The range
of representable values is then:
\begin{equation*}
    -2^{n - 1} \text{ to } 2^{n - 1} - 1.
\end{equation*}
This representation is more efficient than the previous because \mintinline{ca65}{0} has a single representation
and subtraction is performed by adding the two's complement of the subtrahend.
\begin{equation*}
    a - b = a + \left( \text{\textasciitilde} b + 1 \right).
\end{equation*}
\subsection{Logical Operators}
\subsubsection{Boolean Functions}
A Boolean function is a function whose arguments and results assume
values from a two-element set, (usually \(\left\{ 0,\: 1 \right\}\) or
\mintinline{text}{{false, true}}). These functions are also referred to
as \textit{logical functions} when they operate on bits. The most
common logical functions available to microprocessors and most
programming languages are:
\begin{itemize}
    \item Negation: \keywordinline{NOT} \(a\), \textasciitilde\(a\),
          \(\overline{a}\)
    \item Conjunction: \(a\) \mintinline{ca65}{AND} \(b\), \(a\)
          \mintinline{ca65}{&} \(b\), \(a \cdot b\), \(a \land b\)
    \item Disjunction: \(a\) \keywordinline{OR} \(b\), \(a\)
          \mintinline{ca65}{|\vert|} \(b\), \(a + b\), \(a \lor b\)
    \item Exclusive disjunction: \(a\) \keywordinline{XOR} \(b\), \(a\)
          \mintinline{ca65}{^} \(b\), \(a \oplus b\)
\end{itemize}
By convention, we map a bit value of \mintinline{ca65}{0} to \mintinline{ca65}{false}, and a bit value of \mintinline{ca65}{1} to \mintinline{ca65}{true}.
\subsubsection{Negation}
\keywordinline{NOT} is a unary operator used to \textbf{invert} a bit.
\begin{table}[H]
    \centering
    \begin{tabular}{c c}
        \toprule
        \textbf{\(a\)} & \keywordinline{NOT} \(a\) \\
        \midrule
        0              & 1                         \\
        1              & 0                         \\
        \bottomrule
    \end{tabular}
\end{table}
\subsubsection{Conjunction}
\mintinline{ca65}{AND} is a binary operator whose output is true if \textbf{both} inputs are \textbf{true}.
\begin{table}[H]
    \centering
    \begin{tabular}{c c c}
        \toprule
        \textbf{\(a\)} & \textbf{\(b\)} & \textbf{\(a\) \mintinline{ca65}{AND} \(b\)} \\
        \midrule
        0              & 0              & 0                                           \\
        0              & 1              & 0                                           \\
        1              & 0              & 0                                           \\
        1              & 1              & 1                                           \\
        \bottomrule
    \end{tabular}
\end{table}
\subsubsection{Disjunction}
\keywordinline{OR} is a binary operator whose output is true if \textbf{either} input is \textbf{true}.
\begin{table}[H]
    \centering
    \begin{tabular}{c c c}
        \toprule
        \textbf{\(a\)} & \textbf{\(b\)} & \(a\) \keywordinline{OR} \(b\) \\
        \midrule
        0              & 0              & 0                              \\
        0              & 1              & 1                              \\
        1              & 0              & 1                              \\
        1              & 1              & 1                              \\
        \bottomrule
    \end{tabular}
\end{table}
\subsubsection{Exclusive Disjunction}
\keywordinline{XOR} (Exclusive \keywordinline{OR}) is a binary operator whose output is true if \textbf{only one} input is \textbf{true}.
\begin{table}[H]
    \centering
    \begin{tabular}{c c c}
        \toprule
        \textbf{\(a\)} & \textbf{\(b\)} & \(a\) \keywordinline{XOR} \(b\) \\
        \midrule
        0              & 0              & 0                               \\
        0              & 1              & 1                               \\
        1              & 0              & 1                               \\
        1              & 1              & 0                               \\
        \bottomrule
    \end{tabular}
\end{table}
\subsubsection{Bitwise Operations}
When applying logical operators to a sequence of bits, the operation is
performed in a \textbf{bitwise} manner. The result of each operation is
stored in the corresponding bit index also.
\subsection{Bit Manipulation}
Often we need to modify individual bits within a byte, \textbf{without}
modifying other bits. This is accomplished by performing a bitwise
operation on the byte using a \textbf{bit mask} or \textbf{bit field}.
These operations can:
\begin{itemize}
    \item \textbf{Set} specific bits (change value to \mintinline{ca65}{1})
    \item \textbf{Clear} specific bits (change value to \mintinline{ca65}{0})
    \item \textbf{Toggle} specific bits (change values from \(0 \to 1\), or \(1 \to 0\))
\end{itemize}
\subsubsection{Setting Bits}
To \textbf{set} a bit, we take the bitwise \keywordinline{OR} of the
byte, with a bit mask that has a \textbf{1} in each position where the
bit should be set.
\begin{figure}[H]
    \centering
    \includegraphics[height = 4cm, keepaspectratio = true]{figures/bit_set.pdf}
    \caption{Setting bits using the logical or.} % \label{}
\end{figure}
\subsubsection{Clearing Bits}
To \textbf{clear} a bit, we take the bitwise \mintinline{ca65}{AND} of
the byte, with a bit mask that has a \textbf{0} in each position where
the bit should be cleared.
\begin{figure}[H]
    \centering
    \includegraphics[height = 4cm, keepaspectratio = true]{figures/bit_clear.pdf}
    \caption{Clearing bits using the logical and.} % \label{}
\end{figure}
\subsubsection{Toggling Bits}
To \textbf{toggle} a bit, we take the bitwise \keywordinline{XOR} of
the byte, with a bit mask that has a \textbf{1} in each position where
the bit should be toggled.
\begin{figure}[H]
    \centering
    \includegraphics[height = 4cm, keepaspectratio = true]{figures/bit_toggle.pdf}
    \caption{Toggling bits using the logical exclusive or.} % \label{}
\end{figure}
Other bitwise operations act on the entire byte.
\begin{itemize}
    \item One's complement (bitwise \keywordinline{NOT})
    \item Two's complement (bitwise \keywordinline{NOT} + 1)
    \item Shifts
          \begin{itemize}
              \item Logical
              \item Arithmetic (for signed integers)
          \end{itemize}
    \item Rotations
\end{itemize}
\subsubsection{One's Complement}
The one's complement of a byte inverts every bit in the operand. This
is done by taking the bitwise \keywordinline{NOT} of the byte.
Similarly, we can subtract the byte from \mintinline{ca65}{0xFF} to get
the one's complement.
\subsubsection{Two's Complement}
The two's complement of a byte is the one's complement of the byte plus
one. Therefore, we can take the bitwise \keywordinline{NOT} of the
byte, and then add one to it.
\subsubsection{Shifts}
Shifts are used to move bits within a byte. In many programming
languages this is represented by two greater than \mintinline{ca65}{>>}
or two less than \mintinline{ca65}{<<} characters.
\begin{equation*}
    a \gg s
\end{equation*}
shifts the bits in \(a\) by \(s\) places to the right while adding \mintinline{ca65}{0}'s to the MSB.\
\begin{figure}[H]
    \centering
    \includegraphics[height = 2cm, keepaspectratio = true]{figures/logical_right_shift.pdf}
    \caption{Right shift using \mintinline{ca65}{lsr} in AVR Assembly.} % \label{}
\end{figure}
Similarly,
\begin{equation*}
    a \ll s
\end{equation*}
shifts the bits in \(a\) by \(s\) places to the left while adding \mintinline{ca65}{0}'s to the LSB.\
\begin{figure}[H]
    \centering
    \includegraphics[height = 2cm, keepaspectratio = true]{figures/logical_left_shift.pdf}
    \caption{Left shift using \keyword{\ttfamily{lsl}} in AVR Assembly.} % \label{}
\end{figure}
When using signed integers, the arithmetic shift is used to preserve the value of the sign bit when shifting.
\begin{figure}[H]
    \centering
    \includegraphics[height = 2cm, keepaspectratio = true]{figures/arithmetic_right_shift.pdf}
    \caption{Arithmetic right shift using \keyword{\ttfamily{asr}} in AVR Assembly.} % \label{}
\end{figure}
Left shifts are used to multiply numbers by 2, whereas right shifts are used to divide numbers by 2 (with truncation).
\subsubsection{Rotations}
Rotations are used to shift bits with a carry from the previous
instruction. To understand why, calculate the decimal value of the
resulting byte after a shift.
\begin{figure}[H]
    \centering
    \includegraphics[height = 2cm, keepaspectratio = true]{figures/rotate_left.pdf}
    \caption{Rotate left using \mintinline{ca65}{rol} in AVR Assembly.} % \label{}
\end{figure}
\begin{figure}[H]
    \centering
    \includegraphics[height = 2cm, keepaspectratio = true]{figures/rotate_right.pdf}
    \caption{Rotate right using \mintinline{ca65}{ror} in AVR Assembly.} % \label{}
\end{figure}
Here the blue bit is carried from the previous instruction, and the carry bit is updated
to the value of the bit that was shifted out.
Rotations are used to perform multibyte shifts and arithmetic operations.
\subsection{Arithmetic Operations}
\subsubsection{Addition}
Addition is performed using the same process as decimal addition except
we only use two digits, 0 and 1.
\begin{enumerate}
    \item \mintinline{ca65}{0b0 + 0b0 = 0b0}
    \item \mintinline{ca65}{0b0 + 0b1 = 0b1}
    \item \mintinline{ca65}{0b1 + 0b1 = 0b10}
\end{enumerate}
When adding two 1's, we carry the result into the next bit position as we would with a 10 in decimal addition.
In AVR Assembly, we can use the \keywordinline{add} instruction to add two bytes. The following
example adds two bytes.
\begin{minted}{ca65}
; Accumulator
|\keyword{ldi}| r16, 0

; First number
|\keyword{ldi}| r17, 29
|\keyword{add}| r16, r17 ; R16 <- R16 + R17 = 0 + 29 = 29

; Second number
|\keyword{ldi}| r17, 118
|\keyword{add}| r16, r17 ; R16 <- R16 + R17 = 29 + 118 = 147
\end{minted}
Below is a graphical illustration of the above code.
\begin{figure}[H]
    \centering
    \includegraphics[height = 3cm, keepaspectratio = true]{figures/add.pdf}
    \caption{Overflow addition using \keyword{\ttfamily{add}} in AVR Assembly.} % \label{}
\end{figure}
\subsubsection{Overflows}
When the sum of two 8-bit numbers is greater than 8-bit (255), an
\textbf{overflow} occurs. Here we must utilise a second register to
store the high byte so that the result is represented as a 16-bit
number. To avoid loss of information, a \textbf{carry bit} is used to
indicate when an overflow has occurred. This carry bit can be added to
the high byte in the event that an overflow occurs. The following
example shows how to use the \mintinline{ca65}{adc} instruction to
carry the carry bit when an overflow occurs.
\begin{minted}{ca65}
; Low byte
|\keyword{ldi}| r30, 0
; High byte
|\keyword{ldi}| r31, 0

; Empty byte for adding carry bit
|\keyword{ldi}| r29, 0

; First number
|\keyword{ldi}| r16, 0b11111111
; Add to low byte
|\keyword{add}| r30, r16 ; R30 <- R30 + R16 = 0 + 255 = 255, C <- 0
; Add to high byte
adc r31, r29 ; R31 <- R31 + R29 + C = 0 + 0 + 0 = 0

; Second number
|\keyword{ldi}| r16, 0b00000001
; Add to low byte
|\keyword{add}| r30, r16 ; R30 <- R30 + R16 = 255 + 1 = 0, C <- 1
; Add to high byte
adc r31, r29 ; R31 <- R31 + R29 + C = 0 + 0 + 1 = 1
\end{minted}
Therefore, the final result is:
% latexignore
\mintinline{ca65}{R31|:|R30 |=| 0b00000001|:|0b00000001 |=| 256}.
Below is a graphical representation
of the above code.
\begin{figure}[H]
    \centering
    \includegraphics[height = 6cm, keepaspectratio = true]{figures/adc.pdf}
    \caption{Overflow addition using \mintinline{ca65}{adc} in AVR Assembly.} % \label{}
\end{figure}
\subsubsection{Subtraction}
Subtraction is performed using the same process as binary addition,
with the subtrahend in two's complement form. In the case of overflows,
the carry bit is discarded.
\subsubsection{Multiplication}
Multiplication is understood as the sum of a set of partial products,
similar to the process used in decimal multiplication. Here each digit
of the multiplier is multiplied to the multiplicand and each partial
product is added to the result.

Given an \(m\)-bit and an \(n\)-bit number, the product is at most
\((m+n)\)-bits wide.
\begin{align*}
    13 \times 43 & = 00001101_2 \times 00101011_2 \\
                 & =
    \begin{aligned}[t]
         &   & 00001101_2 &  & \times &  & 1_2      \\
         & + & 00001101_2 &  & \times &  & 10_2     \\
         & + & 00001101_2 &  & \times &  & 1000_2   \\
         & + & 00001101_2 &  & \times &  & 100000_2
    \end{aligned}
    \\
                 & =
    \begin{aligned}[t]
         &   & 00001101_2  \\
         & + & 00011010_2  \\
         & + & 01101000_2  \\
         & + & 110100000_2
    \end{aligned}
    \\
                 & = 1000101111
\end{align*}
Using AVR assembly, we can use the \keywordinline{mul} instruction to perform multiplication.
\begin{minted}{ca65}
; First number
|\keyword{ldi}| r16, 13
; Second number
|\keyword{ldi}| r17, 43

; Multiply
|\keyword{mul}| r16, r17 ; R1:R0 <- 0b00000010:0b00101111 = 559
\end{minted}
The result is stored in the register pair \mintinline{text}{R1:R0}.
\subsubsection{Division}
Division, square roots and many other functions are very expensive to
implement in hardware, and thus are typically not found in conventional
ALUs, but rather implemented in software. However, there are other
techniques that can be used to implement division in hardware. By
representing the divisor in reciprocal form, we can try to represent
the number as the sum of powers of 2. For example, the divisor \(6.4\)
can be represented as:
\begin{equation*}
    \frac{1}{6.4} = \frac{10}{64} = 10 \times 2^{-6}
\end{equation*}
so that dividing an integer \(n\) by \(6.4\) is approximately equivalent to:
\begin{equation*}
    \frac{n}{6.4} \approx \left( n \times 10 \right) \gg 6
\end{equation*}
When the divisor is not exactly representable as a power of 2 we can use fractional
exponents to represent the divisor, however this requires a floating point
system implementation which is not provided on the AVR\@.
\part{Microcontroller Fundamentals}
\section{Microcontroller Interfacing}
\subsection{Logic Levels}
\subsubsection{Discretisation}
The process of discretisation translates a continuous signal into a
discrete signal (bits). As an example, we can translate \textbf{voltage
levels} on microcontroller pins into digital \textbf{logic levels}.
\subsubsection{Logic Levels}
For digital input/output (IO), conventionally:
\begin{itemize}
    \item The voltage level of the positive power supply represents a
          \textbf{logical 1}, or the \textbf{high state}, and
    \item \qty{0}{V} (ground) represents a \textbf{logical 0}, or the \textbf{low state}.
\end{itemize}
The QUTy is supplied \qty{3.3}{V} so that when a digital output is high,
the voltage present on the corresponding pin will be around \qty{3.3}{V}.
Because voltage is a continuous quantity, we must discretise the full range of voltages into logical levels using \textbf{thresholds}.
\begin{itemize}
    \item A voltage \textbf{above} the input \textbf{high threshold}
          \(t_H\) is considered \textbf{high}.
    \item A voltage \textbf{below} the input \textbf{low threshold}
          \(t_L\) is considered \textbf{low}.
\end{itemize}
The interpretation of a voltage between these states is determined by \textbf{hysteresis}.
\subsubsection{Hysteresis}
Hysteresis refers to the property of a system whose state is
\textbf{dependent} on its \textbf{history}. In electronic circuits,
this avoids ambiguity in determining the state of an input as it
switches between voltage levels.
\begin{figure}[H]
    \centering
    \includegraphics[height = 5cm, keepaspectratio = true]{figures/hysteresis.pdf}
    \caption{Example of hysteresis.} % \label{}
\end{figure}
Given a transition:
\begin{itemize}
    \item If an input is currently in the \textbf{low state}, it has
          not transitioned to the \textbf{high state} until the voltage
          crosses the \textbf{high input voltage} threshold.
    \item If an input is currently in the \textbf{high state}, it has
          not transitioned to the \textbf{low state} until the voltage
          crosses the \textbf{low input voltage} threshold.
\end{itemize}
It is therefore always preferable to drive a digital input to an unambiguous voltage level.
\subsection{Electrical Quantities}
\subsubsection{Voltage}
\textbf{Voltage} \(v\) is the electrical \textit{potential difference} between two points in a circuit, measured in \textbf{Volts (\unit{V})}.
\begin{itemize}
    \item Voltage is measured across a circuit element, or between two
          points in a circuit, commonly with respect to a \qty{0}{V}
          reference (ground).
    \item It represents the \textbf{potential} of the electrical system
          to do \textbf{work}.
\end{itemize}
\subsubsection{Current}
\textbf{Current} \(i\) is the \textit{rate of flow of electrical charge} through a circuit, measured in \textbf{Amperes (\unit{A})}.
\begin{itemize}
    \item Current is measured through a circuit element.
\end{itemize}
\subsubsection{Power}
\textbf{Power} \(p\) is the rate of energy transferred per unit time, measured in \textbf{Watts (\unit{W})}.
Power can be determined through the equation
\begin{equation*}
    p = v i.
\end{equation*}
\subsubsection{Resistance}
\textbf{Resistance} \(R\) is a property of a material to \textit{resist the flow of current}, measured in \textbf{Ohms (\unit{\ohm})}.
Ohm's law states that the voltage across a component is proportional to the current that flows through it:
\begin{equation*}
    v = i R.
\end{equation*}
Note that not all circuit elements are resistive (or Ohmic), as they
do not follow Ohm's law; this can be seen in diodes.
\begin{figure}[H]
    \centering
    \begin{subfigure}{0.47\linewidth}
        \centering
        \includegraphics[height=4.5cm]{figures/vi_ohmic.pdf}
        \caption{VI curve for Ohmic components.}
    \end{subfigure}
    \begin{subfigure}{0.47\linewidth}
        \centering
        \includegraphics[height=4.5cm]{figures/vi_diode.pdf}
        \caption{VI curve for diodes.}
    \end{subfigure}
    \caption{Voltage-current characteristic curves for various components.}
\end{figure}
Although the wires used to connect a circuit are resistive, we usually assume that they are ideal, that is,
they have zero resistance.
\subsection{Common Electrical Components}
\subsubsection{Resistors}
A \textbf{resistor} is a circuit element that is designed to have a
specific resistance \(R\).
\begin{figure}[H]
    \centering
    \includegraphics[height = 2.5cm, keepaspectratio = true]{figures/resistor.pdf}
    \caption{Resistor circuit symbol.} % \label{}
\end{figure}
\subsubsection{Switches}
A \textbf{switch} is used to connect and disconnect different elements
in a circuit. It can be \textbf{open} or \textbf{closed}.
\begin{itemize}
    \item In the \textbf{open} state, the switch \textbf{will not
          conduct}\footnote{Conductance is a measure of the ability for
          electric charge to flow in a certain path.} current
    \item In the \textbf{closed} state, the switch \textbf{will
          conduct} current
\end{itemize}
Switches can take a variety of forms:
\begin{itemize}
    \item \textbf{Poles} --- the number of circuits the switch can control.
    \item \textbf{Throw} --- the number of output connections each pole can connect its input to.
    \item Momentary or toggle action
    \item Different form factors, e.g., push button, slide, toggle,
          etc.
\end{itemize}
Switches are typically for user input.
\begin{figure}[H]
    \centering
    \begin{subfigure}{0.47\linewidth}
        \centering
        \includegraphics[width=2.5cm]{figures/spst.pdf}
        \caption{Single pole single throw switch.}
    \end{subfigure}
    \begin{subfigure}{0.47\linewidth}
        \centering
        \includegraphics[width=2.5cm]{figures/spdt.pdf}
        \caption{Single pole double throw switch.}
    \end{subfigure}

    \vspace*{5ex}
    \begin{subfigure}{0.47\linewidth}
        \centering
        \includegraphics[width=2.5cm]{figures/dpst.pdf}
        \caption{Double pole single throw switch.}
    \end{subfigure}
    \begin{subfigure}{0.47\linewidth}
        \centering
        \includegraphics[width=2.5cm]{figures/dpdt.pdf}
        \caption{Double pole double throw switch.}
    \end{subfigure}
    \caption{Various types of switches.}
\end{figure}
\subsubsection{Diodes}
A \textbf{diode} is a semiconductor device that conducts current in
only one direction: from the \textbf{anode} to the \textbf{cathode}.
\begin{figure}[H]
    \centering
    \includegraphics[height = 2cm, keepaspectratio = true]{figures/diode.pdf}
    \caption{Diode symbol.} % \label{}
\end{figure}
Diodes are a non-Ohmic device:
\begin{itemize}
    \item When \textbf{forward biased}, a diode \textbf{does} conduct
          current, and the anode-cathode voltage is equal to the diodes
          \textbf{forward voltage}.
    \item When \textbf{reverse biased}, a diode \textbf{does not}
          conduct current, and the cathode-anode voltage is equal to
          the \textbf{applied voltage}.
\end{itemize}
\begin{figure}[H]
    \centering
    \begin{subfigure}{0.47\linewidth}
        \centering
        \includegraphics[height=3.5cm]{figures/diode_forward_bias.pdf}
        \caption{Forward biased diode. \\\(v_{AK} = v_f\) and \(i > 0\).}
    \end{subfigure}
    \begin{subfigure}{0.47\linewidth}
        \centering
        \includegraphics[height=3.5cm]{figures/diode_reverse_bias.pdf}
        \caption{Reverse biased diode. \\\(v_{KA} > 0\) and \(i = 0\).}
    \end{subfigure}
    \caption{Diodes in forward and reverse bias.}
\end{figure}
A diode is only forward biased when the applied anode-cathode voltage \textbf{exceeds} the forward voltage \(v_f\).
A typical forward voltage \(v_f\) for a silicon diode is in the range \qtyrange{0.6}{0.7}{V}, whereas for
Light Emitting Diodes (LEDs), \(v_f\) ranges between \qtyrange{2}{3}{V}.
\subsubsection{Integrated Circuit}
An \textbf{integrated circuit} (IC) is a set of electronic circuits
(typically) implemented on a single piece of semiconductor material,
usually silicon. ICs comprise hundreds to many thousands of
transistors, resistors and capacitors; all implemented on silicon. ICs
are \textbf{packaged}, and connections to the internal circuitry are
exposed via \textbf{pins}. In general, the specific implementation of
the IC is not important, but rather the \textbf{function of the device}
and how it \textbf{interfaces} with the rest of the circuit. Hence, ICs
can be treated as a functional \textbf{black box}. For digital ICs:
\begin{itemize}
    \item \textbf{Input pins} are typically \textbf{high-impedance}, and they appear as an open circuit.
    \item \textbf{Output pins} are typically \textbf{low-impedance}, and will actively drive the voltage
          on a pin and any connected circuitry to a \textbf{high} or \textbf{low} state. They can also
          drive connected loads.
\end{itemize}
\subsection{Digital Outputs}
Digital output interfaces are designed to be able to drive connected
circuitry to a logical high, or logical low; however, the appropriate
technique is \textbf{context specific}. When referring to digital
outputs, we will refer to the states of a net. A \textbf{net} is
defined as the common point of connection of multiple circuit
components. In this section we will consider:
\begin{itemize}
    \item What kind of load the output drives
    \item Could more than one device be attempting to actively drive
          the net to a specific logic level?
\end{itemize}
\subsubsection{Push-Pull Outputs}
A push-pull digital output is the most common form of output used in
digital outputs. The \textbf{output driver} \(A\) \textit{drives} the
\textbf{output state} \(Y\) to:
\begin{itemize}
    \item \textbf{HIGH} by connecting the output net to the supply voltage \(+\unit{V}\).
    \item \textbf{LOW} by connecting the output net to the ground voltage GND (\qty{0}{V}).
\end{itemize}
\begin{figure}[H]
    \centering
    \includegraphics[height = 4cm, keepaspectratio = true]{figures/push_pull.pdf}
    \caption{Push-pull output.} % \label{}
\end{figure}
Hence, the output state \(Y\) is determined by the logic level of the output driver \(A\).
\begin{equation*}
    Y = A.
\end{equation*}
\begin{table}[H]
    \centering
    \begin{tabular}{c | c}
        \toprule
        \textbf{\(A\)} & \textbf{\(Y\)} \\
        \midrule
        LOW            & LOW            \\
        HIGH           & HIGH           \\
        \bottomrule
    \end{tabular}
    \caption{Truth table for a push-pull digital output.} % \label{}
\end{table}
The push-pull output \(Y\) can both source and sink current from the connected net.
\subsubsection{High-Impedance Outputs}
In many instances, a digital output is required to be placed in a
high-impedance (HiZ) state. This is accomplished by using an
\textbf{output enable} (OE) signal.
\begin{figure}[H]
    \centering
    \includegraphics[height = 3cm, keepaspectratio = true]{figures/HiZ.pdf}
    \caption{High-impedance output.} % \label{}
\end{figure}
\begin{itemize}
    \item When the OE signal is \textbf{HIGH}, the output state \(Y\)
          is determined by the output driver \(A\).
    \item When the OE signal is \textbf{LOW}, the output state \(Y\) is
          in a \textbf{high-impedance} state.
\end{itemize}
\begin{table}[H]
    \centering
    \begin{tabular}{c c | c}
        \toprule
        \textbf{\(A\)} & \textbf{OE} & \textbf{\(Y\)} \\
        \midrule
        LOW            & LOW         & HiZ            \\
        HIGH           & LOW         & HiZ            \\
        LOW            & HIGH        & LOW            \\
        HIGH           & HIGH        & HIGH           \\
        \bottomrule
    \end{tabular}
    \caption{Truth table for a push-pull digital output.} % \label{}
\end{table}
When the output is in \textbf{HiZ state}:
\begin{itemize}
    \item The output is an effective \textbf{open circuit}, meaning it
          has \textbf{no effect} on the rest of the circuit.
    \item The voltage on the output net is determined by the
          \textbf{other circuitry} connected to the net.
\end{itemize}
HiZ outputs are typically used when multiple devices share the same wire(s).
\subsubsection{Pull-up and Pull-down Resistors}
When \textbf{no devices} are actively driving a net (e.g., all
connected outputs are in the HiZ state), the state of the net is not
well-defined. Hence, we can use a \textbf{pull-up} or
\textbf{pull-down} resistor to ensure that the state of the pin is
always \textbf{well-defined}.
\begin{multicols}{2}
    \begin{figure}[H]
        \centering
        \includegraphics[height = 4cm, keepaspectratio = true]{figures/pullup_resistor.pdf}
        \caption{Pull-up resistor.} % \label{}
    \end{figure}
    \begin{figure}[H]
        \centering
        \includegraphics[height = 4cm, keepaspectratio = true]{figures/pulldown_resistor.pdf}
        \caption{Pull-down resistor.} % \label{}
    \end{figure}
\end{multicols}
\begin{itemize}
    \item When \textbf{no circuitry} is actively driving the net, the
          resistor will passively pull the voltage to either the
          voltage supply, or ground.
    \item When \textbf{another device} actively drives the net, the
          active device defines the voltage of the net. Hence, the
          current from the resistor is simply sourced or sunk by the
          \textbf{active device}.
\end{itemize}
The resistors used as pull-up and pull-down resistors are typically in the \unit{k\ohm} range.
\subsubsection{Open-Drain Outputs}
Multiple push-pull outputs should never be connected to the same net as
when one output is driven HIGH and another is driven LOW, an effective
short circuit is created and one or more devices may be damaged. While
push-pull outputs with an output enable may be used, the timing must be
carefully managed.

Hence, a more robust solution is to use open-drain outputs.
\begin{figure}[H]
    \centering
    \includegraphics[height = 4cm, keepaspectratio = true]{figures/open_drain.pdf}
    \caption{Open-drain output.} % \label{}
\end{figure}
An open-drain output is either:
\begin{itemize}
    \item In the \textbf{high-impedance} state, where the pull-up
          resistor is used to pull the net to the \textbf{high state}
          when the net is \textbf{not driven low}.
    \item \textbf{Connected to ground}, when the net is \textbf{driven low}.
\end{itemize}
\subsection{Microcontroller Pins}
Microcontrollers are interfaced via their exposed pins. These pins are
the only means to access inputs and outputs, and are used to interface
with other electronic circuits in order to achieve a required
functionality. Pins can be used for:
\begin{itemize}
    \item General purpose input and output (GPIO) --- pin represents a
          digital state
    \item Peripheral functions
    \item Other functions (power supply, reset input, clock input,
          etc.)
\end{itemize}
Pins are typically organised into groups of related IO banks,
referred to as \textbf{ports} on the AVR microcontroller.
These ports and pins are assigned an alphanumeric identifier, (e.g., PB7 for pin 7 on port B).
\begin{figure}[H]
    \centering
    \includegraphics[height = 10cm, keepaspectratio = true]{figures/PORT_block_diagram.pdf}
    \caption{ATtiny1626 PORT block diagram.} % \label{}
\end{figure}
To summarise this diagram:
\begin{itemize}
    \item The data-direction register (DIR) controls the push-pull
          output enable.
    \item The output-driver register (OUT) drives the output state.
    \item The input register (IN) reads the output state.
    \item The internal pull-up resistor is enabled through software.
    \item The physical voltage on the pin can be routed to an analogue
          to digital converter (ADC) for measurement.
    \item Other peripheral functions can override port pin
          configurations and the output state.
\end{itemize}
\subsubsection{Configuring an Output}
\begin{enumerate}
    \item Place the port pin in a \textbf{safe initial state} by
          writing to the OUT register (HIGH or LOW depending on the
          context).
    \item Configure the port pin as an output by \textbf{setting} the
          corresponding bits in the DIR register.
    \item Set the desired pin state by writing to the OUT register.
\end{enumerate}
For example, assume an active HIGH device is connected to pin 5 on port
B. To configure the pin as an output and set the output state to HIGH:
\begin{minted}{ca65}
; Load macros for easy access to port data space addresses.
#include <avr/io.h>

; Bitmask for pin 5
|\keyword{ldi}| r16, PIN5_bm

; Set initial safe state
|\keyword{sts}| PORTB_OUTCLR, r16 ; LOW if active HIGH
|\keyword{sts}| PORTB_OUTSET, r16 ; HIGH if active LOW

; Enable output
|\keyword{sts}| PORTB_DIRSET, r16 ; Enable output on PB5

; Set output state to desired value
|\keyword{sts}| PORTB_OUTSET, r16 ; Set state of PB5 to HIGH
\end{minted}
\subsubsection{Configuring an Input}
\begin{enumerate}
    \item If required, enable the internal pull-up resistor by
          \textbf{setting} the PULLUPEN bit in the \linebreak
          corresponding PINnCTRL register.
    \item Read the IN register to get the current state of the pin.
    \item Isolate the relevant pin using the AND operator.
\end{enumerate}
For example, to read the state of pin 5 on port A:
\begin{minted}{ca65}
#include <avr/io.h>

|\keyword{ldi}| r16, PIN5_bm

; Enable internal pull-up resistor if required
|\keyword{sts}| PORTB_PIN5CTRL, r16

; Read output state from data space
|\keyword{lds}| r17, PORTA_IN
; Read output state using virtual PORT
|\keyword{in}| r17, VPORTA_IN

; Isolate desired pin
|\keyword{andi}| r17, r16
\end{minted}
\subsubsection{Peripheral Multiplexing}
Pins can be used to connect internal peripheral functions to external
devices. As microcontrollers have more peripheral functions than
available pins, peripheral functions are typically multiplexed onto
pins.
\begin{definition}[Multiplexing]
    Multiplexing is a method by which \textbf{multiple peripheral} \linebreak \textbf{functions}
    are mapped to the \textbf{same pin}.
    In this scenario, only one function can be enabled at a time, and the pin
    cannot be used for GPIO\@.
\end{definition}
\begin{itemize}
    \item Peripheral functions can be mapped to different \textbf{sets
          of pins} to provide flexibility and to avoid clashes when
          multiple peripherals are used in an application.
    \item When enabled, peripheral functions \textbf{override} standard
          port functions.
    \item The \textbf{Port Multiplexer} (PORTMUX) is used to select
          which \textbf{pin set} should be used by a peripheral.
    \item Certain peripherals can have their inputs/outputs mapped to
          different \textbf{sets of pins} through the PORTMUX
          peripheral's configuration registers.
\end{itemize}
Note that we cannot re-map a single peripheral function to another pin,
and must consider the entire set.
\subsection{Interfacing to Simple I/O}
\subsubsection{Interfacing to LEDs}
The \textbf{brightness} of an LED is proportional to the
\textbf{current} passing through it. As LEDs are non-Ohmic, we cannot
drive them directly with a voltage as this would result in an
uncontrolled flow of current that may damage the LED or driver.
Instead, LEDs are paired with a \textbf{series resistor} to limit the
flow of current. The appropriate amount of current necessary for this
is dependent on the LED and the capability of the driver device (the
microcontroller). A typical indicator LED requires a current that
ranges from \qtyrange{1}{2}{m.A}.

An LED can be driven in two different configurations from a
microcontroller pin:
\begin{itemize}
    \item \textbf{active high}; in which case the LED is \textbf{lit} when the pin is \textbf{HIGH}.
    \item \textbf{active low}; in which case the LED is \textbf{lit} when the pin is \textbf{LOW}.
\end{itemize}
Both of these configurations have their benefits, and the best
configuration depends entirely on the context.
\pagebreak

\begin{multicols}{2}
    \begin{figure}[H]
        \centering
        \includegraphics[height = 4cm, keepaspectratio = true]{figures/active_high_LED.pdf}
        \caption{Active high configuration.} % \label{}
    \end{figure}
    \begin{figure}[H]
        \centering
        \includegraphics[height = 4cm, keepaspectratio = true]{figures/active_low_LED.pdf}
        \caption{Active low configuration.} % \label{}
    \end{figure}
\end{multicols}
On the QUTy, the LED display is driven in the \textbf{active low} configuration.
This has a number of advantages:
\begin{itemize}
    \item If the internal pull-up resistors are mistakenly enabled, no
          current will flow into the LEDs.
    \item The microcontroller pins can sink higher currents than they
          can source, allowing us to drive the display to a higher
          brightness.
    \item The display used on the QUTy has a common anode
          configuration, hence we must use an active low configuration
          to drive the display segments independently.
\end{itemize}
An LED is an example of a simple \textbf{digital output}, as we can map \textbf{logical states}
to \textbf{LED states} (lit or unlit) for a digital output.
\subsubsection{Interfacing to Switches}
The state of a switch can be used to \textbf{set} the state of a pin.
As the switch has two states (open or closed), these can be mapped
directly to \textbf{logical states}. This can be done by connecting the
switch between the pin and voltage source representing one of the logic
levels (ground or a positive supply).
\begin{multicols}{2}
    \begin{figure}[H]
        \centering
        \includegraphics[height = 5cm, keepaspectratio = true]{figures/active_high_switch.pdf}
        \caption{Active high configuration.} % \label{}
    \end{figure}
    \begin{figure}[H]
        \centering
        \includegraphics[height = 5cm, keepaspectratio = true]{figures/active_low_switch.pdf}
        \caption{Active low configuration.} % \label{}
    \end{figure}
\end{multicols}
\begin{itemize}
    \item When the switch is \textbf{open}, the pull-up/pull-down
          resistor is used to define the state of the switch.
    \item When the switch is \textbf{closed}, the state of the pin is
          defined by the voltage connected to the switch.
\end{itemize}
As with LEDs, we can interface switches to microcontroller pins in two
different configurations:
\begin{itemize}
    \item \textbf{active high}; in which case the pin is \textbf{HIGH} when the switch is \textbf{closed}.
    \item \textbf{active low}; in which case the pin is \textbf{LOW} when the switch is \textbf{closed}.
\end{itemize}
An \textbf{active low} configuration is usually preferred as:
\begin{itemize}
    \item It allows for the utilisation of an \textbf{internal pull-up
          resistor} that is commonly implemented in microcontrollers.
    \item It eliminates the risk of unsafe voltages being applied to
          the pin from the power supply in an active high
          configuration.
    \item It is easier to access a ground reference on a circuit board.
\end{itemize}
\subsubsection{Interfacing to Integrated Circuits}
For digital ICs,
\begin{itemize}
    \item \textbf{Inputs} are typically \textbf{high impedance}
    \item \textbf{Outputs} are typically \textbf{push-pull}
\end{itemize}
This generally means that we can interface an IC by connecting its pins directly to the pins of a microcontroller.
\begin{itemize}
    \item For \textbf{IC inputs}, the microcontroller pin is configured
          as an \textbf{output}, and the \textbf{microcontroller sets}
          the logic level of the net.
    \item For \textbf{IC outputs}, the microcontroller pin is
          configured as an \textbf{input}, and the \textbf{IC sets} the
          logic level of the net.
\end{itemize}
As microcontroller pins are typically configured as \textbf{inputs on reset}, a
pull-up/pull-down resistor may be required if it is important for an IC input to
have a \textbf{known state} prior to the configuration of the relevant microcontroller pins as outputs.
\subsection{Programming a Microcontroller}
There are two main methods of programming a microcontroller. The first
is to use Assembly language, which is the most direct way to interact
with the instruction set of the microcontroller. The second is to use a
higher-level language, such as C, which is more abstracted from the
instruction set of the microcontroller. The choice of language depends
on the complexity and performance requirements of the application.
\part{AVR Assembly Programming}
\section{Introduction to AVR Assembly}
\begin{definition}[Word]
    A word refers to a value that is two bytes in size (16-bit).
\end{definition}
\begin{definition}[Registers]
    A register refers to a memory location that is 1 byte in size (8-bit).
    The ATtiny1626 has 32 registers of which \mintinline{ca65}{r16} to
    \mintinline{ca65}{r31} can be loaded with an immediate value
    (\numrange{0}{255}) using \keywordinline{ldi}.
    \begin{minted}{ca65}
|\keyword{ldi}| r16, 17 ; Load the value 17 into r16
    \end{minted}
    Values are commonly loaded into registers as many other operations
    can be performed on them.
\end{definition}
Instructions in Assembly language are \textbf{mnemonics} that represent
a specific operation that the microcontroller can perform. These
instructions can take a number of \textbf{operands} that specify the
\textbf{parameters} of the operation. The particular syntax for an
instruction and its operands is dependent on the instruction, but
generally, take the following form:
\begin{minted}{text}
mnemonic [operand1, operand2]
\end{minted}
For many instructions, the first operand often corresponds to the
\textbf{destination} of an operation.
\subsection{Unconditional Flow Control}
Most instructions on the AVR Core increment the PC by 1 or 2 (depending
on how many words the opcode occupies) when they are executed, so that
any successive instructions are executed after the first. To divert
execution to a different location, we can utilise \textbf{change of
flow} instructions. The \mintinline{ca65}{jmp} (jump) instruction is
used to simply jump to a different location in the program. By design,
this instruction is capable of jumping to addresses up to \qty{4}{M.B}
in program memory, although the ATtiny1626 does not have this much
memory.
\subsubsection{Labels}
Most change of flow instructions take an \textbf{address} in program
memory as a parameter. Hence, to make this process easier, we can use
labels to refer to locations in program memory (and also RAM).
\begin{minted}{ca65}
jmp new_location ; Jump to the label |\textbf{new\_location}|.
|\keyword{ldi}| r16, 1       ; This instruction is skipped

new_location: ; Label
    |\keyword{push}| r16
\end{minted}
When a label appears in source code, the assembler replaces references
to it with the address of the directive/instruction immediately
following that label. Labels work for both \textbf{absolute} and
\textbf{relative} addresses and the assembler will automatically adjust
the address to the correct type. Labels can also be used as parameters
to other immediate instructions if we store the high and low bytes in
registers and wish to reference the location in an indirect jumping
instruction.
\subsubsection{Absolute and Relative Addresses}
\mintinline{ca65}{jmp} is a 32-bit instruction, that uses 22 bits to
specify an address between \mintinline{ca65}{0x000000} and
\mintinline{ca65}{0x3FFFFF}, or \(2^{23} - 1\) bits of memory
(\qty{8}{MB}). As mentioned earlier, this is much larger
than what a 16-bit PC can address on the ATtiny1626 (\qty{64}{KB}).
As we will only ever need to jump within \qty{64}{KB} of memory, it is
inefficient to use the \mintinline{ca65}{jmp} instruction as it
costs 3 CPU cycles to execute. Therefore, many AVR change of flow
instructions take a value that is \textbf{added} onto the current PC to
calculate the destination address, allowing them to fit within 16 bits.
The \keywordinline{rjmp} (relative jump) instruction is therefore more
suitable as it only requires 2 CPU cycles.

Note the assembler throws an error if the specified address is not
within the range of the PC\@.
\subsection{Conditional Flow Control}
Branching instructions jump to another location in memory based on a
condition, i.e., user input, internal state, or other external factors.
These instructions alter the PC differently based on the value of
register(s) or flags. On the AVR there are two main categories of
branching instructions:
\begin{itemize}
    \item Branch instructions
    \item Skip instructions
\end{itemize}
\subsubsection{Branch Instructions}
Branch instructions use the following logic:
\begin{enumerate}
    \item Check if the specified flag in SREG is cleared/set
    \item If true, jump to the specified address (\(\mathrm{PC}
          \leftarrow \mathrm{PC} + k + 1\))
    \item Otherwise, proceed to the next instruction as normal
          (\(\mathrm{PC} \leftarrow \mathrm{PC} + 1\))
\end{enumerate}
Although there are 20 branch instructions listed in the instruction
set summary, the following two form the basis of all branching
instructions:
\begin{itemize}
    \item \keywordinline{brbc} (branch if bit in SREG is cleared)
    \item \keywordinline{brbs} (branch if bit in SREG is set)
\end{itemize}
All other branching instructions are specific cases of the above
instructions, that are provided to make programming in Assembly easier.
As these instructions check the bits in the SREG, they are usually
preceded by an ALU operation such as \keywordinline{cp} or
\keywordinline{cpi} to trigger the required flags.

As only 7 bits are allocated to the destination address in these
opcodes, branch instructions cannot jump as far as the
\keywordinline{jmp} and \keywordinline{rjmp} instructions.
\subsubsection{Compare Instructions}
Both the \keywordinline{cp} and \keywordinline{cpi} instructions are
used to compare the values in one or two registers. When used, the ALU
performs a subtraction operation on the two operands and updates the
SREG\@. Note that the result is not stored or used in any way as it is
not relevant to the operation.
\begin{itemize}
    \item \mintinline{ca65}{|\keyword{cp}| Rd, Rr} performs \mintinline{ca65}{Rd} \(-\) \mintinline{ca65}{Rr}
    \item \mintinline{ca65}{|\keyword{cpi}| Rd, K} performs \mintinline{ca65}{Rd} \(-\) \mintinline{ca65}{K}
\end{itemize}
\begin{minted}{ca65}
|\keyword{ldi}| r16, 0
|\keyword{ldi}| r19, 10
|\keyword{cp}| r16, r19       ; Compare values in registers r16 and r19
|\keyword{brge}| new_location ; Branch if r16 greater than or equal to r19

new_location:
\end{minted}
Note that many instructions are able to set the Z flag, which is used
to indicate if the result of the operation is zero. In these cases, the
compare instruction may be redundant.
\subsubsection{Skip Instructions}
The skip instructions are less flexible than branch instructions, but
can sometimes be more suitable as they require less space and fewer
cycles. Skip instructions skip the next instruction if a condition is
true. In this example we will skip the line which increments register
16.
\begin{minted}{ca65}
|\keyword{cpse}| r16, r17 ; Skips next instruction if r16 == r17
inc r16       ; This is skipped
\end{minted}
Compare this with the branch instruction:
\begin{minted}{ca65}
|\keyword{cp}| r16, r17
|\keyword{breq}| new_location ; Skips to new_location if r16 == r17
inc r16           ; This is skipped

new_location: ; PC is now here
\end{minted}
Note that the number of cycles for a skip instruction depends on the
size of the instruction being skipped. The \keywordinline{sbrc} and
\keywordinline{sbrs} instructions are used to skip the next instruction
if the specified bit a register is cleared/set.
\begin{minted}{ca65}
|\keyword{ldi}| r16, 0b00101110

|\keyword{sbrc}| r16, 0 ; Skips next instruction if bit 0 of r16 is cleared
inc r16     ; This is skipped
\end{minted}
Comparing this with the branch instruction:
\begin{minted}{ca65}
|\keyword{ldi}| r16, 0b00101110
|\keyword{andi}| r16, 0b00000001 ; Isolate bit 0
|\keyword{breq}| new_location    ; Skips next instruction if r16 == 0
inc r16              ; This is skipped

new_location: ; PC is now here
\end{minted}
The \keywordinline{sbis} and \keywordinline{sbic} instructions are used
to skip the next instruction if the specified bit an I/O register is
set/cleared. For example, if we wish to toggle the decimal point LED
(DISP DP) on the QUTy (PORT B pin 5) when the first button (BUTTON0)
was pressed (PORT A pin 4),
\begin{minted}{ca65}
|\keyword{ldi}| r16, PIN5_bm           ; Bitmask of pin 5
|\keyword{sbis}| VPORTA_IN, 0b00010000 ; Skip next instruction if pin 4 of PORT A is set
|\keyword{sts}| PORTB_OUTTGL, r16      ; Toggle the output driver of pin 5 on PORT B
\end{minted}
Comparing this with the branch instruction:
\begin{minted}{ca65}
|\keyword{in}| r17, VPORTA_IN    ; Read the input register of PORT A
|\keyword{andi}| r17, 0b00010000 ; Isolate pin 4

|\keyword{brne}| new_location ; Skip instructions if r17 != 0

|\keyword{ldi}| r16, PIN5_bm      ; Bitmask of pin 5
|\keyword{sts}| PORTB_OUTTGL, r16 ; Toggle the output driver of pin 5 on PORT B

new_location:
\end{minted}
\subsection{Loops}
By jumping to an earlier address, we can loop over a block of
instructions.
\begin{minted}{ca65}
infinite_loop:
    ; Code to repeat
    |\keyword{rjmp}| infinite_loop
\end{minted}
Loops can also be finite, in which case the loop will terminate when a
counter reaches zero.
\begin{minted}{ca65}
|\keyword{ldi}| r16, 10 ; Set counter to 10
loop:
    dec r16   ; Decrement counter
    |\keyword{brne}| loop ; Branch if counter != 0
\end{minted}
Loops can also be used to repeat until some external event occurs.
\begin{minted}{ca65}
main_loop:
    |\keyword{in}| r17, VPORTA_IN    ; Read the input register of PORT A
    |\keyword{andi}| r17, 0b00010000 ; Isolate pin 4

    |\keyword{brne}| main_loop       ; Branch if counter != 0
    |\keyword{rjmp}| button_pressed

button_pressed:
    ; Execute instructions
    |\keyword{rjmp}| main_loop ; Return to main loop
\end{minted}
\section{Working with AVR Assembly}
\subsection{Delays}
Loops can be utilised to delay the execution of instructions. These
instructions do not execute any useful code. This is useful for when we
wish to wait for an external event to occur.\footnote{Note that this
type of loop is not recommended for time-sharing systems, such as a
personal computer, as the lost CPU cycles cannot be used by other
programs. In these cases, clock interrupts are preferred. However, on a
device such as the ATtiny1626, delay loops can be utilised to precisely
insert delays in a program.}

To create a precisely timed delay, we must take the following values
into account.
\begin{itemize}
    \item The clock speed --- frequency of the clocks oscillations
          (default: \qty{20}{MHz} --- configurable in
          CLKCTLR\_MCLKCTRLA)
    \item The prescaler --- reduces the frequency of the CPU clock
          through division by a specific amount; 12 different settings
          from 1x to 64x (default: 6 --- configurable in
          CLKCTLR\_MCLKCTRLA)
\end{itemize}
The clock oscillates at its effective clock speed:
\begin{equation*}
    \text{effective clock speed} = \text{clock speed} \times \frac{1}{\text{prescaler}}
\end{equation*}
As the default prescaler is 6, the default effective clock speed is
\qty{3.333}{MHz}. The effective clock speed can therefore range between:
\begin{itemize}
    \item Effective maximum clock frequency: \qty{20}{MHz}
          (\qty{20}{MHz} clock \& prescaler 1)\footnote{As the QUTy is
          supplied with \qty{3.3}{V}, it is not safe to go above
          \qty{10}{MHz}.}
    \item Effective minimum clock frequency: \qty{512}{Hz}
          (\qty{32.768}{kHz} clock \& prescaler 64)
\end{itemize}
Therefore to create a delay, we must first determine the required number
of CPU cycles in the body of the loop and iterate until the number of
CPU cycles reaches the required amount. The following examples utilise
counters of various sizes to create delays. Note that \(n\) represents
the number of iterations.
\begin{minted}{ca65}
delay_1:
    |\keyword{ldi}| r16, x ; 1 CPU cycle
    |\keyword{ldi}| r17, 1 ; 1 CPU cycle ; Incrementor

    loop:
        add r16, r17 ; 1 CPU cycle
        |\keyword{brcc}| loop    ; 2 CPU cycles (1 CPU cycle when condition is false)
\end{minted}
The register \mintinline{ca65}{r16} has the following relationship:
\begin{equation*}
    x = \left( 2^8 - 1 \right) - n \iff n = \left( 2^8 - 1 \right) - x
\end{equation*}
with
\begin{align*}
    \text{total cycles} & = 1 + 1 + \left( n + 1 \right) + 2 n + 1 \\
                        & = 3 n + 4
\end{align*}
for a maximum delay of \qty{230.7}{\micro s} (\(\left( 3 \times \left( 2^8 - 1 \right) + 4 \right) T\))\footnote{\(T\) is the period of one CPU cycle (using the default clock configuration): \(T = \frac{1}{\qty{20}{MHz} / 6} = \qty{300}{ns}\).}.
To create larger delays, we can use multiple registers:
\begin{minted}{ca65}
delay_2:
    |\keyword{ldi}| r24, x ; 1 CPU cycle
    |\keyword{ldi}| r25, y ; 1 CPU cycle

    loop:
        |\keyword{adiw}| r24, 1 ; 2 CPU cycles
        |\keyword{brcc}| loop   ; 2 CPU cycles (1 CPU cycle when condition is false)
\end{minted}
The register pair \(\left( y:x \right)\) has the following
relationship:
\begin{align*}
    \left( y:x \right) = \left( 2^{16} - 1 \right) - n \iff n = \left( 2^{16} - 1 \right) - \left( y:x \right)
\end{align*}
with
\begin{align*}
    \text{total cycles} & = 1 + 1 + 2 \left( n + 1 \right) + 2 n + 1 \\
                        & = 4n + 5
\end{align*}
for a maximum delay of \qty{78.644}{ms} (\(\left(4 \times \left( 2^{16} - 1 \right) + 5 \right) T\)).
With three registers,
\begin{minted}{ca65}
delay_3:
    |\keyword{ldi}| r24, x ; 1 CPU cycle
    |\keyword{ldi}| r25, y ; 1 CPU cycle
    |\keyword{ldi}| r26, z ; 1 CPU cycle

    loop:
        |\keyword{adiw}| r24, 1 ; 2 CPU cycles
        |\keyword{adc}| r26, r0 ; 1 CPU cycle (r0 represents a register with value 0)
        |\keyword{brcc}| loop   ; 2 CPU cycles (1 CPU cycle when condition is false)
\end{minted}
The register triplet \(\left( z:y:x \right)\) is determined through:
\begin{align*}
    \left( z:y:x \right) = \left( 2^{24} - 1 \right) - n \iff n = \left( 2^{24} - 1 \right) - \left( z:y:x \right)
\end{align*}
with
\begin{align*}
    \text{total cycles} & = 1 + 1 + 1 + 2 \left( n + 1 \right) + \left( n + 1 \right) + 2 n + 1 \\
                        & = 5n + 7
\end{align*}
for a maximum delay of \qty{25.166}{s} (\(\left(5 \times \left( 2^{24} - 1 \right) + 7 \right) T\)).
This approach can be extended to create delays of any length.
If needed, we can also include the \mintinline{ca65}{nop} (no
operation) instruction which consumes 1 CPU cycle and does nothing. In
addition to this, we can also utilise nested loops, although the timing
is more complex to determine.
\subsection{Memory and IO}
On the AVR Core, as both I/O and SRAM are accessed through the data
space, they can be directly accessed using instructions that read/write
to memory. This approach is known as memory-mapped I/O (MMIO) and is
used to simplify the design of the AVR Core and reduce chip complexity.

In the AVR architecture, programs are located in a separate address
space, (although still accessible through the data space). This is in
contrast to modern CPU architectures.
\begin{figure}[H]
    \centering
    \includegraphics[height = 8cm, keepaspectratio = true]{figures/memory_map.pdf}
    \caption{ATtiny1626 memory map.} % \label{}
\end{figure}
The following instructions may be used to access memory from the data space:
\begin{itemize}
    \item \keywordinline{lds} (load direct from data space to register)
    \item \keywordinline{sts} (store direct from register to data space)
    \item \keywordinline{ld} (load indirect from data space to register)
    \item \keywordinline{st} (store indirect from register to data space)
    \item \keywordinline{push}/\keywordinline{pop} (stack operations in SRAM --- starting at \mintinline{ca65}{0x3800})
    \item \keywordinline{in}/\keywordinline{out} (single cycle I/O register operations)
    \item \keywordinline{sbi}/\keywordinline{cbi} (set/clear bit in I/O register)
\end{itemize}
Note that the \keywordinline{in}/\keywordinline{out} instructions can
only access the low 64 bytes of the I/O register space and the \keywordinline{sbi}/\keywordinline{cbi}
instructions can only access the low 32 bytes of the I/O register space.
As their names suggest, these instructions only require a single CPU
cycle and hence several addresses such as VPORT\{A, B, C\} (virtual
ports) are mapped to this location, for fast access.
\subsubsection{Load/Store Indirect}
While the \keywordinline{lds}/\keywordinline{sts} instructions can be
used to access addresses of bytes, they are generally not suitable for
accessing data structures such as arrays. Instead, we can use the
\keywordinline{ld}/\keywordinline{st} instructions to take advantage of
their 16-bit pointer registers, which support some pointer arithmetic.
\begin{itemize}
    \item \mintinline{ca65}{r26} \(\to\) \keywordinline{XL} (\keywordinline{X}-register low byte)
    \item \mintinline{ca65}{r27} \(\to\) \keywordinline{XH} (\keywordinline{X}-register high byte)
    \item \mintinline{ca65}{r28} \(\to\) \keywordinline{YL} (\keywordinline{Y}-register low byte)
    \item \mintinline{ca65}{r29} \(\to\) \keywordinline{YH} (\keywordinline{Y}-register high byte)
    \item \mintinline{ca65}{r30} \(\to\) \keywordinline{ZL} (\keywordinline{Z}-register low byte)
    \item \mintinline{ca65}{r31} \(\to\) \keywordinline{ZH} (\keywordinline{Z}-register high byte)
\end{itemize}
For example, if we wanted to access a byte in RAM, we can do the following:
\begin{minted}{ca65}
|\keyword{ldi}| XL, lo8(RAMSTART) ; Store address of RAM in X
|\keyword{ldi}| XH, hi8(RAMSTART)

|\keyword{ld}| r16, X ; Load byte from X to r16
; The byte in X is now in r16

|\keyword{ldi}| r17, 24
|\keyword{st}| X, r17 ; Store byte from r16 to X
; The byte in X (and hence at RAMSTART) is now 24
\end{minted}
These pointer registers also support post-increment and pre-decrement
operations:
\begin{itemize}
    \item \mintinline{ca65}{X+} (post-increment pointer address)
    \item \mintinline{ca65}{-X} (pre-decrement pointer address)
\end{itemize}
\begin{minted}{ca65}
|\keyword{ld}| r16, X+ ; Load byte from X to r16, then X <- X + 1
|\keyword{st}| X+, r16 ; Store byte from r16 to X, then X <- X + 1

|\keyword{ld}| r16, -X ; X <- X - 1, then load byte from X to r16
|\keyword{st}| -X, r16 ; X <- X - 1, then store byte from r16 to X
\end{minted}
This operation can be used to copy bytes from one location to another:
\begin{minted}{ca65}
; Copy 10 bytes from RAM to RAM+10
|\keyword{ldi}| XL, lo8(RAMSTART)
|\keyword{ldi}| XH, hi8(RAMSTART)

|\keyword{ldi}| YL, lo8(RAMSTART+10)
|\keyword{ldi}| YH, hi8(RAMSTART+10)

|\keyword{ldi}| r16, 10 ; Loop 10 times
loop:
    |\keyword{ld}| r0, X+ ; Load byte from X to r0, then X <- X + 1
    |\keyword{st}| Y+, r0 ; Store byte from r0 to Y, then Y <- Y + 1
    dec r16
    |\keyword{brne}| loop
\end{minted}
\subsubsection{Load/Store Indirect with Displacement}
In addition to the \keywordinline{ld}/\keywordinline{st} instructions,
the \keywordinline{ldd}/\keywordinline{std} instructions are a special
form that allow us to load/store from/to the address of the pointer
register \textbf{plus} \(q = \left\{ \numrange{0}{63} \right\}\).
\begin{minted}{ca65}
|\keyword{ldi}| YL, lo8(RAMSTART)
|\keyword{ldi}| YH, hi8(RAMSTART)

|\keyword{ldd}| r0, Y+20 ; Load byte from Y+20 to r0
|\keyword{std}| Y+21, r0 ; Store byte from r1 to Y+21
; Note Y still points to RAMSTART
\end{minted}
Note this form is only available for \keywordinline{Y}, and
\keywordinline{Z}.
\subsection{Stack}
The stack is a last-in first-out (LIFO) data structure in SRAM\@. It is
accessed through a register called the stack pointer (SP), which is not
part of the register file like SREG\@.

Upon reset, SP is set to the last available address in SRAM
(\mintinline{ca65}{0x3FFF}), and can be modified through
\keywordinline{push}/\keywordinline{pop} and other methods that are
generally not recommended.
\begin{itemize}
    \item \keywordinline{push} stores a register to SP then decrements SP (\(\mathrm{SP} \leftarrow \mathrm{SP} - 1\))
    \item \keywordinline{pop} increments SP (\(\mathrm{SP} \leftarrow \mathrm{SP} + 1\)) then loads to a register from SP
\end{itemize}
If a particular register is required without modifying other code, we can temporarily
store the value of that register on the stack, and pop it back when we are done:
\begin{minted}{ca65}
|\keyword{push}| ZL ; Temporarily store Z on the stack
|\keyword{push}| ZH
; Z may be used for another purpose
|\keyword{pop}| ZH ; Restore Z from the stack in reverse order
|\keyword{pop}| ZL
\end{minted}
\subsection{Procedures}
Procedures allow us to write modular, reusable code which makes them
powerful when working on complex projects. Although they are usually
associated with high level languages as methods, or functions, they are
also available in assembly.

Procedures begin with a label, and end with the \mintinline{ca65}{ret}
keyword. They must be \textbf{called} using the
\keywordinline{call}/\keywordinline{rcall} instructions.
\begin{minted}{ca65}
procedure:
    ; Procedure body
    |\keyword{ret}| ; Return to caller
\end{minted}
\subsubsection{Saving Context}
To ensure that procedures are maximally flexible and place no
constraints on the caller, we must always restore any modified
registers before returning to the caller. The same is true for the
SREG\@.
\begin{minted}{ca65}
|\keyword{rjmp}| main_loop

procedure:
    |\keyword{push}| r16 ; Save r16 on the stack
    ; Code that possibly modifies r16
    |\keyword{pop}| r16 ; Restore r16 from the stack
    |\keyword{ret}|

main_loop:
    |\keyword{ldi}| r16, 10
    |\keyword{rcall}| procedure ; Call procedure
    |\keyword{push}| r16 ; r16 should still be 10
\end{minted}
\subsubsection{Parameters and Return Values}
Parameters can be passed using registers or the stack depending on the
size of the inputs.
\begin{minted}{ca65}
|\keyword{rjmp}| main_loop

; Calculate the average of two numbers
; Inputs:
;     r16: first number
;     r17: second number
; Outputs:
;     r16: average
average:
    |\keyword{push}| r0 ; Save r0
    |\keyword{in}| r0, CPU_SREG ; Save SREG
    |\keyword{push}| r0

    ; Calculate average
    |\keyword{add}| r16, r17
    ror r16

    |\keyword{pop}| r0 ; Restore SREG
    |\keyword{out}| CPU_SREG, r0
    |\keyword{pop}| r0 ; Restore r0
    |\keyword{ret}|

main_loop:
    ; Arguments
    |\keyword{ldi}| r16, 100
    |\keyword{ldi}| r17, 200
    |\keyword{rcall}| average
\end{minted}
Using the stack:
\begin{minted}{ca65}
|\keyword{rjmp}| main_loop

; Calculate the average of two numbers
; Inputs:
;     top two values on stack
; Outputs:
;     r16: average
average:
    |\keyword{push}| ZL ; Save Z
    |\keyword{push}| ZH
    |\keyword{in}| ZL, CPU_SREG ; Save SREG
    |\keyword{push}| ZL
    |\keyword{push}| r17 ; Save r17
    |\keyword{in}| ZL, CPU_SPL ; Get SP location
    |\keyword{in}| ZH, CPU_SPH

    ; Get numbers number
    |\keyword{ldd}| r16, Z+7
    |\keyword{ldd}| r17, Z+6

    ; Calculate average
    |\keyword{add}| r16, r17
    ror r16

    |\keyword{pop}| r17 ; Restore r17
    |\keyword{pop}| ZL ; Restore SREG
    |\keyword{out}| CPU_SREG, ZL
    |\keyword{pop}| ZH ; Restore Z
    |\keyword{pop}| ZL
    |\keyword{ret}|

main_loop:
    ; Arguments
    |\keyword{ldi}| r16, 100
    |\keyword{push}| r16
    |\keyword{ldi}| r16, 200
    |\keyword{push}| r16
    |\keyword{rcall}| average

    ; Remove arguments from the stack
    |\keyword{pop}| r0
    |\keyword{pop}| r0
\end{minted}
Note that it is preferable to return values using registers.
\part{C Programming}
\section{Introduction to C Programming}
C is a programming language developed in the early 1970s by Dennis
Richie. C is a compiled language, meaning that a separate program is
used to efficiently translate the source code into assembly. Its
compilers are capable of targeting a wide variety of microprocessor
architectures and hence it is used to implement all major operating
system kernels. Compared to many other languages, C is a very efficient
programming language as its constructs map directly onto machine
instructions.
\subsection{Basic Structure}
\subsubsection{The Main Function}
C is a procedural language and hence all code subsides in a procedure
(known as a \textbf{function}). In C, the \mintinline{c}{main} function
is the \textbf{entry point} to the program. Program execution will
generally begin in this function, where we can make calls to other
functions.
\begin{minted}{c}
int main()
{
    // Function body
    return 0;
}
\end{minted}
The purpose of returning a zero at the end of the \mintinline{c}{main}
function is to signify the \textbf{exit status code} of the process. An
exit status of \mintinline{c}{0} is traditionally used to indicate
success, while all non-zero values indicate failure.
\subsubsection{Statements and Comments}
C programs are made up of statements. Statements are placed within
scopes (indicated by braces (\mintinline{c}{{}})) and are executed in
the order they are placed. All statements in C must terminate with a
semicolon (\mintinline{c}{;}). Although assembly instructions translate
to a single opcode, a single C statement can translate to multiple
opcodes.
\begin{minted}{c}
int main()
{
    int x = 3;
    {
        int y = 4;
        x = x + y;
    }
    // x is now 7
    // y is no longer in scope
    return 0;
}
\end{minted}
C supports two styles of comments. The first of these are known as
``C-style comments'', which allow multi-line/block comments. Multi-line
comments use the \mintinline{c}{/* */} syntax.
\begin{minted}{c}
/*
    This is a multi-line comment.
    It can span multiple lines.
*/
\end{minted}
The second style is known as ``C++-style comments'', which allow
single-line comments. These comments are denoted by the
\mintinline{c}{//} syntax.
\begin{minted}{c}
// This is a single-line comment.
int x = 3; // It can be placed after a statement.
\end{minted}
All comments in C are ignored by the compiler.
\subsection{Variables, Literals, and Types}
\subsubsection{Declaring and Initialising Variables}
Variables are used to temporarily store values in memory. Variables
have a \textbf{type} and a \textbf{name} and must be declared before
use. To declare a variable in C, we must specify the type and name of
that variable.
\begin{minted}{c}
int x;
\end{minted}
This variable can then be \textbf{assigned to} using the
\mintinline{c}{=} operator.
\begin{minted}{c}
x = 4;
\end{minted}
To optionally assign a value during declaration, we can apply the
assignment operator after the declaration. This is known as a variable
\textbf{initialisation}, as we are assigning an initial value to the
variable.
\begin{minted}{c}
int x = 4;
\end{minted}
Note that using \textbf{uninitialised variables} results in
\textbf{unspecified behaviour} in C, meaning that the value of such
variables is unpredictable. If we want to assign values to multiple
variables of the same type, we can use the comma (\mintinline{c}{,})
operator.
\begin{minted}{c}
int x = 1, y = 2, z = 3;
\end{minted}
We can also use the assignment (\mintinline{c}{=}) operator to assign
the same value to multiple variables of the same type.
\begin{minted}{c}
int x, y, z;
x = y = z = 5;
\end{minted}
Compound assignment operators perform the operation specified by the
additional operator, then assign the result to the left operand.
\begin{minted}[escapeinside=??]{c}
char x = 0b11001010;
x |= 0b00000001; // x = 0b11001010 | 0b00000001 = 0b11001011

int y = 25;
y += 5; // y = 25 + 5 = 30

char z = 0b10000010;
z <<= 1; // z = 0b10000010 << 1 = 0b00000100
\end{minted}
\subsubsection{Types}
While AVR assembly uses 8-bit registers, C supports larger data types
by treating them as a sequence of bytes. We can also create compound
data types with \mintinline{c}{struct} and \mintinline{c}{union}.

\vspace{1em}
\textbf{Type Specifiers}
\vspace{1em}

Type specifiers in declarations define the type of the variable. The
\mintinline{c}{signed char}, \mintinline{c}{signed int}, and
\mintinline{c}{signed short int}, \mintinline{c}{signed long int}
types, together with their \mintinline{c}{unsigned} variants and
\mintinline{c}{enum}, are all known as \textbf{integral} types.
\mintinline{c}{float}, \mintinline{c}{double}, and
% latexignore
\mintinline{c}{long double} are known as \textbf{floating} or
\textbf{floating-point}
types. The following table summarises various numeric types in C\@:
\begin{table}[H]
    \centering
    \begin{tabular}{c c c}
        \toprule
        \textbf{Description}           & \textbf{Size} & \textbf{Equivalent Definitions}                              \\
        \midrule
        Character data                 & \qty{1}{B}    & \mintinline{c}{signed char c; char c;}                       \\
        Signed short                   & \qty{2}{B}    & \mintinline{c}{signed short int s; signed short s; short s;} \\
        Unsigned short                 & \qty{2}{B}    & \mintinline{c}{unsigned short int us; unsigned short us;}    \\
        Signed integer                 & \qty{4}{B}    & \mintinline{c}{signed int i; signed i; int i;}               \\
        Unsigned integer               & \qty{4}{B}    & \mintinline{c}{unsigned int ui; unsigned ui;}                \\
        Signed long                    & \qty{8}{B}    & \mintinline{c}{signed long int l; signed long l; long l;}    \\
        Unsigned long                  & \qty{8}{B}    & \mintinline{c}{unsigned long int ul; unsigned long ul;}      \\
        Single precision floating      & \qty{4}{B}    & \mintinline{c}{float f;}                                     \\
        Double precision floating      & \qty{8}{B}    & \mintinline{c}{double d;}                                    \\
        Long double precision floating & \qty{16}{B}   & \mintinline{c}{long double ld;}                              \\
        \bottomrule
    \end{tabular}
    % \caption{} % \label{}
\end{table}
Note that the size of these types is not necessarily the same across platforms, hence it is discouraged to use these keywords for
platform specific tasks. \emph{See the section on \hyperref[sec:exact_width_types]{Exact Width Types} for more information}.

\vspace{1em}
\textbf{Type Qualifiers}
\vspace{1em}

Types can be qualified with additional keywords to modify the
properties of the identifier. Three common qualifiers are
\textbf{const}, \textbf{static}, and \textbf{volatile}.
\begin{itemize}
    \item \mintinline{c}{const} --- indicates that the variable is \textbf{constant} and cannot be modified.
    \item \mintinline{c}{static}
          \begin{itemize}
              \item In a global context --- indicates that the variable
                    is only accessible within the file.
              \item In a local context --- indicates that the variable
                    maintains its value between function invocations.
          \end{itemize}
    \item \mintinline{c}{volatile} --- indicates that the variable can be modified or accessed by other programs or hardware.
\end{itemize}

\vspace{1em}
\textbf{Portable Types}
\vspace{1em}

C has a set of standard types that are defined in the language
specification, however the type specifiers shown above may have
different storage sizes depending on the platform. Although this may be
insignificant for most platforms, microcontrollers use specific sizes
for registers, meaning it is important to refer to the correct type
specifiers when declaring a variable.

\vspace{1em}
\textbf{Exact Width Types}\label{sec:exact_width_types}
\vspace{1em}

The standard integer (\mintinline{c}{stdint.h}) library provides
\textbf{exact-width} type definitions that are specific to the
development platform. This ensures that variables can be initialised
with the correct size on any platform.
\begin{minted}{c}
#include <stdint.h>

int8_t i8;
int16_t i16;
int32_t i32;
int64_t i64;

uint8_t ui8;
uint16_t ui16;
uint32_t ui32;
uint64_t ui64;
\end{minted}

\vspace{1em}
\textbf{Floating-Point Types}
\vspace{1em}

The \mintinline{c}{float} and \mintinline{c}{double} types can store
\textbf{floating-point} value types in C. Their implementation allows
for variable levels of precision, i.e., extremely large and small
values. These types are very useful on systems with a floating point
unit (FPU) or equivalent. In most computer systems, floating point
types are represented as 32-bit IEEE 754 single precision floating
point numbers. The \mintinline{c}{float} type is a 32-bit floating
point number and the \mintinline{c}{double} type is a 64-bit floating
point number.

A single precision floating point number has a 1-bit sign, 8-bit
exponent, and 23-bit mantissa. As such, the range of a single precision
floating point number is \(-2^{127} \ldots 2^{127}\). The value of a
floating point number \(f\) is defined as
\begin{equation*}
    f = \left( -1 \right)^s \left( 1 + 2^{-23} m \right) 2^{e - 127}
\end{equation*}
where \(s\) is the sign bit, \(m\) is the mantissa, and \(e\) is the
exponent. Note that values are not equally spaced and there are several
special values that can be represented by floating point numbers.
\begin{itemize}
    \item \(e = 255 \implies 2^{128}\):
          \begin{itemize}
              \item \(m = 0\) (all 0s): \mintinline{c}{INFINITY} if \(s = 0\), \mintinline{c}{-INFINITY} if \(s = 1\)
              \item \(m\) is not all 0s: \mintinline{c}{NAN}
          \end{itemize}
    \item \(e = 0 \implies 2^{-126}\) (de-normalised):
          \begin{itemize}
              \item \(m = 0\) (all 0s): \mintinline{c}{0.0} if \(s = 0\), \mintinline{c}{-0.0} if \(s = 1\)
              \item \(m\) is not all 0s: Subnormal numbers
          \end{itemize}
\end{itemize}
The flexibility of floating point numbers means that arithmetic
operations are expensive if not performed on a Floating Point Unit
(FPU). As the ATtiny1626 does not have an FPU, floating point operations
must be handled using ALU instructions, which can be extremely slow,
especially when compared to integer operations. In addition, floating
point operations require the \mintinline{c}{avr-libc} floating point
library to be linked, which increases the size of the program.

Fixed point mathematics is a technique for performing arithmetic
operations on integers that are scaled by a power of two. This allows
for integer arithmetic to be used instead of floating point arithmetic,
which can be significantly faster at the cost of precision. For common
operations such as sine and cosine, consider using lookup tables.
\subsection{Literals}
\subsubsection{Integer Prefixes}
Integer literals are assumed to be base 10 unless a prefix is
specified. C supports the following prefixes:
\begin{itemize}
    \item \textbf{Binary} (base 2) --- \mintinline{c}{0b}
    \item \textbf{Octal} (base 8) --- \mintinline{c}{0} (note that C does not use the \mintinline{c}{0o} prefix)
    \item \textbf{Decimal} (base 10) --- no prefix
    \item \textbf{Hexadecimal} (base 16) --- \mintinline{c}{0x}
\end{itemize}
\subsubsection{Integer Suffixes}
Integer literals can be suffixed to specify the size/type of the value:
\begin{itemize}
    \item \textbf{Unsigned} --- \mintinline{c}{U}
    \item \textbf{Long} --- \mintinline{c}{L}
    \item \textbf{Long Long} --- \mintinline{c}{LL}
\end{itemize}
Suffixes are generally only required when clarifying ambiguity of values
where the user wishes to use a different type than the default type, or
when an operation may lead to truncation due to an overflow.
\begin{minted}{c}
#include <stdio.h>

printf("%d\n", 2147483648);  // Treated as signed integer and throws warning
printf("%d\n", 2147483648U); // Treated as unsigned integer
\end{minted}
\subsubsection{Floating Point Suffixes}
As with integer types, floating point values can also be suffixed to
specify which type to use.
\begin{itemize}
    \item \textbf{Float} --- \mintinline{c}{f}
    \item \textbf{Double} --- \mintinline{c}{d}
\end{itemize}
\subsection{Flow Control}
\subsubsection{If Statements}
C provides a standard branching control structure known as an
\mintinline{c}{if} statement. This structure tests a condition and
executes a block of code if that condition is \textbf{true}.
\begin{minted}{c}
if (condition)
{
    // Code to execute if condition is true
}
\end{minted}
This structure can be nested and also supports \mintinline{c}{else} and
\mintinline{c}{else if} statements.
\begin{minted}{c}
if (x > 1)
{
    // Code to execute if x is greater than 1
    if (x < 10)
    {
        // Code to execute if x is greater than 1 and less than 10
    }
} else if (x < -1)
{
    // Code to execute if x is less than -1
    if (x > -10)
    {
        // Code to execute if x is less than -1 and greater than -10
    }
} else
{
    // Code to execute if x is not greater than 1 and not less than -1
}
\end{minted}
\subsubsection{While Loops}
The simplest loop structure in C is achieved by using a
\mintinline{c}{while} loop. This loop executes a block of code while
the condition is \textbf{true}.
\begin{minted}{c}
while (condition)
{
    // Code to execute while condition is true
}
\end{minted}
A \mintinline{c}{do} while loop is similar to a \mintinline{c}{while}
loop, but the loop will execute at least once.
\begin{minted}{c}
do
{
    // Code to execute at least once
} while (condition);
\end{minted}
This loop structure is typically accompanied by a looping variable
known as an iterator:
\begin{minted}{c}
int i = 0; // Iterator

// Execute code 10 times
while (i < 10)
{
    // Code to execute while i is less than 10

    i++; // Increment i by 1
}
\end{minted}
\subsubsection{For Loops}
\mintinline{c}{for} loops are similar to \mintinline{c}{while} loops, but they usually result in more understandable code.
\begin{minted}{c}
for (initialisation; condition; increment)
{
    // Code to execute while condition is true
}
\end{minted}
Note the initialisation and increment statements are optional, and
while the condition statement is also optional, we must ensure that the
loop can terminate from within the structure (see next section).
\subsubsection{Break and Continue Statements}
\mintinline{c}{break} and \mintinline{c}{continue} statements are used to terminate a loop early.
\begin{minted}{c}
for (int i = 0; i < 10; i++)
{
    if (i == 5)
    {
        break; // Terminate loop early
    }
    printf("%d\n", i);
}
\end{minted}
\begin{minted}{c}
for (int i = 0; i < 10; i++)
{
    if (i == 5)
    {
        continue; // Skip current iteration and continue with next iteration
    }
    printf("%d\n", i);
}
\end{minted}
If the loop is nested within another loop, the \mintinline{c}{break}
and \mintinline{c}{continue} statements will only terminate the
innermost loop.
\begin{minted}{c}
for (int i = 0; i < 10; i++)
{
    for (int j = 0; j < 10; j++)
    {
        if (j == 5)
        {
            break; // Terminate inner loop early
        }
        printf("%d\n", j);
    }
}
\end{minted}
\subsection{Expressions}
C provides a number of operators which can be used to perform
arithmetic/logical operations on values. C follows the same precedence
rules as mathematics, however caution should be used when comparing
precedence of certain logical and bitwise operations.
\subsubsection{Operation Precedence}
\setminted{escapeinside={?*}{*?}}
\begin{table}[H]
    \centering
    \begin{tabular}{c c c}
        \toprule
        \textbf{Operation}            & \textbf{Operator Symbol}                                                                                                                                                                                                  & \textbf{Associativity}         \\
        \midrule
        Postfix                       & \mintinline{c}{++}, \mintinline{c}{--}                                                                                                                                                                                    & \multirow{5}{*}{Left to right} \\
        Function call                 & \mintinline{c}{()}                                                                                                                                                                                                        &                                \\
        Array sub-scripting           & \mintinline{c}{[]}                                                                                                                                                                                                        &                                \\
        Member access                 & \mintinline{c}{.}                                                                                                                                                                                                         &                                \\
        Member access through pointer & \mintinline{c}{->}                                                                                                                                                                                                        &                                \\
        \midrule
        Prefix                        & \mintinline{c}{++}, \mintinline{c}{--}                                                                                                                                                                                    & \multirow{7}{*}{Right to left} \\
        Unary                         & \mintinline{c}{+}, \mintinline{c}{-}                                                                                                                                                                                      &                                \\
        Logical NOT and bitwise NOT   & \mintinline{c}{!}, \mintinline{c}{~}                                                                                                                                                                                      &                                \\
        Type cast                     & \mintinline{c}{(type)}                                                                                                                                                                                                    &                                \\
        Dereference                   & \mintinline{c}{*}                                                                                                                                                                                                         &                                \\
        Address-of                    & \mintinline[escapeinside=||]{c}{|\&|}                                                                                                                                                                                     &                                \\
        Size-of                       & \mintinline{c}{sizeof}                                                                                                                                                                                                    &                                \\
        \midrule
        Multiplicative                & \mintinline{c}{*}, \mintinline{c}{/}, \mintinline[escapeinside=||]{c}{|\%|}                                                                                                                                               & Left to right                  \\
        Additive                      & \mintinline{c}{+}, \mintinline{c}{-}                                                                                                                                                                                      & Left to right                  \\
        Bitwise shift                 & \mintinline{c}{<<}, \mintinline{c}{>>}                                                                                                                                                                                    & Left to right                  \\
        Relational                    & \mintinline{c}{<}, \mintinline{c}{>}, \mintinline{c}{<=}, \mintinline{c}{>=}                                                                                                                                              & Left to right                  \\
        Equality                      & \mintinline{c}{==}, \mintinline{c}{!=}                                                                                                                                                                                    & Left to right                  \\
        Bitwise AND                   & \mintinline[escapeinside=||]{c}{|\&|}                                                                                                                                                                                     & Left to right                  \\
        Bitwise XOR                   & \mintinline{c}{^}                                                                                                                                                                                                         & Left to right                  \\
        Bitwise OR                    & \mintinline{c}{|}                                                                                                                                                                                                         & Left to right                  \\
        Logical AND                   & \mintinline[escapeinside=||]{c}{|\&\&|}                                                                                                                                                                                   & Left to right                  \\
        Logical OR                    & \mintinline{c}{||}                                                                                                                                                                                                        & Left to right                  \\
        Conditional                   & \mintinline{c}{? :}                                                                                                                                                                                                       & Right to left                  \\
        Assignment                    & \mintinline{c}{=}, \mintinline{c}{+=}, \mintinline{c}{-=}, \mintinline{c}{*=}, \mintinline{c}{/=}, \mintinline[escapeinside=||]{c}{|\%|=}, \mintinline[escapeinside=||]{c}{|\&|=}, \mintinline{c}{^=}, \mintinline{c}{|=} & Right to left                  \\
        Sequential evaluation         & \mintinline{c}{,}                                                                                                                                                                                                         & Left to right                  \\
        \bottomrule
    \end{tabular}
    % \caption{} % \label{}
\end{table}
\subsubsection{Arithmetic Operations}
All arithmetic operations work as expected, noting that integer
division is truncated. If an arithmetic operation causes a type
overflow, the result will depend on the type. For signed integers, the
result of an overflow is \textbf{undefined} in C. For unsigned
integers, the result is truncated to the type size (or the value modulo
the type size).
\subsubsection{Operator Types}
\begin{itemize}
    \item \textbf{Unary} operators --- have a single operand. For example, \mintinline{c}{++} and \mintinline{c}{--}, or \mintinline{c}{+} and \mintinline{c}{-}.
    \item \textbf{Binary} operators --- have two operands. For example, \mintinline{c}{+}, \mintinline{c}{-}, \mintinline{c}{*}, and \mintinline{c}{/}.
    \item \textbf{Ternary} operators --- have three operands. For example, \mintinline{c}{? :}.
\end{itemize}
\subsubsection{Bitwise Operations}
Binary operators behave as expected in C.
\begin{minted}{c}
char x = 0b11001010;
unsigned char y = 0b01100001;

char a = ~x;     // a = ~0b11001010 = 0b00110101
char b = x & y;  // b = 0b11001010 & 0b01100001 = 0b01000000
char c = x | y;  // c = 0b11001010 | 0b01100001 = 0b11101011
char d = x ^ y;  // d = 0b11001010 ^ 0b01100001 = 0b10101011

char e = x << 1; // e = 0b11001010 << 1 = 0b10010100
char f = x >> 1; // f = 0b11001010 >> 1 = 0b11100101
char g = y >> 1; // g = 0b01100001 >> 1 = 0b00110000
\end{minted}
Note that right shifts are automatically sign-extended in C.
\subsubsection{Relational Operations}
Relational operators can be used to compare two values.
\begin{minted}{c}
int x = 5;
int y = 10;
int z = 15;

if (x < y)
{
    printf("x is less than y\n");
}

if (x != 15)
{
    printf("x is not equal to 15\n");
}
\end{minted}
\subsubsection{Logical Operations}
Logical operators can be used to combine two boolean expressions.
\begin{minted}{c}
int x = 5;
int y = 10;
int z = 15;

if (x < y && x != 15)
{
    printf("x is less than y and x is not equal to 15\n");
}
\end{minted}
\subsubsection{Increment and Decrement Operators}
Increment and decrement operators are unary operators that can be used
to increment or decrement a variable by 1.
\begin{minted}{c}
int x = 5;
x++; // x = 6
\end{minted}
The increment and decrement operators can be used as either prefix or
postfix operators.
\begin{minted}{c}
int x = 5;
int y = x++; // y = 5, x = 6
int z = ++x; // z = 7, x = 7
\end{minted}
Prefix operators are evaluated before the statement is executed, while
postfix operators are evaluated after the statement is executed.
\section{Compiling and Linking C Programs}
\subsection{Preprocessing}
The preprocessor processes C source code before it is passed onto the
compiler. The preprocessor strips out comments, handles
\textbf{preprocessor directives}, and replaces macros. Preprocessors
begin with the \mintinline{c}{#} character and no non-whitespace
characters can appear on the line before the preprocessor directive.
The file generated by the preprocessor is called a \textit{translation
unit} (or a \textit{compilation unit}). Two basic preprocessor
directives are \mintinline{c}{#include} and \mintinline{c}{#define}.
\subsubsection{Include Directives}
The \mintinline{c}{#include} directive is used to include the contents
of another file into the current file. This directive has two forms.
\begin{itemize}
    \item \mintinline{c}{#include <filename>} --- include header files for the C standard library and other header files associated with the target platform.
    \item \mintinline{c}{#include "filename"} --- include programmer-defined header files that are typically in the same directory as the file containing the directive.
\end{itemize}
When this directive is used, it is equivalent to copying the contents of the file into the current file,
at the location of the directive. The included file is also preprocessed and may contain other include directives.
\subsubsection{Define Directives}
The \mintinline{c}{#define} directive is used to define
\textbf{preprocessor macros}. Whenever these macros appear in the
source file, they are replaced with the value specified by the macro.
Macros are a simple text replacement mechanism, and thus must be
defined carefully to avoid invalid code from being generated.
\begin{minted}{c}
#include <stdio.h>
#define PI 3.14159265358979

int main()
{
    printf("%f\n", 2 * PI);
    return 0;
}
\end{minted}
Aside from constant values, macros can also be used to create small
compile-time ``functions'', that expand to code:
\begin{minted}{c}
#include <stdio.h>
#define MAX(x, y) ((x) > (y) ? (x) : (y))

int main()
{
    int x = 5;
    int y = 10;

    int z = MAX(x, y);
    printf("%d\n", z);

    return 0;
}
\end{minted}
Note that the semicolon is omitted at the end of the macro definition,
as it would also be substituted into the program. Only a single
preprocessor directive can appear on a line, and the directives must
occupy a single line (note that a backslash
(\mintinline[escapeinside=||]{c}{|\backslash|}) can be used to break
long lines).
\subsection{Compilation}
After it has been preprocessed, a translation unit can be translated
into machine code by a \textbf{compiler}, similar to how assembly code
is translated into machine code by an \textbf{assembler}. Note that
compilers may also emit assembly code as an intermediate step, which is
then passed to an assembler to produce machine code.
\subsubsection{Compilers}
A compiler is a program that translates a translation unit written in a
high-level language such as C. Compilers can be configured to produce
executable code in various ways. Compiling without optimisation will
produce code that closely resembles source code, making the program
easier to debug. This often also results in faster compilation times.
When releasing a program, a compiler can also be configured to optimise
code for minimum code size or maximum speed (or a combination of both).
Compilers offer some advantages over assembly code:
\begin{itemize}
    \item \textbf{Portability} --- high-level languages are more
          portable than assembly code, as they are not tied to a specific
          instruction set architecture. This means that the same source code
          can be used on another platform, granted that a compiler for that
          platform is available.
    \item \textbf{Efficiency} --- compilers allow us to design efficient
          program through the use of compiler optimisations, which can be
          difficult to achieve manually in assembly code.
\end{itemize}
\subsubsection{Assemblers}
Program code that is run directly on a CPU is not designed to be
written by humans. The assembler prioritises performance and size
efficiency, and accounts for simplified chip design. An assembler
allows us to write programs in plain text without needing to memorise
opcodes or manually keep track of memory locations. Assemblers
prioritise performance and size efficiency, and account for simplified
chip design in their operation. Modern assemblers also provide features
such as macros that allow us to write reusable code. While the scope of
an assembler is limited, the programmer can be confident that the code
produced is efficient and optimal, as instructions are directly
translated into machine code. Some advantages of assemblers are
described below:
\begin{itemize}
    \item \textbf{Efficiency} --- assemblers allow for precise control
          over the hardware, which may not be available in a high-level
          language. In such cases, inline assembly can be used to write
          assembly code within a high-level language.
    \item \textbf{Precision} --- precise timing of code is possible due to
          the ability to predict how the assembler generates code. Compilers
          may introduce additional instructions that can make it difficult to
          predict the exact timing of code.
\end{itemize}
\subsection{Object Files}
An object file is the output from the compilation or assembler phase.
Object files mostly contain machine code and information about the
symbols defined in the source code. Large modularised programs which
split source code into multiple files can be compiled into object files
with \textit{unresolved} external references. These references are
resolved during the linking stage to produce a single executable file.
\subsection{Linking}
During the linking stage, the \textbf{linker} combines multiple object
files and links them together to produce a single executable file. The
linker resolves all external references and updates addresses where
required. This step is necessary when source code is split into
multiple files, and is extremely fast as only addresses need to be
updated. The linker can also perform some optimisations such as dead
code elimination, which removes unused code. In assembly, the
\mintinline{ca65}{.global} directive can be used to make labels
available to the linker.
\begin{minted}{ca65}
// function.S
.global function

function:
    ret

// main.S
rcall function
\end{minted}
In this example, both files can be compiled into object files even
though the \mintinline{ca65}{function} label is not defined in
\mintinline{ca65}{main.S}. The linker will resolve the reference to
\mintinline{ca65}{function} and produce a single executable file. In C,
top-level symbols are public by default, but can be made private to the
current translation unit by using the \mintinline{c}{static} keyword.
\begin{minted}{c}
// main.c
static int a = 0;

// file1.c
a = 1; // Error: a is not visible
\end{minted}
To make a symbol visible to other translation units, the
\mintinline{c}{extern} keyword can be used.
\begin{minted}{c}
// main.c
extern int a;
printf("%d\n", a); // Prints 5

// file1.c
int a = 5;
\end{minted}
Any non-static symbols are implicitly global, and can be accessed from
any translation unit.
\subsection{Debugging}
While most microcontrollers are equipped with debugging tools, we are
often presented with no debugging tools at all. Therefore, it is
important to develop strategies to be able to systematically debug
embedded programs with access to basic I/O. Some simple methods include
toggling pins on the microcontroller to indicate the state of the
program and sending formatted strings through the serial port to a
terminal. To route stdin and stdout to/from any serial communications
interface that can read and write characters (e.g., UART, SPI,
I\({}^2\)C, etc.), we can use the \mintinline{c}{stdio.h} library:
\begin{enumerate}
    \item Declare function prototypes for the read/write functions
          (function names are arbitrary):
          \begin{minted}{c}
static int stdio_putchar(char c, FILE *stream);
static int stdio_getchar(FILE *stream);
    \end{minted}
    \item Declare a stream to be used for stdin/stdout, using the
          \mintinline{c}{FDEV_SETUP_STREAM} macro:
          \begin{minted}{c}
static FILE stdio = FDEV_SETUP_STREAM(stdio_putchar, stdio_getchar, _FDEV_SETUP_RW);
\end{minted}
    \item Implement the prototyped functions that read from the serial
          interface (i.e., via UART):
          \begin{minted}{c}
static int stdio_putchar(char c, FILE *stream)
{
    uart_putc(c);
    return c;
}

static int stdio_getchar(FILE *stream)
{
    return uart_getc();
}

void stdio_init(void)
{
    // Assumes serial interface is initialised elsewhere
    stdout = &stdio;
    stdin = &stdio;
}
    \end{minted}
\end{enumerate}
Here we may use the following blocking functions to read and write
characters to the UART interface:
\begin{minted}{c}
uint8_t uart_getc(void) {
    while (!(USART0.STATUS & USART_RXCIF_bm)); // Wait for data

    return USART0.RXDATAL;
}

void uart_putc(uint8_t c) {
    while (!(USART0.STATUS & USART_DREIF_bm)); // Wait for TX.DATA empty

    USART0.TXDATAL = c;
}
\end{minted}
Assembly listings are also useful for debugging in extreme cases, as
they allow us to see exactly what instructions are being executed. This
can be achieved via the \mintinline{text}{avr-objdump} tool.
\section{Advanced C Programming}
\subsection{Pointers}
When a variable is declared, the compiler automatically allocates a
block of memory to store that variable. We can access this block of
memory directly using the identifier for that variable, or indirectly,
using a \textbf{pointer}. Pointers are variables that store the memory
address of another variable and are declared using the following
syntax:
\begin{minted}{c}
uint8_t *ptr; // Uninitialised pointer to uint8_t data
\end{minted}
This code declares a variable \mintinline{c}{ptr} that ``points to'' a
\mintinline{c}{uint8_t}. Internally, as this pointer is simply an
address, it only occupies the size of a \textbf{memory address}, which
is 16 bits on the ATtiny1626.
\subsubsection{Referencing}
We can \textbf{reference} another variable using the following syntax:
\begin{minted}{c}
uint8_t x = 5;
uint8_t *ptr = &x; // Address of x
\end{minted}
The ampersand (\mintinline{c}{&}) operator is used here to return the
\textbf{address of} the variable \mintinline{c}{x}. Notice that the
type of the pointer \mintinline{c}{ptr} must match the type of
\mintinline{c}{x} to ensure that the pointer is correctly dereferenced.
If we know the address of a location in advance, we can also declare a
pointer with a specific address, ensuring that we cast this address to
a pointer type:
\begin{minted}{c}
volatile uint8_t *ptr = (volatile uint8_t *)0x0421; // The address of PORTB DIRSET
\end{minted}
\subsubsection{Dereferencing}
Once we have defined a pointer, we can indirectly access the value it
references using the \textbf{unary dereference} operator
(\mintinline{c}{*}). This is also known as \textbf{indirection}.
\begin{minted}{c}
uint8_t x = 5;
uint8_t *ptr = &x; // Address of x

// Indirectly read and write from x
uint8_t y = *ptr;  // y = 5  (read from x)
*ptr = 10;         // x = 10 (write to x)
\end{minted}
\subsubsection{Using Qualifiers}
Various qualifiers can be used to modify the type of a pointer, and
these qualifiers can apply to both the pointer and the variable it
references. When reading a pointer declaration with qualifiers, we read
from right to left.
\begin{itemize}
    \item \textbf{Non-constant Data} --- We can reference non-constant
          data without any qualifiers, as shown in the examples above.
          \begin{minted}{c}
uint8_t a = 100;
uint8_t b = 200;

uint8_t *ptr = &a; // pointer to a uint8_t

*ptr = 300; // Can indirectly modify referenced variable
ptr = &b;   // Can reassign pointer to another address
\end{minted}
          We can enforce constancy on the referenced variable by
          applying the \mintinline{c}{const} qualifier to the data type
          in the pointer's declaration, even if the data itself is not
          constant.
          \begin{minted}{c}
uint8_t a = 100;
uint8_t b = 200;

const uint8_t *ptr = &a; // pointer to a constant uint8_t

*ptr = 300; // Error: Cannot indirectly modify referenced variable
ptr = &b;   // Can reassign pointer to another address
\end{minted}
    \item \textbf{Constant Data} --- We can reference constant data
          by using the \mintinline{c}{const} qualifier on the referenced
          data type.
          \begin{minted}{c}
const uint8_t a = 100;
uint8_t b = 200;

const uint8_t *ptr = &a; // pointer to a constant uint8_t

*ptr = 300; // Error: Cannot indirectly modify referenced variable
ptr = &b;   // Can reassign pointer to another address
\end{minted}
          If we wish to modify the referenced variable, we can
          reference the variable without the \mintinline{c}{const}
          qualifier:
          \begin{minted}{c}
const uint8_t a = 100;
uint8_t b = 200;

uint8_t *ptr = &a; // pointer to a uint8_t

*ptr = 300; // Can indirectly modify referenced variable
ptr = &b;   // Can reassign pointer to another address
\end{minted}
    \item \textbf{Constant Pointers} --- We can enforce a pointer to be
          constant by applying the \mintinline{c}{const} qualifier to
          the pointer type. This disallows the pointer from being
          reassigned another address.
          \begin{minted}{c}
uint8_t a = 100;
uint8_t b = 200;

uint8_t *const ptr = &a; // constant pointer to a uint8_t

*ptr = 300; // Can indirectly modify referenced variable
ptr = &b;   // Error: Cannot reassign constant pointer
\end{minted}
    \item \textbf{Constant Pointers to Constant Data} --- We can enforce
          both the pointer and the referenced data to be constant by
          applying the \mintinline{c}{const} qualifier to both the
          pointer and the referenced data type.
          \begin{minted}{c}
uint8_t a = 100;
uint8_t b = 200;

const uint8_t *const ptr = &a; // constant pointer to a constant uint8_t

*ptr = 300; // Error: Cannot indirectly modify referenced variable
ptr = &b;   // Error: Cannot reassign constant pointer
\end{minted}
\end{itemize}
\subsubsection{Pointers to Pointers}
Pointers can also reference other pointers.
\begin{minted}{c}
uint8_t a = 100;

uint8_t *ptr = &a;     // Points to `a`
uint8_t **ptr2 = &ptr; // Points to `ptr`
\end{minted}
This can be used to modify the \textbf{value} of a pointer indirectly.
\begin{minted}{c}
uint8_t a = 100;
uint8_t b = 200;

uint8_t *ptr = &a;     // Points to `a`
uint8_t **ptr2 = &ptr; // Points to `ptr`

*ptr2 = &b; // `ptr` now points to `b`
\end{minted}
These examples demonstrate that pointers are simply variables that
store memory addresses. We may increase the level of indirection by
including more asterisks in a pointer declaration, but this is
generally not very common, nor is it particularly useful in most cases.
In the following example, we apply qualifiers to pointers referencing
other pointers:
\begin{minted}{c}
uint8_t a = 100;
uint8_t *ptr = &a;            // Points to `a`
const uint8_t **ptr1 = &ptr;  // Pointer to constant uint8_t
uint8_t * const *ptr2 = &ptr; // Pointer to constant pointer to uint8_t
uint8_t ** const ptr3 = &ptr; // Constant pointer to pointer to uint8_t
\end{minted}
\subsubsection{Pointer Arithmetic}
A common application of pointer types is for iterating through data
structures such as arrays. Here we often want to move a pointer to the
next or previous element in a sequence. This can be achieved through
arithmetic operators such as \mintinline{c}{++} and \mintinline{c}{--}.
Arithmetic on pointers affects the value of the pointer, so that the
pointer references a different memory location. When performing
arithmetic on pointers, the size of an increment is automatically
determined by the type of the referenced variable.
\begin{minted}{c}
uint8_t a = 100;
uint8_t *ptr = &a; // Points to `a`

ptr++; // Increment by 1 byte (size of uint8_t)
// ptr now points to the next byte in memory after `a`
\end{minted}
\subsubsection{Void Pointers}
When a pointer needs to reference a memory address of an unknown type,
it can be declared with the \mintinline{c}{void} keyword.
\begin{minted}{c}
void *ptr;
\end{minted}
Void pointers have no type, and as such, cannot be dereferenced without
first being cast to a specific type. On the other hand, a void pointer
can be assigned to any other pointer type without a cast.
\begin{minted}{c}
uint8_t a = 100;
void *ptr = &a; // Void pointer

*ptr = 200;            // Error: Cannot dereference void pointer
*(uint8_t *)ptr = 200; // Cast void pointer to uint8_t pointer before dereferencing

uint8_t *ptr2 = ptr;   // Pointer value is copied to `ptr2`
\end{minted}
\subsubsection{Size-of}
The \mintinline{c}{sizeof} function can be used to determine the size
of a variable in bytes.
\begin{minted}{c}
uint8_t a = 100;
uint16_t b = 200;
sizeof(a); // Returns 1
sizeof(b); // Returns 2
\end{minted}
\subsection{Array Types}
Array types are used to hold multiple values of the same type in a
contiguous block of memory. Arrays can be declared in the following
ways:
\begin{minted}{c}
uint8_t a[10];            // Array of 10 uint8_t
uint8_t b[10] = {0};      // Array of 10 uint8_t initialised to 0
uint8_t c[] = {1, 2, 3};  // Array of 3 uint8_t initialised to 1, 2, 3
uint8_t d[5] = {1, 2, 3}; // Array of 5 uint8_t initialised to 1, 2, 3, 0, 0
\end{minted}
The brace (\mintinline{c}{{ }}) syntax can only be used to initialise
an array and if the length of the array which is being assigned is less
than the length of the array being assigned to, the remaining values
will be set to 0.
\subsubsection{Indexing}
Array elements can be accessed with the array index operator
(\mintinline{MATLAB}{[ ]}). In C, array indices start at 0.
\begin{minted}{c}
uint8_t a[10] = {0, 1, 2, 3, 4, 5, 6, 7, 8, 9};
a[0];      // Returns 0
a[1] = 10; // a = {0, 10, 2, 3, 4, 5, 6, 7, 8, 9}
\end{minted}
It is undefined behaviour to access an array element which is out of
bounds. However, it is possible to have a pointer to an element one
past the end of an array as long as the pointer is not dereferenced:
\begin{minted}{c}
uint8_t a[10] = {0, 1, 2, 3, 4, 5, 6, 7, 8, 9}
uint8_t *ptr = &a[10];
\end{minted}
To loop through an array, we can use a \mintinline{c}{for} loop.
\begin{minted}{c}
uint8_t a[10];
for (uint8_t i = 0; i < 10; i++) {
    a[i] = i;
}
\end{minted}
\subsubsection{Array Decay}
Expressions with array types can be converted to expressions with
pointers of the same type, allowing us to use pointer arithmetic with
arrays. However, it is important to note that this results in a loss of
information about the size of the array.
\begin{minted}{c}
uint8_t a[10];
sizeof(a); // Returns 10

uint8_t *ptr = a; // Pointer to first element of a
sizeof(ptr);      // Returns 2 (size of the pointer)
\end{minted}
In the declaration of \mintinline{c}{ptr}, the array \textit{decays}
into a pointer. This property is especially useful when accessing
arrays through function parameters, as arrays themselves cannot be
passed by value. Instead, we can pass a reference to the array by
taking advantage of array decay, ensuring that we also pass a second
parameter for the size of the array, as this information is lost.
\begin{minted}{c}
void set_first_to_100(uint8_t *arr) {
    arr[0] = 100;
}

uint8_t a[10] = {0};
set_first_to_100(a); // a = {100, 0, 0, 0, 0, 0, 0, 0, 0, 0}

void print_array(uint8_t *arr, uint8_t len) {
    for (uint8_t i = 0; i < len; i++) {
        printf("%d ", arr[i]);
    }
}

uint8_t b[10] = {0, 1, 2, 3, 4, 5, 6, 7, 8, 9};
print_array(b, 10); // Prints 0 1 2 3 4 5 6 7 8 9
\end{minted}
Note that the parameter syntax \mintinline{c}{uint8_t *arr} is
equivalent to \mintinline{c}{uint8_t arr[]}, however the former is
generally preferred. The syntax \mintinline{MATLAB}{arr[i]} is
equivalent to \mintinline{MATLAB}{*(arr + i)}. This is possible because
arrays are stored contiguously in memory. Note that it is not possible
to change an array's address:
\begin{minted}{c}
uint8_t a[10];
a++; // Error: Cannot change the address of an array
\end{minted}
\subsubsection{Array Length}
The length of an array can be determined with the
\mintinline{c}{sizeof} function.
\begin{minted}{c}
uint8_t a[10];
uint16_t b[5];
sizeof(a) / sizeof(a[0]); // Returns 10
sizeof(b) / sizeof(b[0]); // Returns 5
\end{minted}
We divide by the size of the first element of the array because the
type of the array may be larger than 1 byte.
\subsubsection{Copying Arrays}
Arrays can be copied in one of two ways. The first approach uses a
\mintinline{c}{for} loop.
\begin{minted}{c}
uint8_t a[10];
uint8_t b[10];
for (uint8_t i = 0; i < sizeof(a) / sizeof(a[0]); i++) {
    b[i] = a[i];
}
\end{minted}
The second approach uses the \mintinline{c}{memcpy} function from the
\mintinline{c}{string.h} library, which performs the same operation.
\begin{minted}{c}
uint8_t a[10];
uint8_t b[10];
memcpy(b, a, sizeof(a) / sizeof(a[0]));
\end{minted}
\subsubsection{Multidimensional Arrays}
Multidimensional arrays (or multiple subscript arrays) are used to hold
multidimensional data.
\begin{minted}{c}
uint8_t a[][3] = {
    {1, 2, 3},
    {4, 5, 6},
    {7, 8, 9}
};
\end{minted}
To declare a multidimensional array, all dimensions but the first need
to be specified. The rows of the array must be specified within
additional braces (\mintinline{c}{{ }}). Elements can be accessed by
specifying the index of each dimension.
\begin{minted}{c}
a[0][0]; // Returns 1
a[1][2]; // Returns 6
\end{minted}
These arrays are also stored contiguously in memory, in
\textbf{row-major} order, and hence pointer arithmetic is performed
differently.
\begin{minted}{c}
uint8_t a[][3] = {
    {1, 2, 3},
    {4, 5, 6},
    {7, 8, 9}
};

uint8_t rows = 3;
uint8_t cols = 3;

for (uint8_t i = 0; i < rows; i++) {
    for (uint8_t j = 0; j < cols; j++) {
        // Double indexing
        printf("%d ", a[i][j]);

        // Single indexing
        printf("%d ", a[i * cols + j]);

        // Pointer arithmetic
        printf("%d ", *(*(a + i) + j));
        // Equivalent to: printf("%d ", *(a[i] + j));
        // Each row is a pointer to the first element of that row
    }
}
\end{minted}
\subsection{Functions}
Procedures are called functions in C. Functions can return values and
take arguments. The main function is the entry point of a program.
\begin{minted}{c}
int main(void) {
    return 0;
}
\end{minted}
Functions in C must be declared in the top-level of a C program, and
thus cannot be declared inside other functions. Functions are declared
with the following syntax:
\begin{minted}{c}
return_type function_name(param_type param_name, ...) {
    // Function body
}
\end{minted}
\subsubsection{Parameters}
The parameters of a function are local variables scoped to that
function.
\begin{minted}{c}
uint8_t add(uint8_t a, uint8_t b) { // `a` and `b` are parameters of `add`
    return a + b;
}

int main(void) {
    uint8_t a = 10;
    uint8_t b = 20;

    uint8_t c = add(a, b); // `a` and `b` are arguments to `add`
}
\end{minted}
When a function does not take any arguments, we can use the
\mintinline{c}{void} keyword to prevent parameters from being passed in
accidentally.
\begin{minted}{c}
void func(void) {
}
\end{minted}
\subsubsection{Return Values}
To return a value from a function, we use the \mintinline{c}{return}
keyword. When a function does not return a value, we can also use
\mintinline{c}{void} for the return type. Note that a void function
does not need to use the \mintinline{c}{return} keyword.
\subsubsection{Function Prototypes}
C uses \textbf{single-pass} compilation, meaning that functions need to
be declared before they can be called. Function prototypes are used to
declare a function without having to specify the entire body of the
function.
\begin{minted}{c}
uint8_t add(uint8_t a, uint8_t b); // Function prototype

int main(void) {
    uint8_t a = 10;
    uint8_t b = 20;

    uint8_t c = add(a, b);
}

uint8_t add(uint8_t a, uint8_t b) {
    return a + b;
}
\end{minted}
The compiler uses the function prototype to generate the code required
to call the function without having to know the entire body of the
function. The linker will then resolve all function calls to the
appropriate function definitions. Note that parameter names are not
required in function prototypes.
\subsubsection{Passing by Reference}
As functions only return one value, we can use pointers to pass
multiple values back to the caller. These output values can be passed
as additional parameters.
\begin{minted}{c}
void swap(uint8_t *a, uint8_t *b) {
    uint8_t temp = *a;
    *a = *b;
    *b = temp;
}

int main(void) {
    uint8_t a = 10;
    uint8_t b = 20;

    swap(&a, &b);
}
\end{minted}
\subsubsection{Call Stack}
As functions can call other functions, or even call themselves, local
variables inside functions are stored on the \textbf{stack}. The return
address of where a function is called from is also stored on the stack
so that the program counter can be reverted to that address when the
function returns. Local variables inside functions do not increase the
explicit SRAM usage reported by the compiler. Rather, this memory will
be allocated on the stack when the function is called. Therefore, it is
important to ensure that the stack does not overflow, through recursive
functions or large local variables.
\subsection{Scope}
All variables and other identifiers in C are scoped. Scope affects the
\textbf{visibility} and \textbf{lifecycle} of variables. Scope is
\textbf{hierarchical}, meaning that variables declared in a parent
scope are visible to all child scopes. Variables declared in a child
scope can also hide variables declared in a parent scope declared with
the same name.
\subsubsection{Global Scope}
Variables declared outside any function are declared in the global
scope. Global variables are visible to all functions in a program.
\begin{minted}{c}
uint8_t a = 10; // Global variable

int main(void) {
    uint8_t b = 20; // Local variable

    a++; // `a` is visible to `main`
    return 0;
}
\end{minted}
Global variables are allocated a fixed location in SRAM and do not
exist on the stack.
\subsubsection{Local Scope}
Variables declared inside a function are declared in the local scope.
Their lifetime is limited to the function in which they are declared.
By default, local variables go on the stack.
\subsubsection{Block Scope}
The block scope is a subset of the local scope. Variables declared
inside blocks such as \mintinline{c}{if} statements have their own
scope. These variables are only visible inside the block. We can create
a new scope by using curly braces.
\subsubsection{Static Variables}
When applied to a \textbf{local variable}, the \mintinline{c}{static}
keyword changes the lifetime of a variable to the lifetime of the
program. This means that the variable will not be destroyed when the
function returns, and will retain its value between function calls.
Static variables are allocated in SRAM and not on the stack. When
applied to a \textbf{global variable}, the \mintinline{c}{static}
keyword changes the visibility of the variable to the file in which it
is declared.
\subsection{Advanced Type Techniques}
\subsubsection{Volatile Qualifiers}
As seen in the previous section, we can use the
\mintinline{c}{volatile} keyword to directly reference memory locations
by their address. This is useful for accessing memory mapped IO\@.
\begin{minted}{c}
volatile uint8_t *portb_outclr = 0x0426;
*portb_outclr = 0b00100000;
\end{minted}
The \mintinline{c}{volatile} keyword is important here because this
data is outside the control of the program. The compiler will therefore
not optimise accesses to this variable. The \mintinline{text}{avr/io.h}
header file includes macros and type definitions for accessing various
registers on the AVR microcontroller with similar declarations.
\begin{minted}{c}
#include <avr/io.h>
\end{minted}
\subsubsection{Type Casting}
While C is a statically typed language, it is possible to convert
between types. Some type conversions are performed implicitly, such as
when converting a \mintinline{c}{uint8_t} to a
\mintinline{c}{uint16_t}. However, some implicit type conversions
generate warnings usually because they result in a loss of information,
or because the conversion is not portable across platforms. Therefore,
to explicitly convert a variable to a different type, we can use the
unary type casting operator.
\begin{minted}{c}
volatile uint8_t *portb_outclr = (volatile uint8_t *)0x0426;
\end{minted}
Applying this type cast does not make this code more portable, but
rather it tells the compiler that the programmer is aware of the
conversion being made, and that it is intentional. Some common type
casts are performed between:
\begin{itemize}
    \item \textbf{Integer types}: Conversions between integer types
          (signed or unsigned) will expand or narrow the type, resulting in a
          truncation or zero extension.
          \begin{minted}{c}
(uint8_t)-3 // 253
\end{minted}
    \item \textbf{Floating point types}: Conversions from floating point
          types to integer types results in truncation of the fractional part.
          \begin{minted}{c}
(int16_t)-3.45 // -3
\end{minted}
    \item \textbf{Pointer types}: Conversions on pointer types change
          the pointer's type but do not affect the referenced data. As
          platforms may store data differently, these conversions are
          not portable.
          \begin{minted}{c}
uint16_t a = 12345;
uint8_t *b = (uint8_t *)&a; // likely 57, but may vary on different platforms
\end{minted}
          Casting an integer to a pointer will cause the resulting
          value to be treated as an address.
          \begin{minted}{c}
uint8_t *ptr = (uint8_t *)0x1234;
\end{minted}
          Likewise, casting a pointer to an integer will cause the
          pointer to be treated as an integer, effectively giving us
          the value of the pointer.
          \begin{minted}{c}
uint8_t *ptr = (uint8_t *)0x1234;
uint16_t addr = (uint16_t)ptr; // addr = 0x1234
\end{minted}
          These conversions are not portable, but are often necessary
          when accessing memory mapped IO\@.
\end{itemize}
Type casting can also be used to add/remove qualifiers such as
\mintinline{c}{const} and \mintinline{c}{volatile}, although this can
lead to undefined behaviour if not used correctly. For example, some
platforms store \mintinline{c}{const} variables in read-only memory,
and attempting to modify these variables is undefined behaviour.
\begin{minted}{c}
const uint8_t a = 10;
const uint8_t *b = &a;

*((uint8_t *)b) = 20; // May lead to undefined behaviour
\end{minted}
Casting can also be used to avoid truncation errors when performing
arithmetic.
\begin{minted}{c}
uint16_t a = 25000;
uint16_t b = 10000;

uint32_t c = a * b;           // c = 45696 (incorrect)
uint32_t d = (uint32_t)a * b; // d = 250000000
\end{minted}
\section{Objects}
\subsection{Structures}
Structures are used to group related data together.
\begin{minted}{c}
struct Point {
    uint8_t x;
    uint8_t y;
};

struct Point p;
\end{minted}
The members of a structure can be accessed using the dot operator.
\begin{minted}{c}
p.x = 30;
p.y = 40;
\end{minted}
Struct members can also be initialised using braces as with arrays.
\begin{minted}{c}
struct Point p = { 30, 40 };
\end{minted}
Unlike arrays, structures need not be accessed via pointers and can be
passed between functions, and copied normally.
\begin{minted}{c}
void func(struct Point p) {
    p.x = 50;
}

struct Point p = { 30, 40 };
func(p);

struct Point q = p; // q.x = 50, q.y = 40
\end{minted}
Due to this, structs can contain arrays which can be passed and copied
by placing them in structs.

Along with this, functions can also return structs.
\begin{minted}{c}
struct Point func() {
    struct Point p = { 30, 40 };
    return p;
}

struct Point p = func(); // p.x = 30, p.y = 40
\end{minted}
\subsubsection{Memory Layout}
Struct members are stored in memory in the order they are declared. If
the platform has alignment requirements, the compiler will insert
padding to ensure that the next member is aligned correctly. This is
done to ensure that the compiler can access the members of the struct
efficiently.
\subsubsection{Anonymous Structures}
Structures can be declared without a name if they are only used once.
\begin{minted}{c}
struct {
    uint8_t x;
    uint8_t y;
} p;
\end{minted}
The type of this variable is \mintinline{c}{unnamed}.
\subsubsection{Structures Inside Structures}
Structures can contain other structures.
\begin{minted}{c}
struct Point {
    uint8_t x;
    uint8_t y;
};

struct Rectangle {
    struct Point p1;
    struct Point p2;
};

struct Rectangle r = { { 10, 20 }, { 30, 40 } };
\end{minted}
\subsubsection{Structures and Pointers}
Structures and members of structs can be addressed normally with the
address-of operator.
\begin{minted}{c}
struct Point p = { 30, 40 };
struct Point *ptr = &p;

ptr->x = 50; // Equivalent to (*ptr).x = 50
\end{minted}
When accessing members of structs through pointers, the arrow operator
(\mintinline{c}{->}) can be used. Structures can also contain pointers.
\subsubsection{Typedef}
Typedefs can be used to give a type an alias so that the variables type
is determined by the typedef instead of the actual type. If we want to
use a structure multiple times, we can use a typedef to give it a (new)
name.
\begin{minted}{c}
typedef struct PointStruct {
    uint8_t x;
    uint8_t y;
} Point;

Point p = { 30, 40 }; // Point is an alias to struct PointStruct
\end{minted}
The type of this variable is \mintinline{c}{Point}. The struct also
need not be defined inside of the typedef.
\begin{minted}{c}
struct PointStruct {
    uint8_t x;
    uint8_t y;
};

typedef struct PointStruct Point;

Point p = { 30, 40 };
\end{minted}
This can be useful when the struct is defined in a header file and the
typedef is defined in a source file. In both cases, it is still
possible to use the struct name to declare variables.
\begin{minted}{c}
struct PointStruct p;
\end{minted}
If the struct name is omitted, the type of the struct is
\mintinline{c}{unnamed}.
\begin{minted}{c}
typedef struct {
    uint8_t x;
    uint8_t y;
} Point;

Point p = { 30, 40 }; // Point is an alias to an unnamed struct
\end{minted}
In this case, the struct name cannot be used to declare variables as it
is anonymous. Typedefs can also be used with qualifiers to reduce
unnecessary code.
\subsection{Unions}
Unions have similar syntax to structures, but all the members of a
union share the same overlapping memory. While structs have capacity to
store multiple members, unions only have the capacity to store their
largest member.
\begin{minted}{c}
union Character {
    char character;
    uint8_t integer;
};

union Character c = { 'A' };

printf("%c\n", c.character); // Prints 'A'
printf("%u\n", c.integer); // Prints 65
\end{minted}
This allows us to interpret the same memory location as multiple types
without needing to perform a cast. When used with structs (or other
aggregates), the order of members in those structs is also maintained.
\begin{minted}{c}
struct a {
    uint8_t i;
    float f;
};

struct b {
    uint8_t i;
    char c[4];
};

union u {
    struct a a;
    struct b b;
};

union u u;

u.a.i = 10;
u.a.f = 3.14;

// u.a.i = 10; u.a.f = 3.14

u.b.c[0] = 'A';
u.b.c[1] = 'B';
u.b.c[2] = 'C';
u.b.c[3] = 'D';

// u.b.i = 10; u.b.c = { 'A', 'B', 'C', 'D' }
\end{minted}
\subsection{Bitfields}
Bitfields can be used within structures or unions to specify types of
specific \textbf{bit} sizes.
\begin{minted}{c}
struct {
    uint8_t x : 4;
    uint8_t y : 4;
} bits;

bits.x = 13;
bits.y = 7;

printf("%u\n", bits.x); // Prints 13
printf("%u\n", bits.y); // Prints 7
printf("%lu\n", sizeof(bits)); // Prints 1 (8 bits)
\end{minted}
In this example, the members \mintinline{c}{x} and \mintinline{c}{y}
each occupy 4 bits. Note that the base type of each member must be able
to store the specified number of bits.
\subsubsection{Properties of Bitfields}
The address of a bitfield cannot be taken.
\begin{minted}{c}
struct {
    uint8_t x : 4;
    uint8_t y : 4;
} bits;

uint8_t *ptr = &bits.x; // Error
\end{minted}
A bitfield cannot be an array.
\begin{minted}{c}
struct {
    uint8_t x : 4;
    uint8_t y[4] : 4;
} bits; // Error
\end{minted}
The name of a bitfield can be omitted to introduce padding.
\begin{minted}{c}
struct {
    uint8_t x : 4;
    uint8_t : 4;
} bits;
\end{minted}
A zero-width bitfield can be used to align the next member to the next
word boundary.
\begin{minted}{c}
struct {
    uint8_t x : 4;
    uint8_t : 0;
    uint8_t y;
} bits;

printf("%lu\n", sizeof(bits)); // Prints 2
\end{minted}
\subsection{Strings}
In C, strings are represented as arrays of characters, terminated by a
character of value 0. Strings that are declared using double quotes
(\mintinline{c}{""}) are automatically terminated by this ``null
character''.
\begin{minted}{c}
char *str = "Hello World";

printf("%s\n", str); // Prints "Hello World\n"

// Because str is a pointer, it can be printed directly.
printf(str); // Prints "Hello World"
printf("\n"); // Prints "\n"
\end{minted}
In the example above, the compiler automatically allocates a block of
memory to store the string, which in this case is 12 bytes long (11
characters + null terminator). The pointer \mintinline{c}{str} points
to the first character in the string.
\begin{minted}{c}
char *str = "Hello World";
*str == 'H'; // True
\end{minted}
When using the \mintinline{c}{printf} function, the null terminator is
required to indicate the end of the string. Strings can therefore be
indexed and passed to functions like arrays.
\section{Interrupts}
An interrupt is a signal sent to the processor to indicate that it
should \textit{interrupt} the current code that is being executed to
execute a function called an \textbf{interrupt service routine} (or
\textit{interrupt handler}). Rather than polling for individual events
(such as button presses), interrupts allow the processor to be notified
when an event occurs.
\subsection{Interrupts and the AVR}
On the ATtiny1626, interrupts work as follows:
\begin{enumerate}
    \item An interrupt-worthy event occurs.
    \item The appropriate interrupt flag (\mintinline{c}{INTFLAGS}) in
          the peripheral is set.
    \item If the corresponding interrupt is enabled
          (\mintinline{c}{INTCTRL} field of the peripheral), the
          interrupt is triggered, and we proceed to the following step.
    \item If the global interrupt flag (\mintinline{c}{SREG.I}) is set,
          the interrupt can be executed, and we proceed to the
          following step.
    \item The PC is pushed onto the stack and jumps to the interrupt
          vector (the address of the interrupt handler). See page
          63--64 on the datasheet.
\end{enumerate}
\subsubsection{Interrupt Vectors}
The \textbf{interrupt vector} is a table of addresses that the
processor jumps to when an interrupt is triggered. These addresses are
usually stored at the beginning of program memory.
\subsubsection{Interrupt Service Routine}
The code that handles an interrupt is called an \textbf{interrupt
service routine} (ISR) (or \textit{interrupt handler}). An ISR is
simply a function that is executed as a result of an interrupt. When
configuring an interrupt, we must temporarily disable interrupts
globally to prevent the interrupt from being triggered while we are
configuring it.
\begin{minted}{c}
cli(); // Disable interrupts globally
// Configure interrupts
sei(); // Enable interrupts globally
\end{minted}
It is also important to restore the state of the CPU or registers
before the ISR returns. This is because another interrupt can be
triggered while the ISR is executing. To tackle this, we can use the
\mintinline{c}{ISR} macro from the \mintinline{c}{avr/interrupt.h}
header file which will automatically save and restore the state of the
CPU and registers.
\begin{minted}{c}
#include <avr/interrupt.h>

ISR(TCB0_INT_vect) {
    // Interrupt service routine for TCB0
}
\end{minted}
This header file also sets aside program memory for the interrupt
vector table.
\subsubsection{Interrupt Flags}
Each peripheral has an interrupt flag field (\mintinline{c}{INTFLAGS})
that is set when the conditions for that interrupt occur (even if
interrupts are disabled). The exact format of this field depends on the
type of interrupt, but in general a bit is set for the type of
interrupt. See the datasheet for more information. As some peripherals
have 1 interrupt vector with multiple interrupt sources, the interrupt
flag fields can be used to determine the exact source of the interrupt.
The interrupt flag field is \textit{usually} cleared by writing a 1 to
the corresponding bit.
\subsubsection{Peripheral Interrupts}
Peripherals differ in what causes interrupts to be raised and many have
multiple interrupt sources.
\subsubsection{Port Interrupts}
To configure \mintinline{c}{BUTTON0} as an interrupt source, we must
enable the interrupt in the \mintinline{c}{PORTA.PIN4CTRL} peripheral.
\begin{minted}{c}
// Interrupt service routine for PORTA
ISR(PORTA_PORT_vect) {
    // Check if the interrupt was caused by PORTA.PIN4
    if (PORTA.INTFLAGS & PIN4_bm) {
        // Interrupt service routine for PORTA.PIN4

        // Clear interrupt flag
        VPORTA.INTFLAGS = PIN4_bm;
    }
}

cli();
// Enable pull-up resistor and interrupt on falling edge
PORTA.PIN4CTRL = PORT_PULLUPEN_bm | PORT_ISC_FALLING_gc;
sei();
\end{minted}
\subsubsection{Interrupts and Synchronisation}
ISR's may interact with state used by other code running at the same
time, which can cause problems with synchronisation, similar to those
faced in multithreaded programs. To avoid this, we should make use of
the \mintinline{c}{volatile} keyword so that the compiler does not make
assumptions about variable states. The \mintinline{c}{cli} and
\mintinline{c}{sei} functions can also be used to disable interrupts
and create a memory barrier which prevents instructions from being
reordered by the compiler.
\section{Hardware Peripherals}
Microcontrollers typically include a variety of hardware peripherals
that remove the burden of having to write software for common
functionality such as timers, serial communication, and analogue to
digital conversion. They can provide very precise timing and very fast
(nanosecond) response times. Hardware peripherals can run independent
of the CPU (in parallel) so that:
\begin{itemize}
    \item peripherals can perform tasks without software intervention
    \item peripherals are not subject to timing constraints (execution
          time of instructions, CPU clock speed)
    \item the CPU can be used to perform other computations while the
          peripheral is busy
\end{itemize}
All control of, and communication with peripherals is done through
\textbf{peripheral registers} which the CPU can access via the memory
map. Peripherals also typically have direct access to hardware resources
such as pins.
\subsection{Configuring Hardware Peripherals}
Upon reset, most hardware peripherals are \textbf{disabled by default}
and must be \textbf{configured and enabled} by writing to the
appropriate peripheral registers. This is often done once at the start
of the program, but can also be reconfigured dynamically if required,
depending on the application. Information about peripheral registers is
found in the datasheet and often recommended steps are also provided.
In general:
\begin{enumerate}
    \item Set bits on peripheral registers to configure the peripheral
          in the correct mode
    \item Enable peripheral interrupts and define an associated ISR, if
          required
    \item Enable the peripheral
\end{enumerate}
It is best practice to globally disable interrupts when configuring
peripherals.
\subsection{Timers}
Timers provide precise measurements of \textbf{elapsed clock cycles} in
hardware, independent of software and the CPU\@. Timers are used to
generate periodic events (via an interrupt), measure time between two
events, or generate periodic signals on a pin.
\subsubsection{Timer Implementations}
Most timer implementations use the same basic structure, a
\textbf{counter} which is incremented or decremented by a
clock/event/etc. By comparing the value of this counter, the timer can
perform more complex behaviours such as generating an interrupt or
changing pin state.
\subsubsection{Timer Counters}
The ATtiny1626 has two 16-bit counters, Timer Counter A and B
(TCA/TCB). As both timers are highly configurable, the datasheet should
be consulted for more information. The counter, accessible via the
\mintinline{c}{CNT} register, increments by 1 each clock cycle. The
clock cycle can be reconfigured with a prescaler to increase the
duration of each clock cycle.
% \begin{itemize}
%     \item The capture compare register \mintinline{c}{CCMP} is used to
%           generate an interrupt when the counter reaches a specific
%           value less than the period. The counter continues to increase
%           past this point.
%     \item
% \end{itemize}
% The period register \mintinline{c}{PER} is used to set the maximum value
% of the counter. After reaching this value, the counter overflows to 0.
% The peripheral can be configured to generate an interrupt when this
% overflow occurs.
\subsubsection{Timer Periods}
To generate an event that occurs every \(T\) seconds, we must configure
a timer with a period of \(T\) seconds. This can be done using the
period register, \mintinline{c}{PER}, which is used to set the maximum
value of the counter, so that when the counter reaches this value, it
overflows back to 0. As the counter increments by 1 each clock cycle,
the duration of a single clock cycle \(T_\mathrm{clk}\) is given by:
\begin{equation*}
    T_\mathrm{clk} = \frac{1}{f_\mathrm{main} / \mathrm{prescaler}},
\end{equation*}
where \(f_\mathrm{main}\) is the frequency of the main clock and
\(\mathrm{prescaler}\) is the prescaler applied to the timer peripheral.
Here we assume the main clock frequency to be \(\qty{20}{\mega\hertz}/6\),
where the main clock has its own default prescaler of 6. If we
want to generate an event every \(T\) seconds, we need to set the period
register to be the number of counts of the timer clock required to reach
the period \(T\):
\begin{equation*}
    n = \frac{T}{T_\mathrm{clk}}
\end{equation*}
Here, \(n\) is the value that should be written to the period register,
\mintinline{c}{PER}. The value of the prescaler influences the timer
period and timer resolution, as an increase in the prescaler leads to
both an increase in the period duration and a decrease in the timer
resolution (time elapsed between each count). Therefore, the smallest
prescaler that allows the desired period should be chosen.
\subsubsection{Timer Counter B Example Configuration}
\begin{minted}{c}
#include <avr/io.h>
#include <avr/interrupt.h>

void tcb_init()
{
    TCB0.CTRLB = TCB_CNTMODE_INT_gc; // Configure TCB0 in periodic interrupt mode
    TCB0.CCMP = 3333;                // Set interval for 1 ms (3333 clocks @ 3.333 MHz)
    TCB0.INTCTRL = TCB_CAPT_bm;      // Invoke the CAPT ISR when the counter reaches CCMP
    TCB0.CTRLA = TCB_ENABLE_bm;      // Enable TCB0
}

int main(void)
{
    cli();
    tcb_init();
    sei();

    while (1);
}
\end{minted}
\subsection{Pulse Width Modulation}
Pulse width modulation (PWM) is a technique used to generate a
\textbf{periodic signal} with a variable duty cycle. The \textbf{duty
cycle} \(D\) of a signal is a measure of the ratio of the HIGH time of
the signal compared to the total PWM period.
\begin{equation*}
    D = \frac{T_\mathrm{HIGH}}{T_\mathrm{HIGH} + T_\mathrm{LOW}} = \frac{T_\mathrm{HIGH}}{T}
\end{equation*}
\begin{itemize}
    \item A duty cycle of 0\% corresponds to a signal that is always
          LOW
    \item A duty cycle of 100\% corresponds to a signal that is always
          HIGH
\end{itemize}
PWM can be used as a form of digital to analogue conversion, where a
\textbf{modulating signal} is used to set the duty cycle of a PWM
output. In analogue, a triangular waveform known as the
\textbf{carrier} is compared with this modulating signal so that the
PWM output is HIGH when the modulating signal is greater than the
carrier.
\subsubsection{PWM Implementation}
On the ATtiny1626, the carrier is generated by a timer counter (i.e.,
\mintinline{c}{TCA0.CNT}) and the modulating signal is the compare
value \mintinline{c}{TCA0.CCMP}. By setting the compare value to a
value less than the counter value, the PWM output's duty cycle can be
controlled, via the following equation:
\begin{equation*}
    D_\mathrm{PWM} = \frac{\text{\mintinline{c}{TCA0.CMPn}}}{\text{\mintinline{c}{TCA0.PER}} + 1}
\end{equation*}
\subsubsection{PWM Brightness Control Example}
\begin{minted}{c}
#include <avr/io.h>
#include <avr/interrupt.h>

void tca_init()
{
    // DISP EN
    PORTB.DIRSET = PIN1_bm;

    // Set waveform generation mode to single slope
    // Waveform output controls PA1 PWM (display brightness)
    TCA0.SINGLE.CTRLB = TCA_SINGLE_WGMODE_SINGLESLOPE_gc | TCA_SINGLE_CMP1EN_bm;
    TCA0.SINGLE.PER = 0xFF;                   // Set period to some value
    TCA0.SINGLE.CMP1 = 0xFF;                  // 100% duty cycle
    TCA0.SINGLE.CTRLA = TCA_SINGLE_ENABLE_bm; // Enable TCA0
}

int main(void)
{
    cli();
    tca_init();
    sei();

    while (1);
}
\end{minted}
As the duty cycle of this PWM signal is set to 100\%, the display will
be at full brightness. To dynamically change the brightness of the
display, the compare value can be changed using the buffered register
\mintinline{c}{TCA0.CMP1BUF}. The same approach should be used to
update the period register. This is done to ensure that a new value is
updated only when the counter is at the bottom of the waveform.
\subsection{Analog to Digital Conversion}
Analogue to digital conversion (ADC) is a technique used to convert an
analogue signal to a digital signal. The analogue signal is sampled at
a regular interval and the sampled value is converted to a digital
value. Digital quantities are both discrete in amplitude and time.
\subsubsection{Quantisation}
\textbf{Discretisation in amplitude} is referred to as
\textbf{quantisation}. Each amplitude is assigned a digital
\textbf{code}. The \textbf{code width} determines the \textbf{amplitude
    resolution} and introduces \textbf{quantisation error}.
\subsubsection{Sampling}
\textbf{Discretisation in time} is referred to as \textbf{sampling}. The
\textbf{sampling rate} determines the \textbf{time resolution} and
introduces \textbf{aliasing error}. This rate is typically the period of
the CPU clock.
\subsubsection{ADC Implementation}
Analogue to digital conversion is the process of discretising a
continuous signal (typically a voltage) into a digital code.
Specialised hardware called an \textbf{analogue to digital converter}
(ADC) performs this function. The ADC samples the analogue signal at a
regular interval at an instant in time, and converts the sampled value
to a digital value.
\subsubsection{ADC Potentiometer Example}
\begin{minted}{c}
#include <stdio.h>

#include <avr/io.h>
#include <avr/interrupt.h>

void adc_init()
{
    // Select AIN2 (potentiometer R1)
    ADC0.MUXPOS = ADC_MUXPOS_AIN2_gc;

    // Need 4 CLK_PER cycles @ 3.333 MHz for 1us, select VDD as ref
    ADC0.CTRLC = (4 << ADC_TIMEBASE_gp) | ADC_REFSEL_VDD_gc;
    // Sample duration of 64
    ADC0.CTRLE = 64;
    // Free running
    ADC0.CTRLF = ADC_FREERUN_bm;
    // Select 8-bit resolution, single-ended
    ADC0.COMMAND = ADC_MODE_SINGLE_8BIT_gc | ADC_START_IMMEDIATE_gc;

    // Enable ADC
    ADC0.CTRLA = ADC_ENABLE_bm;
}

int main(void)
{
    cli();
    adc_init();
    sei();

    while (1)
    {
        printf("%u\n", ADC0.RESULT0);
    }
}
\end{minted}
\subsection{Serial Communication}
Serial communication is the process of transmitting data \textbf{one
bit at a time}. On a microcontroller, this is typically done via a
digital I/O pin. The form of serial communication is determined by the
\textbf{protocol} and \textbf{physical interface} used. The protocol
specifies how the data is arranged, and the timing of bit transfers.
The physical interface specifies the electrical characteristics of the
communication medium (i.e., voltage level used to represent binary
values). For two devices to communicate, they must both use the same
protocol and physical interface.
\subsubsection{Serial Communication Terminology}
\begin{itemize}
    \item \textbf{Transmit}: to send data, often abbreviated to \textbf{Tx}.
    \item \textbf{Receive}: to receive data, often abbreviated to \textbf{Rx}.
    \item \textbf{Simplex}: unidirectional communication. Requires one wire as data only flows in one direction.
    \item \textbf{Half-duplex}: bidirectional communication, occurring in one direction at a time. Requires one wire as data flows in one direction at a time.
    \item \textbf{Full-duplex}: bidirectional communication, occurring simultaneously. Requires two wires as data flows in both directions simultaneously.
    \item \textbf{Synchronous}: communication relying on a shared clock. Requires one wire for the clock signal.
    \item \textbf{Asynchronous}: communication that does not rely on a shared clock.
\end{itemize}
There are many serial interfaces that can be used in embedded systems
for communication. Some common interfaces include:
\begin{itemize}
    \item \textbf{UART}: Universal asynchronous receiver/transmitter.
    \item \textbf{SPI}: Serial peripheral interface.
    \item \textbf{I\({}^2\)C}: Inter-integrated circuit.
    \item \textbf{CAN}: Controller area network.
    \item \textbf{I\({}^2\)S}: Inter-IC sound.
\end{itemize}
\subsubsection{UART}
UART is a simple and cost effective serial communication protocol. As
it is asynchronous, its clock is not shared between the two
communicating devices. Instead, the sender and receiver must agree on a
\textbf{baud rate} (the number of bits transmitted per second). This is
typically in the range of
\qtyrange[range-phrase=~to~]{9600}{115200}{baud} (with a
\qty{2}{M.baud} maximum). UART is a frame based protocol, where each
frame is signalled by a start bit (always LOW), and is fixed in length
and format. UART can be used in both full-duplex or half-duplex,
depending on the hardware implementation, where the transmitter and
receiver are fully independent. This means either a 1- or 2-wire mode
is possible (plus 1 for GND).
\subsubsection{UART Frame Format}
The UART frame format is as follows:
\begin{itemize}
    \item \textbf{Start bit}: always LOW\@.
    \item \textbf{Data bits}: 5 to 9 bits of data\@.
    \item \textbf{Parity bit}: optional bit used to detect errors.
    \item \textbf{Stop bit}: always HIGH\@.
    \item \textbf{Idle}: in the idle state, the line is HIGH\@.
\end{itemize}
The parity bit is used to detect errors in the data bits. It allows the receiver to
detect a single-bit error in the frame. The parity bit can be configured to either be
odd or even parity.
\begin{itemize}
    \item For \textbf{even} parity, the total number of 1s in the data
          and parity bits must be even.
    \item For \textbf{odd} parity, the total number of 1s in the data
          and parity bits must be odd.
\end{itemize}
If a parity error is detected in a received frame, the receiver may choose to reject the frame.
\subsubsection{USART0 on the ATtiny1626}
The USART (Universal Synchronous/Asynchronous Receiver/Transmitter)
peripheral is used to configure UART communication on the ATtiny1626.
The transmission operation is as follows:
\begin{enumerate}
    \item The user loads the data for transmission into the
          \mintinline{c}{TXDATA} register.
    \item When the TX shift register is empty, the data will
          immediately be copied into the shift register.
    \item Data is shifted out from the TX shift register, one bit at a
          time, according to the baud rate.
    \item The transmitter is double-buffered so that:
          \begin{itemize}
              \item If a second byte is loaded into the
                    \mintinline{c}{TXDATA} register before the first
                    byte has finished transmitting, this byte will be
                    transferred into the TX buffer and transmitted
                    after the first byte.
              \item Additionally, writing a third byte will cause it to
                    remain in the \mintinline{c}{TXDATA} register,
                    until the previous two bytes have been transmitted.
          \end{itemize}
\end{enumerate}
The reception operation is as follows:
\begin{enumerate}
    \item The start of an incoming frame is detected based on a falling
          edge on the RX line.
    \item Data is shifted into the RX shift register, one bit at a
          time, according to the baud rate.
    \item Once the correct number of data bits have been shifted into
          the shift register, the data will be copied into the
          \mintinline{c}{RXDATA} register.
    \item The receiver is double-buffered so that:
          \begin{itemize}
              \item If a second byte is shifted out of the shift
                    register before the first byte is read, it will be
                    stored in the RX buffer.
              \item Additionally, a third byte will remain in the RX
                    shift register until the RX buffer is empty.
          \end{itemize}
\end{enumerate}
For more information about the USART0 peripheral, see the datasheet.
\subsubsection{USART0 Example Configuration}
\begin{minted}{c}
void uart_init(void)
{
    PORTB.DIRSET = PIN2_bm;                       // Output enable TX pin
    USART0.BAUD = 1389;                           // 9600 baud
    USART0.CTRLA = USART_RXCIE_bm;                // Enable RX interrupt
    USART0.CTRLB = USART_RXEN_bm | USART_TXEN_bm; // Enable RX and TX
}
\end{minted}
\subsubsection{Serial Peripheral Interface}
SPI is a synchronous serial communication protocol where a clock is
transmitted to allow for higher bit rates in the range
\qtyrange{10}{20}{MHz}. SPI is typically used for high-speed, inter-IC
communications (communication with other peripherals) over short
distances. SPI is also \linebreak full-duplex, and can be configured to
use a 2-, 3-, 4-wire modes (plus 1 for GND). It can be used to
communicate with multiple devices simultaneously, using \textbf{chip
select} (CS) (or slave select) lines to select the device to
communicate with. Typically, in a master-slave model, the master device
controls the clock.

The SPI peripheral can be used to interface to devices that produce or
consume a serial bit stream, clocked or otherwise. The clock phase and
polarity can also be modified to suit the device being interfaced to.
It is also possible to have multiple masters on the same bus, but this
requires a careful mechanism to arbitrate access to the bus.
\subsubsection{SPI0 Example Configuration}
\begin{minted}{c}
void spi_init(void)
{
    PORTC.DIRSET = PIN0_bm | PIN2_bm; // Output enable SPI CLK and SPI MOSI

    PORTMUX.SPIROUTEA = PORTMUX_SPI0_ALT1_gc;

    SPI0.CTRLB = SPI_SSD_bm;  // Disable client select line
    SPI0.INTCTRL = SPI_IE_bm; // Enable SPI interrupts (to latch SPI DATA)
    SPI0.CTRLA = SPI_MASTER_bm | SPI_ENABLE_bm; // Enable SPI as master
}
\end{minted}
\subsubsection{Other Serial Protocols}
\begin{itemize}
    \item \textbf{I\({}^2\)C}: Inter-integrated circuit.
          \begin{itemize}
              \item Very common on microcontrollers, suitable for short
                    distances only (typically < \qty{300}{m.m}). Widely
                    used for external peripherals.
              \item Typically, up to \qty{400}{k.baud}.
              \item Half-duplex, synchronous, 2-wire bus, with
                    bidirectional signalling, where devices use
                    open-drain outputs.
          \end{itemize}
    \item \textbf{CAN}: Controller area network.
          \begin{itemize}
              \item Very prevalent automotive and industrial standard,
                    suitable for medium distances
                    (\qtyrange{40}{500}{m})
              \item Robust and reliable (safety critical systems)
              \item Built-in message priority and arbitration
              \item Typically up to \qty{1}{M.baud}
              \item Half-duplex, asynchronous, 2-wire bus (one
                    differential pair)
              \item Very precise timing requirements compared with UART
              \item Complex protocol and controller
          \end{itemize}
\end{itemize}
\subsubsection{Polled vs Interrupt Driven}
In a polled model for serial communication, the CPU is continuously
checking the status of the peripheral to see if it is ready to transmit
or receive data. This is referred to as a \textbf{blocking} read/write,
as the program does not proceed until the read/write completes. This
delay may be significant for a slow serial interface as the CPU cannot
do anything else while waiting for the peripheral to complete the
operation.

Alternatively, we can use an interrupt-driven model, where we can use
interrupts to signal when the peripheral is ready for new data to be
read/written. In this model, no delays are incurred by the CPU, and we
are guaranteed that data is already ready to be read/written. The
read/write operations are also completed in a deterministic amount of
time, and the CPU can do other tasks while waiting for the peripheral
to complete the operation.
\subsection{Serial Communications on the QUTy}
\subsubsection{Virtual COM Port via USB-UART Bridge}
The CP2102N is a USB-UART bridge that allows the QUTy to communicate
with a PC via a USB cable. On the USB side (host/computer), it presents
itself as a virtual COM port (VCP), and on the microcontroller side, it
presents itself as a UART interface. The bytes written via UART are
received by the VCP, and vice versa. Note that the TX pin (output) of
the microcontroller is connected to the RX pin (input) of the VCP, and
vice versa.

The UPDI pin is used to program the flash memory of the QUTy with a
program. It shares the USB-UART bridge with the UART interface, which
necessitates a switch to toggle between the two.
\subsubsection{Controlling the 7-Segment Display}
The 7-segment display is interfaced to the microcontroller by a 74HC595
\textbf{shift register}. A shift register is a device which translates
\textbf{serial} input/output into \textbf{parallel} input/output. On
the QUTy, it takes a serial, 1-bit output from the microcontroller and
uses this to control, in parallel, the 7-segment display, plus a digit
select signal. The 74HC595 takes a \textbf{clocked serial data stream}
as an input, which makes interfacing via the SPI peripheral very
simple.

Q0-Q6 on the shift register control the 7 segments of the display, and
Q7 controls the digit select. The first bit clocked out of the
microcontroller will set the state of Q7, and the last bit clocked out
will set the state of Q0. To latch the data shifted into the shift
register and consequently update the state of Q0-Q7, requires a
\textbf{rising edge} on the DISP LATCH net.
\subsubsection{Time Multiplexing}
As only one side of the 7-segment display can be illuminated at a time,
we can use a \textbf{time multiplexing} scheme, where we periodically
illuminate each digit for a short duration. When a sufficiently large
refresh rate is used, the human eye perceives both digits as being
illuminated at the same time. If we wish to display a number, we can
simply store the state of each digit and switch between them using an
interrupt-driven timer. For example, to configure a \qty{100}{\hertz}
display, we must configure a total period of \qty{10}{ms} (across both
digits). This corresponds to a switch period of \qty{5}{ms}, so that
both digits are illuminated for an equal amount of time, leading to a
uniform brightness across both digits.
\begin{minted}{c}
// Bytes latched onto 7-segment display
volatile uint8_t left_byte = DISP_0 | DISP_LHS;
volatile uint8_t right_byte = DISP_0;

// 5ms interrupt
ISR(TCB0_INT_vect)
{
    static uint8_t current_display_side = 0;

    if (current_display_side)
    {
        SPI0.DATA = left_byte;
    }
    else
    {
        SPI0.DATA = right_byte;
    }

    // Toggle side
    current_display_side ^= 1;

    // Clear interrupt flag
    TCB0.INTFLAGS = TCB_CAPT_bm;
}

ISR(SPI0_INT_vect)
{
    // Latch the byte
    PORTA.OUTCLR = PIN1_bm; // Prepare for rising edge
    PORTA.OUTSET = PIN1_bm; // Generate rising edge

    // Clear interrupt flag
    SPI0.INTFLAGS = SPI_IF_bm;
}
\end{minted}
The display is updated by writing to the variables
\mintinline{c}{left_byte} and \mintinline{c}{right_byte}.
\subsection{Pushbutton Handling}
Pushbuttons have two possible states, \textbf{pressed} and
\textbf{released}. Given an active-low pushbutton,
\begin{itemize}
    \item when a pushbutton is pressed, the corresponding net is pulled
          LOW, and the pins value is \mintinline{c}{0}.
    \item when the pushbutton is released, the corresponding net is
          pulled HIGH, and the pins value \mintinline{c}{1}.
\end{itemize}
The state of individual pushbuttons can be read using bitwise AND
operations with the \mintinline{c}{PORTA.IN} register.
\begin{minted}{c}
#include <avr/io.h>

PORTA.PIN4CTRL = PORT_PULLUPEN_bm;

while (1)
{
    if (PORTA.IN & PIN4_bm)
    {
        // Pushbutton is released
    }
    else
    {
        // Pushbutton is pressed
    }
}
\end{minted}
In this loop structure, the state of the pushbutton will be in the
pressed state until the pushbutton is released. This may not be
desirable if we want to perform a single action when the pushbutton is
pressed. To solve this, we must respond to a \textit{change in state}.
\begin{itemize}
    \item A \textbf{falling edge} is created when the pushbutton is
          pressed (transition from \mintinline{c}{1} to
          \mintinline{c}{0}).
    \item A \textbf{rising edge} is created when the pushbutton is
          released (transition from \mintinline{c}{0} to
          \mintinline{c}{1}).
\end{itemize}
This is known as \textbf{edge detection}. To implement this in C, we can
use the XOR operator (\mintinline{c}{^}) to detect a change in the
signal. To identify the direction of the edge, we can additionally
condition the result with one of the states.
\begin{minted}{c}
uint8_t pb_previous_state = 0xFF;
uint8_t pb_current_state = 0xFF;

while (1)
{
    pb_previous_state = pb_current_state; // save previous state
    pb_current_state = PORTA.IN;          // update with latest measurement

    uint8_t pb_edge = pb_previous_state ^ pb_current_state;
    uint8_t pb_falling_edge = pb_edge & pb_previous_state;
    uint8_t pb_rising_edge = pb_edge & pb_current_state;

    // alternatively if we do not need the edge
    uint8_t pb_falling_edge = pb_previous_state & ~pb_current_state;
    uint8_t pb_rising_edge = ~pb_previous_state & pb_current_state;
}
\end{minted}
\subsubsection{Pushbutton Sampling}
The actuation of mechanical pushbuttons is slow and therefore it is
important to sample pushbutton states fast enough to detect changes in
state. \textbf{Latency} refers to the delay between user input and the
reaction of a system. For user input, latency should be acceptably
small\footnote{What is acceptably small depends on what magnitude of
latency is perceptible and is specific to the application.}. Latency as
low as \qty{2}{ms} is perceptible in particular user input
applications, however latency between \qtyrange{20}{60}{ms} is
acceptable for most applications.
\subsubsection{Switch Bounce}
Switch bounce is an artefact of electromechanical switches where switch
contacts bounce back and forth after a switch is actuated. This results
in a signal that is not stable and can cause false positives when
detecting a change in state. As a digital system can sample a voltage
much faster than a mechanical switch can change state, the system may
detect multiple transitions of the switch state.

To prevent this, we \textbf{debounce} pushbuttons either by using a
debouncing circuit, or through software. To implement this in software,
we can take multiple samples of a pin and only accept a change in state
if these samples are consistent over multiple samples. This can be
implemented using an ISR to capture the state of the pushbutton at
regular intervals.
\begin{minted}{c}
// this variable is used instead of PORTA.IN when detecting edges
volatile uint8_t pb_debounced_state = PIN4_bm;

// Periodic 5ms interrupt
ISR(TCB1_INT_vect)
{
    static uint8_t counter = 3;

    // Capture the state of the pushbutton from the port pin
    uint8_t pb_sample = PORTA.IN & PIN4_bm;
    // Detect a change in state
    uint8_t pb_edge = pb_sample ^ pb_debounced_state;

    if (pb_edge)
    {
        if (counter-- == 0)
        {
            // Update debounced state
            pb_debounced_state = pb_sample;

            // Reset counter
            counter = 3;
        }
    }
    else
    {
        // Reset counter
        counter = 3;
    }

    // Clear interrupt flag
    TCB1.INTFLAGS = TCB_CAPT_bm;
}
\end{minted}
The above implementation can only debounce a single pushbutton, as the
counter corresponds to a single pushbutton, S1. To debounce multiple
pushbuttons, we can declare counters for each pushbutton, however this
can lead to a large amount of duplicate code. Instead, we can utilise
vertical counters.
\subsubsection{Vertical Counters}
Instead of using a counter variable for each pushbutton, we can use the
bits of a single variable to represent the counters for each
pushbutton. Doing so reduces the maximum value of the counter, but this
is not a concern as we only require 3 samples. The following code
implements a vertical counter for 4 pushbuttons.
\begin{minted}{c}
// We can also use PIN4_bm | PIN5_bm | PIN6_bm | PIN7_bm, as we do not care about the other bits
volatile uint8_t pb_debounced_state = 0xFF;

// Periodic 5ms interrupt
ISR(TCB1_INT_vect)

    // Two vertical counters for a total of 4 counter states
    static uint8_t counter0 = 0;
    static uint8_t counter1 = 0;

    // Capture the state of the pushbuttons from the port pin
    uint8_t pb_sample = PORTA.IN;
    // Detect a change in state
    uint8_t pb_edge = pb_sample ^ pb_debounced_state;

    // Update counters
    // If the state of the pushbutton has changed, increment the counter
    counter1 = (counter1 ^ counter0) & pb_edge;
    counter0 = ~counter0 & pb_edge;

    // Update debounced state if counter reaches 3, or immediately on falling edge
    pb_debounced_state ^= (counter1 & counter0) | (pb_edge & pb_previous_state);

    // Clear interrupt flag
    TCB1.INTFLAGS = TCB_CAPT_bm;
}
\end{minted}
This code allows us to debounce up to 8 pushbuttons using a single
8-bit variable. The debounced state of the pushbuttons is stored in
\mintinline{c}{pb_debounced_state} and can be used to perform edge
detection similar to the code in the previous section. This variable is
updated if the state of the pushbutton is consistent over 3 samples, or
if a falling edge is detected. Note that if both falling and rising
edges need to be detected, it is not recommended to immediately update
the debounced state on a falling edge, as this leads to inconsistent
latency between rising and falling edges.
\section{State Machines}
A state machine or finite state machine (FSM) is a mathematical model
of computation in which a machine can only exist in one of a finite
number of states. The machine transitions between states in response to
inputs, and performs actions during transitions. A state machine is
fully defined by its list of states, initial state, and the conditions
for transitioning between states.
\subsection{State Machine Implementation}
To translate a state machine into a C program, we can make use of an
enumerated type. Enumerated types are a special type of data that allow
us to define a set of named constants. Enumerated types can be used to
implement a state machine as follows:
\begin{itemize}
    \item Each enumerator can be used to represent a state.
    \item A \mintinline{c}{switch} statement can be used to implement
          the behaviour in each state.
    \item An \mintinline{c}{if} statement can be used to implement the
          conditions for transitioning between states.
\end{itemize}
\begin{minted}{c}
typedef enum
{
    START,
    STATE1,
    STATE2
} state_t;

// Initial state
state_t state = START;

// Configure outputs for START state

while (1)
{
    // State machine
    switch (state)
    {
        case START:
            if (condition1)
            {
                // Configure outputs for STATE1 state

                // Transition if condition is met
                state = STATE1;
            }
            break;
        case STATE1:
            if (condition2)
            {
                // Configure outputs for STATE2 state

                // Transition if condition is met
                state = STATE2;
            }
            break;
        case STATE2:
            if (condition3)
            {
                // Configure outputs for START state

                // Go back to start if condition is met
                state = START;
            }
            break;
        default: // Invalid state (program should never reach this state)
            // Configure outputs for START state

            // Go back to start
            state = START;
            break;
    }
}
\end{minted}
\subsection{Enumerated Types}
Enumerated types are defined similarly to structures, via the
\mintinline{c}{enum} keyword, and can be anonymous, or named. The
values of an enumerated type are constants, called enumerators, that
are assigned an integer value starting from 0.
\begin{minted}{c}
typedef enum
{
    FALSE,
    TRUE
} boolean_t;

boolean_t b = TRUE; // b is assigned the value 1

b == TRUE; // TRUE is assigned the value 1, so this is true
b == 0; // FALSE is assigned the value 0, so this is false
\end{minted}
While they can be compared to integers, it is recommended to use the
enumerators in comparisons.

Enumerated types can also be defined with explicit values, and can be
used to represent bitmasks.
\begin{minted}{c}
typedef enum
{
    MONDAY = 0b00000001,
    TUESDAY = 0b00000010,
    WEDNESDAY = 0b00000100,
    THURSDAY = 0b00001000,
    FRIDAY = 0b00010000,
    SATURDAY = 0b00100000,
    SUNDAY = 0b01000000,
    WEEKEND = SATURDAY | SUNDAY,
    WEEKDAY = MONDAY | TUESDAY | WEDNESDAY | THURSDAY | FRIDAY
} DAYS;

enum DAYS d = SATURDAY; // d is assigned the value 0b00100000

if (d & WEEKEND) // Check if d is a weekend day
{
    // Do something
}
\end{minted}
\subsection{Switch Statements}
A \mintinline{c}{switch} statement is a control structure that allows
us to select a block of code to execute based on the value of an
expression. When a case is matched, a \mintinline{c}{break} statement
may be used to prevent the program from falling through the case,
however this may be omitted if two states perform the same tasks. The
\mintinline{c}{default} case is executed if no case is matched.
\section{Serial Protocols}
A serial protocol is an agreed-upon standard by which two devices can
communicate with each other, enabling them to exchange data. UART and
SPI are standards for transmitting data, but do not ascribe any meaning
to the data.
\subsection{Serial Protocol Design}
\subsubsection{Requirements for a Serial Protocol}
A serial protocol must:
\begin{itemize}
    \item able to receive data during a transmission
    \item able to recover from errors
    \item engage in flow control
    \item be simple to implement/understand for both the transmitter
          and receiver
\end{itemize}
\subsubsection{Symbols}
A symbol is the fundamental data type used in serial communication
protocols, which can be comprised of several bits. The number of bits
is usually set by the underlying medium and depends on the baud rate.
Smaller symbols are more flexible and allow for more symbols to be
transmitted, whereas larger symbols are more efficient and allow for
more data to be transmitted.
\subsubsection{Messages}
If the information to be exchanged can be entirely encoded within a
single symbol, there is no need for a message structure. However, more
complex protocols require a message structure for large quantities of
data or information of variable length. This is done by dividing the
communication into discrete messages.
\subsubsection{Encoding}
The choice of encoding may also be of concern, depending on the
communication medium, symbol length, and other factors such as human
readability. For example using the entire ASCII character set may not
be desirable as it is not human-readable. Human-readable encoding
schemes usually limit the number of symbols to a small subset of the
ASCII character set:
\begin{itemize}
    \item ASCII \mintinline{c}{32-126} (\mintinline{c}{0x20-0x7E})
          which uses 8-bit symbols
    \item Base64 (\mintinline{c}{0-9}, \mintinline{c}{A-Z},
          \mintinline{c}{a-z}, \mintinline{c}{+}, \mintinline{c}{/})
          which encodes 6 bits into an 8-bit symbol
    \item Hexadecimal (\mintinline{c}{0-9}, \mintinline{c}{A-F}) which
          encodes 4-bits per symbol
\end{itemize}
\subsubsection{Message Structure}
Messages typically contain the following information:
\begin{enumerate}
    \item A \textbf{start sequence} to indicate the beginning of a
          message
    \item An \textbf{identifier} to indicate what type of message is
          being sent (if the protocol requires multiple messages)
    \item A \textbf{payload} containing the data specific to the
          message
    \item A provision for \textbf{escape sequences} to allow for
          special characters (e.g., arbitrary data is not confused with
          start sequences)
    \item A \textbf{checksum} (or message digest), to ensure the
          integrity of the message
    \item A \textbf{stop sequence} to indicate the end of a message
\end{enumerate}
\subsubsection{Start Sequences}
As a serial communication transmits a sequence of symbols with no
structure, there is no guarantee that the entire message is received.
To address this, a start sequence is used to indicate the beginning of
a message. The start sequence is usually a fixed number of unique
symbols that do not appear in the payload, providing a synchronisation
point. If payloads need to contain arbitrary sequences of symbols,
escape sequences may be used.
\subsubsection{Multi-Symbol Start Sequences}
While a single symbol start sequence is simple, a multi-symbol start
sequence has potential benefits:
\begin{itemize}
    \item Reduces need for escape sequences
    \item Reduced likelihood of misinterpreting corrupted data as a
          start sequence
\end{itemize}
\subsubsection{Sub-Symbol Start Sequences}
If messages are encoded with fewer than 8 bits per symbol, the
remaining bits can be used to encode a start sequence. For example, in
UTF-8 encoding, the high bit is always cleared in the first byte of a
sequence.
\subsubsection{Message Identifiers}
Serial protocols often have multiple categories of messages that may be
transmitted. Commonly a fixed-length identifier is transmitted so that
the receiver can respond with the appropriate action, or know when to
expect a payload.
\subsubsection{Payloads}
A payload is used when a message identifier alone is insufficient to
convey the information required. Payloads should be as small as
possible to reduce the overhead of the protocol, as longer payloads
increase the risk of transmission errors, so it may be preferable to
split a large payload into multiple messages.
\subsubsection{Payload Length}
Payloads may be of both fixed and variable length, depending on the
protocol.
\begin{itemize}
    \item For fixed length payloads, the message type itself may define
          the payload length and hence will know when to expect the end
          of the payload.
    \item For variable length payloads, the payload length is encoded
          in the message itself, by either specifying the length within
          the payload, or by using a delimiter to indicate the end of
          the payload.
\end{itemize}
\subsubsection{Variable Length Payloads}
When variable length payloads are expected, two strategies are commonly
used:
\begin{itemize}
    \item Encode the payload length at the start of the payload, either
          with one or two symbols for (1--256) or (1--65536) bytes
          respectively.
    \item Use a \textbf{sentinel} to indicate the end of the payload.
          This is a special symbol that is not used in the payload,
          such as a null character.
\end{itemize}
Note that the second strategy requires the receiver to be able to buffer the entire payload before processing it.
If the payload is too large, this may not be possible.
\subsubsection{Escape Sequences}
When there is potential ambiguity for whether a given symbol or
sequence of symbols is part of a sequence, a payload, or a sentinel for
a payload, escape sequences may be necessary to handle certain
characters. In C, a backslash
(\mintinline[escapeinside=||]{c}{|\backslash|}) is used to escape
characters like the double quote (\mintinline{c}{"}), to tell the
compiler that the character is not the end of the string.

This may not be feasible in a serial protocol as the backslash may be
missed during transmission, and the next character may be treated as a
start sequence. To address this, the escape sequence should not contain
the symbol it is escaping. Instead, an alternate sequence should be
used to represent symbols when they are part of a payload or other
contexts. Note that the escape sequence itself may be part of the
payload, so it is important to account for this also.
\subsubsection{Handshakes}
A protocol where the sender purely transmits data does not know whether
the same information has been received and handled by the receiver. As
such, it is common for protocols where only side is sending information
and the other is receiving, to have the receiving side acknowledge what
it has received (if the serial communications medium is half-duplex or
full-duplex). The most common form of handshakes are \textbf{ACK} and
\textbf{NACK} messages (for acknowledged and not acknowledged).
\begin{itemize}
    \item ACK indicates that the message was received, and that its
          contents were understood.
    \item NACK indicates that the message was received, but that there
          was an error in the message. For example if the message was
          malformed, failed its checksum, or was unable to be
          processed.

          In these situations, the sender may retransmit the message,
          or send a different message.
\end{itemize}
\subsubsection{Message Verification}
As serial communications are prone to transmission errors, it is
important to verify that the message was received correctly. This is
done by including a checksum in which the transmitter computes a value
based on the contents of the message, such as the sum of the bytes, and
transmits it with the message. The receiver then computes a similar
checksum based on what it receives and verifies that it matches the
transmitted checksum. While this simple checksum detects many
transmission errors, it is not guaranteed to handle symbols sent out of
order, or symbols that were corrupted without changing the checksum.
\subsubsection{Flow Control}
When the receiver operates on little power or low storage, it may not
be able to process data when it is transmitted at its usual rate.
Hence, the sender may respond with a flow control message, such as
WAIT, to indicate that it is currently busy processing the previous
message, and RESUME when it is ready to receive more data.
\subsection{Serial Protocol Parsing}
Many serial protocols are designed to be simple to parse, due to the
limitations of hardware used in serial communication. However, it is
important to ensure that concerns around timing, state, and buffers are
addressed to ensure that the parser is robust and reliable. As other
actions may be performed during the parsing of a message, it may not be
possible to use blocking functions such as \mintinline{c}{scanf()}.
Similarly, a periodic interrupt may not be feasible if symbols are not
sent frequently. As such, a better choice may be to use the
\mintinline{c}{UART_RX} interrupt to handle a single character at a
time, using a state machine to handle the parsing of the message.

This state machine can be placed within the interrupt handler, or in a
separate function that is called in the main loop when new characters
are received\footnote{This may be more suitable if the state machine
consumes many CPU cycles.}. The state machine should be designed to
handle the following states:
\begin{enumerate}
    \item \textbf{Idle}: The parser is waiting for a start sequence (and ignores all other symbols)
    \item \textbf{Start Sequence}: The parser is receiving the start sequence
    \item \textbf{Message Identifier}: The parser is receiving the message identifier
    \item \textbf{Payload}: The parser is receiving the payload (may be separate states for various identifiers)
    \item \textbf{Checksum}: The parser is receiving the checksum
\end{enumerate}
\end{document}

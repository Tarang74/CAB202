%!TEX TS-program = xelatex
%!TEX options = -aux-directory=Debug -shell-escape -file-line-error -interaction=nonstopmode -halt-on-error -synctex=1 "%DOC%"
\documentclass[a4paper]{report}
\usepackage[margin = 10mm]{geometry}
% Packages

%% Math enhancements
\usepackage{amsmath} % Misc enhancements to math equations
\usepackage{cancel} % Draw diagonal lines and arrows in math equations
\usepackage{mathtools} % Starred versions of amsmath matrix environments; Multiline, cases, gathered environment
\usepackage{chngcntr} % Reset counter within sections
\usepackage{interval} % Format intervals
\intervalconfig{
    soft open fences
}

%% Symbols
\usepackage{amssymb} % Extended symbol collection - also loads amsfonts
\usepackage{stmaryrd} % Extra symbols

%% Fonts
\usepackage{mathrsfs} % Support \mathcal and \mathscr

%% Environments
\usepackage{amsthm} % Use theorems

%% Tables and arrays
\usepackage{booktabs} % Top and bottom rule for tabular
\usepackage{tabularx} % Advanced Tables

%% Lists
\usepackage{enumitem} % Itemize, enumerate, description environments

%% Page layout
\usepackage{geometry} % Page layout customisation
\usepackage{fancyhdr} % Page headers and footers
\usepackage{float} % Float objects such as figures and tables
\usepackage{tcolorbox} % Create boxed environments

%% Text enhancements
\usepackage[none]{hyphenat} % Disable hyphenation of long text
\usepackage{ragged2e} % Text alignment options

%% Referencing
\usepackage{tocbibind} % Adds bibliography to the Table of Contents
\usepackage{url} % Define urls

%% Graphics
\usepackage{graphicx} % Extension to graphics
% \graphicspath{ {./figures/} }

%% Miscellaneous
\usepackage[outputdir=Debug]{minted} % Typeset programming code
\usepackage{siunitx} % SI units package
\usepackage{derivative} % Derivative notation
\usepackage{pdfpages} % Import PDFs into document

\usepackage[hidelinks]{hyperref} % Handle cross-referencing
\usepackage{bookmark} % New bookmark organisation for hyperref

%% Unicode setup
\usepackage[warnings-off={mathtools-colon, mathtools-overbracket}]{unicode-math}
\setmathfont{Latin Modern Math}
\setmathfont[range={bb, bbit}, Scale=MatchUppercase]{TeX Gyre Pagella Math}
\setmathfont[range={\mathcal, \mathbfcal}, StylisticSet=1]{XITS Math}
\setmathfont[range={\mathscr}]{XITS Math}
\setmathfont[range={"2205}]{XITS Math} % chktex 18

% Preamble

%% Misc Commands

%%% Number Sets
\newcommand*{\N}{\mathbb{N}}
\newcommand*{\Z}{\mathbb{Z}}
\newcommand*{\Q}{\mathbb{Q}}
\newcommand*{\I}{\mathbb{I}}
\newcommand*{\R}{\mathbb{R}}
\newcommand*{\C}{\mathbb{C}}

%%% Empty set character
\let\oldemptyset\emptyset
\let\varnothing\relax
\newcommand{\varnothing}{\char"2205} % chktex 18

%%% Contradiction
\newcommand{\contradiction}{
\hspace{-1em}
{\hbox{
\setbox0=\hbox{\(\mkern-3mu{\times}\mkern-3mu\)}
\setbox1=\hbox to0pt{\hss\copy0\hss}
\copy0\raisebox{0.5\wd0}{\copy1}\raisebox{-0.5\wd0}{\box1}\box0}}
}

%%% Lines for matrices
\newcommand*{\vertbar}{\rule[-1ex]{0.5pt}{2.5ex}}
\newcommand*{\horzbar}{\rule[.5ex]{2.5ex}{0.5pt}}

%% Paired Delimiters
\DeclarePairedDelimiter{\ceil}{\lceil}{\rceil}
\DeclarePairedDelimiter{\floor}{\lfloor}{\rfloor}
\DeclarePairedDelimiter{\abracket}{\langle}{\rangle}
\DeclarePairedDelimiter{\abs}{\lvert}{\rvert}
\DeclarePairedDelimiter{\norm}{\lVert}{\rVert}

%% Probability Functions
\let\Pr\relax
\DeclareMathOperator{\Pr}{Pr}
\DeclareMathOperator{\E}{E}
\DeclareMathOperator{\Var}{Var}
\DeclareMathOperator{\Cov}{Cov}
\DeclareMathOperator{\Corr}{Corr}

\newcommand{\Perm}[2]{\prescript{#1}{}{P}_{#2}}

%% Hyperbolic Functions
\DeclareMathOperator{\arcsinh}{arcsinh}
\DeclareMathOperator{\arccosh}{arccosh}
\DeclareMathOperator{\arctanh}{arctanh}
\DeclareMathOperator{\arccoth}{arccoth}
\DeclareMathOperator{\arcsech}{arcsech}
\DeclareMathOperator{\arccsch}{arccsch}

%% Linear Algebra
%%% Augmented matrices
\makeatletter
\renewcommand*\env@matrix[1][*\c@MaxMatrixCols c]{%
    \hskip -\arraycolsep
    \let\@ifnextchar\new@ifnextchar
    \array{#1}}
\makeatother

%%% Operators
\let\det\relax
\DeclareMathOperator{\det}{det}
\DeclareMathOperator{\Tr}{Tr}
\DeclareMathOperator{\diag}{diag}
\DeclareMathOperator{\adj}{adj}

\DeclareMathOperator{\vspan}{span}
\DeclareMathOperator{\vref}{ref}
\DeclareMathOperator{\vrref}{rref}

\DeclareMathOperator{\vrank}{rank}
\DeclareMathOperator{\vnull}{null}

\DeclareMathOperator{\proj}{proj}

\DeclareMathOperator{\vim}{im}
\DeclareMathOperator{\vcoim}{coim}
\DeclareMathOperator{\vker}{ker}
\DeclareMathOperator{\vcoker}{coker}

\newcommand{\columnspace}[1]{\mathcal{C}\left(\symbf{#1}\right)}
\newcommand{\rowspace}[1]{\mathcal{C}\left(\symbf{#1}^{\top}\right)}
\newcommand{\nullspace}[1]{\mathcal{N}\left(\symbf{#1}\right)}
\newcommand{\leftnullspace}[1]{\mathcal{N}\left(\symbf{#1}^{\top}\right)}

%% Additional operators
\DeclareMathOperator{\erf}{erf}

% Theorems
\theoremstyle{definition}
\newtheorem{definition}{Definition}[section]

\theoremstyle{plain}
\newtheorem{theorem}{Theorem}[subsection]
\newtheorem{corollary}{Corollary}[theorem]
\newtheorem{lemma}{Lemma}[theorem]
\newtheorem{axiom}{Axiom}

\theoremstyle{remark}
\newtheorem{remark}{Remark}
\newtheorem{note}{Note}[subsection]
\newtheorem*{statement}{Statement}

\newenvironment{examples}[1][Examples]{\let\qed\relax\proof[#1]\mbox{}\\*}{\endproof}
\newenvironment{question}[1][Question]{\let\qed\relax\proof[#1]\mbox{}\\*}{\endproof}
\newenvironment{solution}[1][Solution]{\let\qed\relax\proof[#1]\mbox{}\\*}{\endproof}

\newenvironment{proofcase}[1]{\proof[Case #1]\mbox{}}{\endproof}

%% Box styles
\tcbuselibrary{skins}
\newtcolorbox{tcolorboxlarge}[1][]{
    skin=enhanced,
    boxrule=0pt,
    frame hidden,
    sharp corners,
    borderline west={0.5pt}{0pt}{black},
    borderline east={0.5pt}{0pt}{black},
    enlarge left by=10pt,
    width=\linewidth-20pt,
    opacityback=0,
    coltitle=black,
    fonttitle=\large\bfseries,
    #1
}

\newtcolorbox{tcolorboxcols}[1][]{
    skin=enhanced,
    boxrule=0pt,
    frame hidden,
    sharp corners,
    borderline west={0.5pt}{0pt}{black},
    opacityback=0,
    coltitle=black,
    fonttitle=\large\bfseries,
    #1
}

%% Reset counter within subsections
\counterwithin*{equation}{section}
\counterwithin*{equation}{subsection}
\counterwithin*{remark}{subsection}

%% Page layout setup
\pagestyle{fancy}
\setlength\headheight{24pt}
\setlength\parindent{0pt} % Indent first line of new paragraphs


% Additional packages & macros
\usepackage{xcolor}
\newcommand{\keyword}[1]{\textcolor[rgb]{0.00,0.50,0.00}{\textbf{#1}}}
\newcommand{\keywordinline}[1]{\textcolor[rgb]{0.00,0.50,0.00}{\textbf{\mintinline{ca65}{#1}}}}

\pagestyle{plain}

\setlength{\footskip}{0mm}

\setminted{
    escapeinside=||,
    linenos,
    frame=lines
}

\usepackage{subcaption}
\usepackage{multicol}
\usepackage{multirow}

% Header and footer
\newcommand{\unitName}{Microprocessors and Digital Systems}
\newcommand{\unitTime}{Semester 2, 2022}
\newcommand{\unitCoordinator}{Dr Mark Broadmeadow}
\newcommand{\documentAuthors}{Tarang Janawalkar}

% Copyright
\usepackage[
    type={CC},
    modifier={by-nc-sa},
    version={4.0},
    imagewidth={5em},
    hyphenation={raggedright}
]{doclicense}

\date{}

\begin{document}
%
\begin{titlepage}
    \vspace*{\fill}
    \begin{center}
        \LARGE{\textbf{\unitName}} \\[0.1in]
        \normalsize{\unitTime} \\[0.2in]
        \normalsize\textit{\unitCoordinator} \\[0.2in]
        \documentAuthors
    \end{center}
    \vspace*{\fill}
    \doclicenseThis
    \thispagestyle{empty}
\end{titlepage}
\newpage
%
\tableofcontents
\newpage
%
\chapter{Microcontrollers}
\section{Architecture of a Computer}
\begin{definition}[Computer]
    A computer is a digital electronic machine that can be programmed to carry
    out sequences of arithmetic or logical operations (computation) automatically.
\end{definition}
\begin{definition}[Control unit]
    The control unit interprets the instructions and decides what actions to take.
\end{definition}
\begin{definition}[Arithmetic logic unit]
    The arithmetic logic unit (ALU) performs computations required by the control unit.
\end{definition}
\section{Microprocessors \& Microcontrollers}
While a microcontroller puts the CPU and all peripherals onto the same
chip, a microprocessor houses a more powerful CPU on a single chip that
connects to external peripherals. The peripherals include memory, I/O,
and control units. The QUTy uses a microcontroller called ATtiny1626,
that are within a family of microcontrollers called AVRs.
\section{ATtiny1626 Microcontroller}
The ATtiny1626 microcontroller has the following features:
\begin{itemize}
    \item CPU:\@ AVR Core (AVRxt variant)
    \item Memory:\@
          \begin{itemize}
              \item Flash memory (16KB) used to store program
                    instructions in memory
              \item SRAM (2KB) used to store data in memory
              \item EEPROM (256B)
          \end{itemize}
    \item Peripherals:\@ Implemented in hardware (part of the chip) in
          order to offload complexity
\end{itemize}
\subsection{Flash Memory}
\begin{itemize}
    \item Non-volatile --- memory is not lost when power is removed
    \item Inexpensive
    \item Slower than SRAM
    \item Can only erase in large chunks
    \item Typically used to store program data
    \item Generally read-only. Programmed via an external tool, which
          is loaded once and remains static during the lifetime of the
          program
    \item Writing is slow
\end{itemize}
\subsection{SRAM}
\begin{itemize}
    \item Volatile --- memory is lost when power is removed
    \item Expensive
    \item Faster than flash memory and is used to store variables and
          temporary data
    \item Can access individual bytes (large chunk erases are not
          required)
\end{itemize}
\subsection{EEPROM}
\begin{itemize}
    \item Older technology
    \item Expensive
    \item Non-volatile
    \item Can erase individual bytes
\end{itemize}
\section{AVR Core}
\begin{definition}[Computer program]
    A computer program is a sequence or set of instructions in a programming language
    for a computer to execute.
\end{definition}
The main function of the AVR Core Central Processing Unit (CPU) is to ensure correct program execution.
The CPU must, therefore, be able to access memory, perform calculations, control peripherals, and handle interrupts.
Some key characteristics of the AVR Core are:
\begin{itemize}
    \item 8-bit Reduced Instruction Set Computer (RISC)
    \item 32 working registers (R0 to R31)
    \item Program Counter (PC) --- location in memory where the program
          is stored
    \item Status Register (SREG) --- stores key information from
          calculations performed by the ALU (i.e., whether a result is
          negative)
    \item Stack Pointer --- temporary data that doesn't fit into the
          registers
    \item 8-bit core --- all data, registers, and operations, operate within 8-bits
\end{itemize}
\section{Status Register}
The status register is a 8-bit register that stores the result of the
last operation performed by the ALU\@. It has the following flags:
\begin{itemize}
    \item[\textbf{C}] Carry Flag
    \item[\textbf{Z}] Zero Flag
    \item[\textbf{N}] Negative Flag
    \item[\textbf{V}] Two's Complement Overflow Flag
    \item[\textbf{S}] Sign Flag
    \item[\textbf{H}] Half Carry Flag
    \item[\textbf{T}] Transfer Bit
    \item[\textbf{I}] Global Interrupt Enable Bit
\end{itemize}
\section{Program Execution}
At the time of reset, PC = 0. The following steps are then performed:
\begin{enumerate}
    \item Fetch instruction (from memory)
    \item Decode instruction (decode binary instruction)
    \item Execute instruction:
          \begin{itemize}
              \item Execute an operation
              \item Store data in data memory, the ALU, a register, or
                    update the stack pointer
          \end{itemize}
    \item Store result
    \item Update PC (move to next instruction or if instruction is
          longer than 1 word, increment twice. The program can also
          move to another point in the program that has an address
          \(k\), through jumps.)
\end{enumerate}
This is illustrated in the following figure:
\begin{figure}[H]
    \centering
    \includegraphics[height = 12cm, keepaspectratio = true]{figures/AVR_CPU.pdf}
    \caption{Program execution on the ATtiny1626.} % \label{}
\end{figure}
\section{Instructions}
\begin{itemize}
    \item The CPU understands and can execute a limited set of
          instructions --- \textasciitilde88 unique instructions for
          the ATtiny1626
    \item Instructions are encoded in program memory as opcodes. Most
          instructions are two bytes long, but some instructions are
          four bytes long
    \item The AVR Instruction Set Manual describes all of the available
          instructions, and how they are translated into opcodes
    \item Instructions fall loosely into five categories:
          \begin{itemize}
              \item Arithmetic and logic --- ALU
              \item Change of flow --- jumping to different sections of
                    the code or making decisions
              \item Data transfer --- moving data in/out of registers,
                    into the data space, or into RAM
              \item Bit and bit-test --- looking at data in registers
                    (which bits are set or not set)
              \item Control --- changing what the CPU is doing
          \end{itemize}
\end{itemize}
\section{Interacting with memory and peripherals}
\begin{itemize}
    \item The CPU interacts with both memory and peripherals via the
          data space
    \item From the perspective of the CPU, the data space is single
          large array of locations that can be read from, or written
          to, using an address
    \item We control peripherals by reading from, and writing to, their
          registers
    \item Each peripheral register is assigned a unique address in the
          data space
    \item When a peripheral is accessed in this manner we refer to it
          as being memory mapped, as we access them as if they were
          normal memory
    \item Different devices, peripherals and memory can be included in
          a memory map (and sometimes a device can be accessed at
          multiple different addresses)
\end{itemize}
\section{Assembly code}
\begin{itemize}
    \item The opcodes placed into program memory are called machine
          code (i.e., code the machine operates on directly)
    \item We don't write machine code directly as it is:
          \begin{itemize}
              \item Not human readable
              \item Prone to errors (swapping a single bit can
                    completely change the operation)
          \end{itemize}
    \item Instead we can write instructions directly in assembly code
    \item We use instruction mnemonics to specify each instruction:\@
          \keywordinline{ldi}, \keywordinline{add},
          \keywordinline{sts}, \mintinline{ca65}{jmp}, \dots
    \item An assembler takes assembly code and translates it into
          opcodes that can be loaded into program memory
\end{itemize}
\chapter{Digital Representations and Operations}
\section{Representation}
A sequence of \textit{eight} bits is known as a \textbf{byte}, and it
is the most common representation of data in digital systems. A
sequence of \textit{four} bits is known as a \textbf{nibble}. A
sequence of \(n\) bits can represent up to \(2^n\) states.
\subsection{Binary Representation}
Bits are written right-to-left from \textbf{least significant} to
\textbf{most significant} bit.
\begin{itemize}
    \item The \textbf{least significant bit} (LSB) is at bit index 0.
    \item The \textbf{most significant bit} (MSB) is at bit index \(n -
          1\) in an \(n\)-bit sequence.
\end{itemize}
\begin{align*}
    0000_2 & = 0 & 0100_2 & = 4 & 1000_2 & = 8  & 1100_2 = 12  \\
    0001_2 & = 1 & 0101_2 & = 5 & 1001_2 & = 9  & 1101_2 = 13  \\
    0010_2 & = 2 & 0110_2 & = 6 & 1010_2 & = 10 & 1110_2 = 14  \\
    0011_2 & = 3 & 0111_2 & = 7 & 1011_2 & = 11 & 1111_2 = 15
\end{align*}
\subsection{Hexadecimal Representation}
\begin{align*}
    0_{16} & = 0000_2 & 4_{16} & = 0100_2 & 8_{16} & = 1000_2 & C_{16} = 1100_2  \\
    1_{16} & = 0001_2 & 5_{16} & = 0101_2 & 9_{16} & = 1001_2 & D_{16} = 1101_2  \\
    2_{16} & = 0010_2 & 6_{16} & = 0110_2 & A_{16} & = 1010_2 & E_{16} = 1110_2  \\
    3_{16} & = 0011_2 & 7_{16} & = 0111_2 & B_{16} & = 1011_2 & F_{16} = 1111_2
\end{align*}
\subsection{Numeric Literals}
When a fixed value is declared directly in a program, it is referred to
as a \textbf{literal}. Prefixes denote various bases:
\begin{itemize}
    \item \textbf{Binary} notation requires the prefix \mintinline{ca65}{0b}
    \item \textbf{Decimal} notation does not require prefixes
    \item \textbf{Hexadecimal} notation requires the prefix \mintinline{ca65}{0x}
\end{itemize}
\section{Unsigned Integers}
The \textbf{unsigned integers} represent the set of counting (natural)
numbers, starting at 0. In the \textbf{binary system} (base-2) the
unsigned integers are encoded using a sequence of binary digits (0--1).
For example,
\begin{align*}
    10101_2 & = 1 \times 2^4 &  & + 0 \times 2^3 &  & + 1 \times 2^2 &  & + 0 \times 2^1 &  & + 1 \times 2^0 \\
            & = 1 \times 16  &  & + 0 \times 8   &  & + 1 \times 4   &  & + 0 \times 2   &  & + 1 \times 1   \\
            & = 16           &  & + 0            &  & + 4            &  & + 0            &  & + 1            \\
            & = 21_{10}
\end{align*}
The range of values an \(n\)-bit binary number can hold when encoding an unsigned integer is 0 to \(2^n - 1\).
\begin{table}[H]
    \centering
    \begin{tabular}{c c}
        \toprule
        \textbf{No.\ of Bits} & \textbf{Range}                        \\
        \midrule
        8                     & \(0\)--\(255\)                        \\
        16                    & \(0\)--\(\num{65535}\)                \\
        32                    & \(0\)--\(\num{4294967295}\)           \\
        64                    & \(0\)--\(\num{18446744073709551615}\) \\
        \bottomrule
    \end{tabular}
    \caption{Range of available values in binary representations.} % \label{}
\end{table}
\section{Signed Integers}
Signed integers are used to represent integers that can be positive or
negative. The following representations allow us to encode negative
integers using a sequence of binary bits:
\begin{itemize}
    \item Sign-magnitude
    \item One's complement
    \item Two's complement (most common)
\end{itemize}
\subsection{Sign-Magnitude}
In sign-magnitude representation, the most significant bit encodes the
sign of the integer. In an 8-bit sequence, the remaining 7-bits are
used to encode the value of the bit.
\begin{itemize}
    \item If the sign bit is 0, the remaining bits represent a positive
          value,
    \item If the sign bit is 1, the remaining bits represent a negative
          value.
\end{itemize}
As the sign bit consumes one bit from the sequence, the range of values that can be
represented by an \(n\)-bit sign-magnitude encoded bit sequence is:
\begin{equation*}
    -\left( 2^{n - 1} - 1 \right) \text{ to } 2^{n - 1} - 1
\end{equation*}
For 8-bit sequences, this range is: \(-127\) to \(127\).
This presents several issues:
\begin{enumerate}
    \item There are two ways to represent zero:
          \mintinline{ca65}{0b10000000 |=| 0}, or
          \mintinline{ca65}{0b00000000 |=| -0}.
    \item Arithmetic and comparison requires inspecting the sign bit
    \item The range is reduced by 1 (due to the redundant zero
          representation)
\end{enumerate}
\subsection{One's Complement}
In one's complement representation, a negative number is represented by
inverting the bits of a positive number (i.e., \(0 \to 1\) and \(1 \to
0\)). The range of values are still the same:
\begin{equation*}
    -\left( 2^{n - 1} - 1 \right) \text{ to } 2^{n - 1} - 1
\end{equation*}
however, this representation tackles the second problem in the previous representation as
addition is performed via standard binary addition with \textit{end-around carry} (carry bit is added onto result).
\begin{equation*}
    a - b = a + \left( \text{\textasciitilde} b \right) + C.
\end{equation*}
\subsection{Two's Complement}
In two's complement representation, the most significant bit encodes a
negative weighting of \(-2^{n - 1}\). For example, in 8-bit sequences,
index-7 represents a value of \(-128\). The two's complement is
calculated by adding 1 to the one's complement. The range of values
are:
\begin{equation*}
    -2^{n - 1} \text{ to } 2^{n - 1} - 1
\end{equation*}
This representation is more efficient than the previous because \mintinline{ca65}{0} has a single representation
and subtraction is performed by adding the two's complement of the subtrahend.
\begin{equation*}
    a - b = a + \left( \text{\textasciitilde} b + 1 \right).
\end{equation*}
\section{Logical Operators}
\subsection{Boolean Functions}
A Boolean function is a function whose arguments and results assume
values from a two-element set, (usually \(\left\{ 0,\: 1 \right\}\) or
\mintinline{text}{{false, true}}). These functions are also referred to
as \textit{logical functions} when they operate on bits. The most
common logical functions available to microprocessors and most
programming languages are:
\begin{itemize}
    \item Negation: \keywordinline{NOT} \(a\), \textasciitilde\(a\),
          \(\overline{a}\)
    \item Conjunction: \(a\) \mintinline{ca65}{AND} \(b\), \(a\)
          \mintinline{ca65}{&} \(b\), \(a \cdot b\), \(a \land b\)
    \item Disjunction: \(a\) \keywordinline{OR} \(b\), \(a\)
          \mintinline{ca65}{|\vert|} \(b\), \(a + b\), \(a \lor b\)
    \item Exclusive disjunction: \(a\) \keywordinline{XOR} \(b\), \(a\)
          \mintinline{ca65}{^} \(b\), \(a \oplus b\)
\end{itemize}
By convention, we map a bit value of \mintinline{ca65}{0} to \mintinline{ca65}{false}, and a bit value of \mintinline{ca65}{1} to \mintinline{ca65}{true}.
\subsection{Negation}
A unary operator used to \textbf{invert} a bit.
\begin{table}[H]
    \centering
    \begin{tabular}{c c}
        \toprule
        \textbf{\(a\)} & \keywordinline{NOT} \(a\) \\
        \midrule
        0              & 1                         \\
        1              & 0                         \\
        \bottomrule
    \end{tabular}
\end{table}
\subsection{Conjunction}
\mintinline{ca65}{AND} is a binary operator whose output is true if \textbf{both} inputs are \textbf{true}.
\begin{table}[H]
    \centering
    \begin{tabular}{c c c}
        \toprule
        \textbf{\(a\)} & \textbf{\(b\)} & \textbf{\(a\) \mintinline{ca65}{AND} \(b\)} \\
        \midrule
        0              & 0              & 0                                           \\
        0              & 1              & 0                                           \\
        1              & 0              & 0                                           \\
        1              & 1              & 1                                           \\
        \bottomrule
    \end{tabular}
\end{table}
\subsection{Disjunction}
\keywordinline{OR} is a binary operator whose output is true if \textbf{either} input is \textbf{true}.
\begin{table}[H]
    \centering
    \begin{tabular}{c c c}
        \toprule
        \textbf{\(a\)} & \textbf{\(b\)} & \(a\) \keywordinline{OR} \(b\) \\
        \midrule
        0              & 0              & 0                              \\
        0              & 1              & 1                              \\
        1              & 0              & 1                              \\
        1              & 1              & 1                              \\
        \bottomrule
    \end{tabular}
\end{table}
\subsection{Exclusive Disjunction}
\keywordinline{XOR} (Exclusive \keywordinline{OR}) is a binary operator whose output is true if \textbf{only one} input is \textbf{true}.
\begin{table}[H]
    \centering
    \begin{tabular}{c c c}
        \toprule
        \textbf{\(a\)} & \textbf{\(b\)} & \(a\) \keywordinline{XOR} \(b\) \\
        \midrule
        0              & 0              & 0                               \\
        0              & 1              & 1                               \\
        1              & 0              & 1                               \\
        1              & 1              & 0                               \\
        \bottomrule
    \end{tabular}
\end{table}
\subsection{Bitwise Operations}
When applying logical operators to a sequence of bits, the operation is
performed in a \textbf{bitwise} manner. The result of each operation is
stored in the corresponding bit index also.
\section{Bit Manipulation}
Often we need to modify individual bits within a byte, \textbf{without}
modifying other bits. This is accomplished by performing a bitwise
operation on the byte using a \textbf{bit mask} or \textbf{bit field}.

These operations can:
\begin{itemize}
    \item \textbf{Set} specific bits (change value to \mintinline{ca65}{1})
    \item \textbf{Clear} specific bits (change value to \mintinline{ca65}{0})
    \item \textbf{Toggle} specific bits (change values from \(0 \to 1\), or \(1 \to 0\))
\end{itemize}
\subsection{Setting Bits}
To \textbf{set} a bit, we take the bitwise \keywordinline{OR} of the
byte, with a bit mask that has a \textbf{1} in each position where the
bit should be set.
\begin{figure}[H]
    \centering
    \includegraphics[height = 4cm, keepaspectratio = true]{figures/bit_set.pdf}
    \caption{Setting bits using the logical or.} % \label{}
\end{figure}
\subsection{Clearing Bits}
To \textbf{clear} a bit, we take the bitwise \mintinline{ca65}{AND} of
the byte, with a bit mask that has a \textbf{0} in each position where
the bit should be cleared.
\begin{figure}[H]
    \centering
    \includegraphics[height = 4cm, keepaspectratio = true]{figures/bit_clear.pdf}
    \caption{Clearing bits using the logical and.} % \label{}
\end{figure}
\subsection{Toggling Bits}
To \textbf{toggle} a bit, we take the bitwise \keywordinline{XOR} of
the byte, with a bit mask that has a \textbf{1} in each position where
the bit should be toggled.
\begin{figure}[H]
    \centering
    \includegraphics[height = 4cm, keepaspectratio = true]{figures/bit_toggle.pdf}
    \caption{Toggling bits using the logical exclusive or.} % \label{}
\end{figure}
Other bitwise operations act on the entire byte.
\begin{itemize}
    \item One's complement (bitwise \keywordinline{NOT})
    \item Two's complement (bitwise \keywordinline{NOT} + 1)
    \item Shifts
          \begin{itemize}
              \item Logical
              \item Arithmetic (for signed integers)
          \end{itemize}
    \item Rotations
\end{itemize}
\subsection{One's Complement}
The one's complement of a byte inverts every bit in the operand. This
is done by taking the bitwise \keywordinline{NOT} of the byte.
Similarly, we can subtract the byte from \mintinline{ca65}{0xFF} to get
the one's complement.
\subsection{Two's Complement}
The two's complement of a byte is the one's complement of the byte plus
one. Therefore, we can apply take the bitwise \keywordinline{NOT} of
the byte, and then add one to it.
\subsection{Shifts}
Shifts are used to move bits within a byte. In many programming
languages this is represented by two greater than \mintinline{ca65}{>>}
or two less than \mintinline{ca65}{<<} characters.
\begin{equation*}
    a \gg s
\end{equation*}
shifts the bits in \(a\) by \(s\) places to the right while adding \mintinline{ca65}{0}'s to the MSB.\
\begin{figure}[H]
    \centering
    \includegraphics[height = 2cm, keepaspectratio = true]{figures/logical_right_shift.pdf}
    \caption{Right shift using \mintinline{ca65}{lsr} in AVR Assembly.} % \label{}
\end{figure}
Similarly
\begin{equation*}
    a \ll s
\end{equation*}
shifts the bits in \(a\) by \(s\) places to the left while adding \mintinline{ca65}{0}'s to the LSB.\
\begin{figure}[H]
    \centering
    \includegraphics[height = 2cm, keepaspectratio = true]{figures/logical_left_shift.pdf}
    \caption{Left shift using \keyword{\ttfamily{lsl}} in AVR Assembly.} % \label{}
\end{figure}
When using signed integers, the arithmetic shift is used to preserve the value of the sign bit when shifting.
\begin{figure}[H]
    \centering
    \includegraphics[height = 2cm, keepaspectratio = true]{figures/arithmetic_right_shift.pdf}
    \caption{Arithmetic right shift using \keyword{\ttfamily{asr}} in AVR Assembly.} % \label{}
\end{figure}
Left shifts are used to multiply numbers by 2, whereas right shifts are used to divide numbers by 2 (with truncation).
\subsection{Rotations}
Rotatations are used to shift bits with a carry from the previous
instruction.
\begin{figure}[H]
    \centering
    \includegraphics[height = 2cm, keepaspectratio = true]{figures/rotate_left.pdf}
    \caption{Rotate left using \mintinline{ca65}{rol} in AVR Assembly.} % \label{}
\end{figure}
\begin{figure}[H]
    \centering
    \includegraphics[height = 2cm, keepaspectratio = true]{figures/rotate_right.pdf}
    \caption{Rotate right using \mintinline{ca65}{ror} in AVR Assembly.} % \label{}
\end{figure}
Here the blue bit is carried from the previous instruction, and the carry bit is updated
to the value of the bit that was shifted out.
Rotations are used to perform multi-byte shifts and arithmetic operations.
\section{Arithmetic Operations}
\subsection{Addition}
Addition is performed using the same process as decimal addition except
we only use two digits, 0 and 1.
\begin{enumerate}
    \item \mintinline{ca65}{0b0 + 0b0 = 0b0}
    \item \mintinline{ca65}{0b0 + 0b1 = 0b1}
    \item \mintinline{ca65}{0b1 + 0b1 = 0b10}
\end{enumerate}
When adding two 1's, we carry the result into the next bit position as we would with a 10 in decimal addition.
In AVR Assembly, we can use the \keywordinline{add} instruction to add two bytes. The following
example adds two bytes.
\begin{minted}{ca65}
; Accumulator
|\keyword{ldi}| r16, 0

; First number
|\keyword{ldi}| r17, 29
|\keyword{add}| r16, 17 ; R16 <- R16 + R17 = 0 + 29 = 29

; Second number
|\keyword{ldi}| r17, 118
|\keyword{add}| r16, 17 ; R16 <- R16 + R17 = 29 + 118 = 147
\end{minted}
Below is a graphical illustration of the above code.
\begin{figure}[H]
    \centering
    \includegraphics[height = 3cm, keepaspectratio = true]{figures/add.pdf}
    \caption{Overflow addition using \keyword{\ttfamily{add}} in AVR Assembly.} % \label{}
\end{figure}
\subsection{Overflows}
When the sum of two 8-bit numbers is greater than 8-bit (255), an
\textbf{overflow} occurs. Here we must utilise a second register to
store the high byte so that the result is represented as a 16-bit
number.

To avoid loss of information, a \textbf{carry bit} is used to indicate
when an overflow has occurred. This carry bit can be added to the high
byte in the event that an overflow occurs.

The following example shows how to use the \mintinline{ca65}{adc}
instruction to carry the carry bit when an overflow occurs.
\begin{minted}{ca65}
; Low byte
|\keyword{ldi}| r30, 0
; High byte
|\keyword{ldi}| r31, 0

; Empty byte for adding carry bit
|\keyword{ldi}| r29, 0

; First number
|\keyword{ldi}| r16, 0b11111111
; Add to low byte
|\keyword{add}| r30, r16 ; R30 <- R30 + R16 = 0 + 255 = 255, C <- 0
; Add to high byte
adc r31, r29 ; R31 <- R31 + R29 + C = 0 + 0 + 0 = 0

; Second number
|\keyword{ldi}| r16, 0b00000001
; Add to low byte
|\keyword{add}| r30, r16 ; R30 <- R30 + R16 = 255 + 1 = 0, C <- 1
; Add to high byte
adc r31, r29 ; R31 <- R31 + R29 + C = 0 + 0 + 1 = 1
\end{minted}
Therefore the final result is: \mintinline{ca65}{R31|:|R30 |=|
0b00000001|:|0b00000001 |=| 256}. Below is a graphical representation
of the above code.
\begin{figure}[H]
    \centering
    \includegraphics[height = 6cm, keepaspectratio = true]{figures/adc.pdf}
    \caption{Overflow addition using \mintinline{ca65}{adc} in AVR Assembly.} % \label{}
\end{figure}
\subsection{Subtraction}
Subtraction is performed using the same process as binary addition,
with the subtrahend in two's complement form. In the case of overflows,
the carry bit is discarded.
\subsection{Multiplication}
Multiplication is understood as the sum of a set of partial products,
similar to the process used in decimal multiplication. Here each digit
of the multiplier is multiplied to the multipicand and each partial
product is added to the result.

Given an \(m\)-bit and an \(n\)-bit number, the product is at most
\((m+n)\)-bits wide.
\begin{align*}
    13 \times 43 & = 00001101_2 \times 00101011_2                    \\
                 & =
                     \begin{aligned}[t]
                          &   & 00001101_2 &  & \times &  & 1_2       \\
                          & + & 00001101_2 &  & \times &  & 10_2      \\
                          & + & 00001101_2 &  & \times &  & 1000_2    \\
                          & + & 00001101_2 &  & \times &  & 100000_2
                     \end{aligned}
    \\
                 & =
                     \begin{aligned}[t]
                          &   & 00001101_2   \\
                          & + & 00011010_2   \\
                          & + & 01101000_2   \\
                          & + & 110100000_2
                     \end{aligned}
    \\
                 & = 1000101111
\end{align*}
Using AVR assembly, we can use the \keywordinline{mul} instruction to perform multiplication.
\begin{minted}{ca65}
; First number
|\keyword{ldi}| r16, 13
; Second number
|\keyword{ldi}| r17, 43

; Multiply
|\keyword{mul}| r16, r17 ; R1:R0 <- 0b00000010:0b00101111 = 2:47
\end{minted}
The result is stored in the register pair \mintinline{text}{R1:R0}.
\subsection{Division}
Division, square roots and many other functions are very expensive to
implement in hardware, and thus are typically not found in conventional
ALUs, but rather implemented in software.

However, there are other techniques that can be used to implement
division in hardware. By representing the divisor in reciprocal form,
we can try to represent the number as the sum of powers of 2.

For example, the divisor \(6.4\) can be represented as:
\begin{equation*}
    \frac{1}{6.4} = \frac{10}{64} = 10 \times 2^{-6}
\end{equation*}
so that dividing an integer \(n\) by \(6.4\) is approximately equivalent to:
\begin{equation*}
    \frac{n}{6.4} \approx \left( n \times 10 \right) \gg 6
\end{equation*}
When the divisor is not exactly representable as a power of 2 we can use fractional
exponents to represent the divisor, however this requires a floating point
system implementation which is not provided on the AVR\@.
\chapter{Microcontroller Interfacing}
\section{Logic Levels}
\subsection{Discretisation}
The process of discretisation translates a continuous signal into a
discrete signal (bits). As an example, we can translate \textbf{voltage
levels} on microcontroller pins into digital \textbf{logic levels}.
\subsection{Logic Levels}
For digital input/output (IO), conventionally:
\begin{itemize}
    \item The voltage level of the positive power supply represents a
          \textbf{logical 1}, or the \textbf{high state}, and
    \item \qty{0}{V} (ground) represents a \textbf{logical 0}, or the \textbf{low state}.
\end{itemize}
The QUTy is supplied \qty{3.3}{V} so that when a digital output is high,
the voltage present on the corresponding pin will be around \qty{3.3}{V}.
Because voltage is a continuous quantity, we must discretise the full range of voltages into logical levels using \textbf{thresholds}.
\begin{itemize}
    \item A voltage \textbf{above} the input \textbf{high threshold}
          \(t_H\) is considered \textbf{high}.
    \item A voltage \textbf{below} the input \textbf{low threshold}
          \(t_L\) is considered \textbf{low}.
\end{itemize}
The interpretation of a voltage between these states is determined by \textbf{hysteresis}.
\subsection{Hysteresis}
Hysteresis refers to the property of a system whose state is
\textbf{dependent} on its \textbf{history}. In electronic circuits,
this avoids ambiguity in determining the state of an input as it
switches between voltage levels.
\begin{figure}[H]
    \centering
    \includegraphics[height = 5cm, keepaspectratio = true]{figures/hysteresis.pdf}
    \caption{Example of hysteresis.} % \label{}
\end{figure}
Given a transition:
\begin{itemize}
    \item If an input is currently in the \textbf{low state}, it has
          not transitioned to the \textbf{high state} until the voltage
          crosses the \textbf{high input voltage} threshold.
    \item If an input is currently in the \textbf{high state}, it has
          not transitioned to the \textbf{low state} until the voltage
          crosses the \textbf{low input voltage} threshold.
\end{itemize}
It is therefore always preferrable to drive a digital input to an unambiguous voltage level.
\section{Electrical Quantities}
\subsection{Voltage}
\textbf{Voltage} \(v\) is the electrical \textit{potential difference} between two points in a circuit, measured in \textbf{Volts (\unit{V})}.
\begin{itemize}
    \item Voltage is measured across a circuit element, or between two
          points in a circuit, commonly with respect to a \qty{0}{V}
          reference (ground).
    \item It represents the \textbf{potential} of the electrical system
          to do \textbf{work}.
\end{itemize}
\subsection{Current}
\textbf{Current} \(i\) is the \textit{rate of flow of electrical charge} through a circuit, measured in \textbf{Amperes (\unit{A})}.
\begin{itemize}
    \item Current is measured through a circuit element.
\end{itemize}
\subsection{Power}
\textbf{Power} \(p\) is the rate of energy transferred per unit time, measured in \textbf{Watts (\unit{W})}.
Power can be determined through the equation
\begin{equation*}
    p = v i.
\end{equation*}
\subsection{Resistance}
\textbf{Resistance} \(R\) is a property of a material to \textit{resist the flow of current}, measured in \textbf{Ohms (\unit{\ohm})}.
Ohm's law states that the voltage across a component is proportional to the current that flows through it:
\begin{equation*}
    v = i R.
\end{equation*}
Note that not all circuit elements are resistive (or Ohmic), such that they
do not follow Ohm's law, this can be seen in diodes.
\begin{figure}[H]
    \centering
    \begin{subfigure}{0.47\linewidth}
        \centering
        \includegraphics[height=4.5cm]{figures/vi_ohmic.pdf}
        \caption{VI curve for Ohmic components.}
    \end{subfigure}
    \begin{subfigure}{0.47\linewidth}
        \centering
        \includegraphics[height=4.5cm]{figures/vi_diode.pdf}
        \caption{VI curve for diodes.}
    \end{subfigure}
    \caption{Voltage-current characteristic curves for various components.}
\end{figure}
Although the wires used to connect a circuit are resistive, we usually assume that they are ideal, that is,
they have zero resistance.
\section{Electrical Components}
\subsection{Resistors}
A \textbf{resistor} is a circuit element that is designed to have a
specific resistance \(R\).
\begin{figure}[H]
    \centering
    \includegraphics[height = 2.5cm, keepaspectratio = true]{figures/resistor.pdf}
    \caption{Resistor circuit symbol.} % \label{}
\end{figure}
\subsection{Switches}
A \textbf{switch} is used to connect and disconnect different elements
in a circuit. It can be \textbf{open} or \textbf{closed}.
\begin{itemize}
    \item In the \textbf{open} state, the switch \textbf{will not
          conduct}\footnote{Conductance is a measure of the ability for
          electric charge to flow in a certain path.} current
    \item In the \textbf{closed} state, the switch \textbf{will
          conduct} current
\end{itemize}
Switches can take a variety of forms:
\begin{itemize}
    \item \textbf{Poles} --- the number of circuits the switch can control.
    \item \textbf{Throw} --- the number of output connections each pole can connect its input to.
    \item Momentary or toggle action
    \item Different form factors, e.g., push button, slide, toggle,
          etc.
\end{itemize}
Switches are typically for user input.
\begin{figure}[H]
    \centering
    \begin{subfigure}{0.47\linewidth}
        \centering
        \includegraphics[width=2.5cm]{figures/spst.pdf}
        \caption{Single pole single throw switch.}
    \end{subfigure}
    \begin{subfigure}{0.47\linewidth}
        \centering
        \includegraphics[width=2.5cm]{figures/spdt.pdf}
        \caption{Single pole double throw switch.}
    \end{subfigure}

    \vspace*{5ex}
    \begin{subfigure}{0.47\linewidth}
        \centering
        \includegraphics[width=2.5cm]{figures/dpst.pdf}
        \caption{Double pole single throw switch.}
    \end{subfigure}
    \begin{subfigure}{0.47\linewidth}
        \centering
        \includegraphics[width=2.5cm]{figures/dpdt.pdf}
        \caption{Double pole double throw switch.}
    \end{subfigure}
    \caption{Various types of switches.}
\end{figure}
\subsection{Diodes}
A \textbf{diode} is a semiconductor device that conducts current in
only one direction: from the \textbf{anode} to the \textbf{cathode}.
\begin{figure}[H]
    \centering
    \includegraphics[height = 2cm, keepaspectratio = true]{figures/diode.pdf}
    \caption{Diode symbol.} % \label{}
\end{figure}
Diodes are a non-Ohmic device:
\begin{itemize}
    \item When \textbf{forward biased}, a diode \textbf{does} conduct
          current, and the anode-cathode voltage is equal to the diodes
          \textbf{forward voltage}.
    \item When \textbf{reverse biased}, a diode \textbf{does not}
          conduct current, and the cathode-anode voltage is equal to
          the \textbf{applied voltage}.
\end{itemize}
\begin{figure}[H]
    \centering
    \begin{subfigure}{0.47\linewidth}
        \centering
        \includegraphics[height=3.5cm]{figures/diode_forward_bias.pdf}
        \caption{Forward biased diode. \\\(v_{AK} = v_f\) and \(i > 0\).}
    \end{subfigure}
    \begin{subfigure}{0.47\linewidth}
        \centering
        \includegraphics[height=3.5cm]{figures/diode_reverse_bias.pdf}
        \caption{Reverse biased diode. \\\(v_{KA} > 0\) and \(i = 0\).}
    \end{subfigure}
    \caption{Diodes in forward and reverse bias.}
\end{figure}
A diode is only forward biased when the applied anode-cathode voltage \textbf{exceeds} the forward voltage \(v_f\).
A typical forward voltage \(v_f\) for a silicon diode is in the range \qtyrange{0.6}{0.7}{V}, whereas for
Light Emitting Diodes (LEDs), \(v_f\) ranges between \qtyrange{2}{3}{V}.
\subsection{Integrated Circuit}
An \textbf{integrated circuit} (IC) is a set of electronic circuits
(typically) implemented on a single piece of semiconductor material,
usually silicon. ICs comprise of hundreds to many thousands of
transistors, resistors and capacitors; all implemented on silicon. ICs
are \textbf{packaged}, and connections to the internal circuitry are
exposed via \textbf{pins}.

In general, the specific implementation of the IC is not important, but
rather the \textbf{function of the device} and how it
\textbf{interfaces} with the rest of the circuit. Hence ICs can be
treated as a functional \textbf{black box}. For digital ICs:
\begin{itemize}
    \item \textbf{Input pins} are typically \textbf{high-impedance}, and they appear as an open circuit.
    \item \textbf{Output pins} are typically \textbf{low-impedance}, and will actively drive the voltage
          on a pin and any connected circuitry to a \textbf{high} or \textbf{low} state. They can also
          drive connected loads.
\end{itemize}
\section{Digital Outputs}
Digital output interfaces are designed to be able to drive connected
circuitry to one of states, high, or low, however, the appropriate
technique is \textbf{context specific}. When referring to digital
outputs, we will refer to the states of a net. A \textbf{net} is
defined as the common point of connection of multiple circuit
components.

In this section we will consider:
\begin{itemize}
    \item What kind of load the output drives
    \item Could more than one device be attempting to actively drive
          the net to a specific logic level?
\end{itemize}
\subsection{Push-Pull Outputs}
A push-pull digital output is the most common form of output used in
digital outputs. The \textbf{output driver} \(A\) \textit{drives} the
\textbf{output state} \(Y\) to:
\begin{itemize}
    \item \textbf{HIGH} by connecting the output net to the supply voltage \(+\unit{V}\).
    \item \textbf{LOW} by connecting the output net to the ground voltage GND (\qty{0}{V}).
\end{itemize}
\begin{figure}[H]
    \centering
    \includegraphics[height = 4cm, keepaspectratio = true]{figures/push_pull.pdf}
    \caption{Push-pull output.} % \label{}
\end{figure}
Hence the output state \(Y\) is determined by the logic level of the output driver \(A\).
\begin{equation*}
    Y = A.
\end{equation*}
\begin{table}[H]
    \centering
    \begin{tabular}{c | c} % chktex 44
        \toprule
        \textbf{\(A\)} & \textbf{\(Y\)} \\
        \midrule
        LOW            & LOW            \\
        HIGH           & HIGH           \\
        \bottomrule
    \end{tabular}
    \caption{Truth table for a push-pull digital output.} % \label{}
\end{table}
The push-pull output \(Y\) can both source and sink current from the connected net.
\subsection{High-Impedance Outputs}
In many instances, a digital output is required to be placed in a
high-impedance (HiZ) state. This is accomplished by using an
\textbf{output enable} (OE) signal.
\begin{figure}[H]
    \centering
    \includegraphics[height = 3cm, keepaspectratio = true]{figures/HiZ.pdf}
    \caption{High-impedance output.} % \label{}
\end{figure}
\begin{itemize}
    \item When the OE signal is \textbf{HIGH}, the output state \(Y\)
          is determined by the output driver \(A\).
    \item When the OE signal is \textbf{LOW}, the output state \(Y\) is
          in a \textbf{high-impedance} state.
\end{itemize}
\begin{table}[H]
    \centering
    \begin{tabular}{c c | c} % chktex 44
        \toprule
        \textbf{\(A\)} & \textbf{OE} & \textbf{\(Y\)} \\
        \midrule
        LOW            & LOW         & HiZ            \\
        HIGH           & LOW         & HiZ            \\
        LOW            & HIGH        & LOW            \\
        HIGH           & HIGH        & HIGH           \\
        \bottomrule
    \end{tabular}
    \caption{Truth table for a push-pull digital output.} % \label{}
\end{table}
When the output is in \textbf{HiZ state}:
\begin{itemize}
    \item The output is an effective \textbf{open circuit}, meaning it
          has \textbf{no effect} on the rest of the circuit.
    \item The voltage on the output net is determined by the
          \textbf{other circuitry} connected to the net.
\end{itemize}
HiZ outputs are typically used when multiple devices share the same wire(s).
\subsection{Pull-up and Pull-down Resistors}
When \textbf{no devices} are actively driving a net (e.g., all
connected outputs are in the HiZ state), the state of the net is not
well-defined. Hence we can use a \textbf{pull-up} or \textbf{pull-down}
resistor to ensure that the state of the pin is always
\textbf{well-defined}.
\begin{multicols}{2}
    \begin{figure}[H]
        \centering
        \includegraphics[height = 4cm, keepaspectratio = true]{figures/pullup_resistor.pdf}
        \caption{Pull-up resistor.} % \label{}
    \end{figure}
    \begin{figure}[H]
        \centering
        \includegraphics[height = 4cm, keepaspectratio = true]{figures/pulldown_resistor.pdf}
        \caption{Pull-down resistor.} % \label{}
    \end{figure}
\end{multicols}
\begin{itemize}
    \item When \textbf{no circuitry} is actively driving the net, the
          resistor will passively pull the voltage to either the
          voltage supply, or ground.
    \item When \textbf{another device} actively drives the net, the
          active device defines the voltage of the net. Hence the
          current from the resistor is simply sourced or sunk by the
          \textbf{active device}.
\end{itemize}
The resistors used as pull-up and pull-down resistors are typically in the \unit{k\ohm} range.
\subsection{Open-Drain Outputs}
Multiple push-pull outputs should never be connected to the same net as
when one output is driven HIGH and another is driven LOW, an effective
short circuit is created and one or more devices may be damaged. While
push-pull outputs with an output enable may be used, the timing must be
carefully managed.

Hence a more robust solution is to use open-drain outputs.
\begin{figure}[H]
    \centering
    \includegraphics[height = 4cm, keepaspectratio = true]{figures/open_drain.pdf}
    \caption{Open-drain output.} % \label{}
\end{figure}
An open-drain output is either:
\begin{itemize}
    \item In the \textbf{high-impedance} state, where the pull-up
          resistor is used to pull the net to the \textbf{high state}
          when the net is \textbf{not driven low}.
    \item \textbf{Connected to ground}, when the net is \textbf{driven low}.
\end{itemize}
\section{Microcontroller Pins}
Microcontrollers are interfaced via their exposed pins. These pins are
the only means to access inputs and outputs, and they are used to
interface with other electronic circuits in order to achieve a required
functionality. Pins can be used for:
\begin{itemize}
    \item General purpose input and output (GPIO) --- pin represents a
          digital state
    \item Peripheral functions
    \item Other functions (power supply, reset input, clock input,
          etc.)
\end{itemize}
Pins are typically organised into groups of related IO banks,
referred to as \textbf{ports} on the AVR microcontroller.
These ports and pins are assigned an alphanumeric identifier, (e.g., PB7 for pin 7 on port B).
\begin{figure}[H]
    \centering
    \includegraphics[height = 10cm, keepaspectratio = true]{figures/PORT_block_diagram.pdf}
    \caption{ATtiny1626 PORT block diagram.} % \label{}
\end{figure}
To summarise this diagram:
\begin{itemize}
    \item The data direction register (DIR) controls the push-pull
          output enable.
    \item The output driver register (OUT) drives the output state.
    \item The input register (IN) reads the output state.
    \item The internal pull-up register enabled through software.
    \item The physical voltage on the pin can be routed to an analogue
          to digital converter (ADC)
    \item Other peripheral functions can override port pin
          configurations and the output state.
\end{itemize}
\subsection{Configuring an Output in Assembly}
\begin{enumerate}
    \item Place the port pin in a \textbf{safe initial state} by
          writing to the OUT register (HIGH or LOW depending on the
          context).
    \item Configure the port pin as an output by \textbf{setting} the
          corresponding bits in the DIR register.
    \item Set the desired pin state by writing to the OUT register.
\end{enumerate}
\begin{minted}{ca65}
; Load macros for easy access to port data space addresses.
#include <avr/io.h>

; Bitmask for pin 5
|\keyword{ldi}| r16, PIN5_bm

; Set initial safe state
|\keyword{sts}| PORTB_OUTCLR, r16 ; LOW if active HIGH
|\keyword{sts}| PORTB_OUTSET, r16 ; HIGH if active LOW

; Enable output
|\keyword{sts}| PORTB_DIRSET, r16 ; Enable output on PB5

; Set output state to desired value
|\keyword{sts}| PORTB_OUTSET, r16 ; Set state of PB5 to HIGH
\end{minted}
\subsection{Configuring an Input in Assembly}
\begin{enumerate}
    \item If required, enable the internal pull-up resistor by
          \textbf{setting} the PULLUPEN bit in the corresponding
          PINnCTRL register.
    \item Read the IN register to get the current state of the pin.
    \item Isolate the relevant pin using the AND operator.
\end{enumerate}
\begin{minted}{ca65}
#include <avr/io.h>

|\keyword{ldi}| r16, PIN5_bm

; Enable internal pull-up resistor if required
|\keyword{sts}| PORTB_PIN5CTRL, r16

; Read output state from data space
|\keyword{lds}| r17, PORTA_IN
; Read output state using virtual PORT
|\keyword{in}| r17, VPORTA_IN

; Isolate desired pin
|\keyword{andi}| r17, r16
\end{minted}
\subsection{Peripheral Multiplexing}
Pins can be used to connect internal peripheral functions to external
devices. As microcontrollers have more peripheral functions than
available pins, peripheral functions are typically multiplexed onto
pins.
\begin{definition}[Multiplexing]
    Multiplexing is a method by which \textbf{multiple peripheral functions}
    are mapped to the \textbf{same pin}.
    In this scenario, only one function can be enabled at a time, and the pin
    cannot be used for GPIO\@.
\end{definition}
\begin{itemize}
    \item Peripheral functions can be mapped to different \textbf{sets
          of pins} to provide flexibility and to avoid clashes when
          multiple peripherals are used in an application.
    \item When enabled, peripheral functions \textbf{override} standard
          port functions.
    \item The \textbf{Port Multiplexer} (PORTMUX) is used to select
          which \textbf{pin set} should be used by a peripheral.
    \item Certain peripherals can have their inputs/outputs mapped to
          different \textbf{sets of pins} through the PORTMUX\@.
\end{itemize}
Note that we cannot re-map a single peripheral function to another pin, but must consider the entire set.
\section{Interfacing to Simple IO}
\subsection{Driving LEDs}
The \textbf{brightness} of an LED is proportional to the
\textbf{current} passing through it. As LEDs are non-Ohmic, we cannot
drive them directly with a voltage as this would result in an
uncontrolled flow of current that may damage the LED or driver.

Instead, LEDs are paired with a \textbf{series resistor} to limit the
flow of current. The appropriate current is dependent on the specific
LED that is used and the capability of the driver device
(microcontroller). A typical indictor LED requires a current of
\qtyrange{1}{2}{m.A}.
\subsection{Interfacing to LEDs}
An LED can be driven in two different configurations from a
microcontroller pin:
\begin{itemize}
    \item \textbf{active high}; in which case the LED is \textbf{lit} when the pin is \textbf{HIGH}.
    \item \textbf{active low}; in which case the LED is \textbf{lit} when the pin is \textbf{LOW}.
\end{itemize}
Both of these configurations have their benefits, and the best configuration depends entirely on the context.
\begin{multicols}{2}
    \begin{figure}[H]
        \centering
        \includegraphics[height = 4cm, keepaspectratio = true]{figures/active_high_LED.pdf}
        \caption{Active high configuration.} % \label{}
    \end{figure}
    \begin{figure}[H]
        \centering
        \includegraphics[height = 4cm, keepaspectratio = true]{figures/active_low_LED.pdf}
        \caption{Active low configuration.} % \label{}
    \end{figure}
\end{multicols}
On the QUTy, the LED display is driven in the \textbf{active low} configuration.
This has a number of advantages:
\begin{itemize}
    \item If the internal pull-up resistors are mistakenly enabled, no
          current will flow into the LEDs.
    \item The microcontroller pins can sink higher currents than they
          can source, allowing us to drive the display to a higher
          brightness.
    \item The display used on the QUTy has a common anode
          configuration, hence we must use an active low configuration
          to drive the display segments independently.
\end{itemize}
An LED is an example of a simple \textbf{digital output}, as we can map \textbf{logical states}
to \textbf{LED states} (lit or unlit) for a digital output.
\subsection{Switches as Digital Inputs}
The state of a switch can be used to \textbf{set} the state of a pin.
As the switch has two states (open or closed), these can be mapped
directly to \textbf{logical states}. This can be done by connecting the
switch between the pin and voltage source representing one of the logic
levels (ground or a positive supply). \pagebreak
\begin{multicols}{2}
    \begin{figure}[H]
        \centering
        \includegraphics[height = 5cm, keepaspectratio = true]{figures/active_high_switch.pdf}
        \caption{Active high configuration.} % \label{}
    \end{figure}
    \begin{figure}[H]
        \centering
        \includegraphics[height = 5cm, keepaspectratio = true]{figures/active_low_switch.pdf}
        \caption{Active low configuration.} % \label{}
    \end{figure}
\end{multicols}
\begin{itemize}
    \item When the switch is \textbf{open}, the pull-up/pull-down
          resistor is used to define the state of the switch.
    \item When the switch is \textbf{closed}, the state of the pin is
          defined by the voltage connected to via the switch.
\end{itemize}
\subsection{Interfacing to Switches}
As with LEDs, we can interface switches to microcontroller pins in two
different configurations:
\begin{itemize}
    \item \textbf{active high}; in which case the pin is \textbf{HIGH} when the switch is \textbf{closed}.
    \item \textbf{active low}; in which case the pin is \textbf{LOW} when the switch is \textbf{closed}.
\end{itemize}
An \textbf{active low} configuration is usually preferred as:
\begin{itemize}
    \item it allows for the utilisation of an \textbf{internal pull-up
          resistor} that is commonly implemented in microcontrollers.
    \item It eliminates the risk of unsafe voltages being applied to
          the pin from the power supply in an active high
          configuration.
    \item It is easier to access a ground reference on a circuit board.
\end{itemize}
\subsection{Interfacing to Integrated Circuits}
For digital ICs,
\begin{itemize}
    \item \textbf{Inputs} are typically \textbf{high impedance}
    \item \textbf{Outputs} are typically \textbf{push-pull}
\end{itemize}
This generally means that we can interface an IC by connecting its pins directly to the pins of a microcontroller.
\begin{itemize}
    \item For \textbf{IC inputs}, the microcontroller pin is configured
          as an \textbf{output}, and the \textbf{microcontroller sets}
          the logic level of the net.
    \item For \textbf{IC outputs}, the microcontroller pin is
          configured as an \textbf{input}, and the \textbf{IC sets} the
          logic level of the net.
\end{itemize}
As microcontroller pins are typically configured as \textbf{inputs on reset}, a
pull-up/pull-down resistor may be required if it is important for an IC input to
have a \textbf{known state} prior to the configuration of the relevant microcontroller pins as outputs.
\begin{definition}[Word]
    A word refers to a value that is two bytes in size (16-bit).
\end{definition}
\section{Registers}
A register refers to a memory location that is 1 byte in size (8-bit).
The ATtiny1626 has 32 registers of which \mintinline{ca65}{r16} to
\mintinline{ca65}{r31} can be loaded with an immediate value
(\numrange{0}{255}) using \keywordinline{ldi}.
\begin{minted}{ca65}
|\keyword{ldi}| r16, 17 ; Load the value 17 into r16
\end{minted}
Values are commonly loaded into registers as many other operations can
be performed on them.
\section{Flow Control}
Instructions on the AVR Core increment the PC by 1 or 2 (depending on
whether the OPCODE is 1 or 2 words) when they are executed so that any
successive instructions are executed after the first. To divert
execution to a different location, we can utilise \textbf{change of
flow} instructions.

The \mintinline{ca65}{jmp} (jump) instruction is used to simply jump to
a different location in the program. This instruction is capable of
jumping to an address withtin the entire 4M (words) program memory,
however, this is highly excessive for the ATtiny1626.
\section{Labels}
Most change of flow instructions take an \textbf{address} in program
memory as a parameter. Hence to make this process easier, we can use
labels to refer to locations in program memory (and also RAM).
\begin{minted}{ca65}
jmp new_location ; Jump to the label |\textbf{new\_location}|.
|\keyword{ldi}| r16, 1 ; This instruction is skipped

new_location: ; Label
    |\keyword{push}| r16
\end{minted}
When a label appears in source code, the assembler replaces references
to it with the address of the directive/instruction immediately
following that label. Labels work for both \textbf{absolute} and
\textbf{relative} addresses and the assembler will automatically adjust
the address to the correct type.

Additionally, labels can also be used as parameters to other immediate
instructions if we store the high and low bytes in registers and wish
to reference the location in an indirect jumping instruction.
\section{Absolute and Relative Addresses}
\mintinline{ca65}{jmp} is a 32-bit instruction, which uses 22 bits to specify an address between \mintinline{ca65}{0x000000} % chktex 29
and \mintinline{ca65}{0x3FFFFF}, or \(2^{23} - 1\) bits of memory (\qty{8}{MB}). As mentioned earlier, this is much larger % chktex 29
than what the 16-bit PC can address on the ATtiny1626 (\qty{64}{KB}).

As we will only need to jump within \qty{64}{KB} of memory, it is
inefficient to use the \mintinline{ca65}{jmp} instruction as it
requires 3 CPU cycles to execute. Therefore, many AVR change of flow
instructions take a value that is \textbf{added} onto the current PC to
calculate the destination address, allowing them to fit within 16 bits.
The \keywordinline{rjmp} (relative jump) instruction is therefore more
suitable as it only requires 2 CPU cycles.

Note the assembler throw an error if the address is not within the
range of the PC\@.
\section{Branching}
A branching instruction jumps to a different location based on a
condition, i.e., user input, internal state, or other external factors.
Many change of flow instructions are conditional, and will alter the PC
differently based on register value(s) or flags. In AVR there are two
main categories of branching instructions:
\begin{itemize}
    \item Branch instructions
    \item Skip instructions
\end{itemize}
\subsection{Branch Instructions}
Branch instructions use the following logic:
\begin{enumerate}
    \item Check if the specified flag in SREG is cleared/set
    \item If true, jump to the specified address (\(\mathrm{PC}
          \leftarrow \mathrm{PC} + k + 1\))
    \item Otherwise, proceed to the next instruction as normal
          (\(\mathrm{PC} \leftarrow \mathrm{PC} + 1\))
\end{enumerate}
Although there are are 20 branch instructions listed in the instruction set summary, the following two form the basis of all branching instructions:
\begin{itemize}
    \item \keywordinline{brbc} (branch if bit in SREG is cleared)
    \item \keywordinline{brbs} (branch if bit in SREG is set)
\end{itemize}
All other branching instructions are specific cases of the above instructions, that are provided to make programming in Assembly easier.
As these instructions check the bits in the SREG, they are usually preceded by an ALU operation such as \keywordinline{cp} or \keywordinline{cpi} to trigger the required flags.

As only 7 bits are allocated to the destination in the OPCODE, branch
instructions jump shorter distances than relative jumps.
\subsection{Compare Instructions}
Both the \keywordinline{cp} and \keywordinline{cpi} instructions are
used to compare the values in one or two registers. The ALU performs a
subtraction operation whose result is used to update the SREG\@. Note
that the result is not stored or used in any way.
\begin{itemize}
    \item \mintinline{ca65}{|\keyword{cp}| Rd, Rr} performs \mintinline{ca65}{Rd} \(-\) \mintinline{ca65}{Rr}
    \item \mintinline{ca65}{|\keyword{cpi}| Rd, K} performs \mintinline{ca65}{Rd} \(-\) \mintinline{ca65}{K}
\end{itemize}
\begin{minted}{ca65}
|\keyword{ldi}| r16, 0
|\keyword{ldi}| r19, 10
|\keyword{cp}| r16, r19 ; Compare values in registers r16 and r19
|\keyword{brge}| new_location ; Branch if r16 greater than or equal to r19

new_location:
\end{minted}
Note that many instructions are able to set the Z flag, which is used
to indicate if the result of the operation is zero. In these cases, the
compare instruction may be redundant.
\subsection{Skip Instructions}
The skip instructions are less flexible then branch instructions, but
can sometimes require less space or fewer cycles. Skip instructions
skip the next instruction if the condition is true.

In this example we will skip the line which increments register 16.
\begin{minted}{ca65}
|\keyword{cpse}| r16, r17 ; Skips next instruction if r16 == r17
inc r16 ; This is skipped
\end{minted}
Same example which uses a branch instruction:
\begin{minted}{ca65}
|\keyword{cp}| r16, r17
|\keyword{breq}| new_location ; Skips to new_location if r16 == r17
inc r16 ; This is skipped

new_location: ; PC is now here
\end{minted}
Note that the number of cycles for a skip instruction depends on the
size of the instruction being skipped. The \keywordinline{sbrc} and
\keywordinline{sbrs} instructions are used to skip the next instruction
if the specified bit a register is cleared/set.
\begin{minted}{ca65}
|\keyword{ldi}| r16, 0b00101110

|\keyword{sbrc}| r16, 0 ; Skips next instruction if bit 0 of r16 is cleared
inc r16 ; This is skipped
\end{minted}
Comparing with branch instructions
\begin{minted}{ca65}
|\keyword{ldi}| r16, 0b00101110
|\keyword{andi}| r16, 0b00000001 ; Isolate bit 0
|\keyword{breq}| new_location ; Skips next instruction if r16 == 0
inc r16 ; This is skipped

new_location: ; PC is now here
\end{minted}
The \keywordinline{sbis} and \keywordinline{sbic} instructions are used
to skip the next instruction if the specified bit an I/O register is
set/cleared. For example, if we wish to toggle the decimal point LED
(DISP DP) on the QUTy (PORT B pin 5) when the first button (BUTTON0)
was pressed (PORT A pin 4),
\begin{minted}{ca65}
|\keyword{ldi}| r16, PIN5_bm ; Bitmask of pin 5
|\keyword{sbis}| VPORTA_IN, 0b00010000 ; Skip next instruction if pin 4 of PORT A is set
|\keyword{sts}| PORTB_OUTTGL, r16 ; Toggle the output driver of pin 5 on PORT B
\end{minted}
Using branch instructions:
\begin{minted}{ca65}
|\keyword{in}| r17, VPORTA_IN ; Read the input register of PORT A
|\keyword{andi}| r17, 0b00010000 ; Isolate pin 4

|\keyword{brne}| new_location ; Skip instructions if r17 != 0

|\keyword{ldi}| r16, PIN5_bm ; Bitmask of pin 5
|\keyword{sts}| PORTB_OUTTGL, r16 ; Toggle the output driver of pin 5 on PORT B

new_location:
\end{minted}
\section{Loops}
By jumping to an earlier address, we can loop over a block of
instructions.
\begin{minted}{ca65}
infinite_loop:
    ; Code to repeat
    |\keyword{rjmp}| infinite_loop
\end{minted}
Loops can also be finite, in which case the loop will terminate when a
counter reaches zero.
\begin{minted}{ca65}
|\keyword{ldi}| r16, 10 ; Set counter to 10
loop:
    dec r16 ; Decrement counter
    |\keyword{brne}| loop ; Branch if counter != 0
\end{minted}
Loops can also be used to repeat until some external event occurs.
\begin{minted}{ca65}
main_loop:
    |\keyword{in}| r17, VPORTA_IN ; Read the input register of PORT A
    |\keyword{andi}| r17, 0b00010000 ; Isolate pin 4

    |\keyword{brne}| main_loop ; Branch if counter != 0
    |\keyword{rjmp}| button_pressed

button_pressed:
    ; Execute instructions
    |\keyword{rjmp}| main_loop ; Return to main loop
\end{minted}
\section{Delays}
Loops can be utilised to delay the execution of instructions. These
instructions do not execute any useful code. This is useful for when we
wish to wait for an external event to occur.\footnote{Note that this
type of loop is not recommended for time-sharing systems, such as a
personal computer, as the lost CPU cycles cannot be used by other
programs. In these cases, clock interrupts are preferred. However, on a
device such as the ATtiny1626, delay loops can be utilised to precisely
insert delays in a program.}

To create a precisely timed delay, we must take the following values
into account.
\begin{itemize}
    \item The clock speed --- frequency of the clocks oscillations
          (default: \qty{20}{MHz} --- configurable in
          CLKCTLR\_MCLKCTRLA)
    \item The prescaler --- reduces the frequency of the CPU clock
          through division by a specific amount; 12 different settings
          from 1x to 64x (default: 6 --- configurable in
          CLKCTLR\_MCLKCTRLA)
\end{itemize}
The clock oscillates at its effective clock speed:
\begin{equation*}
    \text{effective clock speed} = \text{clock speed} \times \frac{1}{\text{prescaler}}
\end{equation*}
The default prescaler is 6, so the effective clock speed is \qty{3.33}{MHz} by default.
Note that the effective clock speed can therefore range between:
\begin{itemize}
    \item Effective maximum clock frequency: \qty{20}{MHz}
          (\qty{20}{MHz} clock \& prescaler 1)\footnote{As the QUTy is
          supplied with \qty{3.3}{V}, it is not safe to go above
          \qty{10}{MHz}.}
    \item Effective minimum clock frequency: \qty{512}{Hz}
          (\qty{32.768}{kHz} clock \& prescaler 64)
\end{itemize}
Therefore to create a delay, we must first determine the required number of CPU cycles in the body of the loop
and iterate until the number of CPU cycles reaches the required amount.

The following examples utilise counters of various sizes to create
delays. Note that \(n\) represents the number of iterations.
\begin{minted}{ca65}
delay_1:
    |\keyword{ldi}| r16, x ; 1 CPU cycle
    |\keyword{ldi}| r17, 1 ; 1 CPU cycle ; Incrementor

    loop:
        add r16, r17 ; 1 CPU cycle
        |\keyword{brcc}| loop ; 2 CPU cycles (1 CPU cycle when condition is false)
\end{minted}
The register \mintinline{ca65}{r16} has the following relationship:
\begin{equation*}
    x = \left( 2^8 - 1 \right) - n \iff n = \left( 2^8 - 1 \right) - x
\end{equation*}
with
\begin{align*}
    \text{total cycles} & = 1 + 1 + \left( n + 1 \right) + 2 n + 1 \\
                        & = 3 n + 4
\end{align*}
for a maximum delay of \qty{230.7}{\micro s} (\(\left( 3 \times \left( 2^8 - 1 \right) + 4 \right) T\))\footnote{\(T\) is the period of one CPU cycle (using the default clock configuration): \(T = \frac{1}{\qty{20}{MHz} / 6} = \qty{300}{ns}\).}.
To create larger delays, we can use multiple registers:
\begin{minted}{ca65}
delay_2:
    |\keyword{ldi}| r24, x ; 1 CPU cycle
    |\keyword{ldi}| r25, y ; 1 CPU cycle

    loop:
        |\keyword{adiw}| r24, 1 ; 2 CPU cycles
        |\keyword{brcc}| loop ; 2 CPU cycles (1 CPU cycle when condition is false)
\end{minted}
The register pair \(\left( y:x \right)\) has the following
relationship:
\begin{align*}
    \left( y:x \right) = \left( 2^{16} - 1 \right) - n \iff n = \left( 2^{16} - 1 \right) - \left( y:x \right)
\end{align*}
with
\begin{align*}
    \text{total cycles} & = 1 + 1 + 2 \left( n + 1 \right) + 2 n + 1 \\
                        & = 4n + 5
\end{align*}
for a maximum delay of \qty{78.644}{ms} (\(\left(4 \times \left( 2^{16} - 1 \right) + 5 \right) T\)).
With three registers,
\begin{minted}{ca65}
delay_3:
    |\keyword{ldi}| r24, x ; 1 CPU cycle
    |\keyword{ldi}| r25, y ; 1 CPU cycle
    |\keyword{ldi}| r26, z ; 1 CPU cycle

    loop:
        |\keyword{adiw}| r24, 1 ; 2 CPU cycles
        |\keyword{adc}| r26, r0 ; 1 CPU cycle (r0 represents a register with value 0)
        |\keyword{brcc}| loop ; 2 CPU cycles (1 CPU cycle when condition is false)
\end{minted}
The register triplet \(\left( z:y:x \right)\) is determined through:
\begin{align*}
    \left( z:y:x \right) = \left( 2^{24} - 1 \right) - n \iff n = \left( 2^{24} - 1 \right) - \left( z:y:x \right)
\end{align*}
with
\begin{align*}
    \text{total cycles} & = 1 + 1 + 1 + 2 \left( n + 1 \right) + \left( n + 1 \right) + 2 n + 1 \\
                        & = 5n + 7
\end{align*}
for a maximum delay of \qty{25.166}{s} (\(\left(5 \times \left( 2^{24} - 1 \right) + 7 \right) T\)).
This approach can be extended to create delays of any length.

If needed, we can also include the \mintinline{ca65}{nop} (no
operation) instruction which requires 1 CPU cycle and does nothing. In
addition to this, we can also utilise nested loops, however the timing
is more complex to determine.
\section{Memory and IO}
On the AVR Core, as both I/O and SRAM are accessed through the data
space, they can be directly accessed using instructions that read/write
to memory. This approach is known as memory-mapped I/O (MMIO) and it
significantly reduces chip complexity.

In contrast to modern CPU architectures, such as x86, in the AVR
architecture, programs are located in a separate address space
(although the memory is still accessible through the data space).
\begin{figure}[H]
    \centering
    \includegraphics[height = 8cm, keepaspectratio = true]{figures/memory_map.pdf}
    \caption{ATtiny1626 memory map.} % \label{}
\end{figure}
The following instructions may be used to access memory from the data space:
\begin{itemize}
    \item \keywordinline{lds} (load direct from data space to register)
    \item \keywordinline{sts} (store direct from register to data space)
    \item \keywordinline{ld} (load indirect from data space to register)
    \item \keywordinline{st} (store indirect from register to data space)
    \item \keywordinline{push}/\keywordinline{pop} (stack operations in SRAM --- starting at \mintinline{ca65}{0x3800}) % chktex 29
    \item \keywordinline{in}/\keywordinline{out} (single cycle I/O register operations)
    \item \keywordinline{sbi}/\keywordinline{cbi} (set/clear bit in I/O register)
\end{itemize}
Note that the \keywordinline{in}/\keywordinline{out} instructions can
only access the low 64 bytes of the I/O register space and the \keywordinline{sbi}/\keywordinline{cbi}
instructions can only access the low 32 bytes of the I/O register space.
As the name suggests, these instructions only require a single CPU cycle and hence
several addresses such as VPORT\{A, B, C\} (virtual ports) are mapped to this location,
for fast access.
\subsection{Load/Store Indirect}
While the \keywordinline{lds}/\keywordinline{sts} instructions can be
used to access addresses of bytes, they are generally not suitable for
accessing data structures such as arrays. Instead, we can use the
\keywordinline{ld}/\keywordinline{st} instructions to take advantage of
their 16-bit pointer registers, which support some pointer arithmetic.
\begin{itemize}
    \item \mintinline{ca65}{r26} \(\to\) \keywordinline{XL} (\keywordinline{X}-register low byte)
    \item \mintinline{ca65}{r27} \(\to\) \keywordinline{XH} (\keywordinline{X}-register high byte)
    \item \mintinline{ca65}{r28} \(\to\) \keywordinline{YL} (\keywordinline{Y}-register low byte)
    \item \mintinline{ca65}{r29} \(\to\) \keywordinline{YH} (\keywordinline{Y}-register high byte)
    \item \mintinline{ca65}{r30} \(\to\) \keywordinline{ZL} (\keywordinline{Z}-register low byte)
    \item \mintinline{ca65}{r31} \(\to\) \keywordinline{ZH} (\keywordinline{Z}-register high byte)
\end{itemize}
For example, if we wanted to access a byte in RAM, we can do the following:
\begin{minted}{ca65}
|\keyword{ldi}| XL, lo8(RAMSTART) ; Store address of RAM in X
|\keyword{ldi}| XH, hi8(RAMSTART)

|\keyword{ld}| r16, X ; Load byte from X to r16
; The byte in X is now in r16

|\keyword{ldi}| r17, 24
|\keyword{st}| X, r17 ; Store byte from r16 to X
; The byte in X (and hence at RAMSTART) is now 24
\end{minted}
These pointer registers also support post-increment and pre-decrement
operations:
\begin{itemize}
    \item \mintinline{ca65}{X+} (post-increment pointer address)
    \item \mintinline{ca65}{-X} (pre-decrement pointer address)
\end{itemize}
\begin{minted}{ca65}
|\keyword{ld}| r16, X+ ; Load byte from X to r16, then X <- X + 1
|\keyword{st}| X+, r16 ; Store byte from r16 to X, then X <- X + 1

|\keyword{ld}| r16, -X ; X <- X - 1, then load byte from X to r16
|\keyword{st}| -X, r16 ; X <- X - 1, then store byte from r16 to X
\end{minted}
This operation can be used to copy bytes from one location to another:
\begin{minted}{ca65}
; Copy 10 bytes from RAM to RAM+10
|\keyword{ldi}| XL, lo8(RAMSTART)
|\keyword{ldi}| XH, hi8(RAMSTART)

|\keyword{ldi}| YL, lo8(RAMSTART+10)
|\keyword{ldi}| YH, hi8(RAMSTART+10)

|\keyword{ldi}| r16, 10 ; Loop 10 times
loop:
    |\keyword{ld}| r0, X+ ; Load byte from X to r0, then X <- X + 1
    |\keyword{st}| Y+, r0 ; Store byte from r0 to Y, then Y <- Y + 1
    dec r16
    |\keyword{brne}| loop
\end{minted}
\subsection{Load/Store Indirect with Displacement}
In addition to the \keywordinline{ld}/\keywordinline{st} instructions,
the \keywordinline{ldd}/\keywordinline{std} instructions are a special
form that allow us to load/store from/to the address of the pointer
register \textbf{plus} \(q = \left\{ \numrange{0}{63} \right\}\).
\begin{minted}{ca65}
|\keyword{ldi}| YL, lo8(RAMSTART)
|\keyword{ldi}| YH, hi8(RAMSTART)

|\keyword{ldd}| r0, Y+20 ; Load byte from Y+20 to r0
|\keyword{std}| Y+21, r0 ; Store byte from r1 to Y+21
; Note Y still points to RAMSTART
\end{minted}
Note this form is only available for \keywordinline{Y}, and
\keywordinline{Z}.
\section{Stack}
The stack is a last-in first-out (LIFO) data structure in SRAM\@. It is
accessed through a register called the stack pointer (SP), which is not
part of the register file like SREG\@.

Upon reset, SP is set to the last available address in SRAM
(\mintinline{ca65}{0x3FFF}), and can be modified through
\keywordinline{push}/\keywordinline{pop} and other methods that are
generally not recommended.% chktex 29
\begin{itemize}
    \item \keywordinline{push} stores a register to SP then decrements SP (\(\mathrm{SP} \leftarrow \mathrm{SP} - 1\))
    \item \keywordinline{pop} increments SP (\(\mathrm{SP} \leftarrow \mathrm{SP} + 1\)) then loads to a register from SP
\end{itemize}
If a particular register is required without modifying other code, we can temporarily
store the value of that register on the stack, and pop it back when we are done:
\begin{minted}{ca65}
|\keyword{push}| ZL ; Temporarily store Z on the stack
|\keyword{push}| ZH
; Z may be used for another purpose
|\keyword{pop}| ZH ; Restore Z from the stack in reverse order
|\keyword{pop}| ZL
\end{minted}
\section{Procedures}
Procedures allow us to write modular, reusable code which makes them
powerful when working on complex projects. Although they are usually
associated with high level languages as methods, or functions, they are
also available in assembly.

Procedures begin with a label, and end with the \mintinline{ca65}{ret}
keyword. They must be \textbf{called} using the
\keywordinline{call}/\keywordinline{rcall} instructions.
\begin{minted}{ca65}
procedure:
    ; Procedure body
    |\keyword{ret}| ; Return to caller
\end{minted}
\subsection{Saving Context}
To ensure that procedures are maximally flexible and place no
constraints on the caller, we must always restore any modified
registers before returning to the caller. The same is true for the
SREG\@.
\begin{minted}{ca65}
|\keyword{rjmp}| main_loop

procedure:
    |\keyword{push}| r16 ; Save r16 on the stack
    ; Code that possibly modifies r16
    |\keyword{pop}| r16 ; Restore r16 from the stack
    |\keyword{ret}|

main_loop:
    |\keyword{ldi}| r16, 10
    |\keyword{rcall}| procedure ; Call procedure
    |\keyword{push}| r16 ; r16 should still be 10
\end{minted}
\subsection{Parameters and Return Values}
Parameters can be passed using registers or the stack depending on the
size of the inputs.
\begin{minted}{ca65}
|\keyword{rjmp}| main_loop

; Calculate the average of two numbers
; Inputs:
;     r16: first number
;     r17: second number
; Outputs:
;     r16: average
average:
    |\keyword{push}| r0 ; Save r0
    |\keyword{in}| r0, CPU_SREG ; Save SREG
    |\keyword{push}| r0

    ; Calculate average
    |\keyword{add}| r16, r17
    ror r16

    |\keyword{pop}| r0 ; Restore SREG
    |\keyword{out}| CPU_SREG, r0
    |\keyword{pop}| r0 ; Restore r0
    |\keyword{ret}|

main_loop:
    ; Arguments
    |\keyword{ldi}| r16, 100
    |\keyword{ldi}| r17, 200
    |\keyword{rcall}| average
\end{minted}
Using the stack:
\begin{minted}{ca65}
|\keyword{rjmp}| main_loop

; Calculate the average of two numbers
; Inputs:
;     top two values on stack
; Outputs:
;     r16: average
average:
    |\keyword{push}| ZL ; Save Z
    |\keyword{push}| ZH
    |\keyword{in}| ZL, CPU_SREG ; Save SREG
    |\keyword{push}| ZL
    |\keyword{push}| r17 ; Save r17
    |\keyword{in}| ZL, CPU_SPL ; Get SP location
    |\keyword{in}| ZH, CPU_SPH

    ; Get numbers number
    |\keyword{ldd}| r16, Z+7
    |\keyword{ldd}| r17, Z+6

    ; Calculate average
    |\keyword{add}| r16, r17
    ror r16

    |\keyword{pop}| r17 ; Restore r17
    |\keyword{pop}| ZL ; Restore SREG
    |\keyword{out}| CPU_SREG, ZL
    |\keyword{pop}| ZH ; Restore Z
    |\keyword{pop}| ZL
    |\keyword{ret}|

main_loop:
    ; Arguments
    |\keyword{ldi}| r16, 100
    |\keyword{push}| r16
    |\keyword{ldi}| r16, 200
    |\keyword{push}| r16
    |\keyword{rcall}| average

    ; Remove arguments from the stack
    |\keyword{pop}| r0
    |\keyword{pop}| r0
\end{minted}
Note that it is preferrable to return values using registers.
\chapter{Variables}
Variables are used to temporarily store values in memory. Variables
have a \textbf{type} and a \textbf{name} and must be declared before
use.
\section{Declaration}
To declare a variable in C, we must specify the type and name of that
variable.
\begin{minted}{c}
int x;
\end{minted}
This variable can then be \textbf{assigned to} using the
\mintinline{c}{=} operator.
\begin{minted}{c}
x = 4;
\end{minted}
\section{Initialisation}
To optionally assign a value during declaration, we can apply the
assignment operator after the declaration. This is known as a variable
\textbf{initialisation}, as we are assigning an initial value to the
variable.
\begin{minted}{c}
int x = 4;
\end{minted}
Note that using \textbf{uninitialised variables} results in
\textbf{unspecified behaviour} in C, meaning that the value of such
variables is unpredictable.
\section{Types}
While AVR assembly supports 8-bit registers, C supports larger data
types by treating them as a sequence of bytes. We can also create
compound data types with \mintinline{c}{struct} and
\mintinline{c}{union}.
\subsection{Type Specifiers}
Type specifiers in declarations define the type of the variable. The
\mintinline{c}{signed char}, \mintinline{c}{signed int}, and
\mintinline{c}{signed short int}, \mintinline{c}{signed long int}
types, together with their \mintinline{c}{unsigned} variants and
\mintinline{c}{enum}, are all known as \textbf{integral} types.
\mintinline{c}{float}, \mintinline{c}{double}, and \mintinline{c}{long
double} are known as \textbf{floating} or \textbf{floating-point}
types. The following table summarises various numeric types in C\@:
\begin{table}[H]
    \centering
    \begin{tabular}{c c c}
        \toprule
        \textbf{Description}           & \textbf{Size} & \textbf{Equivalent Definitions}                              \\
        \midrule
        Character data                 & \qty{1}{B}    & \mintinline{c}{signed char c; char c;}                       \\
        Signed short                   & \qty{2}{B}    & \mintinline{c}{signed short int s; signed short s; short s;} \\
        Unsigned short                 & \qty{2}{B}    & \mintinline{c}{unsigned short int us; unsigned short us;}    \\
        Signed integer                 & \qty{4}{B}    & \mintinline{c}{signed int i; signed i; int i;}               \\
        Unsigned integer               & \qty{4}{B}    & \mintinline{c}{unsigned int ui; unsigned ui;}                \\
        Signed long                    & \qty{8}{B}    & \mintinline{c}{signed long int l; signed long l; long l;}    \\
        Unsigned long                  & \qty{8}{B}    & \mintinline{c}{unsigned long int ul; unsigned long ul;}      \\
        Single precision floating      & \qty{4}{B}    & \mintinline{c}{float f;}                                     \\
        Double precision floating      & \qty{8}{B}    & \mintinline{c}{double d;}                                    \\
        Long double precision floating & \qty{16}{B}   & \mintinline{c}{long double ld;}                              \\
        \bottomrule
    \end{tabular}
    % \caption{} % \label{}
\end{table}
Note that the size of these types is not necessarily the same across platforms, hence it is discouraged to use these keywords for
platform specific tasks. \emph{See the section on \hyperref[sec:exact_width_types]{Exact Width Types} for more information}.
\subsection{Type Qualifiers}
Types can be qualified with additional keywords to modify the
properties of the identifier. Three common qualifiers are
\textbf{const}, \textbf{static}, and \textbf{volatile}.
\begin{itemize}
    \item \mintinline{c}{const} --- indicates that the variable is \textbf{constant} and cannot be modified.
    \item \mintinline{c}{static} --- indicates that the variable has a global lifetime (maintains value between function invocations).
    \item \mintinline{c}{volatile} --- indicates that the variable can be modified or accessed by other programs or hardware.
\end{itemize}
\subsection{Portable Types}
C has a set of standard types that are defined in the language
specification, however the type specifiers shown above may have
different storage sizes depending on the platform. Although this may be
insignificant for most platforms, microcontrollers use specific sizes
for registers, meaning it is important to refer to the correct type
specifiers when declaring a variable.
\subsection{Exact Width Types}\label{sec:exact_width_types}
The standard integer (\mintinline{c}{stdint.h}) library provides
\textbf{exact-width} type definitions that are specific to the
development platform. This ensures that variables can be initialised
with the correct size on any platform.
\subsection{Floating-Point Types}
The \mintinline{c}{float} and \mintinline{c}{double} types can store
\textbf{floating-point} value types in C. Their implementation allows
for variable levels of precision, i.e., extremely large and extremely
small values. These types are very useful on systems with a floating
point unit (FPU) or equivalent.

As the ATtiny1626 does not have an FPU, arithmetic involving floating
point values is highly inefficient. Therefore, integer arithmetic
should be utilised when possible. Note that a single floating point
number or operation causes the entire floating point library to be
included which can require a large amount of memory.
\chapter{Literals}
\section{Integer Prefixes}
Integer literals are assumed to be base 10 unless a prefix is
specified. C supports all of the following prefixes:
\begin{itemize}
    \item \textbf{Binary} (base 2) --- \mintinline{c}{0b}
    \item \textbf{Octal} (base 8) --- \mintinline{c}{0}
    \item \textbf{Decimal} (base 10) --- no prefix
    \item \textbf{Hexadecimal} (base 16) --- \mintinline{c}{0x}
\end{itemize}
\section{Integer Suffixes}
Integer literals can be suffixed to specify the size/type of the value:
\begin{itemize}
    \item \textbf{Unsigned} --- \mintinline{c}{U}
    \item \textbf{Long} --- \mintinline{c}{L}
    \item \textbf{Long Long} --- \mintinline{c}{LL}
\end{itemize}
Suffixes are generally only required when clarifying ambiguity of values where the user wishes to use a different type than the default type.
\begin{minted}{c}
#include <stdio.h>

printf("%d\n", 2147483648); // Treated as signed integer and throws warning
printf("%d\n", 2147483648U); // Treated as unsigned integer
\end{minted}
\section{Floating Point Suffixes}
As with integer types, floating point values can also be suffixed to
specify which type to use.
\begin{itemize}
    \item \textbf{Float} --- \mintinline{c}{f}
    \item \textbf{Double} --- \mintinline{c}{d}
\end{itemize}
\section{Character and String Literals}
\begin{itemize}
    \item \textbf{Character} --- surrounded by single quotes \mintinline{c}{'A'}
    \item \textbf{String} --- surrounded by double quotes \mintinline{c}{"Hello World"} % chktex 18
\end{itemize}
\subsection{Character Literals}
The encoding of a character is platform-dependent, but typically
characters 0--127 are defined through the American Standard Code for
Information Interchange (ASCII), with slight variations depending on
locale. In this character set, characters 0--31 and 127 are control
characters, and characters 32--126 are printable characters. Characters
128--255 are typically defined by the local character set, and are not
portable between platforms.

A character literal is a single character enclosed in single quotes,
e.g., \mintinline{c}{'a'}, and escape sequences can be used to
represent control characters, e.g., \mintinline{c}{'\n'}. Characters
are stored in memory as 8-bit values, and are typically represented as
unsigned integers in C.
\subsection{String Literals}
A string is a sequence of characters, terminated by a null character
(\mintinline{c}{'\0'}). Strings can be defined using double quotes,
e.g., \mintinline{c}{"Hello, world!"} or as an array of characters.
When defined using double quotes, a null character is implicitly added
to the end of the string, however this is not the case when defined as
an array of characters.% chktex 18

If the size of the character array is specified for a character array,
then
\begin{itemize}
    \item If the size is greater than the length of the string, then
          the remaining elements of the array are initialised to 0.
    \item If the size is less than the length of the string, then the
          string is truncated to fit the array, and no null character
          is added.
\end{itemize}
When accessing strings to perform operations on them, we can utilise the null character to determine the length of the string.

To modify a string, we can utilise pointer arithmetic to modify each
character in-place. The \mintinline{c}{<string.h>} header file contains
a number of functions for manipulating strings, including
\begin{itemize}
    \item \mintinline{c}{strlen()} --- Get length of string
    \item \mintinline{c}{strcpy()}, \mintinline{c}{strncpy()} --- Copy string
    \item \mintinline{c}{memcpy()} --- Copy data
    \item \mintinline{c}{strcmp()} --- Compare strings
    \item \mintinline{c}{strchr()}, \mintinline{c}{strrchr()} --- Find character in string
    \item \mintinline{c}{strstr()} --- Find string in string
    \item \mintinline{c}{strcat()} --- Concatenate two strings
\end{itemize}
\subsection{Standard Input/Output}
The C standard I/O header file \mintinline{c}{<stdio.h>} library
provides a number of functions that read input and write output. Many
of these functions work with the standard input and output devices,
\mintinline{c}{stdin} and \mintinline{c}{stdout}. On a PC, these will
normally read from or write to, a terminal by default. On the QUTy,
these devices are non-functional by default, and must be configured to
read from/write to USART0.

The \mintinline{c}{putchar()} and \mintinline{c}{getchar()} functions
are used to write a single character to the standard output device and
read a single character from the standard input device, respectively.
\subsection{Formatted Output}
The \mintinline{c}{printf()} function (print formatted) is used to
write formatted output to the standard output device. The format string
is a string that contains text and format specifiers, which are used to
specify how the arguments are to be formatted. The format specifiers
are prefixed by a percent sign (\mintinline[escapeinside=||]{c}{|\%|}).
\begin{minted}{c}
#include <stdio.h>

printf("Hello, world!\n"); // Print a string
printf("%d\n", 100); // Print a signed integer (or %i)
printf("%d%%\n", 100); // Escape a percent sign
printf("%u\n", 100); // Print an unsigned integer
printf("%o\n", 100); // Print an unsigned octal
printf("%x\n", 100); // Print an unsigned hexadecimal (%X uppercase)
printf("%p\n", &var); // Print a pointer
\end{minted}
Preceding the format specifier with an optional \textbf{integer length
modifier} will change the size of the argument that is formatted.
\begin{minted}{c}
#include <stdio.h>

printf("%d\n", 100); // Print a signed integer
printf("%hd\n", 100); // Print a signed short integer
printf("%ld\n", 100); // Print a signed long integer
\end{minted}
Additionally, a number before the specifier can be used to specify the
minimum width of the output, and a number after the period can be used
to specify the number of decimal places.
\begin{minted}{c}
#include <stdio.h>

printf("%d\n", 100); // Print a signed integer
printf("%10d\n", 100); // Print a signed integer with a minimum width of 10
printf("%-10d\n", 100); // Print a signed integer with a minimum width of 10 and
                        // justify left
\end{minted}
For floating point numbers, the \mintinline[escapeinside=||]{c}{|\%|f}
specifier can be used to print a fixed-point number, and the
\mintinline[escapeinside=||]{c}{|\%|e} specifier can be used to print a
floating-point number in scientific notation.
\mintinline[escapeinside=||]{c}{|\%|g} prints either depending on the
size. A decimal point can be used to specify the number of decimal
places to print.
\begin{minted}{c}
#include <stdio.h>

printf("%f\n", 100.0); // Print a float
printf("%.2f\n", 100.0); // Print a float with 2 d.p.
printf("%e\n", 100.0); // Print a float in scientific notation
\end{minted}
For performance reasons, \mintinline{c}{printf()} is buffered, and will
store characters in a buffer in SRAM\@. By default
\mintinline{c}{stdout} is line-buffered, meaning that the buffer is
flushed when a newline character is written to the buffer.
\mintinline{c}{fflush(stdout)} can be used to flush the buffer.
\subsection{Formatted Input}
The \mintinline{c}{scanf()} function (scan formatted) is used to read
formatted input from the standard input device. The format string is
similar to that of \mintinline{c}{printf()}, except that the format
specifiers are used to specify how the input is to be read.
\begin{minted}{c}
#include <stdio.h>

int i;
scanf("%d", &i); // Read a signed integer
// A signed integer input will be stored in i
// Otherwise, the function returns 0

char str[10];
scanf("abc%d", str); // Read a string with a prefix
// If the string is "abc123", then str will be "123"
// If the string is "5", then "5" will stay in the buffer and the function returns 0
// If the string is "abcd", then scanf will stop reading and return 0
// Otherwise, the function returns 0
\end{minted}
If two or more format specifiers are used, then the function will read
input until it reaches a whitespace character, and then stop.
\begin{minted}{c}
#include <stdio.h>

int i, j;
scanf("%d %d", &i, &j); // Read two signed integers
// If the input is "123 456", then i will be 123 and j will be 456
// If the input is "123", then i will be 123 and the function returns 0
// Otherwise, the function returns 0
\end{minted}
Whitespace characters are ignored when reading input,
\begin{minted}{c}
#include <stdio.h>

char c;
scanf("%c", &c); // Read a character
// If the input is "a", then c will be 'a'
// If the input is " a", then c will be 'a'
// If the input is "  a", then c will be 'a'
\end{minted}
A width specifier tells the function how many characters to read, this
is useful to prevent buffer overflows.
\begin{minted}{c}
#include <stdio.h>

char str[10];
scanf("%9s", str); // Read a string with a maximum length of 9
// If the input is "123456789", then str will be "123456789"
// If the input is "1234567890", then str will be "123456789"
\end{minted}
The asterisk (\mintinline{c}{*}) can be used to ignore input.
\begin{minted}{c}
#include <stdio.h>

int i;
scanf("%*d %d", &i); // Read a signed integer
// If the input is "123 456", then i will be 456
// If the input is "123", then the function will return 0
// Otherwise, the function returns 0
\end{minted}
A scanset can be used to specify a set of characters that are allowed
to be read.
\begin{minted}{c}
#include <stdio.h>

char c;
scanf("%[abc]", &c); // Read a character
// If the input is "a", then c will be 'a'
// If the input is "b", then c will be 'b'
// If the input is "c", then c will be 'c'
// Otherwise, the function returns 0
\end{minted}
This behaves similarly to a regular expression, and can be negated by
using the caret (\mintinline{c}{^}), ranges can be specified by using a
hyphen (\mintinline{c}{-}), and the backslash
(\mintinline[escapeinside=||]{c}{|\backslash|}) can be used to escape
special characters.

If a width is specified for a scanset, then the function will read
input until it reaches a character that is not in the scanset or until
the maximum width is reached. This character will be left in the
buffer.
\chapter{Expressions}
C provides a number of operators which can be used to perform
arithmetic/logical operations on values. C follows the same precedence
rules as mathematics, however caution should be used when comparing
precedence of certain logical and bitwise operations.
\section{Operation Precedence}
\setminted{escapeinside={?*}{*?}}
\begin{table}[H]
    \centering
    \begin{tabular}{c c c}
        \toprule
        \textbf{Operation}            & \textbf{Operator Symbol}                                                                                                                                                                                                  & \textbf{Associativity}         \\
        \midrule
        Postfix                       & \mintinline{c}{++}, \mintinline{c}{--}                                                                                                                                                                                    & \multirow{5}{*}{Left to right} \\
        Function call                 & \mintinline{c}{()}                                                                                                                                                                                                        &                                \\
        Array subscripting            & \mintinline{c}{[]}                                                                                                                                                                                                        &                                \\
        Member access                 & \mintinline{c}{.}                                                                                                                                                                                                         &                                \\
        Member access through pointer & \mintinline{c}{->}                                                                                                                                                                                                        &                                \\
        \midrule
        Prefix                        & \mintinline{c}{++}, \mintinline{c}{--}                                                                                                                                                                                    & \multirow{7}{*}{Right to left} \\
        Unary                         & \mintinline{c}{+}, \mintinline{c}{-}                                                                                                                                                                                      &                                \\
        Logical NOT and bitwise NOT   & \mintinline{c}{!}, \mintinline{c}{~}                                                                                                                                                                                      &                                \\
        Type cast                     & \mintinline{c}{(type)}                                                                                                                                                                                                    &                                \\
        Dereference                   & \mintinline{c}{*}                                                                                                                                                                                                         &                                \\
        Address-of                    & \mintinline[escapeinside=||]{c}{|\&|}                                                                                                                                                                                     &                                \\
        Size-of                       & \mintinline{c}{sizeof}                                                                                                                                                                                                    &                                \\
        \midrule
        Multiplicative                & \mintinline{c}{*}, \mintinline{c}{/}, \mintinline[escapeinside=||]{c}{|\%|}                                                                                                                                               & Left to right                  \\
        Additive                      & \mintinline{c}{+}, \mintinline{c}{-}                                                                                                                                                                                      & Left to right                  \\
        Bitwise shift                 & \mintinline{c}{<<}, \mintinline{c}{>>}                                                                                                                                                                                    & Left to right                  \\
        Relational                    & \mintinline{c}{<}, \mintinline{c}{>}, \mintinline{c}{<=}, \mintinline{c}{>=}                                                                                                                                              & Left to right                  \\
        Equality                      & \mintinline{c}{==}, \mintinline{c}{!=}                                                                                                                                                                                    & Left to right                  \\
        Bitwise AND                   & \mintinline[escapeinside=||]{c}{|\&|}                                                                                                                                                                                     & Left to right                  \\
        Bitwise XOR                   & \mintinline{c}{^}                                                                                                                                                                                                         & Left to right                  \\
        Bitwise OR                    & \mintinline{c}{|}                                                                                                                                                                                                         & Left to right                  \\
        Logical AND                   & \mintinline[escapeinside=||]{c}{|\&\&|}                                                                                                                                                                                   & Left to right                  \\
        Logical OR                    & \mintinline{c}{||}                                                                                                                                                                                                        & Left to right                  \\
        Conditional                   & \mintinline{c}{? :}                                                                                                                                                                                                       & Right to left                  \\ % chktex 26
        Assignment                    & \mintinline{c}{=}, \mintinline{c}{+=}, \mintinline{c}{-=}, \mintinline{c}{*=}, \mintinline{c}{/=}, \mintinline[escapeinside=||]{c}{|\%|=}, \mintinline[escapeinside=||]{c}{|\&|=}, \mintinline{c}{^=}, \mintinline{c}{|=} & Right to left                  \\
        Sequential evaluation         & \mintinline{c}{,}                                                                                                                                                                                                         & Left to right                  \\
        \bottomrule
    \end{tabular}
    % \caption{} % \label{}
\end{table}
\section{Arithmetic Operations}
All arithmetic operations work as expected, noting that integer
division is truncated.

If an arithmetic operation causes a type overflow, the result will
depend on the type. For signed integers, the result of an overflow is
\textbf{undefined} in C. For unsigned integers, the result is truncated
to the type size (or the value modulo the type size).
\section{Operator Types}
\begin{itemize}
    \item \textbf{Unary} operators --- have a single operand. For example, \mintinline{c}{++} and \mintinline{c}{--}, or \mintinline{c}{+} and \mintinline{c}{-}.
    \item \textbf{Binary} operators --- have two operands. For example, \mintinline{c}{+}, \mintinline{c}{-}, \mintinline{c}{*}, and \mintinline{c}{/}.
    \item \textbf{Ternary} operators --- have three operands. For example, \mintinline{c}{? :}. % chktex 26
\end{itemize}
\section{Assignment}
To assign a value to a variable, use the assignment (\mintinline{c}{=})
operator.
\begin{minted}{c}
int x = 5;
\end{minted}
\section{Multiple Assignment}
If we want to assign values to multiple variables of the same type, we
can use the comma (\mintinline{c}{,}) operator.
\begin{minted}{c}
int x = 1, y = 2, z = 3;
\end{minted}
We can also use the assignment (\mintinline{c}{=}) operator to assign
the same value to multiple variables of the same type.
\begin{minted}{c}
int x, y, z;
x = y = z = 5;
\end{minted}
\section{Compound Assignment}
Compound assignment operators perform the operation specified by the
additional operator, then assign the result to the left operand.
\begin{minted}{c}
char x = 0b11001010;
x |= 0b00000001; // x = 0b11001010 | 0b00000001 = 0b11001011

int y = 25;
y += 5; // y = 25 + 5 = 30

char z = 0b10000010;
z <<= 1; // z = 0b10000010 << 1 = 0b00000100
\end{minted}
\chapter{The Preprocessor}
The preprocessor processes C code before it is passed onto the
compiler. The preprocessor strips out comments, handles
\textbf{preprocessor directives}, and replaces macros. Preprocessors
begin with the \mintinline{c}{#} character and no non-whitespace
characters can appear on the line before the preprocessor directive.
\section{Includes}
The \mintinline{c}{#include} directive is used to include the contents
of another file into the current file. This directive has two forms.
\begin{itemize}
    \item \mintinline{c}{#include <filename>} --- include header files for the C standard library and other header files associated with the target platform.
    \item \mintinline{c}{#include "filename"} --- include programmer-defined header files that are typically in the same directory as the file containing the directive. % chktex 18
\end{itemize}
When this directive is used, it is equivalent to copying the contents of the file into the current file,
at the location of the directive. The included file is also preprocessed and may contain other include directives.
\section{Header Files}
Object files containing \textbf{compiled code} can be linked into a
program to allow programmers to call existing functions. For C to have
knowledge of the functions in this object file, the authors of those
functions should store the function prototypes in a \textbf{header
file}.

Header files end in the \mintinline{text}{.h} extension. They can be
included into the source file using the \mintinline{c}{#include}
directive and can significantly reduce compile times by reducing the
amount of code that needs to be compiled.

In the following example, we will define an \mintinline{c}{add}
function and include it into another C program.
\begin{minted}{c}
// add.c
int add(int x, int y)
{
    return x + y;
}
\end{minted}
This file is compiled to \mintinline{text}{add.o}. To allow the
\mintinline{c}{add} function to be called from the main program, we
need to create a header file containing the function prototype of
\mintinline{c}{add}.
\begin{minted}{c}
// add.h
int add(int x, int y); // The variable names are not required in the prototype
\end{minted}
We can then include this header file into the main program.
\begin{minted}{c}
// main.c
#include <stdio.h> // Include printf definition (and other definitions)
#include "add.h" // Include add function definition

int main()
{
    int x = 5;
    int y = 10;

    int z = add(x, y);
    printf("%d\n", z);

    return 0;
}
\end{minted}
\section{Definitions}
The \mintinline{c}{#define} directive is used to define
\textbf{preprocessor macros}. Whenever these macros appear in the
source file, they are replaced with the value specified by the macro.
Macros are a simple text replacement mechanism, an thus must be defined
carefully to avoid invalid code from being generated.
\begin{minted}{c}
#include <stdio.h>
#define PI 3.14159265358979

int main()
{
    printf("%f\n", 2 * PI);
    return 0;
}
\end{minted}
Aside from constant values, macros can also be used to create small
compile-time ``functions'', that expand to code:
\begin{minted}{c}
#include <stdio.h>
#define MAX(x, y) ((x) > (y) ? (x) : (y))

int main()
{
    int x = 5;
    int y = 10;

    int z = MAX(x, y);
    printf("%d\n", z);

    return 0;
}
\end{minted}
Note that the semi-colon is omitted at the end of the macro definition,
as it would also be substituted into the program. Only a single
preprocessor directive can appear on a line, and the directives must
occupy a single line (note that a backslash
(\mintinline[escapeinside=||]{c}{|\backslash|}) can be used to break
long lines). % chktex 9
\chapter{Pointers}
When a variable is declared, the compiler automatically allocates a
block of memory to store that variable. If we want to access this block
of memory indirectly, we must use a \textbf{pointer}. In C, pointers
are declared as ``pointing to'' an object of another type.
\begin{minted}{c}
uint8_t *ptr; // Pointer to a uint8_t variable
\end{minted}
This code declares a variable \mintinline{c}{ptr} that points to a
\mintinline{c}{uint8_t}. Internally, a pointer contains a
\textbf{memory address}, which on the ATtiny1626 is 16-bit.
\section{Addressing}
When the location we want to access is known in advance, pointers can
be declared with a specific address:
\begin{minted}{c}
volatile uint8_t *ptr = (volatile uint8_t *)0x0421; // The address of PORTB DIRSET
\end{minted}
A more common usage of pointers is to \textit{reference} \textbf{other
variables}.
\begin{minted}{c}
uint8_t x = 5;
uint8_t *ptr = &x; // Address of x
\end{minted}
The amperstand (\mintinline{c}{&}) operator is used to return the
\textbf{address of} the variable \mintinline{c}{x}. Here the pointer
type of \mintinline{c}{ptr} must match the type of \mintinline{c}{x}.
\section{Dereferencing}
Once we have a pointer, we can access the value at the address it
points to using the \textbf{unary dereference} operator
(\mintinline{c}{*}).
\begin{minted}{c}
uint8_t x = 5;
uint8_t *ptr = &x; // Address of x

// Read the value at the address pointed to by ptr
uint8_t y = *ptr; // y = Value at ptr = 5

// Write the value 10 to the address pointed to by ptr
*ptr = 10; // Value at ptr := 10
// x = 10 but y = 5
\end{minted}
This is also known as \textbf{indirection}, as we are
\textit{indirectly accessing} a value through a pointer.
\section{Strings}
In C, strings are represented as arrays of characters, terminated by a
character with the value 0. Strings are declared using double quotes
(\mintinline{c}{""}) and are automatically terminated by a null
character. % chktex 18
\begin{minted}{c}
char *str = "Hello World";

printf("%s\n", str); // Prints "Hello World\n"

// Because str is a pointer, it can be printed directly.
printf(str); // Prints "Hello World"
printf("\n"); // Prints "\n"
\end{minted}
In the example above, the compiler automatically allocates a block of
memory to store the string, which in this case is 12 bytes long (11
characters + null terminator).

The pointer \mintinline{c}{str} points to the first character in the
string.
\begin{minted}{c}
char *str = "Hello World";
*str == 'H'; // True
\end{minted}
When using the \mintinline{c}{printf} function, the null terminator is
required to indicate the end of the string. We will see how to index
into strings in the section on arrays.
\section{Qualifiers}
Various \textbf{qualifiers} can be used to modify the type of a
pointer. Typically these qualifiers apply to the memory pointed to by
the pointer. If the variable which the pointer points to is
\textbf{constant}, the dereference operator cannot be used to reassign
the value of the variable.
\begin{minted}{c}
const uint8_t a = 100; // Constant
uint8_t *ptr = &a; // Points to the constant `a`

*ptr = 200; // Error: Cannot modify `a` because `a` is constant
\end{minted}
If the pointer is declared as \textbf{constant}, the pointer imposes a
\textbf{read-only} restriction on the memory it points to.
\begin{minted}{c}
uint8_t a = 100; // Variable
const uint8_t *ptr = &a; // Points to `a` but treats it as constant

*ptr = 200; // Error: Cannot modify `a` because `ptr` is constant
\end{minted}
Note this does not mean that the pointer itself is constant, only that
the memory it points to is constant. The following is valid:
\begin{minted}{c}
uint8_t a = 100; // Variable
uint8_t b = 200; // Variable

const uint8_t *ptr = &a; // Points to `a` but treats it as constant
ptr = &b; // Valid: `ptr` is not constant
\end{minted}
If the qualifier is placed after the asterisk, the pointer itself is
constant, meaning that it cannot be reassigned to another address.
\begin{minted}{c}
uint8_t a = 100; // Variable
uint8_t b = 200; // Variable

uint8_t *const ptr = &a; // Points to `a` but cannot be reassigned
*ptr = 200; // Valid: `ptr` points to `a` which is not constant
ptr = &b; // Error: Cannot reassign `ptr`
\end{minted}
If we wish, we can apply the qualifiers to both the pointer and the
variable which that pointer points to.
\begin{minted}{c}
uint8_t a = 100; // Variable
uint8_t b = 200; // Variable

const uint8_t *const ptr = &a; // Points to `a` but cannot be reassigned nor modified
ptr = &b; // Error: Cannot reassign `ptr`
*ptr = 200; // Error: Cannot modify `a` because `ptr` is constant
\end{minted}
\subsection{Pointers to Pointers}
Pointers can also point to other pointers.
\begin{minted}{c}
uint8_t a = 100; // Variable

uint8_t *ptr = &a; // Points to `a`
uint8_t **ptr2 = &ptr; // Points to `ptr`
\end{minted}
This can be used to modify the \textbf{address} of a pointer
indirectly.
\begin{minted}{c}
uint8_t a = 100; // Variable
uint8_t b = 200; // Variable

uint8_t *ptr = &a; // Points to `a`
uint8_t **ptr2 = &ptr; // Points to `ptr`

*ptr2 = &b; // `ptr` now points to `b`
\end{minted}
For high levels of indirection, we can use more asterisks, although
this is uncommon. Qualifiers can also be applied to pointers to
pointers:
\begin{minted}{c}
uint8_t a = 100;              // Variable
uint8_t *ptr = &a;            // Points to `a`
const uint8_t **ptr1 = &ptr;  // Pointer to pointer to constant uint8_t
uint8_t * const *ptr2 = &ptr; // Pointer to constant pointer to uint8_t
uint8_t ** const ptr3 = &ptr; // Constant pointer to pointer to uint8_t
\end{minted}
\subsection{Pointer Arithmetic}
Pointers can be changed with arithmetic operators such as
\mintinline{c}{+} and \mintinline{c}{-}. Arithmetic on pointers affects
the address of the pointer, so that the pointer points to another
location. When performing arithmetic on pointers, the size of an
increment is determined by the type of the variable that the pointer is
pointing to.
\begin{minted}{c}
uint8_t a = 100; // Variable
uint8_t *ptr = &a; // Points to `a`
ptr++; // Increment by 1 byte (size of uint8_t)
// ptr now points to the next byte after `a`
\end{minted}
\subsection{Void Pointers}
When a pointer needs to point to a memory address of an unknown type,
it can be declared with the \mintinline{MATLAB}{void} keyword.
\begin{minted}{c}
void *ptr;
\end{minted}
Void pointers have no type, so they cannot be dereferenced. Pointers of
other types can be assigned to void pointers, but not vice versa.
\begin{minted}{c}
uint8_t a = 100;
void *ptr = &a; // Pointer to uint8_t
uint8_t *ptr2 = ptr; // Error: Cannot assign void pointer to uint8_t pointer
\end{minted}
\subsection{Size-of}
The \mintinline{c}{sizeof} function can be used to determine the size
of a variable in bytes.
\begin{minted}{c}
uint8_t a = 100;
uint16_t b = 200;
sizeof(a); // Returns 1
sizeof(b); // Returns 2
\end{minted}
\section{Arrays}
Array types are used to hold multiple values of the same type in a
contiguous block of memory. Arrays can be declared in the following
ways:
\begin{minted}{c}
uint8_t a[10]; // Array of 10 uint8_t
uint8_t b[10] = {0}; // Array of 10 uint8_t initialized to 0
uint8_t c[] = {1, 2, 3}; // Array of 3 uint8_t initialized to 1, 2, 3
uint8_t d[5] = {1, 2, 3}; // Array of 5 uint8_t initialized to 1, 2, 3, 0, 0
\end{minted}
The brace (\mintinline{c}{{ }}) syntax can only be used to initialise
an array and if the length of the array which is being assigned is less
than the length of the array being assigned to, the remaining values
will be set to 0.
\subsection{Character Arrays}
A character array is a special type of array which is used to store
strings. Character arrays can be declared using the
\mintinline{c}{char} keyword.
\begin{minted}{c}
char a[] = "Hello World";
// Equivalent to:
char b[12] = {'H', 'e', 'l', 'l', 'o', ' ',
              'W', 'o', 'r', 'l', 'd', '\0'};
\end{minted}
This method allocates 12 bytes of SRAM and initialises those bytes with
the string \mintinline{c}{"Hello World"}. This means that the string
can be modified later in the program. If we use the
\mintinline{MATLAB}{const} keyword, the string will be stored in flash
memory and cannot be modified.% chktex 18
\begin{minted}{c}
const char a[] = "Hello World";
\end{minted}
\subsection{Indexing}
Array elements can be accessed with the array index operator
(\mintinline{MATLAB}{[ ]}). In C, array indices start at 0.
\begin{minted}{c}
uint8_t a[10] = {0, 1, 2, 3, 4, 5, 6, 7, 8, 9};
a[0]; // Returns 0
a[1] = 10; // a is now {0, 10, 2, 3, 4, 5, 6, 7, 8, 9}
\end{minted}
It is undefined behaviour to access an array element which is out of
bounds. However it is possible to have a pointer to an element one past
the end of an array as long as the pointer is not dereferenced.
\begin{minted}{c}
uint8_t a[10] = {0, 1, 2, 3, 4, 5, 6, 7, 8, 9}
uint8_t *ptr = &a[10];
\end{minted}
To loop through an array, we can use a \mintinline{c}{for} loop.
\begin{minted}{c}
uint8_t a[10];
for (uint8_t i = 0; i < 10; i++) {
    a[i] = i;
}
\end{minted}
\subsection{Pointers and Arrays}
Arrays are implicitly converted to pointers to the first element of the
array.
\begin{minted}{c}
uint8_t a[10];
uint8_t *ptr = a; // Equivalent to `uint8_t *ptr = &a[0]`
*ptr = 100; // `a` is now {100, 0, 0, 0, 0, 0, 0, 0, 0, 0}
\end{minted}
This is especially useful when passing arrays to functions, as arrays
cannot be passed to functions by value, but rather the pointer to that
array can. This lets us index into an array in a function and the
changes will be reflected in the original array.
\begin{minted}{c}
void func(uint8_t *arr) {
    arr[0] = 100;
}

uint8_t a[10];
func(a); // `a` is now {100, 0, 0, 0, 0, 0, 0, 0, 0, 0}
\end{minted}
The syntax \mintinline{MATLAB}{arr[i]} is equivalent to
\mintinline{MATLAB}{*(arr + i)}. This is possible because arrays are
stored contiguously in memory. Note that it is not possible to change
an array's address:
\begin{minted}{c}
uint8_t a[10];
a++; // Error: Cannot change the address of an array
\end{minted}
\subsection{Array Length}
The length of an array can be determined with the
\mintinline{c}{sizeof} function.
\begin{minted}{c}
uint8_t a[10];
uint16_t b[5];
sizeof(a) / sizeof(a[0]); // Returns 10
sizeof(b) / sizeof(b[0]); // Returns 5
\end{minted}
We divide by the size of the first element of the array because the
type of the array may be larger than 1 byte.
\subsection{Copying Arrays}
Arrays can be copied in two ways. The first way is to use a
\mintinline{c}{for} loop.
\begin{minted}{c}
uint8_t a[10];
uint8_t b[10];
for (uint8_t i = 0; i < sizeof(a) / sizeof(a[0]); i++) {
    b[i] = a[i];
}
\end{minted}
The second way is to use the \mintinline{c}{memcpy} function from the
\mintinline{c}{string.h} library.
\begin{minted}{c}
uint8_t a[10];
uint8_t b[10];
memcpy(b, a, sizeof(a) / sizeof(a[0]));
\end{minted}
\subsection{Multidimensional Arrays}
Multi-dimensional arrays (or multiple subscript arrays) are used to
hold multi-dimensional data.
\begin{minted}{c}
uint8_t a[][3] = {
    {1, 2, 3},
    {4, 5, 6},
    {7, 8, 9}
};
\end{minted}
To declare a multi-dimensional array, all dimensions but the first need
to be specified. The rows of the array must be specified within
additional braces (\mintinline{c}{{ }}). Elements can be accessed by
specifying the index of each dimension.
\begin{minted}{c}
a[0][0]; // Returns 1
a[1][2]; // Returns 6
\end{minted}
These arrays are also stored contiguously in memory, in
\textbf{row-major} order, and hence pointer arithmetic is performed
differently.
\begin{minted}{c}
uint8_t a[][3] = {
    {1, 2, 3},
    {4, 5, 6},
    {7, 8, 9}
};

uint8_t rows = 3;
uint8_t cols = 3;

for (uint8_t i = 0; i < rows; i++) {
    for (uint8_t j = 0; j < cols; j++) {
        // Double indexing
        printf("%d ", a[i][j]);

        // Single indexing
        printf("%d ", a[i * cols + j]);

        // Pointer arithmetic
        printf("%d ", *(*(a + i) + j));
        // Equivalent to: printf("%d ", *(a[i] + j));
        // Each row is a pointer to the first element of that row
    }
}
\end{minted}
\section{Functions}
Procedures are called functions in C. Functions can return values and
take arguments. The main function is the entry point of a program.
\begin{minted}{c}
int main(void) {
    return 0;
}
\end{minted}
Functions in C must be declared in the top-level of a C program, and
thus cannot be declared inside other functions. Functions are declared
with the following syntax:
\begin{minted}{c}
return_type function_name(param_type param_name, ...) {
    // Function body
}
\end{minted}
\subsection{Parameters}
The parameters of a function are local variables scoped to that
function.
\begin{minted}{c}
uint8_t add(uint8_t a, uint8_t b) { // `a` and `b` are parameters of `add`
    return a + b;
}

int main(void) {
    uint8_t a = 10;
    uint8_t b = 20;

    uint8_t c = add(a, b); // `a` and `b` are arguments to `add`
}
\end{minted}
To pass an array to a function, we can pass a pointer to that array. To
do so, we must specify the length of the array as well.
\begin{minted}{c}
void print_array(uint8_t *arr, uint8_t len) {
    for (uint8_t i = 0; i < len; i++) {
        printf("%d ", arr[i]);
    }
}

int main(void) {
    uint8_t a[10];

    for (uint8_t i = 0; i < sizeof(a) / sizeof(a[0]); i++) {
        a[i] = i;
    }

    print_array(a, sizeof(a) / sizeof(a[0]));
}
\end{minted}
When a function does not take any arguments, we can specify
\mintinline{c}{void} as the parameter list.
\begin{minted}{c}
void func(void) {
}
\end{minted}
\subsection{Return Values}
To return a value from a function, we use the \mintinline{c}{return}
keyword. When a function does not return a value, we can specify
\mintinline{c}{void} as the return type. Note that a void function does
not need to use the \mintinline{c}{return} keyword.
\subsection{Function Prototypes}
C uses \textbf{single-pass} compilation, meaning that functions need to
be declared before they can be called. Function prototypes are used to
declare a function without having to specify the entire body of the
function.
\begin{minted}{c}
uint8_t add(uint8_t a, uint8_t b); // Function prototype

int main(void) {
    uint8_t a = 10;
    uint8_t b = 20;

    uint8_t c = add(a, b);
}

uint8_t add(uint8_t a, uint8_t b) {
    return a + b;
}
\end{minted}
The compiler uses the function prototype to generate the code required
to call the function without having to know the entire body of the
function. The linker will then resolve all function calls to the
appropriate function definitions. Note that parameter names are not
required in function prototypes.
\subsection{Passing by Reference}
As seen previously, we can pass variables by value and arrays by
reference through pointers.

As functions only return one value, we can use pointers to pass
multiple values back to the caller. These output values are also passed
to the functions parameter list.
\begin{minted}{c}
void swap(uint8_t *a, uint8_t *b) {
    uint8_t temp = *a;
    *a = *b;
    *b = temp;
}

int main(void) {
    uint8_t a = 10;
    uint8_t b = 20;

    swap(&a, &b);
}
\end{minted}
\subsection{Call Stack}
As functions can call other functions, or even call themselves, local
variables inside functions are stored on the \textbf{stack}. The return
address of where a function is called from is also stored on the stack
so that the program counter can be set to that address when the
function returns.

Local variables inside functions do not increase the explicit SRAM
usage reported by the compiler. Rather, this memory will be allocated
on the stack when the function is called. Therefore it is important to
ensure that the stack does not overflow, through recursive functions or
large local variables.
\section{Scope}
Variables and other identifiers in C have scope. Scope affects the
\textbf{visibility} and \textbf{lifecycle} of variables. Scope is
\textbf{hierarchical}, meaning that variables declared in a parent
scope are visible to all child scopes. Variables declared in a child
scope can also hide variables declared in a parent scope declared with
the same name.
\subsection{Global Scope}
Variables declared outside of any function are declared in the global
scope. Global variables are visible to all functions in a program.
\begin{minted}{c}
uint8_t a = 10; // Global variable

int main(void) {
    uint8_t b = 20; // Local variable

    a++; // `a` is visible to `main`
    return 0;
}
\end{minted}
Global variables are allocated a fixed location in SRAM and do not
exist on the stack.
\subsection{Local Scope}
Variables declared inside a function are declared in the local scope.
Their lifetime is limited to the function in which they are declared.
By default, local variables go on the stack.
\subsection{Block Scope}
The block scope is a subset of the local scope. Variables declared
inside blocks such as \mintinline{c}{if} statements have their own
scope. These variables are only visible inside the block. We can create
a new scope by using curly braces.
\subsection{Static Variables}
When applied to a \textbf{local variable}, the \mintinline{c}{static}
keyword changes the lifetime of a variable to the lifetime of the
program. This means that the variable will not be destroyed when the
function returns, and will retain its value between function calls.
Static variables are allocated in SRAM and not on the stack. When
applied to a \textbf{global variable}, the \mintinline{c}{static}
keyword changes the visibility of the variable to the file in which it
is declared.
\chapter{Types}
\section{Accessing Registers}
As seen in the previous chapter, we can use the
\mintinline{c}{volatile} keyword to directly reference memory locations
by address. This is useful for accessing memory mapped IO\@.
\begin{minted}{c}
volatile uint8_t *portb_outclr = 0x0426;
*portb_outclr = 0b00100000;
\end{minted}
The \mintinline{c}{volatile} keyword is important because the variable
is outside the control of the program. The compiler will therefore not
optimize accesses to the variable. The \mintinline{text}{avr/io.h}
header file includes macros and type definitions for accessing various
registers on the AVR microcontroller.
\begin{minted}{c}
#include <avr/io.h>
\end{minted}
\section{Type Casting}
Some type conversions are implicit, such as converting a
\mintinline{c}{uint8_t} to a \mintinline{c}{uint16_t}. However, some
implicit type conversions generate warnings usually because of a loss
of information, or because the conversion is not portable across
platforms.

Therefore, to explicitly convert a variable to a different type, we can
use the unary type casting operator.
\begin{minted}{c}
volatile uint8_t *portb_outclr = (volatile uint8_t *)0x0426;
\end{minted}
This does not make code more portable, but it tells the compiler that
the programmer is aware of the conversion and that it is intentional.
\subsection{Types of Type Casting}
\subsubsection{Numeric Types}
Numeric types (signed or unsigned) will expand or narrow the type,
resulting in the value being truncated or zero extended.
\begin{minted}{c}
(uint8_t)-3 // 253
\end{minted}
\subsubsection{Floating Point Types}
Conversion from floating point to integer will truncate the fractional
part.
\begin{minted}{c}
(int16_t)-3.45 // -3
\end{minted}
\subsubsection{Pointer Types}
Conversion between pointers will change the pointer type but will not
affect the data. Note that these conversions are not portable across
platforms.
\begin{minted}{c}
uint16_t a = 12345;
uint8_t *b = (uint8_t *)&a; // likely 57, but may vary on different platforms
\end{minted}
Casting an integer to a pointer will make the pointer contain the
address in that integer.
\begin{minted}{c}
uint8_t *ptr = (uint8_t *)0x1234;
\end{minted}
Likewise, casting a pointer to an integer will make the integer contain
the address in the pointer.
\begin{minted}{c}
uint8_t *ptr = (uint8_t *)0x1234;
uint16_t addr = (uint16_t)ptr; // addr = 0x1234
\end{minted}
These conversions are not portable, but can be necessary when accessing
memory mapped IO\@.
\subsubsection{Modfying Qualifiers}
Casting can also be used to add/remove qualifiers such as
\mintinline{c}{const} and \mintinline{c}{volatile}, however the
following will not work on the ATtiny1626 because the
\mintinline{c}{const} qualifier usually results in that variable being
stored in SRAM or read-only memory locations.
\begin{minted}{c}
const uint8_t a = 10;
const uint8_t *b = &a;

*((uint8_t *)b) = 20; // May lead to undefined behaviour
\end{minted}
\subsubsection{Avoiding Truncation}
Casting can also be used to avoid truncation errors when performing
arithmetic.
\begin{minted}{c}
uint16_t a = 25000;
uint16_t b = 10000;

uint32_t c = a * b;           // c = 45696 (incorrect)
uint32_t d = (uint32_t)a * b; // d = 250000000
\end{minted}
\section{Floating Point Types}
In C, floating point types are represented as 32-bit IEEE 754 single
precision floating point numbers. The \mintinline{c}{float} type is a
32-bit floating point number and the \mintinline{c}{double} type is a
64-bit floating point number.

A single precision floating point number has a 1-bit sign, 8-bit
exponent, and 23-bit mantissa. As such, the range of a single precision
floating point number is \(-2^{127} \ldots 2^{127}\). A floating point
number \(f\) can be represented as
\begin{equation*}
    f = \left( -1 \right)^s \left( 1 + 2^{-23} m \right) 2^{e - 127}
\end{equation*}
where \(s\) is the sign bit, \(m\) is the mantissa, and \(e\) is the exponent.
Note that values are not equally spaced. There are several special values that
can be represented by floating point numbers.
\begin{itemize}
    \item \(e = 255 \implies 2^{128}\):
          \begin{itemize}
              \item \(m = 0\) (all 0s): \mintinline{c}{INFINITY} if \(s = 0\), \mintinline{c}{-INFINITY} if \(s = 1\)
              \item \(m\) is not all 0s: \mintinline{c}{NAN}
          \end{itemize}
    \item \(e = 0 \implies 2^{-126}\) (denormalised):
          \begin{itemize}
              \item \(m = 0\) (all 0s): \mintinline{c}{0.0} if \(s = 0\), \mintinline{c}{-0.0} if \(s = 1\)
              \item \(m\) is not all 0s: Subnormal numbers
          \end{itemize}
\end{itemize}
The flexibility of floating point numbers means that arithmetic operations
are expensive if not performed on a Floating Point Unit (FPU). As AVR does not have an FPU, floating point operations
must be handled using the ALU instructions which can be significantly slower than integer operations.
In addition, floating point operations require the \mintinline{c}{avr-libc} floating point library to be linked
which increases the size of the program.
\subsection{Fixed Point Math}
Fixed point math is a technique for performing arithmetic operations on
integers that are scaled by a power of two. This allows for integer
arithmetic to be used instead of floating point arithmetic, which can
be significantly faster at the cost of precision. For common operations
such as sine and cosine, consider using lookup tables.
\chapter{Objects}
\section{Structures}
Structures are a way to group related data together.
\begin{minted}{c}
struct Point {
    uint8_t x;
    uint8_t y;
};

struct Point p;
\end{minted}
The members of a structure can be accessed using the dot operator.
\begin{minted}{c}
p.x = 30;
p.y = 40;
\end{minted}
Struct members can also be initialized using braces as with arrays.
\begin{minted}{c}
struct Point p = { 30, 40 };
\end{minted}
Unlike arrays, structures need not be accessed via pointers and can be
passed between functions, and copied normally.
\begin{minted}{c}
void func(struct Point p) {
    p.x = 50;
}

struct Point p = { 30, 40 };
func(p);

struct Point q = p; // q.x = 50, q.y = 40
\end{minted}
Due to this, structs can contain arrays which can be passed and copied
by placing them in structs.

Along with this, functions can also return structs.
\begin{minted}{c}
struct Point func() {
    struct Point p = { 30, 40 };
    return p;
}

struct Point p = func(); // p.x = 30, p.y = 40
\end{minted}
\subsection{Memory Layout}
Struct members are stored in memory in the order they are declared. If
the platform has alignment requirements, the compiler will insert
padding to ensure that the next member is aligned correctly. This is
done to ensure that the compiler can access the members of the struct
efficiently.
\subsection{Anonymous Structures}
Structures can be declared without a name if they are only used once.
\begin{minted}{c}
struct {
    uint8_t x;
    uint8_t y;
} p;
\end{minted}
The type of this variable is \mintinline{c}{unnamed}.
\subsection{Structures Inside Structures}
Structures can contain other structures.
\begin{minted}{c}
struct Point {
    uint8_t x;
    uint8_t y;
};

struct Rectangle {
    struct Point p1;
    struct Point p2;
};

struct Rectangle r = { { 10, 20 }, { 30, 40 } };
\end{minted}
\subsection{Structures and Pointers}
Structures and members of structs can be addressed normally with the
address-of operator.
\begin{minted}{c}
struct Point p = { 30, 40 };
struct Point *ptr = &p;

ptr->x = 50; // Equivalent to (*ptr).x = 50
\end{minted}
When accessing members of structs through pointers, the arrow operator
(\mintinline{c}{->}) can be used.

Structures can also contain pointers.
\subsection{Typedef}
Typedefs can be used to give a type an alias so that the variables type
is determined by the typedef instead of the actual type. If we want to
use a structure multiple times, we can use a typedef to give it a (new)
name.
\begin{minted}{c}
typedef struct PointStruct {
    uint8_t x;
    uint8_t y;
} Point;

Point p = { 30, 40 }; // Point is an alias to struct PointStruct
\end{minted}
The type of this variable is \mintinline{c}{Point}. The struct also
need not be defined inside of the typedef.
\begin{minted}{c}
struct PointStruct {
    uint8_t x;
    uint8_t y;
};

typedef struct PointStruct Point;

Point p = { 30, 40 };
\end{minted}
This can be useful when the struct is defined in a header file and the
typedef is defined in a source file. In both cases, it is possible to
use the struct name to declare variables.
\begin{minted}{c}
struct PointStruct p;
\end{minted}
If the struct name is omitted, the type of the struct is
\mintinline{c}{unnamed}.
\begin{minted}{c}
typedef struct {
    uint8_t x;
    uint8_t y;
} Point;

Point p = { 30, 40 }; // Point is an alias to an unnamed struct
\end{minted}
In this case, the struct name cannot be used to declare variables as it
is anonymous.

Typedefs can also be used with qualifiers to reduce the amount of
typing.
\section{Unions}
Unions are similar to structures, however the members of a union share
the same overlapping memory location. While structs have capacity to
store multiple values, unions only have the capacity to store its
largest value.
\begin{minted}{c}
union Character {
    char character;
    uint8_t integer;
};

union Character c = { 'A' };

printf("%c\n", c.character); // Prints 'A'
printf("%u\n", c.integer); // Prints 65
\end{minted}
This use allows us to access the same memory location interpretted as a
different type without the need of casting pointers.

When used with structs (or other aggregates), the order members in
those structs is also maintained.
\begin{minted}{c}
struct a {
    uint8_t i;
    float f;
};

struct b {
    uint8_t i;
    char c[4];
};

union u {
    struct a a;
    struct b b;
};

union u u;

u.a.i = 10;
u.a.f = 3.14;

// u.a.i = 10; u.a.f = 3.14

u.b.c[0] = 'A';
u.b.c[1] = 'B';
u.b.c[2] = 'C';
u.b.c[3] = 'D';

// u.b.i = 10; u.b.c = { 'A', 'B', 'C', 'D' }
\end{minted}
\section{Bitfields}
Bitfields can be used within structures or unions to specify types of
specific \textbf{bit} sizes.
\begin{minted}{c}
struct {
    uint8_t x : 4;
    uint8_t y : 4;
} bits;

bits.x = 13;
bits.y = 7;

printf("%u\n", bits.x); // Prints 13
printf("%u\n", bits.y); // Prints 7
printf("%lu\n", sizeof(bits)); // Prints 1 (8 bits)
\end{minted}
In this example, the \mintinline{c}{x} and \mintinline{c}{y} members
are 4 bits each. The base type of each member must be able to store the
specified number of bits.
\subsection{Properties of Bitfields}
The address of a bitfield cannot be taken.
\begin{minted}{c}
struct {
    uint8_t x : 4;
    uint8_t y : 4;
} bits;

uint8_t *ptr = &bits.x; // Error
\end{minted}
A bitfield cannot be an array.
\begin{minted}{c}
struct {
    uint8_t x : 4;
    uint8_t y[4] : 4;
} bits; // Error
\end{minted}
The name of a bitfield can be omitted, this will introduce padding.
\begin{minted}{c}
struct {
    uint8_t x : 4;
    uint8_t : 4;
} bits;
\end{minted}
A zero-width bitfield can be used to align the next member to the next
word boundary.
\begin{minted}{c}
struct {
    uint8_t x : 4;
    uint8_t : 0;
    uint8_t y;
} bits;

printf("%lu\n", sizeof(bits)); // Prints 2
\end{minted}
\chapter{Interrupts}
An interrupt is a signal sent to the processor to indicate that it
should \textit{interrupt} the current code that is being executed to
execute a function called an \textbf{interrupt service routine} (or
\textit{interrupt handler}). Rather than polling for individual events
(such as button presses), interrupts allow the processor to be notified
when an event occurs.
\section{Interrupts and the AVR}
On the ATtiny1626, interrupts work as follows:
\begin{enumerate}
    \item An interrupt-worthy event occurs.
    \item The appropriate interrupt flag (\mintinline{c}{INTFLAGS}) in
          the peripheral is set.
    \item If the corresponding interrupt is enabled
          (\mintinline{c}{INTCTRL} field of the peripheral), the
          interrupt is triggered, and we proceed to the following step.
    \item If the global interrupt flag (\mintinline{c}{SREG.I}) is set,
          the interrupt can be executed, and we proceed to the
          following step.
    \item The PC is pushed onto the stack and jumps to the interrupt
          vector (the address of the interrupt handler). See page
          63--64 on the datasheet.
\end{enumerate}
\subsection{Interrupt Vectors}
The \textbf{interrupt vector} is a table of addresses that the
processor jumps to when an interrupt is triggered. These addresses are
usually at the beginning of the program memory.
\subsection{Interrupt Service Routine}
The code that handles the interrupt is called the \textbf{interrupt
service routine} (ISR) (or \textit{interrupt handler}). The ISR is a
function that is executed as a result of the interrupt.

When configuring an interrupt, we must temporarily disable interrupts
globally to prevent the interrupt from being triggered while we are
configuring it.
\begin{minted}{c}
cli(); // Disable interrupts globally
// Configure interrupts
sei(); // Enable interrupts globally
\end{minted}
It is also important to restore the state of the CPU or registers
before the ISR returns. This is because another interrupt can be
triggered while the ISR is executing. To tackle this, we can use the
\mintinline{c}{ISR} macro from the \mintinline{c}{avr/interrupt.h}
header file which will automatically save and restore the state of the
CPU and registers.
\begin{minted}{c}
#include <avr/interrupt.h>

ISR(TCB0_INT_vect) {
    // Interrupt service routine for TCB0
}
\end{minted}
This header file also sets aside program memory for the interrupt
vector table.
\subsection{Interrupt Flags}
Each peripheral has an interrupt flag field (\mintinline{c}{INTFLAGS})
that is set when the conditions for that interrupt occur (even if
interrupts are disabled). The exact format of this field depends on the
type of interrupt, but in general a bit is set for the type of
interrupt, see the datasheet for more information.

As some peripherals have 1 interrupt vector with multiple interrupt
sources, the interrupt flag fields can be used to determine the exact
source of the interrupt.

The interrupt flag field is cleared by writing a 1 to the corresponding
bit.
\subsection{Peripheral Interrupts}
Peripherals differ in what causes interrupts to be raised and many have
multiple interrupt sources.
\subsubsection{Port Interrupts}
To configure \mintinline{c}{BUTTON0} as an interrupt source, we must
enable the interrupt in the \mintinline{c}{PORTA.PIN4CTRL} peripheral.
\begin{minted}{c}
ISR(PORTA_PORT_vect) {
    // Interrupt service routine for PORTA
    VPORTA.INTFLAGS = PIN4_bm;
}

cli();
// Enable pull-up resistor and interrupt on falling edge
PORTA.PIN4CTRL |= PORT_PULLUPEN_bm | PORT_ISC_FALLING_gc;
sei();
\end{minted}
\subsection{Interrupts and Synchronisation}
ISR's may interact with state used by other code running at the same
time, which can cause problems with synchronisation, similar to those
faced with multithreaded programming. To avoid this, we should make use
of the \mintinline{c}{volatile} keyword so that the compiler does not
make assumptions about variable states. The \mintinline{c}{cli} and
\mintinline{c}{sei} functions can also be used to disable interrupts
and create a memory barrier which prevents instructions from being
reordered by the compiler.
\chapter{Compilation}
C source code is translated into machine code through a
\textbf{compiler}, whereas assembly code is translated into machine
code through an \textbf{assembler}. While some compilers emit assembly
code, others emit machine code directly.
\section{Assembler}
Program code run directly on CPUs is not designed to be written by
humans. The assembler prioritises performance and size efficiency, and
accounts for simplified chip design.

They allow us to write programs in plain text without needing to
memorise opcodes, or manually keep track of memory locations. Modern
assemblers also provide features such as macros that assist in writing
code. While the scope of an assembler is limited, the programmer
decides exactly which instructions are used and the assembler simply
translates them into machine code.
\section{Compiler}
A compiler is a program that translates source code written in a
high-level language such as C. In a high level programming language,
the desired program is described algorithmically by the programmer and
the compiler produces an equivalent program which results in the same
\textbf{side effects} when executed.
\subsection{Compilation Process}
Compilers can be configured to produce code in various ways. Compiling
without optimisation will produce code which closely resembles the
source code, so that it is easier to debug. This also means that code
generation is faster. Compilers can also be configured to optimise code
for minimum code size or maximum speed (or a combination of both).
\subsection{Advantages of Compilers}
High level languages are more portable than assembly code, as they are
not tied to a specific instruction set architecture. This means that
the same source code can be used on another platform, granted that a
compiler for that platform is available. Compilers also allow us to
write efficient code through the use of compiler optimisations, which
can be difficult to achieve manually in assembly code.
\subsection{Disadvantages of Compilers}
Some disadvantages of compilers are that hardware-specific features may
only be available from assembly requiring the use of inline assembly or
assembly language macros. Precise timing of code is also difficult due
to the inability to predict how the compiler generates code.
\section{Object Files}
An object file is the output from the compilation or assembler phase.
Object files mostly contain machine code, and also contain information
about the symbols defined in the source code.

Large modularised programs which split source code into multiple files
can be compiled into object files with external references unresolved.
This allows the linker to resolve the references and produce a single
executable file.
\section{Linker}
The linker is a program that combines multiple object files and links
them together to produce a single executable file. The linker resolves
all external references and addresses are updated as required. The
linking step is extremely fast compared to the compilation step, as
only addresses need to be updated. The linker also performs some
optimisations such as dead code elimination, which removes unused code.

In assembly, the \mintinline{ca65}{.global} directive can be used to
make labels available to the linker.
\subsection{Linker in Assembly}
\begin{minted}{ca65}
// function.S
.global function

function:
    ret

// main.S
rcall function
\end{minted}
In this example, both files can be compiled into object files even
though the \mintinline{ca65}{function} label is not defined in
\mintinline{ca65}{main.S}. The linker will resolve the reference to
\mintinline{ca65}{function} and produce a single executable file.
\subsection{Linker in C}
In C, top-level symbols are public by default, but can be made private
to the current translation unit by using the \mintinline{c}{static}
keyword.
\begin{minted}{c}
// main.c
static int a = 0;

// file1.c
a = 1; // Error: a is not visible
\end{minted}
To make a symbol visible to other translation units, the
\mintinline{c}{extern} keyword can be used.
\begin{minted}{c}
// main.c
extern int a;
printf("%d\n", a); // Prints 5

// file1.c
int a = 5;
\end{minted}
Any non-static symbols are implicitly global, and can be accessed from
any translation unit.
\section{Debugging}
While most microcontrollers are equipped with debugging tools, we are
often presented with no debugging tools at all. Therefore it is
important to develop strategies to be able to systematically debug
embedded programs with access to basic I/O. Some simple methods include
toggling pins on the microcontroller to indicate the state of the
program and sending formatted strings through the serial port to a
terminal.

To route stdin and stdout to/from any serial communications interface
that can read and write characters (e.g., UART, SPI, I\({}^2\)C, etc.),
we can use the \mintinline{c}{stdio.h} library:
\begin{enumerate}
    \item Declare function prototypes for the read/write functions
          (names can be anything):
          \begin{minted}{c}
static int stdio_putchar(char c, FILE *stream);
static int stdio_getchar(FILE *stream);
    \end{minted}
    \item Declare a stream to be used for stdin/stdout, using the
          \mintinline{c}{FDEV_SETUP_STREAM} macro:
          \begin{minted}{c}
static FILE stdio = FDEV_SETUP_STREAM(stdio_putchar, stdio_getchar, _FDEV_SETUP_RW);
    \end{minted}
    \item Implement the prototyped functions that read from the serial
          interface (i.e., via UART):
          \begin{minted}{c}
static int stdio_putchar(char c, FILE *stream)
{
    uart_putc(c);
    return c;
}

static int stdio_getchar(FILE *stream)
{
    return uart_getc();
}

void stdio_init(void)
{
    // Assumes serial interface is initialised elsewhere
    stdout = &stdio;
    stdin = &stdio;
}
    \end{minted}
\end{enumerate}
Here we use the following blocking functions to read and write characters:
\begin{minted}{c}
uint8_t uart_getc(void) {
    while (!(USART0.STATUS & USART_RXCIF_bm)); // Wait for data

    return USART0.RXDATAL;
}

void uart_putc(uint8_t c) {
    while (!(USART0.STATUS & USART_DREIF_bm)); // Wait for TX.DATA empty

    USART0.TXDATAL = c;
}
\end{minted}
Assembly listings are also useful for debugging in extreme cases, as
they allow us to see exactly what instructions are being executed. This
can be achieved via the \mintinline{text}{avr-objdump} tool.
\chapter{Hardware Peripherals}
Microcontrollers typically include a variety of hardware peripherals
that remove the burden of having to write software for common
functionality such as timers, serial communication, and analogue to
digital conversion.

They can provide very precise timing and very fast (nanosecond)
response times.

Hardware peripherals can run independent of the CPU (in parallel) so
that:
\begin{itemize}
    \item peripherals can perform tasks without software intervention
    \item peripherals are not subject to timing constraints (execution
          time of instructions, CPU clock speed)
    \item the CPU can be used to perform other computations while the
          peripheral is busy
\end{itemize}
All control of, and communication with peripherals is done through \textbf{peripheral registers}
which the CPU can access via the memory map. Peripherals also typically have direct access
to hardware resources such as pins.
\section{Configuring Hardware Peripherals}
Upon reset, most hardware peripherals are \textbf{disabled by default}
and must be \textbf{configured and enabled} by writing to the
appropriate peripheral registers. This is often done once at the start
of the program, but can also be reconfigured dynamically if required,
depending on the application. Information about peripheral registers is
found in the datasheet and often recommended steps are also provided.

In general:
\begin{enumerate}
    \item Set bits on peripheral registers to configure the peripheral
          in the correct mode
    \item Enable peripheral interrupts and define an associated ISR, if
          required
    \item Enable the peripheral
\end{enumerate}
It is best practice to globally disable interrupts when configuring peripherals.
\section{Timers}
Timers provide precise measurements of \textbf{elapsed clock cycles} in
hardware, independent of software and the CPU\@. Timers are used to
generate periodic events (via an interrupt), measure time between two
events, generate periodic signals on a pin that are frequency of pulse
width modulated etc.
\subsection{Timer Implementations}
Most timer implementations use the same basic structure, a
\textbf{counter} which is incremented or decremented by a
clock/event/etc. By comparing the value of this counter, the timer can
perform more complex behaviours such as generating an interrupt,
changing pin state etc.
\subsection{Timer Counters}
The ATtiny1626 has two timers, Timer Counter A and B (TCA/TCB) that are
both 16-bit counters. As both timers are highly configurable, the
datasheet should be consulted for more information.

In general, the counter, \mintinline{c}{CNT}, increments by 1 each
clock cycle. The clock cycle can be configured with a prescaler to
increase the duration of a timer.

The capture compare register \mintinline{c}{CCMP} can be used to
generate an interrupt when the counter reaches a certain value.

The period register \mintinline{c}{PER} can also be used to set the
maximum value of the counter. This register can also generate
interrupts when the counter reaches the maximum value (overflow).
\subsection{Timer Periods}
The period \(T\) is the duration of a timer cycle, that is, the time
before the counter resets to 0. To configure the timer counter to cycle
according to this period, we must configure the register
\mintinline{c}{PER}, which is the \textbf{number of counts} \(n\) of
the timer clock, depending on the \textbf{clock frequency}
\(f_\mathrm{clk}\) of the timer clock.

One period of the timer clock is given by:
\begin{equation*}
    T_\mathrm{clk} = \frac{1}{f_\mathrm{clk} / \mathrm{prescaler}}
\end{equation*}
so that the number of timer clock cycles in a period \(T\) is given by:
\begin{equation*}
    n = \frac{T}{T_\mathrm{clk}}.
\end{equation*}
The prescalared used to configure the timer clock leads to a trade-off between
the timer period and the timer resolution, as the interval between timer clock cycles \(T_\mathrm{clk}\)
is increased.
\subsection{Timer Counter B Example Configuration}
\begin{minted}{c}
#include <avr/io.h>
#include <avr/interrupt.h>

void tcb_init() {
    TCB0.CTRLB = TCB_CNTMODE_INT_gc; // Configure TCB0 in periodic interrupt mode
    TCB0.CCMP = 3333;                // Set period to some value
    TCB0.INTCTRL = TCB_CAPT_bm;      // Invoke the CAPT ISR when the counter reaches CCMP
    TCB0.CTRLA = TCB_ENABLE_bm;      // Enable TCB0
}

int main(void)
{
    cli();
    tcb_init();
    sei();

    while (1);
}
\end{minted}
\section{Pulse Width Modulation}
Pulse width modulation (PWM) is a technique used to generate a
\textbf{periodic signal} with a variable duty cycle. The \textbf{duty
cycle} \(D\) of a signal is a measure of the ratio of the HIGH time of
the signal compared to the total PWM period.
\begin{equation*}
    D = \frac{T_\mathrm{HIGH}}{T_\mathrm{HIGH} + T_\mathrm{LOW}} = \frac{T_\mathrm{HIGH}}{T}
\end{equation*}
A duty cycle of 0\% means that the signal net is always LOW, while a duty cycle of 100\%
means that the signal net is always HIGH\@.

PWM can be used as a form of digital to analogue conversion, where a
\textbf{modulating signal} is used to set the duty cycle of a PWM
output. In analogue, a triangular waveform known as the
\textbf{carrier} is compared with this modulating signal so that the
PWM output is high when the modulating signal is greater than the
carrier.
\subsection{PWM Implementation}
On the ATtiny1626, the carrier is generated by a timer counter (i.e.,
\mintinline{c}{TCA0.CNT}) and the modulating signal is the compare
value \mintinline{c}{TCA0.CCMP}. By setting the compare value to a
value less than the counter value, the PWM output's duty cycle can be
controlled, via the following equation:
\begin{equation*}
    D_\mathrm{PWM} = \frac{\text{\mintinline{c}{TCA0.CCMP}}}{\text{\mintinline{c}{TCA0.CNT}} + 1}
\end{equation*}
\subsection{PWM Brightness Control Example}
\begin{minted}{c}
#include <avr/io.h>
#include <avr/interrupt.h>

void tca_init() {
    // DISP EN
    PORTB.DIR = PIN1_bm;

    // Set waveform generation mode to single slope
    // Waveform output controls PA1 PWM (display brightness)
    TCA0.SINGLE.CTRLB = TCA_SINGLE_WGMODE_SINGLESLOPE_gc | TCA_SINGLE_CMP1EN_bm;
    TCA0.SINGLE.PER = 0xFF;                   // Set period to some value
    TCA0.SINGLE.CMP1 = 0xFF;                  // 100% duty cycle
    TCA0.SINGLE.CTRLA = TCA_SINGLE_ENABLE_bm; // Enable TCA0
}

int main(void)
{
    cli();
    tca_init();
    sei();

    while (1);
}
\end{minted}
To dynamically change the brightness of the display, the compare value
can be changed using the buffered register
\mintinline{c}{TCA0.SINGLE.CMP1BUF}. The same applies to the period.
\section{Analog to Digital Conversion}
Analog to digital conversion (ADC) is a technique used to convert an
analogue signal to a digital signal. The analogue signal is sampled at
a regular interval and the sampled value is converted to a digital
value. Digital quantities are both discrete in amplitude and time.
\subsection{Quantisation}
When we \textbf{discretise in amplitude}, this is referred to as
\textbf{quantisation}.

Each amplitude is assigned a digital \textbf{code}. The \textbf{code
width} determines the \textbf{amplitude resolution} and introduces
\textbf{quantisation error}.
\subsection{Sampling}
When we \textbf{discretise in time}, this is referred to as
\textbf{sampling}. The \textbf{sampling rate} determines the
\textbf{time resolution} and introduces \textbf{aliasing error}. This
rate is typically the period of the CPU clock.
\subsection{ADC Implementation}
Analogue to digital conversion is the process of discretising a
continuous signal (typically a voltage) into a digital code.
Specialised hardware called an \textbf{analogue to digital converter}
(ADC) performs this function. The ADC samples the analogue signal at a
regular interval at an instant in time, and converts the sampled value
to a digital value.
\subsection{ADC Potentiometer Example}
\begin{minted}{c}
#include <stdio.h>

#include <avr/io.h>
#include <avr/interrupt.h>

void adc_init() {
    // Select AIN2 (potentiometer R1)
    ADC0.MUXPOS = ADC_MUXPOS_AIN2_gc;

    // Need 4 CLK_PER cycles @ 3.3 MHz for 1us, select VDD as ref
    ADC0.CTRLC = (4 << ADC_TIMEBASE_gp) | ADC_REFSEL_VDD_gc;
    // Sample duration of 64
    ADC0.CTRLE = 64;
    // Free running
    ADC0.CTRLF = ADC_FREERUN_bm;
    // Select 8-bit resolution, single-ended
    ADC0.COMMAND = ADC_MODE_SINGLE_8BIT_gc | ADC_START_IMMEDIATE_gc;

    // Enable ADC
    ADC0.CTRLA = ADC_ENABLE_bm;
}

int main(void)
{
    cli();
    adc_init();
    sei();

    while (1)
    {
        printf("%u\n", ADC0.RESULT0);
    }
}
\end{minted}
\section{Serial Communication}
Serial communication is the process of transmitting data \textbf{one
bit at a time}. On a microcontroller, this is typically done via a
digital I/O pin.

The form of serial communication is determined by the \textbf{protocol}
used (how the data is arranged, timing etc.) and the \textbf{physical
interface} used (i.e., voltage level used to represent bits). For two
devices to communicate, they must both use the same protocol and
physical interface.
\subsection{Serial Communication Terminology}
\begin{itemize}
    \item \textbf{Transmit}: to send data. Often abbreviated to \textbf{Tx}.
    \item \textbf{Receive}: to receive data. Often abbreviated to \textbf{Rx}.
    \item \textbf{Full-duplex}: bidirectional communication, occurring simultaneously.
    \item \textbf{Half-duplex}: bidirectional communication, occurring in one direction at a time.
    \item \textbf{Simplex}: Unidirectional communication.
    \item \textbf{Synchronous}: Communication relying on a shared clock.
    \item \textbf{Asynchronous}: Communication that does not rely on a shared clock.
\end{itemize}
There are many serial interfaces used in embedded systems, including:
\begin{itemize}
    \item \textbf{UART}: Universal asynchronous receiver/transmitter.
    \item \textbf{SPI}: Serial peripheral interface.
    \item \textbf{I\({}^2\)C}: Inter-integrated circuit.
    \item \textbf{CAN}: Controller area network.
    \item \textbf{I\({}^2\)S}: Inter-IC sound.
\end{itemize}
\subsection{UART}
UART is a simple and cost effective serial communication protocol. As
it is asynchronous, its clock is not shared between the two
communicating devices. Instead, the sender and receiver must agree on a
\textbf{baud rate} (the number of bits transmitted per second). This is
typically in the range of
\qtyrange[range-phrase=~to~]{9600}{115200}{baud} (with a \qty{2}{Mbaud}
maximum).

UART is a frame based protocol, where each frame is signalled by a
start bit (always LOW), and is fixed length and format. UART can be
used in both full-duplex or half-duplex, depending on the hardware
implementation, where the transmitter and receiver are fully
independent. This means either a 1- or 2-wire mode is possible (plus 1
for GND).
\subsubsection{UART Frame Format}
The UART frame format is as follows:
\begin{itemize}
    \item \textbf{Start bit}: always LOW\@.
    \item \textbf{Data bits}: 5 to 9 bits of data\@.
    \item \textbf{Parity bit}: optional bit used to detect errors.
    \item \textbf{Stop bit}: always HIGH\@.
    \item \textbf{Idle}: in the idle state, the line is HIGH\@.
\end{itemize}
The parity bit is used to detect errors in the data bits. It allows the receiver to
detect a single-bit error in the frame. The parity bit can be configured to either be
odd or even parity.
\begin{itemize}
    \item For \textbf{even} parity, the total number of 1s in the data
          and parity bits must be even.
    \item For \textbf{odd} parity, the total number of 1s in the data
          and parity bits must be odd.
\end{itemize}
If a parity error is detected in a received frame, the receiver may choose to reject the frame.
\subsubsection{USART0 on the ATtiny1626}
The USART (Universal Synchronous/Asynchronous Receiver/Transmitter)
peripheral is used to implement UART on the ATtiny1626. The
transmission operation is as follows:
\begin{enumerate}
    \item The user loads the data for transmission into the
          \mintinline{c}{TXDATA} register.
    \item When the TX shift register is empty, the data will
          immediately be copied into the shift register.
    \item Data is shifted out from the TX shift register, one bit at a
          time, according to the baud rate.
    \item The transmitter is double-buffered so that:
          \begin{itemize}
              \item If a second byte is loaded into the
                    \mintinline{c}{TXDATA} register before the first
                    byte has finished transmitting, this byte will be
                    transferred into the TX buffer and transmitted
                    after the first byte.
              \item Additionally, writing a third byte will cause it to
                    remain in the \mintinline{c}{TXDATA} register,
                    until the previous two bytes have been transmitted.
          \end{itemize}
\end{enumerate}
The reception operation is as follows:
\begin{enumerate}
    \item The start of an incoming frame is detected based on a falling
          edge on the RX line.
    \item Data is shifted into the RX shift register, one bit at a
          time, according to the baud rate.
    \item Once the correct number of data bits have been shifted the
          shift register, the data will be copied into the
          \mintinline{c}{RXDATA} register.
    \item The receiver is double-buffered so that:
          \begin{itemize}
              \item If a second byte is shifted out of the shift
                    register before the first byte is read, it will be
                    stored in the RX buffer.
              \item Additionally, a third byte will remain in the RX
                    shift register until the RX buffer is empty.
          \end{itemize}
\end{enumerate}
For more information about the USART0 peripheral, see the datasheet.
\subsubsection{USART0 Example Configuration}
\begin{minted}{c}
void uart_init(void) {
    PORTB.DIRSET = PIN2_bm; // Output enable TX pin
    USART0.BAUD = 1389; // 9600 baud
    USART0.CTRLA = USART_RXCIE_bm; // Enable RX interrupt
    USART0.CTRLB = USART_RXEN_bm | USART_TXEN_bm; // Enable RX and TX
}
\end{minted}
The baud rate is calculated as follows:
\begin{equation*}
    \text{USART.BAUD} = \frac{64 f_{CLK\_PER}}{S f_{BAUD}}
\end{equation*}
where \(S\) is the number of samples per bit:
\begin{itemize}
    \item Asynchronous normal mode: \(S = 16\)
    \item Asynchronous double speed mode: \(S = 8\)
    \item Synchronous mode: \(S = 2\)
\end{itemize}
\subsection{Serial Peripheral Interface}
SPI is a synchronous serial communication protocol where a clock is
transmitted to allow for higher bit rates in the range
\qtyrange{10}{20}{MHz}. SPI is typically used for high-speed, inter-IC
communications (communication with other peripherals) over short
distances. SPI is also full-duplex, and can be configured to use a 2-,
3-, 4-wire modes (plus 1 for GND). It can be used to communicate with
multiple devices simultaneously, using \textbf{chip select} (CS) (or
slave select) lines to select the device to communicate with. Typically
in a master slave model, the master device controls the clock.

The SPI peripheral can be used to interface to devices that produce or
consume a serial bit stream, clocked or otherwise. The clock phase and
polarity can also be modified to suit the device being interfaced to.
It is also possible to have multiple masters on the same bus, but this
requires a careful mechanism to arbitrate access to the bus.
\subsubsection{SPI0 Example Configuration}
\begin{minted}{c}
void spi_init(void) {
    VPORTC.OUT = PIN0_bm | PIN2_bm; // Drive output to LOW on SPI CLK and SPI MOSI
    VPORTC.DIR = PIN0_bm | PIN2_bm; // Output enable SPI CLK and SPI MOSI

    PORTMUX.SPIROUTEA = PORTMUX_SPI0_ALT1_gc;

    SPI0.CTRLB = SPI_SSD_bm; // Disable client select line
    SPI0.INTCTRL = SPI_IE_bm; // Enable SPI interrupts (to latch SPI DATA)
    SPI0.CTRLA = SPI_MASTER_bm | SPI_ENABLE_bm; // Enable SPI as master
}
\end{minted}
\subsection{Other Serial Protocols}
\begin{itemize}
    \item \textbf{I\({}^2\)C}: Inter-integrated circuit.
          \begin{itemize}
              \item Very common on microcontrollers, suitable for short
                    distances only (typically < \qty{300}{mm}). Widely
                    used for external peripherals.
              \item Typically up to \qty{400}{kbaud}.
              \item Half-duplex, synchronous, 2-wire bus, with
                    bidirectional signalling, where devices use
                    open-drain outputs.
          \end{itemize}
    \item \textbf{CAN}: Controller area network.
          \begin{itemize}
              \item Very prevalent automotive and industrial standard,
                    suitable for medium distances
                    (\qtyrange{40}{500}{m})
              \item Robust and reliable (safety critical systems)
              \item Built-in message priority and arbitration
              \item Typically up to \qty{1}{Mbaud}
              \item Half-duplex, asynchronous, 2-wire bus (one
                    differential pair)
              \item Very precise timing requirements compared with UART
              \item Complex protocol and controller
          \end{itemize}
\end{itemize}
\subsection{Polled vs Interrupt Driven}
In a polled model for serial communication, the CPU is continuously
checking the status of the peripheral to see if it is ready to transmit
or receive data. This is referred to as a \textbf{blocking} read/write,
as the program does not proceed until the read/write completes. This
delay may be significant for a slow serial interface as the CPU cannot
do anything else while waiting for the peripheral to complete the
operation.

Alternatively, we can use an interrupt-driven model, where we can use
interrupts to signal when the peripheral is ready for new data to be
read/written. In this model, no delays are incurred by the CPU, and we
are guaranteed that data is already ready to be read/written. The
read/write operations are also completed in a deterministic amount of
time, and the CPU can do other tasks while waiting for the peripheral
to complete the operation.
\section{Serial Communications on the QUTy}
\subsection{Virtual COM Port via USB-UART Bridge}
The CP2102N is a USB-UART bridge that allows the QUTy to communicate
with a PC via a USB cable. On the USB (host, e.g., computer) side, it
presents itself as a virtual COM port (VCP), and on the microcontroller
side, it presents itself as a UART interface. The bytes written via the
UART are received by the VCP, and vice versa. Note that the TX pin
(output) of the microcontroller is connected to the RX pin (input) of
the VCP, and vice versa.

The UPDI pin is used to program the flash memory of the QUTy with a
program. It share the USB-UART bridge with the UART interface, which
necessitates a switch to select between the two.
\subsection{Controlling the 7-Segment Display}
The 7-segment display is interfaced to the microcontroller by a 74HC595
\textbf{shift register}. A shift register is a device which translates
\textbf{serial} input/output into \textbf{parallel} input/output. On
the QUTy, it takes a serial, 1-bit output from the microcontroller and
uses this to control, in parallel, the 7-segment display, plus a digit
select signal. The 74HC595 takes a \textbf{clocked serial data stream}
as an input, which makes interfacing via the SPI peripheral very
simple.

Q0-Q6 on the shift register control the 7 segments of the display, and
Q7 controls the digit select. The first bit clocked out of the
microcontroller will set the state of Q7, and the last bit clocked out
will set the state of Q0.

To latch the data shifted into the shift register into the output
register (and consequently update the state of Q0-Q7) requires a
\textbf{rising edge} on the DISP LATCH net.
\subsection{Time Multiplexing}
As only one side of the 7-segment display can be illuminated at a time,
we need to multiplex the display to show both digits. This is done by
using a \textbf{time multiplexing} scheme, where we illuminate each
digit for a short period of time, and then switch to the other digit.
As this is done rapidly, the human eye perceives both digits as being
illuminated at the same time. If we wish to display a number, we can
simply store the state of each digit and switch between them using an
interrupt-driven timer.

If we choose a refresh rate of \qty{50}{Hz}, we must choose a period of
\qty{20}{ms} for both digits. This means that we must switch digits
every \qty{10}{ms} which will ensure that each digit is displayed for
an equal amount of time.
\begin{minted}{c}
// Bytes latched onto 7-segment display
volatile int8_t left_byte = DISP_0 | DISP_LHS;
volatile int8_t right_byte = DISP_0;

// Current display side (alternates between left and right)
volatile uint8_t current_display_side = 0;

// 10ms interrupt
ISR(TCB0_INT_vect)
{
    if (current_display_side == 1)
        SPI0.DATA = left_byte;
    else
        SPI0.DATA = right_byte;

    // Toggle display side (flip lsb)
    current_display_side ^= 1;

    TCB0.INTFLAGS = TCB_CAPT_bm;
}

ISR(SPI0_INT_vect)
{
    // Latch the digit
    PORTA.OUTCLR = PIN1_bm; // Drive HIGH
    PORTA.OUTSET = PIN1_bm; // Drive back to LOW

    // Clear the interrupt flag (undocumented behaviour)
    SPI0.INTFLAGS = SPI_IF_bm;
}
\end{minted}
The display is updated by writing to the variables
\mintinline{c}{left_byte} and \mintinline{c}{right_byte}.
\section{Pushbutton Handling}
Pushbuttons have two possible states, \textbf{pressed} and
\textbf{released}. Given an active-low pushbutton,
\begin{itemize}
    \item When the pushbutton is pressed, the pin is pulled low,
          corresponding to a \mintinline{c}{0} value.
    \item When the pushbutton is released, the pin is pulled high,
          corresponding to a \mintinline{c}{1} value.
\end{itemize}
The state of pushbuttons can be read through a bitwise AND operation with the \mintinline{c}{PORTA.IN} register.
\begin{minted}{c}
#include <avr/io.h>

PORTA.PIN4CTRL = PORT_PULLUPEN_bm;

while (1) {
    if (PORTA.IN & PIN4_bm) {
        // Pushbutton is released
    } else {
        // Pushbutton is pressed
    }
}
\end{minted}
In this loop structure, the state of the pushbutton will be in the
pressed state until the pushbutton is released. This may not be
desirable if we want to perform a single action when the pushbutton is
pressed. To solve this, we must respond to a \textit{change in state}.
\begin{itemize}
    \item A \textbf{falling edge} is created when the pushbutton is
          pressed (transition from \mintinline{c}{1} to
          \mintinline{c}{0}).
    \item A \textbf{rising edge} is created when the pushbutton is
          released (transition from \mintinline{c}{0} to
          \mintinline{c}{1}).
\end{itemize}
This is known as \textbf{edge detection}.

To implement this in C, we can use the XOR operator (\mintinline{c}{^})
to detect a change in the signal.
\begin{minted}{c}
uint8_t pb_prev = 0xFF;
uint8_t pb_state = 0xFF;

while (1)
{
    pb_prev = pb_state;
    pb_state = PORTA.IN;
    uint8_t pb_edge = pb_state ^ pb_prev;

    uint8_t pb_falling_edge = pb_edge & pb_prev;
    uint8_t pb_rising_edge = pb_edge & pb_state;
}
\end{minted}
To additionally detect a falling/rising edge, we can use the AND
operator (\mintinline{c}{&}) to detect a change in the signal.
\subsection{Pushbutton Sampling}
The actuation of mechanical pushbuttons is slow and therefore it is
important to sample pushbutton states fast enough to detect changes in
state. Latency refers to the delay between user input and the reaction
of a system. For user input, latency should be acceptably small; what
is acceptably small depends on what magnitude of latency is perceptible
and is specific to the application. Latency as low as \qty{2}{ms} is
perceptible in particular user input applications, however latency
betwen \qtyrange{20}{60}{ms} is acceptable for most applications.
\subsection{Switch Bounce}
Switch bounce is an artefact of electromechanical switches where the
switch contacts bounce back and forth when the switch is actuated. This
results in a signal that is not stable and can cause false positives
when detecting a change in state.

As a digital system can sample a voltage much faster than a mechanical
switch can change state, the system may detect multiple transitions of
the switch state.

To prevent this, we can \textbf{debounce} the pushbuttons and either by
using a debouncing circuit, or software. To implement this in software,
we can take multiple samples and only detect a change in state if the
samples are consistent over multiple samples. Due to this, it is better
to use an ISR to capture the state of the pushbutton at regular
intervals.
\begin{minted}{c}
volatile uint8_t pb_debounced_state = PIN4_bm;
volatile uint8_t pb_falling_edge = 0;
volatile uint8_t pb_rising_edge = 0;

// Periodic 4ms interrupt
ISR(TCB0_INT_vect)
{
    static uint8_t counter = 3;

    // Capture the state of the pushbutton from the port pin
    uint8_t pb_sample = PORTA.IN & PIN4_bm;
    // Detect a change in state
    uint8_t pb_edge = pb_sample ^ pb_debounced_state;

    if (pb_edge)
    {
        if (counter-- == 0)
        {
            // Save previous debounced state
            uint8_t pb_previous_state = pb_debounced_state;

            // Update debounced state
            pb_debounced_state = pb_sample;

            // Update falling/rising edge
            pb_falling_edge = pb_edge & pb_previous_state;
            pb_rising_edge = pb_edge & pb_debounced_state;

            // Reset counter
            counter = 3;
        }
    }
    else
        // Reset counter
        counter = 3;

    TCB0.INTFLAGS = TCB_CAPT_bm;
}
\end{minted}
The above implementation only debounces a single pushbutton, as the
counter only applies to S1. To debounce multiple pushbuttons, it is
possible to use a counter for each pushbutton, however this will lead
to a large amount of code. Instead, we can utilise vertical counting.
\subsection{Vertical Counters}
Instead of using a counter variable for each pushbutton, we can use the
bits of a single variable to represent the counters for each
pushbutton. Doing so will reduce the maximum value of the counter,
however this is not a problem as we only require the counter to reach a
value of 3. The following code implements a vertical counter for 4
pushbuttons.
\begin{minted}{c}
// We can use PIN4_bm | PIN5_bm | PIN6_bm | PIN7_bm, but we do not care about the other bits
volatile uint8_t pb_debounced_state = 0xFF;
volatile uint8_t pb_falling_edge = 0;
volatile uint8_t pb_rising_edge = 0;

// Periodic 4ms interrupt
ISR(TCB0_INT_vect)
{
    // Two vertical counters for a total of 4 counter states
    static uint8_t counter0 = 0;
    static uint8_t counter1 = 0;

    // Capture the state of the pushbuttons from the port pin
    uint8_t pb_sample = PORTA.IN;
    // Detect a change in state
    uint8_t pb_edge = pb_sample ^ pb_debounced_state;

    // Update counters
    // If the state of the pushbutton has changed, increment the counter
    counter1 = (counter1 ^ counter0) & pb_edge;
    counter0 = ~counter0 & pb_edge;

    // Save previous debounced state
    uint8_t pb_previous_state = pb_debounced_state;

    // Update debounced state if counter is 3 or immediately on falling edge
    pb_debounced_state ^= (counter1 & counter0) | (pb_edge & pb_previous_state);

    // Update falling/rising edge
    pb_falling_edge = pb_edge & pb_previous_state;
    pb_rising_edge = pb_edge & pb_debounced_state;

    TCB0.INTFLAGS = TCB_CAPT_bm;
}
\end{minted}
This code allows us to debounce up to 8 pushbuttons using a single
8-bit variable. The debounced state of the pushbuttons is stored in
\mintinline{c}{pb_debounced_state}. This variable updates if the state
of the pushbutton is consistent over 3 samples, or if a falling edge is
detected.
\chapter{State Machines}
A state machine or finite state machine (FSM) is a mathematical model
of computation in which a machine can only exist in one of a finite
number of states. The machine transitions between states in response to
inputs, and can perform actions during transitions. A state machine is
fully defined by its list of states, initial state, and the conditions
for transitioning between states.
\section{Moore and Mealy Machines}
In a Moore machine, outputs are determined only by the current state of
the machine. This is preferred in hardward due to the simplicity of the
implementation.

In a Mealy machine, outputs are determined by the current state and the
inputs. This is preferred in software due to its flexibility and
because it leads to a simpler state machine. Outputs are produced as a
result of a transition, and not as a result of a state.
\section{State Machine Implementation}
To translate a state machine into a C program, we can make use of an
enumerated type. Enumerated types are a special type of data type that
allows us to define a set of named constants.

Enumerated types can be used to implement a state machine as follows:
\begin{itemize}
    \item Each enumerator can be used to represent a state.
    \item A \mintinline{c}{switch} statement can be used to implement
          the behaviour in each state.
    \item An \mintinline{c}{if} statement can be used to implement the
          conditions for transitioning between states.
\end{itemize}
\begin{minted}{c}
typedef enum
{
    START,
    STATE1,
    STATE2
} state_t;

// Initial state
state_t state = START;

while (1)
{
    // State machine
    switch (state)
    {
        case START:
            if (condition1)
                // Transition if condition is met
                state = STATE1;
            break;
        case STATE1:
            if (condition2)
                // Transition if condition is met
                state = STATE2;
            break;
        case STATE2:
            if (condition3)
                // Go back to start if condition is met
                state = START;
            break;
        default:
            // Invalid state, reset to start
            state = START;
            break;
    }
}
\end{minted}
\section{Enumerated Types}
Enumerated types are defined similarly to structures, via the
\mintinline{c}{enum} keyword, and can be anonymous, or named. The
values of an enumerated type are constants, called enumerators, that
are assigned an integer value starting from 0.
\begin{minted}{c}
typedef enum
{
    FALSE,
    TRUE
} boolean_t;

boolean_t b = TRUE; // b is assigned the value 1

b == TRUE; // TRUE is assigned the value 1, so this is true
b == 0; // FALSE is assigned the value 0, so this is false
\end{minted}
While they can be compared to integers, it is recommended to use the
enumerators in comparisons.

Enumerated types can also be defined with explicit values, and can be
used to represent bitmasks.
\begin{minted}{c}
typedef enum
{
    MONDAY = 0b00000001,
    TUESDAY = 0b00000010,
    WEDNESDAY = 0b00000100,
    THURSDAY = 0b00001000,
    FRIDAY = 0b00010000,
    SATURDAY = 0b00100000,
    SUNDAY = 0b01000000,
    WEEKEND = SATURDAY | SUNDAY,
    WEEKDAY = MONDAY | TUESDAY | WEDNESDAY | THURSDAY | FRIDAY
} DAYS;

enum DAYS d = SATURDAY; // d is assigned the value 0b00100000

if (d & WEEKEND) // Check if d is a weekend day
    // Do something
\end{minted}
\section{Switch Statements}
A \mintinline{c}{switch} statement is a control structure that allows
us to select a block of code to execute based on the value of an
expression. When a case is matched, a \mintinline{c}{break} statement
may be used to prevent the program from falling through the case,
however this may be omitted if two states perform the same tasks. The
\mintinline{c}{default} case is executed if no case is matched.
\chapter{Serial Protocols}
A serial protocol is an agreed-upon standard by which two devices can
communicate with each other, enabling them to exchange data. UART and
SPI are standards for transmitting data, but do not ascribe any meaning
to the data.
\section{Serial Protocol Design}
\subsection{Requirements for a Serial Protocol}
A serial protocol must:
\begin{itemize}
    \item able to receive data during a transmission
    \item able to recover from errors
    \item engage in flow control
    \item be simple to implement/understand for both the transmitter
          and receiver
\end{itemize}
\subsection{Symbols}
A symbol is the fundamental data type used in serial communication
protocols, which can be comprised of several bits. The number of bits
is usually set by the underlying medium and depends on the baud rate.

Smaller symbols are more flexible and allow for more symbols to be
transmitted, whereas larger symbols are more efficient and allow for
more data to be transmitted.
\subsection{Messages}
If the information to be exchanged can be entirely encoded within a
single symbol, there is no need for a message structure. However, more
complex protocols require a message structure for large quantities of
data or information of variable length. This is done by dividing the
communication into discrete messages.
\subsection{Encoding}
The choice of encoding may also be of concern, depending on the
communication medium, symbol length, and other factors such as human
readability. For example using the entire ASCII character set may not
be desireable as it is not human readable. Human readable encoding
schemes usually limit the number of symbols to a small subset of the
ASCII character set:
\begin{itemize}
    \item ASCII \mintinline{c}{32-126} (\mintinline{c}{0x20-0x7E})
          which uses 8-bit symbols % chktex 29, chktex 8
    \item Base64 (\mintinline{c}{0-9}, \mintinline{c}{A-Z},
          \mintinline{c}{a-z}, \mintinline{c}{+}, \mintinline{c}{/})
          which encodes 6 bits into an 8-bit symbol % chktex 8
    \item Hexadecimal (\mintinline{c}{0-9}, \mintinline{c}{A-F}) which
          encodes 4-bits per symbol % chktex 8
\end{itemize}
\subsection{Message Structure}
Messages typically contain the following information:
\begin{enumerate}
    \item A \textbf{start sequence} to indicate the beginning of a
          message
    \item An \textbf{identifier} to indicate what type of message is
          being sent (if the protocol requires multiple messages)
    \item A \textbf{payload} containing the data specific to the
          message
    \item A provision for \textbf{escape sequences} to allow for
          special characters (e.g., arbitrary data is not confused with
          start sequences)
    \item A \textbf{checksum} (or message digest), to ensure the
          integrity of the message
    \item A \textbf{stop sequence} to indicate the end of a message
\end{enumerate}
\subsection{Start Sequences}
As a serial communication transmits a sequence of symbols with no
structure, there is no guarantee that the entire message is received.
To address this, a start sequence is used to indicate the beginning of
a message. The start sequence is usually a fixed number of unique
symbols that do not appear in the payload, providing a synchronisation
point. If payloads need to contain arbitrary sequences of symbols,
escape sequences may be used.
\subsection{Multi-Symbol Start Sequences}
While a single symbol start sequence is simple, a multi-symbol start
sequence has potential benefits:
\begin{itemize}
    \item Reduces need for escape sequences
    \item Reduced likelihood of misinterpreting corrupted data as a
          start sequence
\end{itemize}
\subsection{Sub-Symbol Start Sequences}
If messages are encoded with fewer than 8 bits per symbol, the
remaining bits can be used to encode a start sequence. For example, in
UTF-8 encoding, the high bit is always cleared in the first byte of a
sequence.
\subsection{Message Identifiers}
Serial protocols often have multiple categories of messages that may be
transmitted. Commonly a fixed-length identifier is transmitted so that
the receiver can respond with the appropriate action, or know when to
expect a payload.
\subsection{Payloads}
A payload is used when a message identifier alone is insufficient to
convey the information required. Payloads should be as small as
possible to reduce the overhead of the protocol, as longer payloads
increase the risk of transmission errors, so it may be preferrable to
split a large payload into multiple messages.
\subsubsection{Payload Length}
Payloads may be of both fixed and variable length, depending on the
protocol.
\begin{itemize}
    \item For fixed length payloads, the message type itself may define
          the payload length and hence will know when to expect the end
          of the payload.
    \item For variable length payloads, the payload length is encoded
          in the message itself, by either specifying the length within
          the payload, or by using a delimiter to indicate the end of
          the payload.
\end{itemize}
\subsubsection{Variable Length Payloads}
When variable length payloads are expected, two strategies are commonly
used:
\begin{itemize}
    \item Encode the payload length at the start of the payload, either
          with one or two symbols for (1--256) or (1--65536) bytes
          respectively.
    \item Use a \textbf{sentinel} to indicate the end of the payload.
          This is a special symbol that is not used in the payload,
          such as a null character.
\end{itemize}
Note that the second strategy requires the receiver to be able to buffer the entire payload before processing it.
If the payload is too large, this may not be possible.
\subsection{Escape Sequences}
When there is potential ambiguity as to whether a given symbol or
sequence of symbols is part of a sequence, a payload, or a sentinel for
a payload, escape sequences may be necessary to handle certain
characters.

In C, a backslash (\mintinline[escapeinside=||]{c}{|\backslash|}) is
used to escape characters like the double quote (\mintinline{c}{"}) to
tell the compiler that the character is not the end of the string. % chktex 18

In a serial protocol, this may not appropriate as the backslash may be
missed during transmission, and the next character may be treated as a
start sequence for example. To address this, the escape sequence should
not contain the symbol it is escaping. Instead, an alternate sequence
is used to represent symbols when they are part of a payload or other
contexts. Note that the escape sequence itself may be part of the
payload, so it is important to also account for this.
\subsection{Handshakes}
A protocol where the sender purely transmits data does not know whether
the same information has been received and handled by the receiver. As
such, it is common for protocols where only side is sending information
and the other is receiving, to have the receiving side acknowledge what
it has received (if the serial communications medium is half-duplex or
full-duplex).

The most common form of handshakes are \textbf{ACK} and \textbf{NACK}
messages (for acknowledged and not acknowledged).
\begin{itemize}
    \item ACK indicates that the message was received, and that the
          contents were understood.
    \item NACK indicates that the message was received, but that there
          was an error in the message. For example, the message was
          malformed, failed its checksum, or was unable to be
          processed.

          In this situation, the sender may retransmit the message, or
          send a different message.
\end{itemize}
\subsection{Message Verification}
As serial communications are prone to transmission errors, it is
important to verify that the message was received correctly. This is
done by including a checksum in which the transmitter computes a value
based on the contents of the message, such as the sum of the bytes, and
transmits it with the message. The receiver then computes a similar
checksum based on what it receives and verifies that it matches the
transmitted checksum.

This simple checksum detects many transmission errors however, it is
not guaranteed to handle symbols sent out of order, or symbols that are
corrupted without changing the checksum.
\subsection{Flow Control}
When the receiver operates on little power or low storage, the sender
may not be able to send data quickly, as the receiver may not be able
to process it. Hence the sender may response with a flow control
message, such as WAIT, to indicate that it is currently processing the
previous message, and RESUME when it is ready to receive more data.
\section{Serial Protocol Parsing}
Many serial protocols are designed to be simple to parse, due to the
limitations of hardware used in serial communication. However, it is
important to ensure that concerns around timing, state and buffers are
addressed to ensure that the parser is robust and reliable.

As other actions may be performed during the parsing of a message, it
may not be possible to use blocking functions such as
\mintinline{c}{scanf()}. Similarly, a periodic interrupt may not be
feasible if the symbols are not sent frequently. As such, a better
choice may be to use the \mintinline{c}{UART_RX} interrupt to handle a
single character at a time, and then use a state machine to handle the
parsing of the message.

This state machine can be placed within the interrupt handler, or in a
separate function that is called in the main loop. The state machine
can be implemented to consider the following:
\begin{enumerate}
    \item \textbf{Idle}: The parser is waiting for a start sequence (and ignores all other symbols)
    \item \textbf{Start Sequence}: The parser is receiving the start sequence
    \item \textbf{Message Identifier}: The parser is receiving the message identifier
    \item \textbf{Payload}: The parser is receiving the payload (could be separate states for various identifiers)
    \item \textbf{Checksum}: The parser is receiving the checksum
\end{enumerate}
\end{document}
